\laborator{Основы программирования в пакете Mathcad}

\goal ознакомиться с возможностями языка программирования математического пакета Mathcad; рассмотреть основные операторы и приемы программирования в Mathcad.

\subsubsection{Теория}
Язык программирования Mathcad содержит все элементы языка высокого уровня, необходимые для математических расчетов. Кроме того, он содержит дополнительно сотни встроенных функций и операторов системы Mathcad, имеет возможности численного и символьного расчета различных величин, и поэтому по эффективности не уступает профессиональным системам программирования.

Все операторы и элементы языка программирования Mathcad расположены на рабочей панели «Программирование». Данная панель содержит восемь пиктограмм, соответствующих операторам языка программирования Mathcad. 
Чтобы написать программу, прежде всего для нее должен быть создан специальный обособленный от остального документа блок. Выглядит он как черная вертикальная линия с маркерами, в которые заносятся те или иные выражения алгоритма. Чтобы построить единичный элемент программного блока, нажмите пиктограмму «Добавить строку программы» панели «Программирование» (клавишей \keys{]}). При этом в области курсора появится следующий объект:
\begin{center}
	\includegraphics{old-prog-f1.png}
\end{center}

Обычно программа содержит больше чем две строки, поэтому лучше сразу задать блок из 5-6 маркеров. Сделать это можно, последовательно нажав нужное количество раз соответствующую кнопку панели «Программирование» или \keys{]}.

Чтобы добавить строку к уже созданному блоку, поставьте курсор в ту строку блока, выше или ниже которой должна быть введена строка. Положение вставляемого маркера определяется положением синей вертикальной черты курсора. Если она находится слева от выделенного выражения, то маркер будет добавлен выше выделенной строки, если справа – то ниже. Чтобы развернуть курсор в нужную сторону, нажмите клавишу \keys{Insert}.

Программный блок можно создать и внутри уже заданного блока. Для этого используйте один из стандартных способов, поставив курсор в маркер любого из операторов программирования:
\begin{center}
	\includegraphics{old-prog-f2.png}
\end{center}

Созданный таким образом блок выглядит как параллельная главному блоку линия. Выражения, внесенные в него, будут обособлены от остальной программы, и выполнение соответствующих действий будет связано только с оператором, к которому относится внутренний блок.
Для присвоения значений переменным и функциям в программах Mathcad используется специальный оператор: «$\leftarrow$» --- «Локальное определение» (панель «Программирование») или сочетание \keys{\shift + [} . Использовать оператор обычного присваивания «:=» в программах нельзя.

Присваивание значений в программах имеет ряд особенностей. Присвоение величин используемым алгоритмом функциям и переменным может быть произведено как в самой программе, так и выше нее. Если значение переменной или функции присваивается в программе посредством оператора «$\leftarrow$», то такая переменная или функция будет являться локальной, то есть она будет видимой только в рамках программы. Как-то повлиять на объекты вне программы она не сможет (равно как извне к ней нельзя будет получить доступ). Если переменная или функция задаются выше программы с помощью оператора «:=», то она будет обладать глобальной видимостью, то есть такая переменная или функция будет доступна любому нижележащему объекту, в том числе и коду программ. Однако программа может только прочитать значение глобальной переменной или вызвать глобальную функцию. Как-то изменить значение глобальной переменной или функции программа не может. Если программа должна осуществлять какую-то модификацию объекта (например, возводить все элементы массива в квадрат), то результат своей работы она должна возвращать.

Локальные переменные и функции имеют приоритет над глобальными в рамках «родной» программы. Это означает, что если имеется локальная и глобальная переменные (или функции) с одним именем, то обращение по этому имени будет адресоваться к локальной переменной (или функции).

\primer{Задайте переменную \mc{А}, далее создайте программный блок, где этой же переменной присвойте новое значение и выведите его на печать. После выполнения блока распечатайте значение переменной. Значение переменной, полученной в результате выполнения программы, присвойте новой переменной, что бы она была доступна вне программы. }

\begin{center}
	\includegraphics{old-prog-z1.png}
\end{center}

\primer{Значение переменной, полученной в результате выполнения предыдущей программы, присвойте новой переменной, чтобы она была доступна вне программы. }

\begin{center}
	\includegraphics{old-prog-z2.png}
\end{center}

Как видно из задания, в результате выполнения программы значение переменной \mc{А} изменено не было.

Mathcad позволяет в теле программы задать локальную пользовательскую функцию. Создаются локальные функции точно так же, как обычные (только в качестве оператора присваивания используется ). Вызвать локальную функцию можно только из нижележащих строк программы. Вне программы она не доступна.

\primer{Задайте пользовательскую функцию $f(x,y,z) = sin(x)+sin(y)+sin(z)$. Обратитесь в теле программы к данной функции несколько раз и распечатайте полученное значение.}
\begin{center}
	\includegraphics{old-prog-z3.png}
\end{center}

Если необходимо ввести большое число локальных переменных, то это можно сделать, используя следующие приемы:
\begin{enumerate}
	\item С помощью матрицы строки. В маркере программного блока создается матрица-строка из n элементов, после этого определяется каждая переменная в маркерах данной матрицы.
	\begin{center}
		\includegraphics{old-prog-z3-1.png}
	\end{center}
	
	\item Проведение присваивания в строке через запятую. Для этого поставьте курсор в маркер программного блока и последовательным нажатием клавиши \keys{,} введите необходимое количество маркеров, после чего в каждом из них задайте переменную либо пропишите требуемое действие.
	\begin{center}
		\includegraphics{old-prog-z3-2.png}
	\end{center}
	
\end{enumerate}

Данные приемы позволяют сократить длину программы.
Построение программ проводится с использованием специальных управляющих операторов, вроде оператора цикла \mc{for} или оператора условия \mc{if}. Чтобы задать нужный оператор, используйте соответствующие кнопки панели <<Программирование>>. Просто набрать оператор с клавиатуры нельзя – он будет воспринят системой Mathcad как неизвестная функция.

Такие операторы, как \mc{if}, \mc{for}, \mc{while}, активируют код, помещенный в их левый маркер, в том случае, если выполняется условие в правом. Для задания условия используются такие операторы панели <<Булева алгебра>>, как <<$=$>>, <<$>$>>, <<$<$>> и т.д. Можно задать и комплекс условий, задействовав оператор логического И <<$\land$>> панели «Булева алгебра», или оператор логического ИЛИ <<$\lor$>> .

\subsection*{Операторы цикла (\mc{for}, \mc{while})}

Оператор простого цикла \mc{for} позволяет организовать выполнение операции или проверку условия для ряда конкретных значений переменной. Оператор for задается с помощью команды панели «Программирование» или сочетанием клавиш \keys{\ctrl + \shift + '}. Оператор \mc{for} имеет три маркера:
\begin{center}
	\includegraphics{old-prog-f3.png}
\end{center}


В двух верхних маркерах, соединенных символом принадлежности, задается имя переменной, по которой организуется цикл, и ряд принимаемых ею значений. В нижнем маркере определяется операция или комплекс операций, которые должны быть выполнены для каждого значения переменной. Ряд значений переменной обычно представляет собой последовательность целых чисел, которая задается с помощью ранжированной переменной. Для этого в правый верхний маркер вводится оператор ранжированной переменной (панель «Матрица»), по умолчанию ряд будет содержать целые значения с шагом 1. Если значения переменной должны изменяться с меньшим или большим шагом, то это можно сделать, введя в правом маркере оператора ранжированной переменной через запятую первое и второе значения в ряде переменной (разница между ними и задаст шаг).
\begin{center}
	\includegraphics{old-prog-f4.png}
\end{center}


Если операция или комплекс операций должны быть просчитаны при ряде некоторых конкретных значений переменной, причем ряд этот нельзя задать математически в общем виде, его можно непосредственно определить в правом верхнем маркере оператора \mc{for} в виде вектора:
\begin{center}
	\includegraphics{old-prog-f5.png}
\end{center}

С помощью второго оператора цикла \mc{while} (Пока) (сочетание клавиш \keys{\ctrl+]} ) можно организовать цикл, который будет работать до тех пор, пока выполняется некоторое условие. Оператор \mc{while} имеет два маркера, в которые вводятся соответственно условие работы цикла и выражения операций, которые должны быть проделаны на каждом его витке:
\begin{center}
	\includegraphics{old-prog-f6.png}
\end{center}


В цикле \mc{while} количество его витков не нужно определять явно. Итерации будут совершаться до тех пор, пока будет выполняться условие в правом маркере.

\begin{center}
	\includegraphics{old-prog-f7.png}
\end{center}

Если возникает необходимость прервать работу цикла, то можно использовать оператор \mc{break} (Прервать). Ввиду того, что цикл бывает нужно остановить при выполнении некоторого условия, оператор \mc{break} почти всегда используется с условным оператором \mc{if}.

\primer{Используя операторы цикла, замените значения элементов произвольного массива A размерностью 3х3 их квадратами. Полученные значения присвойте новой переменной.}
\begin{center}
	\includegraphics{old-prog-z4.png}
\end{center}


\subsection*{Условные операторы (\mc{if}, \mc{otherwise})}
Условный оператор \mc{if} имеет два маркера:
\begin{center}
	\includegraphics{old-prog-f8.png}
\end{center}
В правый маркер вводится условие, в левый --- операция, которая должна быть проделана в случае, если условие выполнится (если же оно не выполняется, система просчитывает программу, пропуская данный фрагмент). Как уже говорилось, в маркер оператора может быть внесено несколько условий.

Оператор \mc{otherwise} (Иначе) предназначен для определения того действия, которое должно быть выполнено, если условие оператора \mc{if} (Если) окажется неистинным.

Если по условию необходимо выполнить не одну, а несколько операций, то в этом случае курсор помещается в левый маркер и нажатием пиктограммы «Добавить строку программы» или клавиши \keys{]} добавляется необходимое количество строк.
\begin{center}
	\includegraphics{old-prog-f9.png}
\end{center}

В этом случае операции, которые необходимо выполнить, записываются в блок после оператора \mc{if} (или \mc{otherwise}).

\primer{Используя условные операторы, создайте блок программы для определения переменных \mc{А} и \mc{B}, которые будут принимать значения в зависимости от переменной \mc{N}. Если \mc{N>0}, то \mc{А=10}; \mc{В=20}; если \mc{N<0}, то \mc{А=5}, \mc{В=15}. Результат присвоить переменой \mc{С}.}
\begin{center}
	\includegraphics{old-prog-f10.png}
\end{center}


Вопросы для самоконтроля:
\begin{enumerate}
	\item В какой пиктограмме содержатся операторы и элементы языка программирования Mathcad?
	\item Чем визуально отличается блок программы от остального документа?
	\item Какой оператор используется для присвоения значений переменным в программных блоках?
	\item Какие операторы цикла есть в пакете Mathcad? В чем их отличие?
	\item Какие условные операторы цикла есть в пакете Mathcad?
\end{enumerate}
