\laborator{Определение условий фазовых равновесий пар-жидкость}

\goal на основе экспериментальных данных по давлению насыщенных паров чистых компонентов определить условия фазовых равновесий пар-жидкость бинарной идеальной смеси при различных термодинамических условиях; результаты представить в виде диаграмм фазового равновесия ($у-х$; $Р-х, у$; $Т-х, у$).

Теория
Химической промышленности в основном приходится иметь дело с системами, представляющими собой смеси газов и жидкостей, которые необходимо разделять. При проектировании процессов разделения подобных систем необходимо иметь данные о равновесных свойствах смесей. Условия фазового равновесия удобно представлять в виде фазовой диаграммы, которая описывает влияние температуры, давления и состава на вид и число фаз, которые могут сосуществовать при данных условиях. Число фаз определяется согласно правилу фаз Гиббса. Вид фаз, которые могут сосуществовать в конкретных условиях, зависит от химической природы компонентов. Представление фазового равновесия более удобно в графическом виде, чем в табличном, так как позволяет охватить взаимные связи между переменными (рис. 5.1).

В соответствии с правилом фаз Гиббса для любой термодинамически равновесной системы число параметров состояния $С$ (степеней свободы), которые можно изменять, сохраняя число существующих фаз $Ф$ неизмененным, определяется выражением

,

где $К$ – число компонентов системы, $N$ – число параметров состояния системы, имеющих одно и то же значение во всех фазах (обычно температура $Т$ и давление $Р$, $N$ = 2). Величина $С$ определяет число параметров состояния, которые нужно задать для однозначного определения состояния системы.

Из правила фаз следует, что в однокомпонентной системе ($К$ = 1) в однофазной области ($Ф$ = 1) состояние системы определяется двумя параметрами ($С$ = 2) $Т$ и $Р$ (можно произвольно и одновременно изменять оба параметра до тех пор, пока система не окажется на одной из ограничивающих область линий). На линиях фазового равновесия ($Ф$ = 2) состояние системы определяется одним параметром ($С$ = 1) (произвольно можно менять только один из параметров – $Р$ или $Т$ (если изменяется $Т$, то $Р$ будет изменяться в соответствии с ходом кривой, и наоборот)). В трехфазной области ($Ф$ = 3) число степеней свободы равно нулю ($С$ = 0), что соответствует тройной точке (в этой точке ни один из параметров не может быть изменен, равновесное сосуществование трех фаз однокомпонентной системы возможно только при строго определенных значениях $Т$ и $Р$.). Для любой системы число фаз максимально, когда $С$ = 0. При увеличении числа компонентов $К$ в системе растет число параметров состояния $С$, необходимых для однозначного определения состояния системы.

Как было показано выше, для однокомпонентной системы максимальное число степеней свободы равно двум, соответственно фазовое равновесие такой системы можно представить на плоскости в виде двухкоординатной фазовой диаграммы (рис. 5.1). Для двухкомпонентной системы число степеней свободы равно трем. Взаимозависимость трех переменных можно отразить лишь посредством пространственных диаграмм (рис. 5.2). На практике работать с трехмерными диаграммами уже сложнее, чем с двухмерными. Для трехкомпонентной системы число степеней свободы равно четырем и представить фазовую диаграмму такой системы графически уже затруднительно. 

Однако возможен альтернативный подход, позволяющий упростить построение пространственных диаграмм. Например, в случае двухкомпонентной смеси можно использовать серию плоскостных диаграмм при фиксированном значении третьей переменной. Пример таких диаграмм показан на рис. 5.2, это плоскости Р1, Р2, Р3 и т.д.

Обычно используют следующие виды диаграмм:
\begin{itemize}
\item в виде изобарических сечений $(Р = const)$, которые демонстрируют влияние $Т$ и общего состава смеси на состояние фаз системы (пример такой диаграммы для бинарной смеси представлен на рис. 5.3); 
\item в виде изотермических сечений $(Т = const)$, которые демонстрируют влияние $Р$ и общего состава смеси на состояние фаз системы;
\item в виде изоплет (диаграммы постоянного состава), которые демонстрируют влияние $Р$ и $Т$ на состояние фаз системы при постоянном составе смеси.	
\end{itemize} 

При построении диаграммы для удобства обычно ограничиваются показом отношений между определенным числом фаз, например пар~-- жидкость, жидкость~-- жидкость, жидкость~-- твердая фаза. Для решения практических задач необходимы лишь ограниченные области $Т$ и $Р$, число рассматриваемых фаз также ограничено. При проектировании большинства процессов химической технологии наибольшей интерес представляет набор диаграмм для систем пар – жидкость. В этом случае, например диаграмма фазового равновесия (см. рис. 5.3) является избыточной, так как интерес представляет только ее верхняя часть, где представлена система пар – жидкость. 

На рис. 5.4 представлены основные типы фазовых диаграмм систем пар – жидкость. Представленный вид диаграмм характерен для систем с плавным изменением температур кипения растворов в диапазоне температур между температурами кипения чистых жидкостей, включая системы, подчиняющиеся закону Рауля. Вид этих зависимостей определяется свойствами компонентов смеси. Например, для идеальной смеси зависимость $Р~-- х$ при $T=const$ в соответствии с законом Рауля будет прямой линией.

Для технических расчетов наиболее важной является диаграмма $T~-- x, у$ --- зависимость температур кипения жидкости и конденсации паров от составов жидкой и паровой фаз, так как процессы перегонки в промышленных аппаратах протекают в изобарных условиях. Для проведения расчетов использование данных о фазовом равновесии в графическом виде не всегда удобно, особенно при решении задач оптимизации. В этой связи условия фазовых равновесий удобно представлять в виде системы уравнений, решение которой позволит определить искомые параметры равновесия. 

Условия равновесия двух фаз $n$ --- компонентной системы при заданной температуре Т определяются следующей системой уравнений:

где $\mu$ – химический потенциал (верхний индекс обозначает фазу, нижний – компонент). Чтобы определить условия фазового равновесия, необходимо решить систему уравнений (1). Численное решение данной системы можно получить, если выражения для химического потенциала и давления представлены в явном виде.
В однокомпонентном случае химический потенциал определяется следующим образом:

Отсюда можно получить явный вид для химического потенциала, однако это возможно только если известно уравнение состояния. В случае использования уравнения состояния идеального газа  выражение будет иметь вид

\begin{enumerate}
	\item Виды уравнений регрессии. Обобщенная регрессия.
	\item Методы определения параметров уравнения регрессии.
	\item Определение остаточной ошибки уравнения регрессии.
	\item Как определяется число независимых параметров, полностью определяющих состояние системы?
	\item Какие существуют типы диаграмм фазового равновесия?
	\item Методы расчета фазовых равновесий при различных термодинамических условиях (P-x, y, T-x, y)
\end{enumerate}
1. 