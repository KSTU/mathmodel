\laborator{Основы математического пакета Mathcad}

\goal ознакомиться с пользовательским интерфейсом математического пакета Mathcad; изучить способы задания различного типа переменных и функций; освоить приемы работы с графическим и текстовым редакторами.

Теория
Mathcad – программное средство, являющееся средой для выполнения на компьютере разнообразных расчетов. Mathcad включает в свой состав три редактора – формульный (по умолчанию), текстовый и графический. Благодаря им обеспечивается принятый в математике способ записи функций и выражений и получение результатов вычислений, произведенных компьютером в виде таблиц и графиков. Взаимодействие пользователя с компьютером осуществляется с помощью удобного графического интерфейса, включающего пиктограммы, диалоговые окна, меню, опции и другие «инструменты», располагаемые на экране дисплея. Mathcad включает множество операторов, встроенных функций и алгоритмов решения разнообразных математических задач.
Пользовательский интерфейс системы создан так, что пользователь, имеющий элементарные навыки работы с Windows-приложениями, может сразу начать работу с Mathcad. Работа с документами Mathcad обычно не требует обязательного использования возможностей главного меню, так как основные из них дублируются пиктограммами управления. Пиктограммы управления представляют собой перемещаемые наборные панели, которые содержат заготовки из шаблонов математических знаков (цифр, знаков арифметических операций, матриц, знаков интегралов, производных и т.д.). Наборные панели могут быть выведены на экран все сразу или в нужном количестве. Кнопки на наборных панелях вводят в месте расположения курсора общепринятые и специальные математические знаки и операторы. На рис.1.1 представлены наборные панели Mathcad, вызываемые из панели инструментов «Математическая» (Вид  Панели инструментов  Математическая):
Калькулятор – содержит арифметические инструменты;
Графики – содержит инструменты графиков;
Матрицы – содержит векторные и матричные операции;
Вычисления – содержит инструменты вычислений;
Высшая математика – содержит операторы математического анализа;
Булева алгебра – содержит инструменты булевой алгебры;
Программирование – содержит инструменты программирования;
Греческий алфавит – содержит символы греческого алфавита;
Символьно – содержит символические операторы.


Рис.1.1. Наборные панели пакета Mathcad

По умолчанию ввод осуществляется в вычислительный блок. Для запуска формульного редактора достаточно установить курсор мыши в любом свободном месте окна редактирования и щелкнуть левой клавишей. Появится визир в виде маленького красного крестика. Его можно перемещать клавишами перемещения курсора. Визир указывает место, с которого можно начинать набор формул – вычислительных блоков. Подготовка вычислительных блоков облегчается благодаря выводу шаблона при задании того или иного оператора при помощи наборных панелей (см. рис.1.1). Для ввода данных можно указать курсором мыши на нужный шаблон данных и щелчком левой ее клавиши ввести его.
Чтобы определить переменную, необходимо выполнить следующие действия:
– набрать имя переменной (регистр имеет значение);
– ввести оператор присваивания «:=», сделать это можно нажатием кнопки «Определение» панели «Калькулятор» или «Вычисления» либо с помощью сочетания клавиш «Shift»+ «:»;
– на место черного маркера, появившегося справа от оператора присваивания, ввести значение переменной ().
Mathcad позволяет работать со следующими типами переменных:
- скалярная величина;
- вектор (который также может быть задан с помощью оператора ранжированной переменной);
- матрица.
Если переменная определяется как скалярная величина, то ее численное значение вводится с клавиатуры на место черного маркера. Работа в Mathcad осуществляется аналогично программированию на языках интерпретаторах, т.е. программа выполняется слева направо и сверху вниз. Это означает, что переменные должны быть определены в тексте программы левее или выше места их использования.

Задание 1

Используя формульный редактор, задайте переменную А, присвойте ей значение, целая часть которого – день вашего рождения, десятичная – месяц (например 1 января), и распечатайте ее экране используя оператор числового вычисления «=» на панели «Калькулятор»:



Попробуйте распечатать переменную А левее и выше ее места определения и посмотрите на результат.

Массивы (векторы, матрицы), по принципу задания их элементов, можно разделить на две группы:
– векторы и матрицы, при задании которых не существует прямой связи между величиной элемента и его индексами;
– ранжированные переменные – векторы, величина элементов которых напрямую определяется индексом.
В Mathcad реализовано несколько способов задания массивов:
– задание матрицы или вектора вручную с помощью команды «Вставить матрицу»;
– определение матрицы последовательным заданием каждого элемента;
– использование ранжированных переменных;
– создание таблицы данных;
– чтение из внешнего файла, и др.
Наиболее простым способом задания матрицы является использование специального окна «Вставить матрицу» (вызывается нажатием кнопки «Матрица или вектор» на наборной панели «Матрицы» или сочетанием клавиш «Ctrl»+«М»). Параметры создаваемой матрицы или вектора можно определить в окошках «Строк» и «Столбцов». В результате в документ будет вставлена заготовка с черными маркерами вместо элементов, в которые необходимо ввести нужные значения:



Элементы матрицы могут быть как числами, так и выражениями. Если среди выражений или символов, выступающих в качестве элементов матрицы, есть неизвестные или параметры, то они обязательно должны быть численно определены выше. В противном случае матрица должна быть задана как функция.




Задание 2

2.1. Задайте вектор В, содержащий три произвольных численных значения, и распечатайте его:

		

2.2. Задайте квадратную матрицу М, содержащую девять произвольных значений, и распечатайте ее:

		

В случае заданной матрицы всегда можно получить значение любого его элемента, используя его матричные индексы. Матричные индексы равняются номеру строки и столбца, на пересечении которых элемент находится. В математике отсчет строк и столбцов принято начинать с единицы. В программировании же начальные индексы обычно равняются нулю. По умолчанию в Mathcad строки и столбцы тоже начинаются с нуля. В том случае если такая система не удобна, то точку отсчета можно изменить на вкладке «Опции рабочего листа» (вкладка основного меню «Инструменты») изменив параметр «Начальный индекс» на единицу.
Итак, чтобы получить значение какого-то матричного элемента, нужно ввести имя матрицы с соответствующими индексами и поставить «=». Для задания индексов на панели «Матрицы» имеется специальная кнопка «Нижний индекс», которой соответствует клавиша «[». Нажав ее, вы увидите, что на месте будущего индекса, чуть ниже текста имени матрицы, появится черный маркер . В него через запятую следует ввести значения индексов. На первом месте при этом должен стоять номер строки, а на втором – номер столбца. При выделении элемента вектора нужно задать только индекс строки. Индексы также могут быть определены и через выражения или специальные функции.
Помимо одного элемента можно очень просто выделить и целые матричные столбцы. Чтобы это сделать, нужно использовать специальный оператор панели «Матрицы» «Столбец матрицы» (также вводится сочетанием «Ctrl»+«6»)  и в черный маркер ввести требуемый номер столбца. В том случае, если требуется выделить строку, матрицу необходимо транспонировать (оператор «Транспонирование матрицы» той же рабочей панели).

Задание 3

3.1. Обратитесь к центральному элементу матрицы М и распечатайте его значение:



3.2. Обратитесь к первому столбцу матрицы М и распечатайте его значение:



3.3. Обратитесь к второму элементу вектора В и распечатайте его значение:



3.4. Измените начало отсчета векторов и матриц с 0 на 1 и повторите задания 3.1. – 3.3. (после выполнения задания верните начало отсчета на 0).

Ранжированные переменные. Одной из разновидностей задания массивов является использование так называемых ранжированных переменных. Ранжированная переменная – это разновидность вектора, особенностью которого является непосредственная связь между индексом элемента и его величиной. В Mathcad ранжированные переменные очень активно используются как аналог программных операторов цикла (например, при построении графиков).
Простейшим примером ранжированной переменной является вектор, значение элементов которого совпадает с их индексами. Для задания такой ранжированной переменной выполните следующую последовательность действий.
1.  Введите имя переменной и оператор присваивания.
2.  Поставив курсор в маркер значения переменной, нажмите кнопку «Диапазон переменных» панели «Матрицы». При этом будет введена заготовка в виде двух маркеров, разделенных точками:

Данную заготовку можно вставить с помощью клавиши «;».
3.  В левый черный маркер заготовки ранжированной переменной введите ее первое значение, в правый – последнее.

Шаг изменения ранжированной переменной при ее задании с помощью описанного способа постоянен и равен единице. Однако при необходимости его можно сделать и произвольным. Дли этого нужно, поставив после левой границы интервала запятую, ввести второе значение ранжированной переменной. Разность между первым и вторым ее значением и определит шаг. 
Использование ранжированных переменных во многом основано на том, что большинство математических действий в Mathcad над векторами осуществляется точно так же, как над простыми числами. Так, например, существует возможность вычисления значений практически любой встроенной и пользовательской функции от вектора. При этом в качестве результата будет выдан вектор, составленный из значений функции при величинах переменных, равных соответствующим элементам исходного вектора.

Задание 4

4.1. Задайте ранжированную переменную С от 0 до 3 (шаг изменения переменной по умолчанию) и распечатайте ее значения:



4.2. Задайте новый шаг изменения ранжированной переменной Е (например 0.5) и распечатайте ее значения:



Таблицы. Все экспериментальные данные обрабатываются в Mathcad в виде матриц. Однако использовать описанные выше стандартные методы задания массивов для этого крайне неудобно. В этом случае можно использовать так называемую таблицу ввода. Чтобы ее вызвать, задействуйте команду Вставить  Данные  Таблица главного меню (соответствующая этой команде кнопка имеется и на панели «Стандартные»). В документ будет введена следующая заготовка:



Присвоив будущей матрице определенное имя (вводится в черный маркер слева от знака присваивания), попробуйте определиться с ее размерами. Если она не очень большая, можно сразу расширить пустую таблицу до нужной величины. Для этого следует использовать специальные черные маркеры, появляющиеся на контуре таблицы при ее выделении. Никаких ограничений на размеры таблица ввода не имеет. Создание таблицы повторяет заполнение обычных матриц, однако в таблицах нельзя использовать формулы.
Так как таблицы являются для Mathcad такими же матрицами, как заданные стандартными способами, с ними можно проводить все те же преобразования, что и со стандартными массивами. Если отобразить содержание таблицы через ее имя, оно визуализируется (при стандартных настройках) как простая матрица.

Задание 5

Задайте таблицу Т, состоящую из 2 столбцов и 5 строк (нумерацию массивов начать с 0). Распечатать отдельно каждый столбец таблицы и один из элементов таблицы:



Операторы
Операторы – это символ или последовательность символов, обозначающих то или иное математическое действие. Операторы, выполняющие все основные арифметические действия, расположены на панели «Калькулятор» (рис.1.1). Ввод основных арифметических операторов может быть также осуществлен с клавиатуры.

Функции
Функции в Mathcad делятся на две группы:
- функции пользователя;
- встроенные функции.
Техника использования функций обоих типов абсолютно идентична, а вот задание отличается принципиально. Для задания функции пользователя необходимо выполнить следующие действия:
- ввести имя функции;
- после имени ввести пару круглых скобок, в которых через запятую необходимо указать все переменные, от которых зависит функция;
- ввести оператор присваивания «:=»;
- на место черного маркера, появившегося справа от оператора присваивания, необходимо задать вид функции.
В выражение определяемой функции могут входить как непосредственно переменные, так и другие встроенные и пользовательские функции. 

Задание 6

6.1. Задайте пользовательскую функцию Y=Х2+Х-1, подставьте в качестве аргумента ранее определенные переменные А и С, а также выражение 2А:


6.2. Задайте вектор D как функцию от переменных Х и Y. Подставьте вместо переменных Х и Y произвольные значения и распечатайте вектор D.

		

Встроенные функции – это функции, заданные в Mathcad изначально. Чтобы их использовать, достаточно корректно набрать имена функций с клавиатуры. Наиболее распространенные из них можно ввести с панели «Калькулятор», это синус, косинус, тангенс, натуральный и десятичный логарифмы, экспонента. Для того чтобы задать все остальные встроенные функций Mathcad, нужно открыть специальное окно «Вставить функции».

Задание 7

7.1. Задайте пользовательскую функцию , 
подставьте в качестве аргумента ранее определенные переменные А и С, а также вектор D:


Графический редактор
Все основные типы графиков и инструменты работы с ними расположены на рабочей панели «Графики» (рис.1.1). Здесь можно найти ссылки на семь типов графиков. В курсе математического моделирования потребуются только график кривой в двумерной декартовой системе координат. Соответствующая графическая заготовка вызывается из наборной панели «Графики»  X-Y график :

 

На рисунке область ввода значения аргумента и функции обозначена черными метками по центру соответствующих осей. По оси абсцисс откладывается аргумент, например Х, по оси ординат функция – например Y(Х). Границы по осям выставляются автоматически, но предусмотрено их изменение вручную. Переход к редактированию осуществляется постановкой курЛабораторная работа 1

Основы математического пакета Mathcad

Цель работы: ознакомиться с пользовательским интерфейсом математического пакета Mathcad; изучить способы задания различного типа переменных и функций; освоить приемы работы с графическим и текстовым редакторами.

Теория
Mathcad – программное средство, являющееся средой для выполнения на компьютере разнообразных расчетов. Mathcad включает в свой состав три редактора – формульный (по умолчанию), текстовый и графический. Благодаря им обеспечивается принятый в математике способ записи функций и выражений и получение результатов вычислений, произведенных компьютером в виде таблиц и графиков. Взаимодействие пользователя с компьютером осуществляется с помощью удобного графического интерфейса, включающего пиктограммы, диалоговые окна, меню, опции и другие «инструменты», располагаемые на экране дисплея. Mathcad включает множество операторов, встроенных функций и алгоритмов решения разнообразных математических задач.
Пользовательский интерфейс системы создан так, что пользователь, имеющий элементарные навыки работы с Windows-приложениями, может сразу начать работу с Mathcad. Работа с документами Mathcad обычно не требует обязательного использования возможностей главного меню, так как основные из них дублируются пиктограммами управления. Пиктограммы управления представляют собой перемещаемые наборные панели, которые содержат заготовки из шаблонов математических знаков (цифр, знаков арифметических операций, матриц, знаков интегралов, производных и т.д.). Наборные панели могут быть выведены на экран все сразу или в нужном количестве. Кнопки на наборных панелях вводят в месте расположения курсора общепринятые и специальные математические знаки и операторы. На рис.1.1 представлены наборные панели Mathcad, вызываемые из панели инструментов «Математическая» (Вид  Панели инструментов  Математическая):
Калькулятор – содержит арифметические инструменты;
Графики – содержит инструменты графиков;
Матрицы – содержит векторные и матричные операции;
Вычисления – содержит инструменты вычислений;
Высшая математика – содержит операторы математического анализа;
Булева алгебра – содержит инструменты булевой алгебры;
Программирование – содержит инструменты программирования;
Греческий алфавит – содержит символы греческого алфавита;
Символьно – содержит символические операторы.


Рис.1.1. Наборные панели пакета Mathcad

По умолчанию ввод осуществляется в вычислительный блок. Для запуска формульного редактора достаточно установить курсор мыши в любом свободном месте окна редактирования и щелкнуть левой клавишей. Появится визир в виде маленького красного крестика. Его можно перемещать клавишами перемещения курсора. Визир указывает место, с которого можно начинать набор формул – вычислительных блоков. Подготовка вычислительных блоков облегчается благодаря выводу шаблона при задании того или иного оператора при помощи наборных панелей (см. рис.1.1). Для ввода данных можно указать курсором мыши на нужный шаблон данных и щелчком левой ее клавиши ввести его.
Чтобы определить переменную, необходимо выполнить следующие действия:
– набрать имя переменной (регистр имеет значение);
– ввести оператор присваивания «:=», сделать это можно нажатием кнопки «Определение» панели «Калькулятор» или «Вычисления» либо с помощью сочетания клавиш «Shift»+ «:»;
– на место черного маркера, появившегося справа от оператора присваивания, ввести значение переменной ().
Mathcad позволяет работать со следующими типами переменных:
- скалярная величина;
- вектор (который также может быть задан с помощью оператора ранжированной переменной);
- матрица.
Если переменная определяется как скалярная величина, то ее численное значение вводится с клавиатуры на место черного маркера. Работа в Mathcad осуществляется аналогично программированию на языках интерпретаторах, т.е. программа выполняется слева направо и сверху вниз. Это означает, что переменные должны быть определены в тексте программы левее или выше места их использования.

Задание 1

Используя формульный редактор, задайте переменную А, присвойте ей значение, целая часть которого – день вашего рождения, десятичная – месяц (например 1 января), и распечатайте ее экране используя оператор числового вычисления «=» на панели «Калькулятор»:



Попробуйте распечатать переменную А левее и выше ее места определения и посмотрите на результат.

Массивы (векторы, матрицы), по принципу задания их элементов, можно разделить на две группы:
– векторы и матрицы, при задании которых не существует прямой связи между величиной элемента и его индексами;
– ранжированные переменные – векторы, величина элементов которых напрямую определяется индексом.
В Mathcad реализовано несколько способов задания массивов:
– задание матрицы или вектора вручную с помощью команды «Вставить матрицу»;
– определение матрицы последовательным заданием каждого элемента;
– использование ранжированных переменных;
– создание таблицы данных;
– чтение из внешнего файла, и др.
Наиболее простым способом задания матрицы является использование специального окна «Вставить матрицу» (вызывается нажатием кнопки «Матрица или вектор» на наборной панели «Матрицы» или сочетанием клавиш «Ctrl»+«М»). Параметры создаваемой матрицы или вектора можно определить в окошках «Строк» и «Столбцов». В результате в документ будет вставлена заготовка с черными маркерами вместо элементов, в которые необходимо ввести нужные значения:



Элементы матрицы могут быть как числами, так и выражениями. Если среди выражений или символов, выступающих в качестве элементов матрицы, есть неизвестные или параметры, то они обязательно должны быть численно определены выше. В противном случае матрица должна быть задана как функция.




Задание 2

2.1. Задайте вектор В, содержащий три произвольных численных значения, и распечатайте его:



2.2. Задайте квадратную матрицу М, содержащую девять произвольных значений, и распечатайте ее:



В случае заданной матрицы всегда можно получить значение любого его элемента, используя его матричные индексы. Матричные индексы равняются номеру строки и столбца, на пересечении которых элемент находится. В математике отсчет строк и столбцов принято начинать с единицы. В программировании же начальные индексы обычно равняются нулю. По умолчанию в Mathcad строки и столбцы тоже начинаются с нуля. В том случае если такая система не удобна, то точку отсчета можно изменить на вкладке «Опции рабочего листа» (вкладка основного меню «Инструменты») изменив параметр «Начальный индекс» на единицу.
Итак, чтобы получить значение какого-то матричного элемента, нужно ввести имя матрицы с соответствующими индексами и поставить «=». Для задания индексов на панели «Матрицы» имеется специальная кнопка «Нижний индекс», которой соответствует клавиша «[». Нажав ее, вы увидите, что на месте будущего индекса, чуть ниже текста имени матрицы, появится черный маркер . В него через запятую следует ввести значения индексов. На первом месте при этом должен стоять номер строки, а на втором – номер столбца. При выделении элемента вектора нужно задать только индекс строки. Индексы также могут быть определены и через выражения или специальные функции.
Помимо одного элемента можно очень просто выделить и целые матричные столбцы. Чтобы это сделать, нужно использовать специальный оператор панели «Матрицы» «Столбец матрицы» (также вводится сочетанием «Ctrl»+«6»)  и в черный маркер ввести требуемый номер столбца. В том случае, если требуется выделить строку, матрицу необходимо транспонировать (оператор «Транспонирование матрицы» той же рабочей панели).

Задание 3

3.1. Обратитесь к центральному элементу матрицы М и распечатайте его значение:



3.2. Обратитесь к первому столбцу матрицы М и распечатайте его значение:



3.3. Обратитесь к второму элементу вектора В и распечатайте его значение:



3.4. Измените начало отсчета векторов и матриц с 0 на 1 и повторите задания 3.1. – 3.3. (после выполнения задания верните начало отсчета на 0).

Ранжированные переменные. Одной из разновидностей задания массивов является использование так называемых ранжированных переменных. Ранжированная переменная – это разновидность вектора, особенностью которого является непосредственная связь между индексом элемента и его величиной. В Mathcad ранжированные переменные очень активно используются как аналог программных операторов цикла (например, при построении графиков).
Простейшим примером ранжированной переменной является вектор, значение элементов которого совпадает с их индексами. Для задания такой ранжированной переменной выполните следующую последовательность действий.
1.  Введите имя переменной и оператор присваивания.
2.  Поставив курсор в маркер значения переменной, нажмите кнопку «Диапазон переменных» панели «Матрицы». При этом будет введена заготовка в виде двух маркеров, разделенных точками:

Данную заготовку можно вставить с помощью клавиши «;».
3.  В левый черный маркер заготовки ранжированной переменной введите ее первое значение, в правый – последнее.

Шаг изменения ранжированной переменной при ее задании с помощью описанного способа постоянен и равен единице. Однако при необходимости его можно сделать и произвольным. Дли этого нужно, поставив после левой границы интервала запятую, ввести второе значение ранжированной переменной. Разность между первым и вторым ее значением и определит шаг. 
Использование ранжированных переменных во многом основано на том, что большинство математических действий в Mathcad над векторами осуществляется точно так же, как над простыми числами. Так, например, существует возможность вычисления значений практически любой встроенной и пользовательской функции от вектора. При этом в качестве результата будет выдан вектор, составленный из значений функции при величинах переменных, равных соответствующим элементам исходного вектора.

Задание 4

4.1. Задайте ранжированную переменную С от 0 до 3 (шаг изменения переменной по умолчанию) и распечатайте ее значения:



4.2. Задайте новый шаг изменения ранжированной переменной Е (например 0.5) и распечатайте ее значения:



Таблицы. Все экспериментальные данные обрабатываются в Mathcad в виде матриц. Однако использовать описанные выше стандартные методы задания массивов для этого крайне неудобно. В этом случае можно использовать так называемую таблицу ввода. Чтобы ее вызвать, задействуйте команду Вставить  Данные  Таблица главного меню (соответствующая этой команде кнопка имеется и на панели «Стандартные»). В документ будет введена следующая заготовка:



Присвоив будущей матрице определенное имя (вводится в черный маркер слева от знака присваивания), попробуйте определиться с ее размерами. Если она не очень большая, можно сразу расширить пустую таблицу до нужной величины. Для этого следует использовать специальные черные маркеры, появляющиеся на контуре таблицы при ее выделении. Никаких ограничений на размеры таблица ввода не имеет. Создание таблицы повторяет заполнение обычных матриц, однако в таблицах нельзя использовать формулы.
Так как таблицы являются для Mathcad такими же матрицами, как заданные стандартными способами, с ними можно проводить все те же преобразования, что и со стандартными массивами. Если отобразить содержание таблицы через ее имя, оно визуализируется (при стандартных настройках) как простая матрица.

Задание 5

Задайте таблицу Т, состоящую из 2 столбцов и 5 строк (нумерацию массивов начать с 0). Распечатать отдельно каждый столбец таблицы и один из элементов таблицы:



Операторы
Операторы – это символ или последовательность символов, обозначающих то или иное математическое действие. Операторы, выполняющие все основные арифметические действия, расположены на панели «Калькулятор» (рис.1.1). Ввод основных арифметических операторов может быть также осуществлен с клавиатуры.

Функции
Функции в Mathcad делятся на две группы:
- функции пользователя;
- встроенные функции.
Техника использования функций обоих типов абсолютно идентична, а вот задание отличается принципиально. Для задания функции пользователя необходимо выполнить следующие действия:
- ввести имя функции;
- после имени ввести пару круглых скобок, в которых через запятую необходимо указать все переменные, от которых зависит функция;
- ввести оператор присваивания «:=»;
- на место черного маркера, появившегося справа от оператора присваивания, необходимо задать вид функции.
В выражение определяемой функции могут входить как непосредственно переменные, так и другие встроенные и пользовательские функции. 

Задание 6

6.1. Задайте пользовательскую функцию Y=Х2+Х-1, подставьте в качестве аргумента ранее определенные переменные А и С, а также выражение 2А:


6.2. Задайте вектор D как функцию от переменных Х и Y. Подставьте вместо переменных Х и Y произвольные значения и распечатайте вектор D.



Встроенные функции – это функции, заданные в Mathcad изначально. Чтобы их использовать, достаточно корректно набрать имена функций с клавиатуры. Наиболее распространенные из них можно ввести с панели «Калькулятор», это синус, косинус, тангенс, натуральный и десятичный логарифмы, экспонента. Для того чтобы задать все остальные встроенные функций Mathcad, нужно открыть специальное окно «Вставить функции».

Задание 7

7.1. Задайте пользовательскую функцию , 
подставьте в качестве аргумента ранее определенные переменные А и С, а также вектор D:


Графический редактор
Все основные типы графиков и инструменты работы с ними расположены на рабочей панели «Графики» (рис.1.1). Здесь можно найти ссылки на семь типов графиков. В курсе математического моделирования потребуются только график кривой в двумерной декартовой системе координат. Соответствующая графическая заготовка вызывается из наборной панели «Графики»  X-Y график :



На рисунке область ввода значения аргумента и функции обозначена черными метками по центру соответствующих осей. По оси абсцисс откладывается аргумент, например Х, по оси ординат функция – например Y(Х). Границы по осям выставляются автоматически, но предусмотрено их изменение вручную. Переход к редактированию осуществляется постановкой курсора в соответствующий регистр по осям абсцисс и ординат, на рисунке это черные метки в начале и конце каждой оси, отмеченные по краям снизу черными угловыми линиями. 
В ряде случаев (например, если в графике приходится очень часто менять диапазоны по осям или необходимо выделить конкретную область) можно задать векторы данных самостоятельно. Сделать это можно с помощью оператора ранжированной переменной (задание ранжированной переменной рассматривалось выше). В этом случае аргумент функции задается в виде вектора, т.е. переменная и функция будут заданы в виде двух соразмерных векторов, по которым будет построен график.
Mathcad позволяет отображать до 16 графических зависимостей в одной системе координат, что удобно для визуального контроля и сравнения полученных результатов. Чтобы добавить к уже имеющемуся графику еще один, выполните следующую последовательность действий.
1. Установите курсор справа от выражения, определяющего координаты последнего ряда данных по оси Y (предварительно выделив его).
2. Нажмите клавишу «,». При этом курсор опустится на строку ниже и появится чистый маркер.
3. В появившийся маркер введите выражение для новой функции или имя функции.
С помощью описанного метода можно построить графики функций одной переменной. Если же кривые, которые нужно отобразить на одной области, зависят от различных переменных, то их, аналогично добавлению новых функций, следует ввести через запятую в нижний маркер в том же порядке, что и соответствующие им функции.
Изменение настроек отображения осей и графиков (если необходимо изменить тип или цвет кривой) осуществляется с помощью диалогового окна «Форматирование выбранной декартовой плоскости», вызвать которое можно, дважды щелкнув левой кнопкой мыши на области графика.

Задание 8

8.1. Построите графическую зависимость для функции Y(X), определенной в задании 7:



8.2. Измените диапазон графика путем определения аргумента как ранжированной переменной от 0 до 5 (шаг по умолчанию):



8.3. Вручную измените диапазон по оси X или Y, а также вид кривой; добавьте на график вторую функцию и также измените вид ее кривой. Измените шаг ранжированной переменной (для сглаживания функции):




Текстовый редактор
Текстовый режим в Mathcad позволяет создавать всевозможные комментарии и качественно оформлять решенные задачи.
Чтобы ввести текстовую область, нажмите сочетание клавиш Shift+ «‘» (курсор ввода при этом должен располагаться на чистом участке документа). При этом появится специальная рамка, а курсор ввода приобретет вид красной вертикальной линии.
Набирается текст в Mathcad точно так же, как в любом текстовом редакторе. Если известно, сколько места на листе займет комментарий, то можно сразу растянуть текстовую область до нужных размеров. Чаще же текст просто вводят в область, обрывая строки с помощью клавиши «Enter», когда они достигают нужной длины (это приходится делать, так как Mathcad не выполняет автоматически переносов слов).

Задание 9сора в соответствующий регистр по осям абсцисс и ординат, на рисунке это черные метки в начале и конце каждой оси, отмеченные по краям снизу черными угловыми линиями. 
В ряде случаев (например, если в графике приходится очень часто менять диапазоны по осям или необходимо выделить конкретную область) можно задать векторы данных самостоятельно. Сделать это можно с помощью оператора ранжированной переменной (задание ранжированной переменной рассматривалось выше). В этом случае аргумент функции задается в виде вектора, т.е. переменная и функция будут заданы в виде двух соразмерных векторов, по которым будет построен график.
Mathcad позволяет отображать до 16 графических зависимостей в одной системе координат, что удобно для визуального контроля и сравнения полученных результатов. Чтобы добавить к уже имеющемуся графику еще один, выполните следующую последовательность действий.
1. Установите курсор справа от выражения, определяющего координаты последнего ряда данных по оси Y (предварительно выделив его).
2. Нажмите клавишу «,». При этом курсор опустится на строку ниже и появится чистый маркер.
3. В появившийся маркер введите выражение для новой функции или имя функции.
С помощью описанного метода можно построить графики функций одной переменной. Если же кривые, которые нужно отобразить на одной области, зависят от различных переменных, то их, аналогично добавлению новых функций, следует ввести через запятую в нижний маркер в том же порядке, что и соответствующие им функции.
Изменение настроек отображения осей и графиков (если необходимо изменить тип или цвет кривой) осуществляется с помощью диалогового окна «Форматирование выбранной декартовой плоскости», вызвать которое можно, дважды щелкнув левой кнопкой мыши на области графика.

Задание 8

8.1. Построите графическую зависимость для функции Y(X), определенной в задании 7:



8.2. Измените диапазон графика путем определения аргумента как ранжированной переменной от 0 до 5 (шаг по умолчанию):



8.3. Вручную измените диапазон по оси X или Y, а также вид кривой; добавьте на график вторую функцию и также измените вид ее кривой. Измените шаг ранжированной переменной (для сглаживания функции):

 		


Текстовый редактор
Текстовый режим в Mathcad позволяет создавать всевозможные комментарии и качественно оформлять решенные задачи.
Чтобы ввести текстовую область, нажмите сочетание клавиш Shift+ «‘» (курсор ввода при этом должен располагаться на чистом участке документа). При этом появится специальная рамка, а курсор ввода приобретет вид красной вертикальной линии.
Набирается текст в Mathcad точно так же, как в любом текстовом редакторе. Если известно, сколько места на листе займет комментарий, то можно сразу растянуть текстовую область до нужных размеров. Чаще же текст просто вводят в область, обрывая строки с помощью клавиши «Enter», когда они достигают нужной длины (это приходится делать, так как Mathcad не выполняет автоматически переносов слов).

Задание 9