\laborator{Моделирование реакций в реакторах с различной структурой потоков}

\goal для различных моделей структуры потока составить математическую модель протекающих в аппарате химических реакций; на основе полученных моделей провести моделирование работы реактора; результаты оценки эффективности для различных вариантов аппарата представить графически (в виде зависимости селективности и степени превращения от времени пребывания в аппарате).

Теория
Основными критериями эффективности проведения химических реакций являются конверсия и селективность. 
Конверсия (степень превращения) --- это величина, характеризующая превращение сырья в результате реакции. Она представляет собой отношение общего количества исходного реагента, вступившего в реакцию $m_A^R$ , к количеству реагента, взятого для проведения реакции $m_A^0$:
\begin{equation}\label{eq:rea.alpha}
	\alpha_A = \dfrac{m_A^R}{m_A^0}
\end{equation}
В случае, если в реакцию вступает несколько веществ, конверсию рассчитывают по наиболее ценному веществу.  

Селективность --- это отношение количества исходного реагента, расходуемого на целевую реакцию $m_A^Ц$, к общему количеству реагента, пошедшего на реакцию $m_A^R$:
\begin{equation}\label{eq:rea.s}
	S_B= \dfrac{m_A^Ц}{m_A^R}
\end{equation}

Выход реакции --- отношение количества продукта $m_B^Ц$ к максимальному, теоретически возможному количеству продукта $m_B^{Max}$:
\begin{equation}\label{eq:rea.beta}
	\beta_B=\dfrac{m_B^Ц}{m_B^{Max}}
\end{equation}
Теоретически максимальное значение количества продукта рассчитывается из материального баланса при следующих условиях: скорость побочных реакций равна нулю, обратимые целевые реакции проходят до условий равновесия.

Выбор этих показатели для оценки эффективности связан с тем, что:
\begin{itemize}
	\item в себестоимости процессов химического превращения стоимость сырья составляет порядка 50\ --\ 60\ \%.
	\item энергетические затраты, связанные с нагревом, охлаждением, перемещением сырья и продуктов реакции, зависят от того, сколько сырья приходится возвращать на повторную переработку. При конверсии 100\ \% все сырье пропускается через реактор один раз, при меньшей конверсии часть сырья приходится снова возвращать реактор;
	\item продукты реакции из реактора необходимо разделять и для их разделения потребуется немало разнообразного оборудования, многочисленные циклы нагрев -- охлаждение, испарение -- конденсация... Следовательно, дополнительные затраты энергии, а значит, и средств.
\end{itemize}

Селективность и конверсия подчиняются строгим законам, а именно законам химической термодинамики и кинетики. Это справедливо на этапе создания технологии. Однако когда мы говорим о проектировании аппаратов для реализации данных процессов, то на величину селективности, а особенно степени превращения, большое влияние оказывает время пребывания молекулы вещества в реакторе. Оценочное значение среднего времени пребывания получают из отношения объема реактора к объемному расходу поступающего вещества. В действительности же траектория движения элементов потока в аппарате может быть чрезвычайно сложной, что может приводить к существенным отклонениям времени пребывания конкретного элементарного объема от среднего значения. Структура потоков в аппарате зависит от конструкции реактора и гидродинамических условий. 

При математическом моделировании работы аппаратов используются различные модели структуры потоков, основными из них являются следующие: 
\begin{itemize}
\item модель идеального вытеснения;
\item модель идеального смешения;
\item диффузионная модель;
\item ячеечная модель.
\end{itemize}

Необходимо отметить, что данные модели представляют собой упрощенную картину реальной структуры потоков в аппаратах, которая значительно сложнее. Например, многие химические реакции сопряжены с процессами теплообмена, которые вследствие неоднородности поля температур (наблюдаемой при теплообмене) дополнительно усложняют характер движения элементов потока, например за счет конвективных токов, вызванных естественной конвекцией.

\subsection*{Изотермический реактор идеального вытеснения}
Наибольшее распространение на производстве получили изотермические реакторы непрерывного действия. В них температура поддерживается на постоянном уровне как во времени, так и в пространстве, что обеспечивает постоянство константы скорости химической реакции (на оптимальном уровне) и, следовательно, упрощает задачу проведения процесса. Рассмотрим реактор данного типа, в котором протекают следующие реакции:
\begin{equation*} 
\begin{aligned} 
A+B \xleftrightarrow[k_2]{k_1} C + \Delta H_1\\
C \xrightarrow{k_3} D + \Delta H_2
\end{aligned} 
\end{equation*}
где $k$ --- константы скорости реакции.
Математические модели аппаратов составляются на основе записи уравнений кинетики, материального и теплового баланса. В случае изотермического реактора тепловым балансом можно пренебречь. Для реактора данного типа математические модели для различных структур потока представлены в работе \cite{klinov-mm2009}. Рассмотрим варианты идеализированных моделей реактора.

Изменением температуры можно пренебречь при малых значениях теплового эффекта реакций ($\Delta H_1 \approx 0$, $\Delta H_2 \approx 0$) или компенсации выделенного~/~поглощенного тепла за счет внешнего теплообмена .
В стационарных условиях изменение концентрации компонентов описывается уравнением:
\begin{equation} \label{eq:rea.miv1}
\nu \dfrac{d C_A}{d x} = i
\end{equation}
где $\nu$ --- скорость движения среды, $C_A$ --- концентрация компонента A, $x$ --- длинна реактора, $i$ --- сток или приход массы. В случае отсутствия межфазного массопереноса, сток или приход массы обуславливается только лишь химической реакцией. Компонент А участвует в 1 и 2 (обратимой реакции). В первой реакции компонент А расходуется, следовательно это сток массы и скорость 1 реакции будет со знаком --. Обратная реакция увеличивает концентрация компонента А, и скорость реакции запишем со знаком +. Суммарное выражение изменения концентрации запишется в виде $i=-r_1+r_2$.

Скорость реакции равна произведению константы скорости на концентрацию компонентов, вступающих в реакцию, так скорость первой реакции равна: 
\begin{equation} \label{eq:rea.react1}
r_1=k_1 C_A C_B
\end{equation}
Используя определение скорости $\nu=\frac{d x}{d \tau}$ и выражение \eqref{eq:rea.react1}
перепишем уравнение \eqref{eq:rea.miv1} для описания изменения концентрации в зависимости от времени пребывания компонента в реакторе $\tau$:
\begin{equation}
\dfrac{d C_A}{d \tau} = -k_1 C_A C_B +k_2 C_C
\end{equation}
Составляя аналогичные уравнения для остальных веществ, получим систему дифференциальных уравнений:
\begin{equation}\label{eq:rea.sys1}
\left\lbrace 
\begin{gathered} 
\dfrac{d C_A} {d \tau} = -k_1 C_A C_B +k_2 C_C \\
\dfrac{d C_B} {d \tau} = -k_1 C_A C_B +k_2 C_C \\
\dfrac{d C_C} {d \tau} = k_1 C_A C_B -k_2 C_C - k_3 C_C \\
\dfrac{d C_D} {d \tau} = k_3 C_C \\
\end{gathered} 
\right.
\end{equation}

В результате решения системы уравнений \eqref{eq:rea.sys1} получим концентрации каждого из компонентов в конкретный момент времени. Полученные таким образом векторы решений будут представлять собой зависимость концентраций исходных компонентов и продуктов реакции от времени пребывания в аппарате.

По результатам моделирования можно произвести расчеты показателей конверсии \eqref{eq:rea.alpha}, селективности \eqref{eq:rea.s} и выхода \eqref{eq:rea.beta}, на основе которых производится выбор оптимального типа реактора для проведения процесса. 

\subsection*{Неизотермический реактор идеального вытеснения}
В действительности реакций с нулевым тепловым эффектом практически не существует. Рассмотрим неизотермические реакторы на примере тех же реакций.
Изменение температуры в реакторе идеального вытеснения определяется по выражению:
\begin{equation}\label{eq:rea.dtmiv}
	\nu \rho c_p \dfrac{d T}{d x} = \sum q_i
\end{equation}
где $q_i$ --- подводимый или отводимый поток тепла. Суммарный поток тепла складывается из нескольких составляющих: подвода или отвода теплоты за счет теплоносителя (реакторы с рубашкой, змеевиком и т.д.) и теплоты химических реакций. 

В случае адиабатического реактора все тепло химической реакции идет на изменение температуры в реакторе. Поток тепла за счет химической реакции определяется по выражению:
\begin{equation} \label{eq:rea.qreak}
	{q_r}_q=r_1 \Delta H_1 = k_1 C_A C_B \Delta H_1
\end{equation}

По закону Гесса тепловой эффекта 2 реакции (обратной) будет равен $-\Delta H_1$. Используя определение скорости $\nu=\frac{d x }{d \tau}$ и выражения \cref{eq:rea.qreak,eq:rea.dtmiv} можно записать выражение изменения температуры во времени:
\begin{equation}\label{eq:rea.dtmiv1}
	\dfrac{d T}{d \tau}= \dfrac {k_1(T) C_A C_B \Delta H_1 -k_2(T) C_C \Delta H_1 +k_3(T) C_C \Delta H_2}{\rho c_p} 
\end{equation}
Данным выражением необходимо дополнить систему уравнений \cref{eq:rea.sys1}. Следует заметить, что в выражениях для  неизотермического реактора константа скорости будет функцией температуры и описываться уравнением Аррениуса:
\begin{equation}
	k_i={k_0}_i \exp \left( \dfrac{-E_i}{RT}\right)
\end{equation}
где $k_0$ --- предэкспоненциальный множитель,  $E$ --- энергия активации.

В случае подвода или отвода тепла в выражение \eqref{eq:rea.dtmiv} необходимо добавить член, описывающий тепловой поток $q=K(T-T_T)P$, где $К$ --- коэффициент теплопередачи, $P$ --- периметр сечения поверхности теплопередачи, $T_T$ --- температура теплоносителя. Таким образом изменение температуры можно описать дифференциальным уравнением:
\begin{eqnarray}\label{eq:rea.dtmiv2}
\dfrac{d T}{d \tau}= \dfrac {k_1(T) C_A C_B \Delta H_1 -k_2(T) C_C \Delta H_1 +k_3(T) C_C \Delta H_2}{\rho c_p} + \nonumber \\ + \dfrac{K(T_T-T)P}{\rho c_p} \quad
\end{eqnarray}

Постоянство температуры теплоносителя возможно поддерживать за счет фазового перехода, так, при обогреве реакционной смеси паром, по всей длине реактора температура конденсата будет постоянно. В случае, если температура теплоносителя изменяется по длине реактора, систему дифференциальных уравнений необходимо дополнить изменением температуры теплоносителя. Например если структура потока теплоносителя описывается моделью идеального вытеснения и потоки движутся прямотоком изменение температуры будет записано в виде:
\begin{equation}
	\dfrac{d T_T}{d \tau} = -\dfrac{K(T_T-T)P}{\rho_T {c_p}_T}
\end{equation}
где индексы $T$ --- обозначают свойства теплоносителя.


\subsection*{Изотермический реактор идеального смешения}
В реакторе идеального смешения происходит мгновенное выравнивание полей концентрации и температуры. Уравнения для описания изменения концентрации во времени можно получить исходя из модели идеального вытеснения соответствующей реакции (уравнение \eqref{eq:rea.sys1}) с учетом постоянства концентрации компонентов и температуры в реакторе. Интегрируя уравнение при заданных начальных концентрациях  $C_{0_i}$  для рассматриваемой реакции получим систему алгебраических уравнений:

\begin{equation}\label{eq:rea.syssmis}
\left\lbrace 
\begin{gathered} 
\dfrac{C_A - {C_A}_0} {\tau} = -k_1 C_A C_B +k_2 C_C \\
\dfrac{C_B - {C_B}_0} {\tau} = -k_1 C_A C_B +k_2 C_C \\
\dfrac{C_C - {C_C}_0} {\tau} = k_1 C_A C_B -k_2 C_C - k_3 C_C \\
\dfrac{C_D - {C_D}_0} {\tau} = k_3 C_C \\
\end{gathered} 
\right.
\end{equation}

\subsection*{Незотермический реактор идеального смешения}


%\zd 
 
 
Вопросы для самоконтроля
\begin{enumerate}
	\item По каким критериям оценивается эффективность проведения химических реакций?
	\item Какие факторы оказывают влияние на критерии эффективности проведения химических реакций?
	\item Что понимают под структурой потока в аппарате?
	\item Какие существуют модели структуры потоков?
	\item На основе чего составляются математические модели реакторов?
	\item Как провести сравнение эффективности работы реакторов?
\end{enumerate}