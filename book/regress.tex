\laborator{Регрессионный анализ, методы аппроксимации}

\goal ознакомится с возможностями математического пакета Mathcad при решении задач регрессионного анализа; ознакомится с процедурами численного решения алгебраических уравнений и систем уравнений, реализованными в данном пакете. 

Теория
Регрессионный анализ – статистический метод исследования зависимости между зависимой переменной $y$ и одной или несколькими независимыми переменными $x_1$,$x_2$,...,$x_n$.
В статистике для оценки силы корреляционной зависимости двух случайных величин используется коэффициент корреляции $r_{xy}$. Определяется он отношением математического ожидания произведения отклонений случайных величин от их средних значений к произведению среднеквадратичных отклонений этих величин:
\begin{equation}
	r_{xy}=\dfrac{\dfrac{1}{n} \sum\limits_{i=1}^{n} (x_i-\bar{x}) (y_i-\bar{y}), }{\sigma_x \sigma_y}
\end{equation}
где $x$ и $y$ --- среднеквадратичные отклонения, определяемые следующим образом:
\begin{equation}
	\sigma_x=\sqrt{\dfrac{1}{n} \sum\limits_{i=1}^{n}(x_i-\bar{x})^2},
\end{equation}
где $\bar{x}=\dfrac{1}{n} \sum\limits_{i=1}^{n} x_i$.
Аналогичные выражения записываются и для y.

В Mathcad коэффициент корреляции двух выборок по данной формуле можно подсчитать с помощью встроенной функции \mc(corr(x,y)) (где x и y --- векторы, между которыми определяется коэффициент корреляции). Если коэффициент корреляции равен по модулю единице, то между случайными величинами существует линейная зависимость. Если же он равен нулю, то случайные величины независимы. Промежуточные значения $r_ху$ говорят о том, что две выборки коррелируют в некоторой степени.

Если две выборки коррелируют, то можно установить зависимость между ними. Для вычисления регрессии в Mathcad имеется ряд функций. Обычно эти функции создают кривую или поверхность определенного типа, которая минимизирует ошибку между собой и имеющимися данными. Функции отличаются прежде всего типом кривой или поверхности, которую они используют, чтобы аппроксимировать данные.

Конечный результат регрессии --- функция, с помощью которой можно оценить значения в промежутках между заданными точками. Расхождение полученной функции регрессии с экспериментальными данными можно оценить через относительную среднюю и максимальную ошибку:

\begin{equation}
	err_i= \dfrac{\left| y_i^э - y(x^э_i) \right|}{y_i^э} 100\ \% ,
\end{equation}
\begin{equation}
	err_{av}=\dfrac{1}{n}\sum\limits_{i=1}^{n} err_i ,
\end{equation}
где $y_i^э$, $y(x_i^э)$ --- экспериментальное и расчетное значение функции; $n$ --- число экспериментальных точек; $err_i$ --- ошибка в i–й точке (из них определяется максимальная);  --- средняя ошибка функции регрессии.

систем уравнений и оператор численного вывода (знак «=»). При вводе слова \mc{Find} можно использовать шрифт любого размера, произвольный стиль, прописные и строчные буквы. В скобках через запятую задайте переменные в том порядке, в котором должны быть расположены в ответе соответствующие им корни. Если результаты решения требуется использовать в дальнейших расчетах, то тогда их необходимо присвоить некоторой переменной .

Различают следующие виды функций регрессии:
Линейная регрессия – эти функции возвращают наклон и смещение линии, которая наилучшим образом аппроксимирует данные.
Если поместить значения $x$ в вектор \mc{VX} и соответствующие значения Y в VY, то линия определяется в виде:

\mc{Y = slope(VX, VY)X + intercept(VX, VY)},

где \mc{slope(VX, VY)} - возвращает скаляр: наклон линии регрессии для данных из VX и VY;
intercept(VX, VY) - возвращает скаляр: смещение по оси ординат линии регрессии для данных из VX и VY.


Вопросы для самоконтроля
\begin{enumerate}
\item Что такое регрессионный анализ?
\item Что такое коэффициент корреляции?
\item Какие виды функций регрессии существуют, в чем их различие?
\item Какие функции решения уравнения с одним неизвестным и системы уравнений используются в MathCad?
\item Что является результатом решения системы нелинейных уравнений?
\item Какие символы должны использоваться в качестве знаков равенства или неравенства при записи уравнений в вычислительном блоке Given-Find?
\end{enumerate}