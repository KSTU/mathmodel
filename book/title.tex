\begin{center}
Министерство образования и науки Российской Федерации

Государственное образовательное учреждение высшего образования

«Казанский национальный исследовательский технологический университет»
\vspace{3 cm}

А.В. Клинов, А.В. Малыгин, Анашкин И.П., Минибаева Л.Р.
\vspace{3 cm}

\textsc{Лабораторный практикум по математическому моделированию химико-технологических процессов}
\vspace{1 cm}

Учебное пособие
\vspace{4 cm}

Казань

КНИТУ

\number\year
\end{center}
\thispagestyle{empty}
\newpage

УДК 66.011:51.74+519.2

ББК 35.11:22.1

К 49

\textbf{Клинов А.В.}

Лабораторный практикум по математическому моделированию химико-технологических процессов: учебное пособие / А.В. Клинов, А.В. Малыгин; М-во образ. и науки РФ, Казан. гос. технол. ун-т. – Казань: КГТУ, 2011. – 100 с.
ISBN



Рассмотрены некоторые задачи математического моделирования химико-технологических процессов: описание свойств веществ и условий фазового равновесия; моделирование процессов разделения и химических превращений в аппаратах. Разобраны математические методы, используемые при решении этих задач, а также их реализация в среде математического пакета Mathcad. 
Предназначено для студентов всех форм обучения, изучающих дисциплину «Математическое моделирование химико-технологических процессов».
Подготовлено на кафедре «Процессы и аппараты химической технологии».

Печатается по решению редакционно-издательского совета Казанского государственного технологического университета





Рецензенты: д-р техн. наук, проф. А.Г. Лаптев
	         канд. техн. наук, доц. Д.А. Шапошников

ISBN 				 Клинов А.В., Малыгин А.В., 2011
				 Казанский государственный
технологический университет, 2011
\thispagestyle{empty}
\newpage
\tableofcontents
\newpage