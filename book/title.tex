\begin{center}
	
Министерство науки и высшего образования Российской Федерации

Федеральное государственное бюджетное образовательное учреждение высшего образования

«Казанский национальный исследовательский технологический университет»
\vspace{3 cm}

А.\,В.~Клинов, А.\,В.~Малыгин, И.\,П.~Анашкин, Л.\,Р.~Минибаева
\vspace{3 cm}

\textsc{Лабораторный практикум по математическому моделированию химико-технологических процессов}
\vspace{1 cm}

Учебное пособие
\vspace{4 cm}

Казань

КНИТУ

\number\year
\end{center}
\thispagestyle{empty}
\newpage

УДК 66.011:51.74+519.2

ББК 

К 

\textbf{Клинов А.В.}


Рассмотрены некоторые задачи математического моделирования химико-технологических процессов: описание свойств веществ и условий фазового равновесия; моделирование процессов разделения и химических превращений в аппаратах. Разобраны математические методы, используемые при решении этих задач, а также их реализация в среде математического пакета Mathcad. 
Предназначено для студентов всех форм обучения, изучающих дисциплину «Моделирование химико"=технологических процессов».
Подготовлено на кафедре «Процессы и аппараты химической технологии».


\thispagestyle{empty}
\newpage
\tableofcontents
\newpage