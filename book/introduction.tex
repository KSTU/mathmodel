\addcontentsline{toc}{section}{Введение}
\section*{Введение}
Математическое моделирование является эффективным ин\-стру\-мен\-том инженерных и научно-исследовательских расчетов, который позволяет решать задачи проектирования новых и повышения эффективности существующих процессов. При этом в качестве объекта исследования выступает не сам аппарат или процесс, а его математическая модель. Надежность получаемых результатов моделирования во многом зависит от теоретических основ, на которых строится модель, и закладываемых в нее допущений.

На химико-технологические процессы влияет большое количество элементарных явлений, таких как гидродинамика, кинетика, фазовое равновесие и т.д., которые, к тому же, и взаимосвязаны. Поэтому эти явления нужно уметь корректно описывать, чтобы правильно отобразить процессы, протекающие в аппарате, и соответственно, получить достоверный результат.

В данном пособии предлагаются примеры моделирования типовых явлений, присутствующих в процессах химической технологии. Рассмотрены задачи описания фазового равновесия в идеальных и неидеальных системах, идентификация параметров модели, моделирование массообменных процессов и химических реакций, протекающих в аппаратах с заданной структурой потока.

Представленное учебное пособие является приложением к пособию «Математическое моделирование химико~-- технологических процессов» \cite{klinov-mm2009}, подготовленному на кафедре ПАХТ. Задачей данного пособия является обучение студентов навыкам составления математических моделей процессов и разработки алгоритмов их численного решения. Для численного решения этих задач предлагается использовать математический пакет Mathcad, который является средой для выполнения на компьютере разнообразных расчетов. Рассматриваются основы работы с данным математическим пакетом. Разобраны основные функции (нахождение коэффициентов регрессии, решение системы простых и дифференциальных уравнений), необходимые для реализации алгоритмов численного решения математических моделей.

В конце каждой лабораторной работы представлены примеры заданий для выполнения. Для увеличения вариативности заданий на языке программирования GNU octave была написана программа (свидетельство о государственной регистрации программы для ЭВМ №2020662252) генерации вариантов заданий для каждой лабораторной работы. Исходный текст программы может быть найден по ссылке https://github.com/kstu/mathmodel/.