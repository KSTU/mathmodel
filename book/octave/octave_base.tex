\laborator{Основы математического пакета GNU Octave}

GNU Octave является высокоуровневым языком программирования, ориентированным на численное решение математических задач. Данный пакет разрабатывался как свободная альтернатива популярному пакету Matlab. В связи с этим синтаксис программ этих двух языках идентичен. Несмотря на меньшее количество подключаемых библиотек, более низкую производительность (отчасти за счет отсутствия поддержки многопотоковых вычислений) пакет Ocatve обладает всем необходимым функционалом как для академического, так и для использования на практике при инженерных расчетах.
На момент написания методического указания, Octave и подключаемый набор библиотек активно развиваются.

\section*{Установка}
Пакет Octave распространяется под лицензией GNU, позволяющих модификацию исходного кода, что позволяет скомпилировать его под множество операционных систем. Для большинства популярных Linux дистрибутивов пакет Octave уже включен в стандартные репозитории. Существуют сборки для Windows, MacOS, BSD (https://www.gnu.org/software/octave/download.html). Также существует портированная на android версия  

\section*{Графическое приложение}



\section*{Переменные}

\section*{Массивы}

\section*{Функции}

\section*{Программирование}

\section*{Работа с файлами}

\section*{Графики}

