\laborator{Основы математического пакета Mathcad}

\goal ознакомиться с пользовательским интерфейсом математического пакета Mathcad; изучить способы задания различного типа переменных и функций; освоить приемы работы с графическим и текстовым редакторами.

Mathcad --- программное средство, являющееся средой для выполнения на компьютере разнообразных расчетов. Mathcad включает в свой состав три редактора – формульный (по умолчанию), текстовый и графический. Благодаря им обеспечивается принятый в математике способ записи функций и выражений и получение результатов вычислений, произведенных компьютером в виде таблиц и графиков. Взаимодействие пользователя с компьютером осуществляется с помощью удобного графического интерфейса, включающего пиктограммы, диалоговые окна, меню, опции и другие «инструменты», располагаемые на экране дисплея. Mathcad включает множество операторов, встроенных функций и алгоритмов решения разнообразных математических задач.

Пользовательский интерфейс системы создан так, что пользователь, имеющий элементарные навыки работы с Windows-приложениями, может сразу начать работу с Mathcad. Элементы управления расположены в на меню в виде ленты, широко применяемой в офисных программных продуктах.

По умолчанию ввод осуществляется в вычислительный блок. Для запуска формульного редактора достаточно установить курсор мыши в любом свободном месте окна редактирования и щелкнуть левой клавишей. Появится визир в виде крестика. Его можно перемещать клавишами перемещения курсора. Визир указывает место, с которого можно начинать набор формул --- вычислительных блоков. Для ввода данных можно указать курсором мыши на нужный шаблон данных и щелчком левой ее клавиши ввести его.



\subsubsection*{Переменные}
Используемые в расчете данные хранятся в переменных. Чтобы определить переменную, необходимо выполнить следующие действия:
\begin{itemize}
	\item набрать имя переменной (регистр имеет значение);
	\item ввести оператор присваивания «:=», сделать через меню  \menu[,]{Математика, Операторы, Определение},  либо с помощью сочетания клавиш \keys{\shift+:};
	\item на место маркера, появившегося справа от оператора присваивания, ввести значение переменной;
	\item также можно ввести размерность переменной.
\end{itemize}
Mathcad позволяет работать со следующими типами переменных:
\begin{itemize}
\item скалярная величина;
\item вектор (который также может быть задан с помощью оператора ранжированной переменной);
\item матрица.
\end{itemize}

Работа в Mathcad осуществляется аналогично программированию на языках интерпретаторах, т.е. программа выполняется слева направо и сверху вниз. Это означает, что переменные должны быть определены в тексте программы левее или выше места их использования.

При задании имен переменных рекомендуется исходить из обозначения данной величины в используемых формулах, и не использовать кириллических символов. При этом не использовать одно и то же обозначение для различных переменных. В качестве примера можно привести следующее: для плотности газа $\rho_g$ и плотности жидкости $\rho_l$ можно использовать индексы. Греческие символы можно нати в меню  \menu[,]{Математика, Символы}. Индекс для скалярной величины можно задать в пункте меню \menu[,]{Математика, Стиль, Нижний индекс} при этом в данном случае индекс будет восприниматься программой как часть имени, и не следует путать его с индексом векторов и матриц.


\subsubsection*{Единицы измерения}
В пакете Mathcad имеется полный набор единиц измерения по международной системе СИ,  американской системе единиц  (USCS) и системе "сантиметр-грамм-секунда" (СГС). Использование единиц измерения для значений в расчете позволяет автоматически проводить определение размерности результирующей величины. Это позволяет значительно снизить ошибки, возникающие при переводе единиц измерения. Поэтому, при решении практических задач рекомендуется задавать размерности для всех используемых величин.

Для задания размерности необходимо при присвоении значения переменной дописать размерность данной величины. Список единиц измерения можно посмотреть в меню \menu[,]{Математика, ЕИ, ЕИ} 

\subsubsection*{Матрицы и таблицы}
Массивы (векторы, матрицы), по принципу задания их элементов, можно разделить на две группы:
\begin{itemize}
\item векторы и матрицы, при задании которых не существует прямой связи между величиной элемента и его индексами;
\item ранжированные переменные --- векторы, величина элементов которых напрямую определяется индексом.
\end{itemize}
В Mathcad реализовано несколько способов задания массивов:
\begin{itemize}
	\item задание матрицы или вектора вручную с помощью команды \menu[,]{Матрицы, Вставить матрицу};
	\item определение матрицы последовательным заданием каждого элемента;
	\item использование ранжированных переменных;
	\item создание таблицы данных;
	\item чтение из внешнего файла, и др.
\end{itemize}

Наиболее простым способом задания матрицы является использованием меню \menu[,]{Матрицы, Вставить матрицу} или сочетанием клавиш \keys{\ctrl+M}. Изменять количество рядов и строк в матрице можно соответствующими командами в меню \menu[,]{Матрицы, Строки и столбцы}.

Элементы матрицы могут быть как числами, так и выражениями. Если среди выражений или символов, выступающих в качестве элементов матрицы, есть неизвестные или параметры, то они обязательно должны быть численно определены выше.

В случае заданной матрицы всегда можно получить значение любого его элемента, используя его матричные индексы. Матричные индексы равняются номеру строки и столбца, на пересечении которых элемент находится. В математике отсчет строк и столбцов принято начинать с единицы. В программировании же начальные индексы обычно равняются нулю. По умолчанию в Mathcad строки и столбцы тоже начинаются с нуля. В том случае если такая система не удобна, то точку отсчета можно изменить переменную \mc{ORIGIN} в меню \menu[,]{Матрицы, Строки и столбцы} или переопределив переменную в самом документе.

Итак, чтобы получить значение какого-то матричного элемента, нужно ввести имя матрицы с соответствующими индексами и поставить «=». Для задания индексов  в меню \menu[,]{Матрицы, Операторы с матрицами, Индекс матрицы}, которой соответствует клавиша \keys{[}. Нажав ее, вы увидите, что на месте будущего индекса, чуть ниже текста имени матрицы, появится маркер ввода индекса. В него через запятую следует ввести значения индексов. На первом месте при этом должен стоять номер строки, а на втором --- номер столбца. При выделении элемента вектора нужно задать только индекс строки. Индексы также могут быть определены и через выражения или специальные функции.

Помимо одного элемента можно очень просто выделить и целые матричные столбцы или строки. Чтобы это сделать, нужно использовать специальные операторы в меню \menu[,]{Матрицы, Операторы с матрицами} (также вводится сочетанием \keys{\ctrl+\shift+C} и \keys{\ctrl+\shift+R}  соответственно)  и в маркер ввести требуемый номер столбца или строки.

При вычислении значений по одной и той же формуле, необходимо использовать оператор векторизации расположенный в меню \menu[,]{Матрицы, Операторы с матрицами} или по горячим клавишам \keys{\ctrl+\shift+\textasciicircum}. При этом при вычислении выражение записанное под знаком векторизаии будет проводится поэлементно. Без оператора векторизации соответственно будут проводится матричные операции.

%пример

Одной из разновидностей задания массивов является использование так называемых ранжированных переменных. Ранжированная переменная --- это разновидность вектора, особенностью которого является непосредственная связь между индексом элемента и его величиной. В Mathcad ранжированные переменные очень активно используются как аналог программных операторов цикла (например, при построении графиков).

Простейшим примером ранжированной переменной является вектор, значение элементов которого совпадает с их индексами. Для задания такой ранжированной переменной выполните следующую последовательность действий:
\begin{itemize}
	\item Введите имя переменной и оператор присваивания.
	\item  Поставив курсор в маркер значения переменной, нажмите кнопку «Диапазон переменных» панели «Матрицы». При этом будет введена заготовка в виде двух маркеров, разделенных точками.	Данную заготовку можно вставить дважды нажав на клавишу \keys{.}.
	\item В левый  маркер заготовки ранжированной переменной введите ее первое значение, в правый – последнее.
\end{itemize}

Шаг изменения ранжированной переменной при ее задании с помощью описанного способа постоянен и равен единице. Однако при необходимости его можно сделать и произвольным. Дли этого нужно, поставив после левой границы интервала запятую, ввести второе значение ранжированной переменной. Разность между первым и вторым ее значением и определит шаг. 

Использование ранжированных переменных во многом основано на том, что большинство математических действий в Mathcad над векторами осуществляется точно так же, как над простыми числами. Так, например, существует возможность вычисления значений практически любой встроенной и пользовательской функции от вектора. При этом в качестве результата будет выдан вектор, составленный из значений функции при величинах переменных, равных соответствующим элементам исходного вектора.

\subsubsection*{Таблицы}
Все экспериментальные данные обрабатываются в Mathcad в виде матриц. Однако использовать описанные выше стандартные методы задания массивов для этого крайне неудобно. В этом случае можно использовать так называемую таблицу ввода. Чтобы ее вызвать, задействуйте команду Вставитьца главного меню (соответствующая этой команде кнопка имеется и на панели «Стандартные»). В документ будет введена следующая заготовка:



Присвоив будущей матрице определенное имя (вводится в черный маркер слева от знака присваивания), попробуйте определиться с ее размерами. Если она не очень большая, можно сразу расширить пустую таблицу до нужной величины. Для этого следует использовать специальные черные маркеры, появляющиеся на контуре таблицы при ее выделении. Никаких ограничений на размеры таблица ввода не имеет. Создание таблицы повторяет заполнение обычных матриц, однако в таблицах нельзя использовать формулы.
Так как таблицы являются для Mathcad такими же матрицами, как заданные стандартными способами, с ними можно проводить все те же преобразования, что и со стандартными массивами. Если отобразить содержание таблицы через ее имя, оно визуализируется (при стандартных настройках) как простая матрица.

\subsubsection*{Функции}
Функции в Mathcad делятся на две группы:
\begin{itemize}
\item функции пользователя;
\item встроенные функции.
\end{itemize}
Техника использования функций обоих типов абсолютно идентична, а вот задание отличается принципиально. Для задания функции пользователя необходимо выполнить следующие действия:
\begin{itemize}
	\item ввести имя функции;
	\item после имени ввести пару круглых скобок, в которых через запятую необходимо указать все переменные, от которых зависит функция;
	\item ввести оператор присваивания \mc{:=};
	\item на место черного маркера, появившегося справа от оператора присваивания, необходимо задать вид функции.
\end{itemize}

В выражение определяемой функции могут входить как непосредственно переменные, так и другие встроенные и пользовательские функции. 

Встроенные функции --- это функции, заданные в Mathcad изначально. Чтобы их использовать, достаточно корректно набрать имена функций с клавиатуры. Наиболее распространенные из них можно ввести с панели «Калькулятор», это синус, косинус, тангенс, натуральный и десятичный логарифмы, экспонента. Для того чтобы задать все остальные встроенные функций Mathcad, нужно открыть специальное окно «Вставить функции».

\subsubsection*{Текстовый редактор}

Текстовый режим в Mathcad позволяет создавать всевозможные комментарии и качественно оформлять решенные задачи.
Чтобы ввести текстовую область, нажмите сочетание клавиш Shift+ «‘» (курсор ввода при этом должен располагаться на чистом участке документа). При этом появится специальная рамка, а курсор ввода приобретет вид красной вертикальной линии.
Набирается текст в Mathcad точно так же, как в любом текстовом редакторе. Если известно, сколько места на листе займет комментарий, то можно сразу растянуть текстовую область до нужных размеров. Чаще же текст просто вводят в область, обрывая строки с помощью клавиши «Enter», когда они достигают нужной длины (это приходится делать, так как Mathcad не выполняет автоматически переносов слов).

\subsubsection*{Численное решение уравнений и систем уравнений}
Для численного поиска решений уравнений с одним неизвестным в Mathcad существует специальная встроенная функция \mc{root}. Функция эта может использоваться в двух различных формах, при этом реализуются разные численные алгоритмы. Так, если определена только одна точка приближения к корню, поиск решений будет осуществляться методом секущих. Если же задан интервал, на котором предположительно локализовано решение, то поиск его будет осуществлен с применением двух модификаций метода бисекции.

Если необходимо найти корень некоторого уравнения, причем известен интервал, в котором он локализован, проще всего использовать функцию \mc{root} с четырьмя аргументами: \mc{rоot (f(x), x, a, b)}, где \mc{f(x)} --- функция, определяющая уравнение, \mc{х} --- переменная, \mc{а} и \mc{b} --- границы интервала локализации. Обязательным условием является то, что значения функции на концах интервала должны быть противоположных знаков. Это связано с особенностью используемых \mc{root} алгоритмов. Если нарушить это условие, система выдаст сообщение об ошибке.

В тех случаях, когда определить границы такой локализации невозможно, следует применять функцию \mc{root} с одной точкой приближения: \mc{rоot(f(x), x)}. В этом случае необходимо перед вызовом функцию \mc{root} задать для переменной \mc{x} начальное приближение.

Важной характеристикой решения является его точность. В Mathcad можно регулировать величину погрешности решения, изменяя значение специальной системной переменной \mc{TOL}. В общем случае, чем меньше \mc{TOL}, тем точнее будет найден корень, но и тем больше времени уйдет на его определение (а также будет выше риск, что численный метод не сойдется к решению).

%пример решения

Для численного решения систем уравнений в Mathcad служит блок \mc{Given-Find}. Используя блок \mc{Given-Find}, можно решать системы, содержащие до 250 нелинейных уравнений и до 1000 линейных. Результатом решения системы будет численное значение искомых корней.

Для решения системы уравнений с помощью блока \mc{Given~-- Find} необходимо выполнить следующее:
\begin{itemize}
	\item Задайте начальные приближения для всех неизвестных, входящих в систему уравнений. Mathcad решает уравнения при помощи итерационных методов. На основе начального приближения строится последовательность, сходящаяся к искомому решению.
	\item Напечатайте в рабочем окне Mathcad ключевое слово \mc{Given}. Оно указывает Mathcad, что далее следует система уравнений. При вводе слова \mc{Given} можно использовать любой шрифт, прописные и строчные буквы. Убедитесь, что при этом Вы не находитесь в текстовой области.
	\item Введите уравнения и неравенства в любом порядке ниже ключевого слова \mc{Given}.  Булева алгебра либо с помощью сочетания клавиш \keys{\ctrl+=}. Между левыми и правыми частями неравенств может стоять любой из логических символов и (панель Булева алгебра).
	\item Введите функцию решения систем уравнений и оператор численного вывода (знак =). При вводе слова \mc{Find}. В скобках через запятую задайте переменные в том порядке, в котором должны быть расположены в ответе соответствующие им корни. Если результаты решения требуется использовать в дальнейших расчетах, то тогда их необходимо присвоить некоторой переменной .
\end{itemize}

%пример решения


\subsubsection*{Вопросы для самоконтроля}
\begin{enumerate}
	\item Для каких целей предназначен математический пакет Mathcad?
	\item С какими типами переменных позволяет работать Mathcad?
	\item В чем разница между оператором присваивания и оператором численного решения?
	\item Какие типы функций есть в Mathcad, в чем их отличие?
	\item Для чего предназначен графический редактор?
	\item Для чего предназначен текстовый редактор?
	\item Что является результатом решения системы нелинейных уравнений?
	\item Какие символы должны использоваться в качестве знаков равенства или неравенства при записи уравнений в вычислительном блоке Given-Find?
\end{enumerate}



