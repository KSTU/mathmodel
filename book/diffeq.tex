\laborator{Решение дифференциальных уравнений}

\goal ознакомиться с возможностями математического пакета Mathcad при решении дифференциальных уравнений в различных вариантах постановки задачи (задача Коши, краевая задача).

Теория
Дифференциальные уравнения позволяют выразить соотношения между изменениями физических величин, и потому они имеют большое значение в прикладных задачах. Обыкновенным дифференциальным уравнением порядка $r$ называется уравнение:
\begin{equation} \label{dif.mdif}
	F(x,y(x),y^\prime(x),y^{\prime \prime}(x), ... , y^r(x))=0,
\end{equation}
которое связывает независимую переменную $x$, искомую функцию и ее производные. Решение (интегрирование) дифференциального уравнения \ref{dif.mdif} заключается в отыскании функций (интегралов), которые удовлетворяют этому уравнению для всех значений  в определенном конечном  или бесконечном интервале. Решения могут быть проверены подстановкой в уравнение \ref{dif.mdif}.

Общее решение обыкновенного дифференциального уравнения порядка $r$ имеет вид:
\begin{equation}
	y=y(x,C_1,C_2, ... ,C_r),
\end{equation}
где $С_1$, $С_2$, ... , $С_r$ --- произвольные постоянные (постоянные интегрирования). Каждый частный выбор этих постоянных дает частное решение. В задаче Коши (начальной задаче) требуется найти частное решение, удовлетворяющее $r$ начальным условиям:
\begin{equation}
	y(x_0)=y_0, \newline
	y^{\prime}(x_0)=y_0^{\prime}, ... , y^{r}(x_0)=y^r_0,
\end{equation}
по которым определяются $r$ постоянных $С_1$, $С_2$, ... , $С_r$. В краевой задаче на $y(x)$ и ее производные накладываются $r$ краевых условий в точках $x=a$ и $x=b$.

Методы численного решения обыкновенных дифференциальных уравнений в форме задачи Коши разработаны досконально \cite{shipachevvs2005}. Самыми распространенными из них являются алгоритмы Рунге~-- Кутта, успешно используемые для решения подавляющего большинства дифференциальных уравнений. 

В Mathcad имеются специальные встроенные функции, позволяющие находить решения как линейных, так и нелинейных систем дифференциальных уравнений. Несмотря на различные методы поиска решения, каждая из этих функций требует, чтобы были заданы, по крайней мере, следующие величины, необходимые для поиска решения:
\begin{itemize}
	\item начальные условия; 
	\item набор точек, в которых нужно найти решение;
	\item само дифференциальное уравнение, записанное в некотором специальном виде.
\end{itemize}
Для качественного и быстрого решения подавляющего большинства систем дифференциальных уравнений используется функция \mc{ rkfixed(у0, t0, t1, М, D)}. Данная функция решает задачу Коши с помощью алгоритма на основе метода Рунге – Кутта 4-го порядка с фиксированным шагом.
При использовании функции \mc{rkfixed} для решения системы дифференциальных уравнении первого порядка:
\begin{equation}\label{dif.tosol1}
	y_i^{\prime}(x)=F(x,y_i(x)),\quad i=1, ... ,N
\end{equation}
она должна быть записана в векторном виде:
\begin{equation}\label{dif.tosol2}
	y(x)=D(y(x),x),
\end{equation}
где $y(x)$ --- вектор первых производных системы; $D(y(x),x)$ --- вектор~-- функция, каждая строка которой содержит правую часть соответствующего уравнения системы \ref{dif.tosol1}.

Параметры функции \mc{rkfixed} определяются следующим образом:
 \begin{itemize}[label={}]
 	\item \mc{y0} --- вектор значений искомых функций на левой границе интервала изменения переменной. Размерность вектора определяется порядком дифференциального уравнения или числом уравнений в системе (если решается система уравнений). Для дифференциального уравнения первого порядка вектор начальных значений вырождается в одну точку $y_0 = y(x_0)$;
 	\item \mc{t0} и \mc{t1} – начальная и конечная точки интервала, на котором ищется решение системы дифференциальных уравнений;
 	\item \mc{М} --- число точек, за исключением начальной точки, в которых будет определяться решение системы дифференциальных уравнений. Длина шага вычисляется делением интервала \mc{t1 - t0}, на число шагов \mc{М}. Величина mc{M} влияет на точность и трудоемкость численного решения системы дифференциальных уравнений. Большой шаг снижает точность и трудоемкость решения, маленький шаг, наоборот, повышает точность, но одновременно и трудоемкость. Данный факт необходимо учитывать при выборе значения \mc{М}. Число \mc{М} определяет число строк в полученной матрице решений, которое равно \mc{М+1};
 	\item \mc{D(x,y)} --- вектор-функция, содержащая правые части уравнений системы \ref{dif.tosol1}. Должна быть задана как функция двух переменных: скаляра \mc{х} (аргумента функции) и вектора \mc{у} (все искомые функции системы должны быть представлены как элементы одного вектора \mc{у}).
 \end{itemize}
 
Результатом работы функции \mc{rkfixed} является матрица, в первом столбце которой содержатся значения переменной \mc{t} (от \mc{t0} до \mc{t1}), а в остальных --- значения неизвестных функций системы, рассчитанные в заданных точках. При этом порядок расположения столбцов искомых функций определяется последовательностью, в которой они были занесены в вектор \mc{у}.

Если дифференциальные уравнения системы содержат производные от неизвестных функций выше первого порядка, то все уравнения, содержащие такие производные, необходимо преобразовать. Любое уравнение вида \ref{dif.mdif}, содержащее производные выше первого порядка посредством замены:
\begin{equation}
	y_1(x)=y(x), \\ y_2(x)=y^{\prime}(x),  ... , y_r(x)=^r(x)
\end{equation}
может быть приведено к совокупности уравнений:
\begin{equation}
	y_1^{\prime}(x)=y_2(x), \\ y_2^{\prime}(x)=y_3(x), ... , y_r^{\prime}(x)=F(x,y_1(x),y_2(x, ... , y_r(x))
\end{equation}
В приведенных выше уравнениях уже нет производных выше первого порядка. Преобразовав подобным образом каждое из уравнений, входящих в исходную систему, получим новую систему с большим количеством неизвестных функций, но с производными только первого порядка. Следовательно, для решения такой системы дифференциальных уравнений можно использовать функцию \mc{rkfixed}, как описано выше.

Примеры решения:

Решение краевых задач

Краевые задачи отличаются от задач Коши тем, что начальные условия для них задаются на обеих границах интервала поиска решений. Краевая форма для дифференциальных уравнений и их систем используется в основном в физике и технике в тех случаях, когда определить все начальные значения на одной границе интервала невозможно. 

Найти решение для заданных таким образом дифференциальных уравнений возможно только на основе алгоритма, в котором многократно решается задача Коши. Суть этого алгоритма заключается в подборе (естественно, направленном) недостающих параметров на одной из границ интервала, исходя из того условия, что соответствующие решения полученной задачи Коши в противоположной точке интервала должны совпадать с исходными краевыми условиями с определенным уровнем точности.

Для решения краевых задач в Mathcad имеется встроенная функция \mc{sbval(z.t0,t1,D,load,score)}. Данная функция требует определения следующих параметров:
\begin{itemize}[label={}]
	\item \mc{z} --- вектор приближений, в котором необходимо определить исходные значения для недостающих на левой границе условий. Выбор начального приближения оказывает влияние на сходимость и время поиска решения;
	\item \mc{t0} и \mc{t1} --- начальная и конечная точки интервала, на котором ищется решение системы дифференциальных уравнений;
	\item \mc{D(x,y)} --- вектор-функция, описывающая дифференциальное уравнение или систему уравнений. Задается аналогично рассмотренной ранее встроенной функции \mc{rk\-fix\-ed};
	\item \mc{load(t0,z)} --- векторная функция двух переменных, описывающая значение функции на левой границе интервала. Представляет собой вектор из \mc{N} элементов (\mc{N} соответствует количеству уравнений системы), каждый из которых является значением соответствующей функции вектора \mc{y} в точке \mc{t0}. Если начальное значение некоторой функции неизвестно, в качестве элемента \mc{load} следует использовать величину из вектора приближений \mc{z};
	\mc{score(t1,y)} --- векторная функция, служащая для задания правых граничных значений. Элементы вектора \mc{score} должны быть заданы как разности известных начальных значений в точке \mc{t1} и соответствующих им значений функций \mc{y}, возвращаемых функцией \mc{sbval}. Алгоритм, лежащий в основе функции \mc{sbval}, использует текущие величины \mc{score} в качестве меры точности подобранных приближений. 
\end{itemize}

Результатом работы функции \mc{sbval} является вектор с найденными значениями начальных условий, недостающих для представления системы дифференциальных уравнений в форме задачи Коши. Определив начальные условия, можно решить данную систему дифференциальных уравнений используя встроенную функцию \mc{rkfixed}.  При этом  в качестве  ее параметра \mc{у0} можно использовать уже определенную выше функцию \mc{load} (обозначив вектор результата как \mc{z}).