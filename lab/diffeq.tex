\textsc{\textbf{Лабораторная работа <<Решение дифференциальных уравнений>>}}

\textsc{\textbf{Вариант 1}}
\begin{enumerate}
\item (1 балл) Решить численно дифференциальное уравнение $\dfrac{dy}{dx}-2xy=2xy^3              $ с начальными значениями $y(     2)=     1$ на интервале от $x=     2$ до $x=     5$. Построить график функции.\item (1 балл) Решить численно систему дифференциальных уравнений:
 \begin{equation*}
\left\{
\begin{gathered}
\dfrac{dy}{dx}=\dfrac{x+y}{1+z}      \\
\dfrac{dz}{dx}=x-2y                  
\end{gathered}
\right.
\end{equation*}
на интервале от $x= 5$ от $x=15$ с граничными условиями: $y( 5)=3.22$, $z( 5)=3.71$. Построить график функции. 
\item (1 балл) Решить численно систему дифференциальных уравнений:
 \begin{equation*}
\left\{
\begin{gathered}
\dfrac{dy}{dx}=sin(x-y+z)\\
\dfrac{dz}{dx}=x-2y
\end{gathered}
\right.
\end{equation*}
на интервале от $x= 9$ от $x=11$ с граничными условиями: $y( 9)=4.84$, $z(11)=1.86$.  Построить график функции. 
\item (1 балл)  Записать дифференциальное уравнение распределения температуры вдоль стенки, материал которой имеет следующую зависимость теплопроводности от темепратуры: $\lambda=42.1+0.9T$. Построить распределение темпарутры по толщине стенки толщиной 14.0 см при температуре стенки с одной стороны равной  783 K и тепловом потоке 419.6 Вт/м. Определить температуру с другой стороны стенки.

 \end{enumerate}
\textsc{\textbf{Вариант 2}}
\begin{enumerate}
\item (1 балл) Решить численно дифференциальное уравнение $\dfrac{dy}{y}=\dfrac{dx}{x-1}         $ с начальными значениями $y(     3)=     3$ на интервале от $x=     3$ до $x=     7$. Построить график функции.\item (1 балл) Решить численно систему дифференциальных уравнений:
 \begin{equation*}
\left\{
\begin{gathered}
\dfrac{dy}{dx}=x                     \\
\dfrac{dz}{dx}=sin(x+y+z)            
\end{gathered}
\right.
\end{equation*}
на интервале от $x= 4$ от $x=13$ с граничными условиями: $y( 4)=3.57$, $z( 4)=0.78$. Построить график функции. 
\item (1 балл) Решить численно систему дифференциальных уравнений:
 \begin{equation*}
\left\{
\begin{gathered}
\dfrac{dy}{dx}=y\\
\dfrac{dz}{dx}=\dfrac{y}{z}
\end{gathered}
\right.
\end{equation*}
на интервале от $x= 1$ от $x= 5$ с граничными условиями: $y( 1)=0.73$, $z( 5)=9.10$.  Построить график функции. 
\item (1 балл)  Тело массой 8.9 кг и площадью поверхности 1.6 м$^2$ охлаждается в помещении с постоянной температурой воздуха 25.8 $^\circ\mathrm{C}$. Теплоемкость материала 2595.0 $\frac{Дж}{кг град}$. Записать уравнение описывающее изменение температуры тела во времени (при условии высокой теплопроводности материала тела). Определить за какое время температура тела опустится с 101.2 $^\circ\mathrm{C}$ до 31.0 $^\circ\mathrm{C}$ при постоянном коэффициенте теплопередачи равным   307 $\frac{Вт}{м^2 град.}$ 

\end{enumerate}
\textsc{\textbf{Вариант 3}}
\begin{enumerate}
\item (1 балл) Решить численно дифференциальное уравнение $x\dfrac{dy}{dx}+y=0                   $ с начальными значениями $y(     4)=     5$ на интервале от $x=     4$ до $x=    11$. Построить график функции.\item (1 балл) Решить численно систему дифференциальных уравнений:
 \begin{equation*}
\left\{
\begin{gathered}
\dfrac{dy}{dx}=z^{2/3}               \\
\dfrac{dz}{dx}=sin(x+y+z)            
\end{gathered}
\right.
\end{equation*}
на интервале от $x= 1$ от $x= 8$ с граничными условиями: $y( 1)=1.97$, $z( 1)=0.70$. Построить график функции. 
\item (1 балл) Решить численно систему дифференциальных уравнений:
 \begin{equation*}
\left\{
\begin{gathered}
\dfrac{dy}{dx}=\sqrt{x-y+z}\\
\dfrac{dz}{dx}=x
\end{gathered}
\right.
\end{equation*}
на интервале от $x= 2$ от $x=12$ с граничными условиями: $y( 2)=2.73$, $z(12)=70.72$.  Построить график функции. 
\item (1 балл)  В сосуд, содержащий 10.871 л воды подают 3.06 раствор соли концентрацией 0.46 кг/л. Поступающий в сосуд раствор моментально равномерно перемешивается с водой (модель идеального смешения), и смесь вытекает с таким же расхдом. Составить дифференциальное уравнение изменения массы соли в сосуде. Построить график изменения массы соли во времени. Сколько соли будет в сосуде через  10 минут?

\end{enumerate}
