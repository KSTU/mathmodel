
\documentclass[a4paper,12pt]{article}

%%% Работа с русским языком
\usepackage{cmap}					% поиск в PDF
\usepackage{mathtext} 				% русские буквы в формулах
\usepackage[T2A]{fontenc}			% кодировка
\usepackage[utf8]{inputenc}			% кодировка исходного текста
\usepackage[english,russian]{babel}	% локализация и переносы

%%% Дополнительная работа с математикой
\usepackage{amsmath,amsfonts,amssymb,amsthm,mathtools} % AMS
\usepackage{icomma} % "Умная" запятая: $0,2$ --- число, $0, 2$ --- перечисление

%% Номера формул
%\mathtoolsset{showonlyrefs=true} % Показывать номера только у тех формул, на которые есть \eqref{} в тексте.
%\usepackage{leqno} % Нумерация формул слева

%% Свои команды
\DeclareMathOperator{\sgn}{\mathop{sgn}}

%% Перенос знаков в формулах (по Львовскому)
\newcommand*{\hm}[1]{#1\nobreak\discretionary{}
{\hbox{$\mathsurround=0pt #1$}}{}}

%%% Работа с картинками
\usepackage{graphicx}  % Для вставки рисунков
%\graphicspath{{images/}{images2/}}  % папки с картинками
\setlength\fboxsep{3pt} % Отступ рамки \fbox{} от рисунка
\setlength\fboxrule{1pt} % Толщина линий рамки \fbox{}
\usepackage{wrapfig} % Обтекание рисунков текстом

%%% Работа с таблицами
\usepackage{array,tabularx,tabulary,booktabs} % Дополнительная работа с таблицами
\usepackage{longtable}  % Длинные таблицы
\usepackage{multirow} % Слияние строк в таблице

%%% Теоремы
\theoremstyle{plain} % Это стиль по умолчанию, его можно не переопределять.
\newtheorem{theorem}{Теорема}[section]
\newtheorem{proposition}[theorem]{Утверждение}
 
\theoremstyle{definition} % "Определение"
\newtheorem{corollary}{Следствие}[theorem]
\newtheorem{problem}{Задача}[section]
 
\theoremstyle{remark} % "Примечание"
\newtheorem*{nonum}{Решение}

%%% Программирование
\usepackage{etoolbox} % логические операторы

%%% Заголовок
\author{\LaTeX{} в Вышке}
\title{Модели для описания селективности зависящей от толщины мембраны}
\date{\today}

\begin{document}
% Этот шаблон документа разработан в 2014 году
% Данилом Фёдоровых (danil@fedorovykh.ru) 
% для использования в курсе 
% <<Документы и презентации в \LaTeX>>, записанном НИУ ВШЭ
% для Coursera.org: http://coursera.org/course/latex .
% Исходная версия шаблона --- 
% https://www.writelatex.com/coursera/latex/3.1

\documentclass[a5paper,12pt]{article}
\usepackage[left=1.5cm, top=1cm, right=1cm, bottom=20mm]{geometry}

%%% Работа с русским языком
\usepackage{cmap}					% поиск в PDF
\usepackage{mathtext} 				% русские буквы в формулах
\usepackage[T2A]{fontenc}			% кодировка
\usepackage[utf8]{inputenc}			% кодировка исходного текста
\usepackage[english,russian]{babel}	% ло кализация и переносы
%%\usepackage{textcomp}

%%% Дополнительная работа с математикой
\usepackage{amsmath,amsfonts,amssymb,amsthm,mathtools} 
\usepackage{icomma} % "Умная" запятая: $0,2$ --- число, $0, 2$ --- перечисление

%% Номера формул
\usepackage{xymtexpdf}
\usepackage{chmst-pdf}


%% Свои команды
\DeclareMathOperator{\sgn}{\mathop{sgn}}

%% Перенос знаков в формулах (по Львовскому)
\newcommand*{\hm}[1]{#1\nobreak\discretionary{}
{\hbox{$\mathsurround=0pt #1$}}{}}

%%% Работа с картинками
\usepackage{graphicx}  % Для вставки рисунков
\usepackage{rotating}
\graphicspath{{fig/}{images2/}}  % папки с картинками
\setlength\fboxsep{3pt} % Отступ рамки \fbox{} от рисунка
\setlength\fboxrule{1pt} % Толщина линий рамки \fbox{}
\usepackage{wrapfig} % Обтекание рисунков текстом

%%% Работа с таблицами
\usepackage{array,tabularx,tabulary,booktabs} % Дополнительная работа с таблицами
\usepackage{longtable}  % Длинные таблицы
\usepackage{multirow} % Слияние строк в таблице

%%% Теоремы
\theoremstyle{plain} % Это стиль по умолчанию, его можно не переопределять.
\newtheorem{theorem}{Теорема}[section]
\newtheorem{proposition}[theorem]{Утверждение}
 
\theoremstyle{definition} % "Определение"
\newtheorem{corollary}{Следствие}[theorem]
\newtheorem{problem}{Задача}[section]
 
\theoremstyle{remark} % "Примечание"
\newtheorem*{nonum}{Решение}

%%% Программирование
\usepackage{etoolbox} % логические операторы


%%%tikz
\usepackage{tikz}
\usetikzlibrary{calc}
\usetikzlibrary{patterns}

%%% Заголовок
\author{Ivan Anashkin}
\title{Методическое указание}
\date{\today}

\begin{document}

%Вставление кусков
%\input{lab1.tex}
%\input{lab2.tex}
%\input{lab3.tex}
%\input{lab4.tex}
%\input{lab5.tex}

\section*{Лабораторная работа моделирование тепломассообменных процессов}

В связи с тем, что модель идеального вытеснения имеет наибольшую движущую силу, в промышленности наиболее распространение получили аппараты наиболее близкие к данной структуре потока: кожухотрубчатые, "труба в трубе", пластинчатые и другие. На рисунке представлена схема потоков в прямоточном теплообменнике. 

Тепловой баланс холодного теплоносителя:
\begin{equation}\label{eq:tbal-cold}
	G_1 T c_{p1}-G_1 (T_1+\Delta T_1) + q \Delta F=0
\end{equation}
Поток тепла направлен от холодного теплоносителя к горячему, поэтому в выражении теплового баланса будет с отрицательным знаком:
\begin{equation}\label{eq:tbal-hot}
	G_2 T_2 c_{p2} -G_2 (T_2 - \Delta T_2) -q \Delta F =0
\end{equation}

Поток тепла можно выразить уравнением теплопередачи:
\begin{equation}\label{eq:teploper}
	q=K(T_2-T_1)
\end{equation}
где K --- коэффициент теплопередачи. Расписывая площадь поверхности теплопередачи через периметр сечения теплопередачи $P$ и длину элементарного объема $\Delta x$ как $\Delta F = \Delta x P$, с использованием выражения \ref{eq:teploper}, уравнения \ref{eq:tbal-cold} и \ref{eq:tbal-hot} можно переписать в виде системы:

 \begin{equation}
 \left\{
 \begin{aligned}
 &\dfrac{dT_1}{dx}=\dfrac{K(T_2-T_1)P}{G_1 c_{p1}}        \\
 &\dfrac{dT_2}{dx}=-\dfrac{K(T_2-T_1)P}{G_2 c_{p2}}            
 \end{aligned}
 \right.
 \end{equation}

В зависимости от поставленной задачи граничные условия могут задаваться по разному. Обычно известны расходы теплоносителей и температуры на входе теплообменника.

Для противоточного теплообменника можно записать следующую систему дифференциальных уравнений:
 \begin{equation}
 \left\{
 \begin{aligned}
 &\dfrac{dT_1}{dx}=\dfrac{K(T_2-T_1)P}{G_1 c_{p1}}        \\
 &\dfrac{dT_2}{dx}=\dfrac{K(T_2-T_1)P}{G_2 c_{p2}}            
 \end{aligned}
 \right.
 \end{equation}
 
 В случае противоточного направления и известных входящих потоках необходимо решать краевую задачу.
 
 Для плоской стенки коэффициент определяется как:
 \begin{equation}
 	K=\dfrac{1}{\dfrac{1}{\alpha_1} + \sum \dfrac{\delta}{\lambda} + \dfrac{1}{\alpha_2}}
 \end{equation}
 где $\alpha$ --- коэффициент теплоотдачи, $\delta$ --- толщина стенки, $\lambda$ --- коэффициент теплопроводности материала стенки, суммирование $\frac{\delta}{\lambda}$ проводится в случае если стенка состоит из нескольких слоев различного материала. Коэффициент теплоотдачи описывается критериальными уравнениями и зависит от многих факторов: конструкции аппарата, скорости движения жидкости физико-химических свойств. Можно привести следующие примеры:
 \begin{itemize}
	 \item 
	 \item Теплоотдача при перемешивании жидкостей мешалками:
	 \begin{equation}
		Nu=C Re^m Pr^{0.33} \left(\dfrac{\mu}{\mu_{ст}}\right)^{0.14} \dfrac{\lambda}{D}
	 \end{equation}
	 где критерий Рейнольдса определяется как $Re=\frac{ \rho n d_m^2}{\mu}$, $D$ --- диаметр сосуда, $n$ --- частота вращения мешалки, $\mu_{ст}$ --- динамический коэффициент вязкости жидкости при температуре стенки змеевика или рубашки, $\mu$ --- коэффициент вязкости при средней температуре жидкости, определяемой $(t_{ср.ж}+t_{ст})/2$. Для аппаратов с рубашками: $C=0.36$, $m=0.67$
\end{itemize}

В предлагаемых задачах можно считать теплофизические свойства не зависящими от температуры и $\frac{Pr}{Pr_{ст}}=\frac{\mu}{\mu_{ст}}=1$

\newpage
Задачка 1 Простая
Смоделировать процесс теплообмена в прямоточном теплообменнике типа "Труба в трубе". Использовать модель идеального вытеснения для обоих потоков. Размеры теплообменника: длинна 10 м, диаметр внешней трубы 50 мм, диаметр внутренней трубы 20 мм, толщина стенки 2 мм, теплопроводность материала стенки 500 Вт/(м К). Горячий теплоноситель направляется во внешнюю трубу и имеет следующие параметры: температура $T_г=200 C$, теплоемкость $\lambda_г=1600\frac{Дж}{кг \cdot С}$, плотность $\rho_г=1200 \frac{кг}{м3^3}$, расход $G_г=2 \frac{т}{ч}$. Холодный теплоноситель направляется в межтрубное пространство и имеет следующие параметры: температура $T_х=10 C$, теплоемкость $\lambda_х=1200\frac{Дж}{кг \cdot С}$, плотность $\rho_х=1000 \frac{кг}{м3^3}$, расход $G_х=2 \frac{т}{ч}$.
Определить распределение температуры теплоносителя по длине теплообменника. 

Задачка 2 Средняя
Смоделировать процесс теплообмена в прямоточном теплообменнике типа "Труба в трубе". Использовать модель идеального вытеснения для обоих потоков. Размеры теплообменника: длинна 10 м, диаметр внешней трубы 50 мм, диаметр внутренней трубы 20 мм, толщина стенки 2 мм, теплопроводность материала стенки 500 Вт/(м К). Во внутреннюю трубу подается насыщенный пар. Холодный теплоноситель направляется в межтрубное пространство и имеет следующие параметры: температура $T_х=10 C$, теплоемкость $\lambda_х=1200\frac{Дж}{кг \cdot С}$, плотность $\rho_х=1000 \frac{кг}{м3^3}$, расход $G_х=2 \frac{т}{ч}$.
Определить температуру пара, требуемую для нагрева теплоносителя до температуры 100 градусов. 

Задачка 3 Сложная
Смоделировать процесс теплообмена при нагревании жидкости в аппарате с мешалкой и змеевиком. Для жидкости в змеевике использовать модель идеального вытеснения, для жидкости в аппарате использовать модель идеального смешения. Горячий теплоноситель имеет температуру на входе $T_г=150 С$, вязкость $\mu_г=0.1 Па\cdot с$, плотность  $\rho_г= 1400 \frac{кг}{м^3}$, теплоемкость $c_{p\ г}=1200 \frac{Дж}{кг С}$, расход $G_г=1000 \frac{кг}{ч}$. Холодный теплоноситель имеет температуру на входе $T_х=150 С$, вязкость $\mu_х=0.1 Па\cdot с$, плотность  $\rho_х= 1400 \frac{кг}{м^3}$, теплоемкость $c_p=1200 \frac{Дж}{кг С}$, расход $G_г=1000 \frac{кг}{ч}$. Определить температуру холодного и горячего теплоносителей на выходе из аппарата.

Задачка 4 Сложная то же самое что и в 3 но с кипением

Задачка 5 Средня То же самое что и 1 но с противотоком

Задачка 6 Простая То же самое что и 2 но с противотоком

привет лиля радиковна
dfs gdfg sdfg sdf gsdf gsdf 

\end{document}

\end{document}
