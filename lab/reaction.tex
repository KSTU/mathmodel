\textsc{\textbf{Вариант 1}}

 В реакторе идеального вытеснения протекает реакция: \begin{equation*} \begin{aligned} A+B \xleftrightarrow[k_2]{k_1} C + \Delta H_1 \\ B + C \xrightarrow{k_3} D + \Delta H_2 \end{aligned} \end{equation*}                                На вход  реактор подается смесь при температуре $ T_н =  227 K$, теплоемкость смеси $c_p= 3400 \frac{Дж}{моль \cdot K}$, состав подаваемой смеси: $c_A=20.8 \frac{моль}{л}$, $c_B=0.3 \frac{моль}{л}$. Параметры реакций: энергии активации $E_{a1}=10.7 \frac{кДж}{моль}$, $E_{a2}=18.9  \frac{кДж}{моль}$, $E_{a3}=13.9  \frac{кДж}{моль}$, предэкспоненциальный множитель $k_{01}=        32$,$k_{02}=       629$,$k_{03}=        86$, тепловой эффект $\Delta H_1=  9.7  \frac{кДж}{моль}$, $\Delta H_2=35.8 \frac{кДж}{моль}$.\begin{itemize} \item (3 балла) Составить математическую модель изотермического реактора. Определить распределение концентрации компонентов по времени. Определить изменение конверсии по компоненту A, селективности и выхода по компоненту B. \item (+1 балл) Составить математическую модель адиабатического реактора, определить изменение концентрации и температуры во времени. Сравнить результаты для адиабатического и изотермического реактора. \item (+1 балл) Составить математическую модель реактора идеального смешения. Сравнить результаты для модели идеального вытеснения. \end{itemize}

\textsc{\textbf{Вариант 2}}

 В реакторе идеального вытеснения протекает реакция: \begin{equation*} \begin{aligned} A \xleftrightarrow{k_1} B + \Delta H_1 \\ A \xrightarrow{k_2} C + \Delta H_2 \\ A \xleftrightarrow{k_3} D + \Delta H_3 \end{aligned} \end{equation*} На вход  реактор подается смесь при температуре $ T_н =  218 K$, теплоемкость смеси $c_p= 2773 \frac{Дж}{моль \cdot K}$, состав подаваемой смеси: $c_A=22.9 \frac{моль}{л}$, $c_B=0.3 \frac{моль}{л}$. Параметры реакций: энергии активации $E_{a1}=10.6 \frac{кДж}{моль}$, $E_{a2}=16.8  \frac{кДж}{моль}$, $E_{a3}=13.6  \frac{кДж}{моль}$, предэкспоненциальный множитель $k_{01}=        42$,$k_{02}=       463$,$k_{03}=       170$, тепловой эффект $\Delta H_1= 20.3 \frac{кДж}{моль}$, $\Delta H_2=16.8 \frac{кДж}{моль}$, $\Delta H_3 = 34.9 \frac{кДж}{моль}$\begin{itemize} \item (3 балла) Составить математическую модель изотермического реактора. Определить распределение концентрации компонентов по времени. Определить изменение конверсии по компоненту A, селективности и выхода по компоненту B. \item (+1 балл) Составить математическую модель адиабатического реактора, определить изменение концентрации и температуры во времени. Сравнить результаты для адиабатического и изотермического реактора. \item (+1 балл) Составить математическую модель реактора идеального смешения. Сравнить результаты для модели идеального вытеснения. \end{itemize}

\textsc{\textbf{Вариант 3}}

 В реакторе идеального вытеснения протекает реакция: \begin{equation*} \begin{aligned} A+B \xleftrightarrow[k_2]{k_1} C +\Delta H_1 \\ A \xrightarrow{k_3} B + \Delta H_2 \end{aligned} \end{equation*}                                     На вход  реактор подается смесь при температуре $ T_н =  268 K$, теплоемкость смеси $c_p= 3413 \frac{Дж}{моль \cdot K}$, состав подаваемой смеси: $c_A=29.4 \frac{моль}{л}$, $c_B=0.4 \frac{моль}{л}$. Параметры реакций: энергии активации $E_{a1}=13.4 \frac{кДж}{моль}$, $E_{a2}=21.5  \frac{кДж}{моль}$, $E_{a3}=18.7  \frac{кДж}{моль}$, предэкспоненциальный множитель $k_{01}=        53$,$k_{02}=       711$,$k_{03}=       280$, тепловой эффект $\Delta H_1= 12.7  \frac{кДж}{моль}$, $\Delta H_2=13.1 \frac{кДж}{моль}$.\begin{itemize} \item (3 балла) Составить математическую модель изотермического реактора. Определить распределение концентрации компонентов по времени. Определить изменение конверсии по компоненту A, селективности и выхода по компоненту B. \item (+1 балл) Составить математическую модель адиабатического реактора, определить изменение концентрации и температуры во времени. Сравнить результаты для адиабатического и изотермического реактора. \item (+1 балл) Составить математическую модель реактора идеального смешения. Сравнить результаты для модели идеального вытеснения. \end{itemize}

\textsc{\textbf{Вариант 4}}

 В реакторе идеального вытеснения протекает реакция: \begin{equation*} \begin{aligned} A+B \xleftrightarrow[k_2]{k_1} C +\Delta H_1 \\ A \xrightarrow{k_3} B + \Delta H_2 \end{aligned} \end{equation*}                                     На вход  реактор подается смесь при температуре $ T_н =  319 K$, теплоемкость смеси $c_p= 2940 \frac{Дж}{моль \cdot K}$, состав подаваемой смеси: $c_A=33.5 \frac{моль}{л}$, $c_B=0.4 \frac{моль}{л}$. Параметры реакций: энергии активации $E_{a1}=21.0 \frac{кДж}{моль}$, $E_{a2}=35.0  \frac{кДж}{моль}$, $E_{a3}=21.6  \frac{кДж}{моль}$, предэкспоненциальный множитель $k_{01}=       343$,$k_{02}=     24772$,$k_{03}=       329$, тепловой эффект $\Delta H_1= 38.1  \frac{кДж}{моль}$, $\Delta H_2=12.8 \frac{кДж}{моль}$.\begin{itemize} \item (3 балла) Составить математическую модель изотермического реактора. Определить распределение концентрации компонентов по времени. Определить изменение конверсии по компоненту A, селективности и выхода по компоненту B. \item (+1 балл) Составить математическую модель адиабатического реактора, определить изменение концентрации и температуры во времени. Сравнить результаты для адиабатического и изотермического реактора. \item (+1 балл) Составить математическую модель реактора идеального смешения. Сравнить результаты для модели идеального вытеснения. \end{itemize}

\textsc{\textbf{Вариант 5}}

 В реакторе идеального вытеснения протекает реакция: \begin{equation*} \begin{aligned} A \xleftrightarrow[k_2]{k_1} C + \Delta H_1 \\ A \xrightarrow{k_3} B + \Delta H_2 \end{aligned} \end{equation*}                                      На вход  реактор подается смесь при температуре $ T_н =  274 K$, теплоемкость смеси $c_p= 2477 \frac{Дж}{моль \cdot K}$, состав подаваемой смеси: $c_A=15.4 \frac{моль}{л}$, $c_B=0.2 \frac{моль}{л}$. Параметры реакций: энергии активации $E_{a1}=13.5 \frac{кДж}{моль}$, $E_{a2}=27.8  \frac{кДж}{моль}$, $E_{a3}=17.2  \frac{кДж}{моль}$, предэкспоненциальный множитель $k_{01}=        38$,$k_{02}=      6264$,$k_{03}=       204$, тепловой эффект $\Delta H_1= 16.9  \frac{кДж}{моль}$, $\Delta H_2=16.2 \frac{кДж}{моль}$.\begin{itemize} \item (3 балла) Составить математическую модель изотермического реактора. Определить распределение концентрации компонентов по времени. Определить изменение конверсии по компоненту A, селективности и выхода по компоненту B. \item (+1 балл) Составить математическую модель адиабатического реактора, определить изменение концентрации и температуры во времени. Сравнить результаты для адиабатического и изотермического реактора. \item (+1 балл) Составить математическую модель реактора идеального смешения. Сравнить результаты для модели идеального вытеснения. \end{itemize}

\textsc{\textbf{Вариант 6}}

 В реакторе идеального вытеснения протекает реакция: \begin{equation*} \begin{aligned} A+B \xleftrightarrow[k_2]{k_1} C + \Delta H_1 \\ B + C \xrightarrow{k_3} D + \Delta H_2 \end{aligned} \end{equation*}                                На вход  реактор подается смесь при температуре $ T_н =  254 K$, теплоемкость смеси $c_p= 3562 \frac{Дж}{моль \cdot K}$, состав подаваемой смеси: $c_A=26.6 \frac{моль}{л}$, $c_B=0.4 \frac{моль}{л}$. Параметры реакций: энергии активации $E_{a1}=13.7 \frac{кДж}{моль}$, $E_{a2}=21.0  \frac{кДж}{моль}$, $E_{a3}=18.2  \frac{кДж}{моль}$, предэкспоненциальный множитель $k_{01}=        86$,$k_{02}=       599$,$k_{03}=       474$, тепловой эффект $\Delta H_1= -15.2  \frac{кДж}{моль}$, $\Delta H_2=14.5 \frac{кДж}{моль}$.\begin{itemize} \item (3 балла) Составить математическую модель изотермического реактора. Определить распределение концентрации компонентов по времени. Определить изменение конверсии по компоненту A, селективности и выхода по компоненту B. \item (+1 балл) Составить математическую модель адиабатического реактора, определить изменение концентрации и температуры во времени. Сравнить результаты для адиабатического и изотермического реактора. \item (+1 балл) Составить математическую модель реактора идеального смешения. Сравнить результаты для модели идеального вытеснения. \end{itemize}

\textsc{\textbf{Вариант 7}}

 В реакторе идеального вытеснения протекает реакция: \begin{equation*} \begin{aligned} A \xleftrightarrow[k_2]{k_1} B + \Delta H_1 \\ B \xrightarrow{k_3} C + \Delta H_2 \end{aligned} \end{equation*}                                      На вход  реактор подается смесь при температуре $ T_н =  290 K$, теплоемкость смеси $c_p= 2171 \frac{Дж}{моль \cdot K}$, состав подаваемой смеси: $c_A=17.3 \frac{моль}{л}$, $c_B=0.2 \frac{моль}{л}$. Параметры реакций: энергии активации $E_{a1}=18.4 \frac{кДж}{моль}$, $E_{a2}=22.9  \frac{кДж}{моль}$, $E_{a3}=15.8  \frac{кДж}{моль}$, предэкспоненциальный множитель $k_{01}=       230$,$k_{02}=       651$,$k_{03}=        58$, тепловой эффект $\Delta H_1= 13.3  \frac{кДж}{моль}$, $\Delta H_2=42.8 \frac{кДж}{моль}$.\begin{itemize} \item (3 балла) Составить математическую модель изотермического реактора. Определить распределение концентрации компонентов по времени. Определить изменение конверсии по компоненту A, селективности и выхода по компоненту B. \item (+1 балл) Составить математическую модель адиабатического реактора, определить изменение концентрации и температуры во времени. Сравнить результаты для адиабатического и изотермического реактора. \item (+1 балл) Составить математическую модель реактора идеального смешения. Сравнить результаты для модели идеального вытеснения. \end{itemize}

\textsc{\textbf{Вариант 8}}

 В реакторе идеального вытеснения протекает реакция: \begin{equation*} \begin{aligned} A \xleftrightarrow[k_2]{k_1} C + \Delta H_1 \\ A \xrightarrow{k_3} B + \Delta H_2 \end{aligned} \end{equation*}                                      На вход  реактор подается смесь при температуре $ T_н =  319 K$, теплоемкость смеси $c_p= 2172 \frac{Дж}{моль \cdot K}$, состав подаваемой смеси: $c_A=24.4 \frac{моль}{л}$, $c_B=0.3 \frac{моль}{л}$. Параметры реакций: энергии активации $E_{a1}=16.2 \frac{кДж}{моль}$, $E_{a2}=28.7  \frac{кДж}{моль}$, $E_{a3}=29.5  \frac{кДж}{моль}$, предэкспоненциальный множитель $k_{01}=        69$,$k_{02}=      2295$,$k_{03}=      3618$, тепловой эффект $\Delta H_1= -26.6  \frac{кДж}{моль}$, $\Delta H_2=23.2 \frac{кДж}{моль}$.\begin{itemize} \item (3 балла) Составить математическую модель изотермического реактора. Определить распределение концентрации компонентов по времени. Определить изменение конверсии по компоненту A, селективности и выхода по компоненту B. \item (+1 балл) Составить математическую модель адиабатического реактора, определить изменение концентрации и температуры во времени. Сравнить результаты для адиабатического и изотермического реактора. \item (+1 балл) Составить математическую модель реактора идеального смешения. Сравнить результаты для модели идеального вытеснения. \end{itemize}

\textsc{\textbf{Вариант 9}}

 В реакторе идеального вытеснения протекает реакция: \begin{equation*} \begin{aligned} A+B \xleftrightarrow[k_2]{k_1} C + \Delta H_1 \\ B + C \xrightarrow{k_3} D + \Delta H_2 \end{aligned} \end{equation*}                                На вход  реактор подается смесь при температуре $ T_н =  355 K$, теплоемкость смеси $c_p= 3249 \frac{Дж}{моль \cdot K}$, состав подаваемой смеси: $c_A=15.2 \frac{моль}{л}$, $c_B=0.3 \frac{моль}{л}$. Параметры реакций: энергии активации $E_{a1}=22.9 \frac{кДж}{моль}$, $E_{a2}=35.2  \frac{кДж}{моль}$, $E_{a3}=35.2  \frac{кДж}{моль}$, предэкспоненциальный множитель $k_{01}=       296$,$k_{02}=      5960$,$k_{03}=      8231$, тепловой эффект $\Delta H_1= -42.4  \frac{кДж}{моль}$, $\Delta H_2=-30.6 \frac{кДж}{моль}$.\begin{itemize} \item (3 балла) Составить математическую модель изотермического реактора. Определить распределение концентрации компонентов по времени. Определить изменение конверсии по компоненту A, селективности и выхода по компоненту B. \item (+1 балл) Составить математическую модель адиабатического реактора, определить изменение концентрации и температуры во времени. Сравнить результаты для адиабатического и изотермического реактора. \item (+1 балл) Составить математическую модель реактора идеального смешения. Сравнить результаты для модели идеального вытеснения. \end{itemize}

\textsc{\textbf{Вариант 10}}

 В реакторе идеального вытеснения протекает реакция: \begin{equation*} \begin{aligned} A \xleftrightarrow[k_2]{k_1} C + \Delta H_1 \\ A \xrightarrow{k_3} B + \Delta H_2 \end{aligned} \end{equation*}                                      На вход  реактор подается смесь при температуре $ T_н =  301 K$, теплоемкость смеси $c_p= 3180 \frac{Дж}{моль \cdot K}$, состав подаваемой смеси: $c_A=24.4 \frac{моль}{л}$, $c_B=0.3 \frac{моль}{л}$. Параметры реакций: энергии активации $E_{a1}=18.5 \frac{кДж}{моль}$, $E_{a2}=32.6  \frac{кДж}{моль}$, $E_{a3}=17.9  \frac{кДж}{моль}$, предэкспоненциальный множитель $k_{01}=       138$,$k_{02}=      9678$,$k_{03}=        83$, тепловой эффект $\Delta H_1= 27.1  \frac{кДж}{моль}$, $\Delta H_2=-33.7 \frac{кДж}{моль}$.\begin{itemize} \item (3 балла) Составить математическую модель изотермического реактора. Определить распределение концентрации компонентов по времени. Определить изменение конверсии по компоненту A, селективности и выхода по компоненту B. \item (+1 балл) Составить математическую модель адиабатического реактора, определить изменение концентрации и температуры во времени. Сравнить результаты для адиабатического и изотермического реактора. \item (+1 балл) Составить математическую модель реактора идеального смешения. Сравнить результаты для модели идеального вытеснения. \end{itemize}

\textsc{\textbf{Вариант 11}}

 В реакторе идеального вытеснения протекает реакция: \begin{equation*} \begin{aligned} A \xleftrightarrow[k_2]{k_1} B + \Delta H_1 \\ B \xrightarrow{k_3} C + \Delta H_2 \end{aligned} \end{equation*}                                      На вход  реактор подается смесь при температуре $ T_н =  354 K$, теплоемкость смеси $c_p= 2466 \frac{Дж}{моль \cdot K}$, состав подаваемой смеси: $c_A=23.6 \frac{моль}{л}$, $c_B=0.3 \frac{моль}{л}$. Параметры реакций: энергии активации $E_{a1}=27.8 \frac{кДж}{моль}$, $E_{a2}=49.7  \frac{кДж}{моль}$, $E_{a3}=34.8  \frac{кДж}{моль}$, предэкспоненциальный множитель $k_{01}=      1692$,$k_{02}=    569388$,$k_{03}=      9904$, тепловой эффект $\Delta H_1=  8.8  \frac{кДж}{моль}$, $\Delta H_2=17.0 \frac{кДж}{моль}$.\begin{itemize} \item (3 балла) Составить математическую модель изотермического реактора. Определить распределение концентрации компонентов по времени. Определить изменение конверсии по компоненту A, селективности и выхода по компоненту B. \item (+1 балл) Составить математическую модель адиабатического реактора, определить изменение концентрации и температуры во времени. Сравнить результаты для адиабатического и изотермического реактора. \item (+1 балл) Составить математическую модель реактора идеального смешения. Сравнить результаты для модели идеального вытеснения. \end{itemize}

\textsc{\textbf{Вариант 12}}

 В реакторе идеального вытеснения протекает реакция: \begin{equation*} \begin{aligned} A+B \xleftrightarrow[k_2]{k_1} C + \Delta H_1 \\ B + C \xrightarrow{k_3} D + \Delta H_2 \end{aligned} \end{equation*}                                На вход  реактор подается смесь при температуре $ T_н =  202 K$, теплоемкость смеси $c_p= 3178 \frac{Дж}{моль \cdot K}$, состав подаваемой смеси: $c_A=28.7 \frac{моль}{л}$, $c_B=0.2 \frac{моль}{л}$. Параметры реакций: энергии активации $E_{a1}= 6.8 \frac{кДж}{моль}$, $E_{a2}=18.2  \frac{кДж}{моль}$, $E_{a3}=16.3  \frac{кДж}{моль}$, предэкспоненциальный множитель $k_{01}=         7$,$k_{02}=      1220$,$k_{03}=       658$, тепловой эффект $\Delta H_1= 41.0  \frac{кДж}{моль}$, $\Delta H_2=-39.4 \frac{кДж}{моль}$.\begin{itemize} \item (3 балла) Составить математическую модель изотермического реактора. Определить распределение концентрации компонентов по времени. Определить изменение конверсии по компоненту A, селективности и выхода по компоненту B. \item (+1 балл) Составить математическую модель адиабатического реактора, определить изменение концентрации и температуры во времени. Сравнить результаты для адиабатического и изотермического реактора. \item (+1 балл) Составить математическую модель реактора идеального смешения. Сравнить результаты для модели идеального вытеснения. \end{itemize}

\textsc{\textbf{Вариант 13}}

 В реакторе идеального вытеснения протекает реакция: \begin{equation*} \begin{aligned} A+B \xleftrightarrow[k_2]{k_1} C +\Delta H_1 \\ A \xrightarrow{k_3} B + \Delta H_2 \end{aligned} \end{equation*}                                     На вход  реактор подается смесь при температуре $ T_н =  393 K$, теплоемкость смеси $c_p= 2009 \frac{Дж}{моль \cdot K}$, состав подаваемой смеси: $c_A=23.7 \frac{моль}{л}$, $c_B=0.3 \frac{моль}{л}$. Параметры реакций: энергии активации $E_{a1}=26.8 \frac{кДж}{моль}$, $E_{a2}=63.0  \frac{кДж}{моль}$, $E_{a3}=38.9  \frac{кДж}{моль}$, предэкспоненциальный множитель $k_{01}=       636$,$k_{02}=   6159439$,$k_{03}=     12933$, тепловой эффект $\Delta H_1= -36.3  \frac{кДж}{моль}$, $\Delta H_2=-12.3 \frac{кДж}{моль}$.\begin{itemize} \item (3 балла) Составить математическую модель изотермического реактора. Определить распределение концентрации компонентов по времени. Определить изменение конверсии по компоненту A, селективности и выхода по компоненту B. \item (+1 балл) Составить математическую модель адиабатического реактора, определить изменение концентрации и температуры во времени. Сравнить результаты для адиабатического и изотермического реактора. \item (+1 балл) Составить математическую модель реактора идеального смешения. Сравнить результаты для модели идеального вытеснения. \end{itemize}

\textsc{\textbf{Вариант 14}}

 В реакторе идеального вытеснения протекает реакция: \begin{equation*} \begin{aligned} A+B \xleftrightarrow[k_2]{k_1} C +\Delta H_1 \\ A \xrightarrow{k_3} B + \Delta H_2 \end{aligned} \end{equation*}                                     На вход  реактор подается смесь при температуре $ T_н =  246 K$, теплоемкость смеси $c_p= 2115 \frac{Дж}{моль \cdot K}$, состав подаваемой смеси: $c_A=21.1 \frac{моль}{л}$, $c_B=0.3 \frac{моль}{л}$. Параметры реакций: энергии активации $E_{a1}=10.3 \frac{кДж}{моль}$, $E_{a2}=18.7  \frac{кДж}{моль}$, $E_{a3}=14.4  \frac{кДж}{моль}$, предэкспоненциальный множитель $k_{01}=        13$,$k_{02}=       248$,$k_{03}=        52$, тепловой эффект $\Delta H_1= 32.5  \frac{кДж}{моль}$, $\Delta H_2=-9.9 \frac{кДж}{моль}$.\begin{itemize} \item (3 балла) Составить математическую модель изотермического реактора. Определить распределение концентрации компонентов по времени. Определить изменение конверсии по компоненту A, селективности и выхода по компоненту B. \item (+1 балл) Составить математическую модель адиабатического реактора, определить изменение концентрации и температуры во времени. Сравнить результаты для адиабатического и изотермического реактора. \item (+1 балл) Составить математическую модель реактора идеального смешения. Сравнить результаты для модели идеального вытеснения. \end{itemize}

\textsc{\textbf{Вариант 15}}

 В реакторе идеального вытеснения протекает реакция: \begin{equation*} \begin{aligned} A+B \xleftrightarrow[k_2]{k_1} C + \Delta H_1 \\ B + C \xrightarrow{k_3} D + \Delta H_2 \end{aligned} \end{equation*}                                На вход  реактор подается смесь при температуре $ T_н =  369 K$, теплоемкость смеси $c_p= 3001 \frac{Дж}{моль \cdot K}$, состав подаваемой смеси: $c_A=21.1 \frac{моль}{л}$, $c_B=0.4 \frac{моль}{л}$. Параметры реакций: энергии активации $E_{a1}=25.6 \frac{кДж}{моль}$, $E_{a2}=47.0  \frac{кДж}{моль}$, $E_{a3}=46.1  \frac{кДж}{моль}$, предэкспоненциальный множитель $k_{01}=       742$,$k_{02}=    159775$,$k_{03}=    154223$, тепловой эффект $\Delta H_1= -33.5  \frac{кДж}{моль}$, $\Delta H_2=35.2 \frac{кДж}{моль}$.\begin{itemize} \item (3 балла) Составить математическую модель изотермического реактора. Определить распределение концентрации компонентов по времени. Определить изменение конверсии по компоненту A, селективности и выхода по компоненту B. \item (+1 балл) Составить математическую модель адиабатического реактора, определить изменение концентрации и температуры во времени. Сравнить результаты для адиабатического и изотермического реактора. \item (+1 балл) Составить математическую модель реактора идеального смешения. Сравнить результаты для модели идеального вытеснения. \end{itemize}

\textsc{\textbf{Вариант 16}}

 В реакторе идеального вытеснения протекает реакция: \begin{equation*} \begin{aligned} A+B \xleftrightarrow[k_2]{k_1} C + \Delta H_1 \\ B + C \xrightarrow{k_3} D + \Delta H_2 \end{aligned} \end{equation*}                                На вход  реактор подается смесь при температуре $ T_н =  389 K$, теплоемкость смеси $c_p= 3757 \frac{Дж}{моль \cdot K}$, состав подаваемой смеси: $c_A=28.0 \frac{моль}{л}$, $c_B=0.3 \frac{моль}{л}$. Параметры реакций: энергии активации $E_{a1}=32.6 \frac{кДж}{моль}$, $E_{a2}=41.0  \frac{кДж}{моль}$, $E_{a3}=31.0  \frac{кДж}{моль}$, предэкспоненциальный множитель $k_{01}=      1872$,$k_{02}=     14866$,$k_{03}=       755$, тепловой эффект $\Delta H_1= -31.4  \frac{кДж}{моль}$, $\Delta H_2=20.8 \frac{кДж}{моль}$.\begin{itemize} \item (3 балла) Составить математическую модель изотермического реактора. Определить распределение концентрации компонентов по времени. Определить изменение конверсии по компоненту A, селективности и выхода по компоненту B. \item (+1 балл) Составить математическую модель адиабатического реактора, определить изменение концентрации и температуры во времени. Сравнить результаты для адиабатического и изотермического реактора. \item (+1 балл) Составить математическую модель реактора идеального смешения. Сравнить результаты для модели идеального вытеснения. \end{itemize}

\textsc{\textbf{Вариант 17}}

 В реакторе идеального вытеснения протекает реакция: \begin{equation*} \begin{aligned} A \xleftrightarrow{k_1} B + \Delta H_1 \\ A \xrightarrow{k_2} C + \Delta H_2 \\ A \xleftrightarrow{k_3} D + \Delta H_3 \end{aligned} \end{equation*} На вход  реактор подается смесь при температуре $ T_н =  363 K$, теплоемкость смеси $c_p= 3613 \frac{Дж}{моль \cdot K}$, состав подаваемой смеси: $c_A=16.5 \frac{моль}{л}$, $c_B=0.2 \frac{моль}{л}$. Параметры реакций: энергии активации $E_{a1}=19.5 \frac{кДж}{моль}$, $E_{a2}=50.9  \frac{кДж}{моль}$, $E_{a3}=28.2  \frac{кДж}{моль}$, предэкспоненциальный множитель $k_{01}=        67$,$k_{02}=    460332$,$k_{03}=      1253$, тепловой эффект $\Delta H_1= 31.4 \frac{кДж}{моль}$, $\Delta H_2=20.1 \frac{кДж}{моль}$, $\Delta H_3 = -30.2 \frac{кДж}{моль}$\begin{itemize} \item (3 балла) Составить математическую модель изотермического реактора. Определить распределение концентрации компонентов по времени. Определить изменение конверсии по компоненту A, селективности и выхода по компоненту B. \item (+1 балл) Составить математическую модель адиабатического реактора, определить изменение концентрации и температуры во времени. Сравнить результаты для адиабатического и изотермического реактора. \item (+1 балл) Составить математическую модель реактора идеального смешения. Сравнить результаты для модели идеального вытеснения. \end{itemize}

\textsc{\textbf{Вариант 18}}

 В реакторе идеального вытеснения протекает реакция: \begin{equation*} \begin{aligned} A+B \xleftrightarrow[k_2]{k_1} C +\Delta H_1 \\ A \xrightarrow{k_3} B + \Delta H_2 \end{aligned} \end{equation*}                                     На вход  реактор подается смесь при температуре $ T_н =  266 K$, теплоемкость смеси $c_p= 2296 \frac{Дж}{моль \cdot K}$, состав подаваемой смеси: $c_A=30.0 \frac{моль}{л}$, $c_B=0.4 \frac{моль}{л}$. Параметры реакций: энергии активации $E_{a1}=14.2 \frac{кДж}{моль}$, $E_{a2}=25.7  \frac{кДж}{моль}$, $E_{a3}=27.1  \frac{кДж}{моль}$, предэкспоненциальный множитель $k_{01}=        95$,$k_{02}=      4676$,$k_{03}=      9044$, тепловой эффект $\Delta H_1= -13.6  \frac{кДж}{моль}$, $\Delta H_2=31.4 \frac{кДж}{моль}$.\begin{itemize} \item (3 балла) Составить математическую модель изотермического реактора. Определить распределение концентрации компонентов по времени. Определить изменение конверсии по компоненту A, селективности и выхода по компоненту B. \item (+1 балл) Составить математическую модель адиабатического реактора, определить изменение концентрации и температуры во времени. Сравнить результаты для адиабатического и изотермического реактора. \item (+1 балл) Составить математическую модель реактора идеального смешения. Сравнить результаты для модели идеального вытеснения. \end{itemize}

\textsc{\textbf{Вариант 19}}

 В реакторе идеального вытеснения протекает реакция: \begin{equation*} \begin{aligned} A \xleftrightarrow[k_2]{k_1} B + \Delta H_1 \\ B \xrightarrow{k_3} C + \Delta H_2 \end{aligned} \end{equation*}                                      На вход  реактор подается смесь при температуре $ T_н =  312 K$, теплоемкость смеси $c_p= 2204 \frac{Дж}{моль \cdot K}$, состав подаваемой смеси: $c_A=32.7 \frac{моль}{л}$, $c_B=0.3 \frac{моль}{л}$. Параметры реакций: энергии активации $E_{a1}=19.0 \frac{кДж}{моль}$, $E_{a2}=40.5  \frac{кДж}{моль}$, $E_{a3}=25.6  \frac{кДж}{моль}$, предэкспоненциальный множитель $k_{01}=       266$,$k_{02}=    122013$,$k_{03}=      2062$, тепловой эффект $\Delta H_1= -10.2  \frac{кДж}{моль}$, $\Delta H_2=-22.7 \frac{кДж}{моль}$.\begin{itemize} \item (3 балла) Составить математическую модель изотермического реактора. Определить распределение концентрации компонентов по времени. Определить изменение конверсии по компоненту A, селективности и выхода по компоненту B. \item (+1 балл) Составить математическую модель адиабатического реактора, определить изменение концентрации и температуры во времени. Сравнить результаты для адиабатического и изотермического реактора. \item (+1 балл) Составить математическую модель реактора идеального смешения. Сравнить результаты для модели идеального вытеснения. \end{itemize}

\textsc{\textbf{Вариант 20}}

 В реакторе идеального вытеснения протекает реакция: \begin{equation*} \begin{aligned} A \xleftrightarrow[k_2]{k_1} B + \Delta H_1 \\ B \xrightarrow{k_3} C + \Delta H_2 \end{aligned} \end{equation*}                                      На вход  реактор подается смесь при температуре $ T_н =  263 K$, теплоемкость смеси $c_p= 2321 \frac{Дж}{моль \cdot K}$, состав подаваемой смеси: $c_A=31.3 \frac{моль}{л}$, $c_B=0.4 \frac{моль}{л}$. Параметры реакций: энергии активации $E_{a1}=11.7 \frac{кДж}{моль}$, $E_{a2}=18.5  \frac{кДж}{моль}$, $E_{a3}=10.1  \frac{кДж}{моль}$, предэкспоненциальный множитель $k_{01}=        12$,$k_{02}=       223$,$k_{03}=        10$, тепловой эффект $\Delta H_1= -39.2  \frac{кДж}{моль}$, $\Delta H_2=13.5 \frac{кДж}{моль}$.\begin{itemize} \item (3 балла) Составить математическую модель изотермического реактора. Определить распределение концентрации компонентов по времени. Определить изменение конверсии по компоненту A, селективности и выхода по компоненту B. \item (+1 балл) Составить математическую модель адиабатического реактора, определить изменение концентрации и температуры во времени. Сравнить результаты для адиабатического и изотермического реактора. \item (+1 балл) Составить математическую модель реактора идеального смешения. Сравнить результаты для модели идеального вытеснения. \end{itemize}

\textsc{\textbf{Вариант 21}}

 В реакторе идеального вытеснения протекает реакция: \begin{equation*} \begin{aligned} A+B \xleftrightarrow[k_2]{k_1} C +\Delta H_1 \\ A \xrightarrow{k_3} B + \Delta H_2 \end{aligned} \end{equation*}                                     На вход  реактор подается смесь при температуре $ T_н =  259 K$, теплоемкость смеси $c_p= 3906 \frac{Дж}{моль \cdot K}$, состав подаваемой смеси: $c_A=18.4 \frac{моль}{л}$, $c_B=0.2 \frac{моль}{л}$. Параметры реакций: энергии активации $E_{a1}=10.9 \frac{кДж}{моль}$, $E_{a2}=23.3  \frac{кДж}{моль}$, $E_{a3}=27.5  \frac{кДж}{моль}$, предэкспоненциальный множитель $k_{01}=        26$,$k_{02}=      2689$,$k_{03}=     13834$, тепловой эффект $\Delta H_1= 39.4  \frac{кДж}{моль}$, $\Delta H_2=-36.6 \frac{кДж}{моль}$.\begin{itemize} \item (3 балла) Составить математическую модель изотермического реактора. Определить распределение концентрации компонентов по времени. Определить изменение конверсии по компоненту A, селективности и выхода по компоненту B. \item (+1 балл) Составить математическую модель адиабатического реактора, определить изменение концентрации и температуры во времени. Сравнить результаты для адиабатического и изотермического реактора. \item (+1 балл) Составить математическую модель реактора идеального смешения. Сравнить результаты для модели идеального вытеснения. \end{itemize}

\textsc{\textbf{Вариант 22}}

 В реакторе идеального вытеснения протекает реакция: \begin{equation*} \begin{aligned} A \xleftrightarrow[k_2]{k_1} B + \Delta H_1 \\ B \xrightarrow{k_3} C + \Delta H_2 \end{aligned} \end{equation*}                                      На вход  реактор подается смесь при температуре $ T_н =  263 K$, теплоемкость смеси $c_p= 2579 \frac{Дж}{моль \cdot K}$, состав подаваемой смеси: $c_A=22.9 \frac{моль}{л}$, $c_B=0.4 \frac{моль}{л}$. Параметры реакций: энергии активации $E_{a1}=13.5 \frac{кДж}{моль}$, $E_{a2}=20.0  \frac{кДж}{моль}$, $E_{a3}=13.3  \frac{кДж}{моль}$, предэкспоненциальный множитель $k_{01}=        47$,$k_{02}=       418$,$k_{03}=        46$, тепловой эффект $\Delta H_1= -14.9  \frac{кДж}{моль}$, $\Delta H_2=37.0 \frac{кДж}{моль}$.\begin{itemize} \item (3 балла) Составить математическую модель изотермического реактора. Определить распределение концентрации компонентов по времени. Определить изменение конверсии по компоненту A, селективности и выхода по компоненту B. \item (+1 балл) Составить математическую модель адиабатического реактора, определить изменение концентрации и температуры во времени. Сравнить результаты для адиабатического и изотермического реактора. \item (+1 балл) Составить математическую модель реактора идеального смешения. Сравнить результаты для модели идеального вытеснения. \end{itemize}

\textsc{\textbf{Вариант 23}}

 В реакторе идеального вытеснения протекает реакция: \begin{equation*} \begin{aligned} A+B \xleftrightarrow[k_2]{k_1} C +\Delta H_1 \\ A \xrightarrow{k_3} B + \Delta H_2 \end{aligned} \end{equation*}                                     На вход  реактор подается смесь при температуре $ T_н =  388 K$, теплоемкость смеси $c_p= 3472 \frac{Дж}{моль \cdot K}$, состав подаваемой смеси: $c_A=21.8 \frac{моль}{л}$, $c_B=0.2 \frac{моль}{л}$. Параметры реакций: энергии активации $E_{a1}=26.1 \frac{кДж}{моль}$, $E_{a2}=50.4  \frac{кДж}{моль}$, $E_{a3}=27.6  \frac{кДж}{моль}$, предэкспоненциальный множитель $k_{01}=       579$,$k_{02}=    157247$,$k_{03}=       614$, тепловой эффект $\Delta H_1= -12.4  \frac{кДж}{моль}$, $\Delta H_2=41.8 \frac{кДж}{моль}$.\begin{itemize} \item (3 балла) Составить математическую модель изотермического реактора. Определить распределение концентрации компонентов по времени. Определить изменение конверсии по компоненту A, селективности и выхода по компоненту B. \item (+1 балл) Составить математическую модель адиабатического реактора, определить изменение концентрации и температуры во времени. Сравнить результаты для адиабатического и изотермического реактора. \item (+1 балл) Составить математическую модель реактора идеального смешения. Сравнить результаты для модели идеального вытеснения. \end{itemize}

\textsc{\textbf{Вариант 24}}

 В реакторе идеального вытеснения протекает реакция: \begin{equation*} \begin{aligned} A+B \xleftrightarrow[k_2]{k_1} C + \Delta H_1 \\ B + C \xrightarrow{k_3} D + \Delta H_2 \end{aligned} \end{equation*}                                На вход  реактор подается смесь при температуре $ T_н =  388 K$, теплоемкость смеси $c_p= 3804 \frac{Дж}{моль \cdot K}$, состав подаваемой смеси: $c_A=17.4 \frac{моль}{л}$, $c_B=0.3 \frac{моль}{л}$. Параметры реакций: энергии активации $E_{a1}=19.8 \frac{кДж}{моль}$, $E_{a2}=50.4  \frac{кДж}{моль}$, $E_{a3}=54.1  \frac{кДж}{моль}$, предэкспоненциальный множитель $k_{01}=        72$,$k_{02}=    149949$,$k_{03}=    605841$, тепловой эффект $\Delta H_1=  5.7  \frac{кДж}{моль}$, $\Delta H_2= 7.2 \frac{кДж}{моль}$.\begin{itemize} \item (3 балла) Составить математическую модель изотермического реактора. Определить распределение концентрации компонентов по времени. Определить изменение конверсии по компоненту A, селективности и выхода по компоненту B. \item (+1 балл) Составить математическую модель адиабатического реактора, определить изменение концентрации и температуры во времени. Сравнить результаты для адиабатического и изотермического реактора. \item (+1 балл) Составить математическую модель реактора идеального смешения. Сравнить результаты для модели идеального вытеснения. \end{itemize}

\textsc{\textbf{Вариант 25}}

 В реакторе идеального вытеснения протекает реакция: \begin{equation*} \begin{aligned} A \xleftrightarrow[k_2]{k_1} B + \Delta H_1 \\ B \xrightarrow{k_3} C + \Delta H_2 \end{aligned} \end{equation*}                                      На вход  реактор подается смесь при температуре $ T_н =  392 K$, теплоемкость смеси $c_p= 3752 \frac{Дж}{моль \cdot K}$, состав подаваемой смеси: $c_A=25.4 \frac{моль}{л}$, $c_B=0.3 \frac{моль}{л}$. Параметры реакций: энергии активации $E_{a1}=31.2 \frac{кДж}{моль}$, $E_{a2}=42.2  \frac{кДж}{моль}$, $E_{a3}=43.8  \frac{кДж}{моль}$, предэкспоненциальный множитель $k_{01}=      1224$,$k_{02}=     19871$,$k_{03}=     23974$, тепловой эффект $\Delta H_1= 22.4  \frac{кДж}{моль}$, $\Delta H_2=33.5 \frac{кДж}{моль}$.\begin{itemize} \item (3 балла) Составить математическую модель изотермического реактора. Определить распределение концентрации компонентов по времени. Определить изменение конверсии по компоненту A, селективности и выхода по компоненту B. \item (+1 балл) Составить математическую модель адиабатического реактора, определить изменение концентрации и температуры во времени. Сравнить результаты для адиабатического и изотермического реактора. \item (+1 балл) Составить математическую модель реактора идеального смешения. Сравнить результаты для модели идеального вытеснения. \end{itemize}

\textsc{\textbf{Вариант 26}}

 В реакторе идеального вытеснения протекает реакция: \begin{equation*} \begin{aligned} A+B \xleftrightarrow[k_2]{k_1} C +\Delta H_1 \\ A \xrightarrow{k_3} B + \Delta H_2 \end{aligned} \end{equation*}                                     На вход  реактор подается смесь при температуре $ T_н =  233 K$, теплоемкость смеси $c_p= 3858 \frac{Дж}{моль \cdot K}$, состав подаваемой смеси: $c_A=28.2 \frac{моль}{л}$, $c_B=0.2 \frac{моль}{л}$. Параметры реакций: энергии активации $E_{a1}=10.4 \frac{кДж}{моль}$, $E_{a2}=19.2  \frac{кДж}{моль}$, $E_{a3}=19.3  \frac{кДж}{моль}$, предэкспоненциальный множитель $k_{01}=        33$,$k_{02}=      1058$,$k_{03}=      1029$, тепловой эффект $\Delta H_1= -25.3  \frac{кДж}{моль}$, $\Delta H_2=-7.1 \frac{кДж}{моль}$.\begin{itemize} \item (3 балла) Составить математическую модель изотермического реактора. Определить распределение концентрации компонентов по времени. Определить изменение конверсии по компоненту A, селективности и выхода по компоненту B. \item (+1 балл) Составить математическую модель адиабатического реактора, определить изменение концентрации и температуры во времени. Сравнить результаты для адиабатического и изотермического реактора. \item (+1 балл) Составить математическую модель реактора идеального смешения. Сравнить результаты для модели идеального вытеснения. \end{itemize}

\textsc{\textbf{Вариант 27}}

 В реакторе идеального вытеснения протекает реакция: \begin{equation*} \begin{aligned} A \xleftrightarrow[k_2]{k_1} C + \Delta H_1 \\ A \xrightarrow{k_3} B + \Delta H_2 \end{aligned} \end{equation*}                                      На вход  реактор подается смесь при температуре $ T_н =  265 K$, теплоемкость смеси $c_p= 2131 \frac{Дж}{моль \cdot K}$, состав подаваемой смеси: $c_A=16.7 \frac{моль}{л}$, $c_B=0.4 \frac{моль}{л}$. Параметры реакций: энергии активации $E_{a1}=12.9 \frac{кДж}{моль}$, $E_{a2}=14.7  \frac{кДж}{моль}$, $E_{a3}=14.6  \frac{кДж}{моль}$, предэкспоненциальный множитель $k_{01}=        23$,$k_{02}=        34$,$k_{03}=        68$, тепловой эффект $\Delta H_1= 14.2  \frac{кДж}{моль}$, $\Delta H_2=-35.5 \frac{кДж}{моль}$.\begin{itemize} \item (3 балла) Составить математическую модель изотермического реактора. Определить распределение концентрации компонентов по времени. Определить изменение конверсии по компоненту A, селективности и выхода по компоненту B. \item (+1 балл) Составить математическую модель адиабатического реактора, определить изменение концентрации и температуры во времени. Сравнить результаты для адиабатического и изотермического реактора. \item (+1 балл) Составить математическую модель реактора идеального смешения. Сравнить результаты для модели идеального вытеснения. \end{itemize}

\textsc{\textbf{Вариант 28}}

 В реакторе идеального вытеснения протекает реакция: \begin{equation*} \begin{aligned} A+B \xleftrightarrow[k_2]{k_1} C +\Delta H_1 \\ A \xrightarrow{k_3} B + \Delta H_2 \end{aligned} \end{equation*}                                     На вход  реактор подается смесь при температуре $ T_н =  279 K$, теплоемкость смеси $c_p= 3555 \frac{Дж}{моль \cdot K}$, состав подаваемой смеси: $c_A=26.2 \frac{моль}{л}$, $c_B=0.4 \frac{моль}{л}$. Параметры реакций: энергии активации $E_{a1}=17.9 \frac{кДж}{моль}$, $E_{a2}=17.6  \frac{кДж}{моль}$, $E_{a3}=23.8  \frac{кДж}{моль}$, предэкспоненциальный множитель $k_{01}=       146$,$k_{02}=       105$,$k_{03}=       991$, тепловой эффект $\Delta H_1= -32.0  \frac{кДж}{моль}$, $\Delta H_2=21.5 \frac{кДж}{моль}$.\begin{itemize} \item (3 балла) Составить математическую модель изотермического реактора. Определить распределение концентрации компонентов по времени. Определить изменение конверсии по компоненту A, селективности и выхода по компоненту B. \item (+1 балл) Составить математическую модель адиабатического реактора, определить изменение концентрации и температуры во времени. Сравнить результаты для адиабатического и изотермического реактора. \item (+1 балл) Составить математическую модель реактора идеального смешения. Сравнить результаты для модели идеального вытеснения. \end{itemize}

\textsc{\textbf{Вариант 29}}

 В реакторе идеального вытеснения протекает реакция: \begin{equation*} \begin{aligned} A \xleftrightarrow{k_1} B + \Delta H_1 \\ A \xrightarrow{k_2} C + \Delta H_2 \\ A \xleftrightarrow{k_3} D + \Delta H_3 \end{aligned} \end{equation*} На вход  реактор подается смесь при температуре $ T_н =  266 K$, теплоемкость смеси $c_p= 2056 \frac{Дж}{моль \cdot K}$, состав подаваемой смеси: $c_A=30.0 \frac{моль}{л}$, $c_B=0.3 \frac{моль}{л}$. Параметры реакций: энергии активации $E_{a1}=10.4 \frac{кДж}{моль}$, $E_{a2}=23.1  \frac{кДж}{моль}$, $E_{a3}=23.7  \frac{кДж}{моль}$, предэкспоненциальный множитель $k_{01}=        10$,$k_{02}=      1499$,$k_{03}=      1470$, тепловой эффект $\Delta H_1= -33.2 \frac{кДж}{моль}$, $\Delta H_2=-18.2 \frac{кДж}{моль}$, $\Delta H_3 = 15.2 \frac{кДж}{моль}$\begin{itemize} \item (3 балла) Составить математическую модель изотермического реактора. Определить распределение концентрации компонентов по времени. Определить изменение конверсии по компоненту A, селективности и выхода по компоненту B. \item (+1 балл) Составить математическую модель адиабатического реактора, определить изменение концентрации и температуры во времени. Сравнить результаты для адиабатического и изотермического реактора. \item (+1 балл) Составить математическую модель реактора идеального смешения. Сравнить результаты для модели идеального вытеснения. \end{itemize}

\textsc{\textbf{Вариант 30}}

 В реакторе идеального вытеснения протекает реакция: \begin{equation*} \begin{aligned} A+B \xleftrightarrow[k_2]{k_1} C +\Delta H_1 \\ A \xrightarrow{k_3} B + \Delta H_2 \end{aligned} \end{equation*}                                     На вход  реактор подается смесь при температуре $ T_н =  268 K$, теплоемкость смеси $c_p= 3421 \frac{Дж}{моль \cdot K}$, состав подаваемой смеси: $c_A=21.0 \frac{моль}{л}$, $c_B=0.2 \frac{моль}{л}$. Параметры реакций: энергии активации $E_{a1}=15.2 \frac{кДж}{моль}$, $E_{a2}=21.4  \frac{кДж}{моль}$, $E_{a3}=23.1  \frac{кДж}{моль}$, предэкспоненциальный множитель $k_{01}=        85$,$k_{02}=       735$,$k_{03}=      1380$, тепловой эффект $\Delta H_1=  8.1  \frac{кДж}{моль}$, $\Delta H_2=29.3 \frac{кДж}{моль}$.\begin{itemize} \item (3 балла) Составить математическую модель изотермического реактора. Определить распределение концентрации компонентов по времени. Определить изменение конверсии по компоненту A, селективности и выхода по компоненту B. \item (+1 балл) Составить математическую модель адиабатического реактора, определить изменение концентрации и температуры во времени. Сравнить результаты для адиабатического и изотермического реактора. \item (+1 балл) Составить математическую модель реактора идеального смешения. Сравнить результаты для модели идеального вытеснения. \end{itemize}

