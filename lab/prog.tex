\textsc{\textbf{Лабораторная работа <<Основы программирования>>}}

\textsc{\textbf{Вариант 1}}
\begin{enumerate}
\item  Создать программу, возвращающую значение $\sum\limits_{n=1}^{N} {\dfrac{1}{n} -\dfrac{1}{n^2}}$, если N=4. 
\item  Создать программу, возвращающую значение $\sum\limits_{e^{n-2}<  23} {n}^{0.3}                     $. 
\item Создать функцию, аргументом которой является массив $A$ произвольной размерностью, возвращающую значение $D_{i,j}=\begin{cases} A_{i,j}^2, & \text{если } \sqrt{A_{i,j}}<5 \\ \sqrt{A_{i,j}}, & \text{если } \sqrt{A_{i,j}}>5 \end{cases}              $ 
\item Машина трогается с места и за 11.2 минуты разгоняется с постоянным ускорением до 92.4 ${\frac{км}{ч}}$, далее машина   22 минут едет с постоянной скоростью, после чего начинает торможение с постоянным ускорением и через   3 минут останавливается. Определить путь пройденный машиной.  \item Реализовать алгоритм определения корня уравнения методом деления отрезка пополам.                                                                                                                                                                                              

\end{enumerate}
\textsc{\textbf{Вариант 2}}
\begin{enumerate}
\item  Создать программу, возвращающую значение $\sum\limits_{n=1}^{N} {\dfrac{1}{n} +sin(n)}        $, если N=5. 
\item  Создать программу, возвращающую значение $\sum\limits_{n+4<      84} n-\sin(n)                     $. 
\item Создать функцию, аргументом которой является массив $A$ произвольной размерностью, возвращающую значение $D_{i,j}=\begin{cases} A_{i,j}+4, & \text{если } {A_{i,j}}<\sin(A_{i,j}) \\  A_{i,j}-1, & \text{если } {A_{i,j}}>\sin(A_{i,j}) \end{cases}    $ 
\item Машина трогается с места и за  9.1 минуты разгоняется с постоянным ускорением до 128.8 ${\frac{км}{ч}}$, далее машина   36 минут едет с постоянной скоростью, после чего начинает торможение с постоянным ускорением и через   2 минут останавливается. Определить путь пройденный машиной.  \item Реализовать алгоритм определения корня уравнения методом деления отрезка пополам.                                                                                                                                                                                              

\end{enumerate}
\textsc{\textbf{Вариант 3}}
\begin{enumerate}
\item  Создать программу, возвращающую значение $\sum\limits_{n=1}^{N} \dfrac{n}{(n+1)^2}            $, если N=8. 
\item  Создать программу, возвращающую значение $\sum\limits_{n+4<      36} {\dfrac{1}{n} e^{-n}}         $. 
\item Создать функцию, аргументом которой является массив $A$ произвольной размерностью, возвращающую значение $D_{i,j}=\begin{cases} \sin(A_{i,j}), & \text{если } A_{i,j}<\pi \\ \cos(A_{i,j}), & \text{если } A_{i,j}>\pi \end{cases}                     $ 
\item Машина трогается с места и за 12.6 минуты разгоняется с постоянным ускорением до 83.8 ${\frac{км}{ч}}$, далее машина   42 минут едет с постоянной скоростью, после чего начинает торможение с постоянным ускорением и через   2 минут останавливается. Определить путь пройденный машиной.  \item Составить программу определения количества четных чисел в произвольной матрице.                                                                                                                                                                                                  

\end{enumerate}
\textsc{\textbf{Вариант 4}}
\begin{enumerate}
\item  Создать программу, возвращающую значение $\sum\limits_{n=1}^{N} {n (n+1)}                     $, если N=9. 
\item  Создать программу, возвращающую значение $\sum\limits_{n^{1.5}<  19} n^{2-n}                       $. 
\item Создать функцию, аргументом которой является массив $A$ произвольной размерностью, возвращающую значение $D_{i,j}=\begin{cases} i-j, & \text{если } {A_{i,j}}<i+j \\  A_{i,j}^j, & \text{если } {A_{i,j}}>i+j \end{cases}                              $ 
\item Машина трогается с места и за  9.3 минуты разгоняется с постоянным ускорением до 60.3 ${\frac{км}{ч}}$, далее машина   53 минут едет с постоянной скоростью, после чего начинает торможение с постоянным ускорением и через   3 минут останавливается. Определить путь пройденный машиной.  \item Реализовать алгоритм определения корня уравнения методом деления отрезка пополам.                                                                                                                                                                                              

\end{enumerate}
\textsc{\textbf{Вариант 5}}
\begin{enumerate}
\item  Создать программу, возвращающую значение $\sum\limits_{n=0}^{N} \dfrac{n+3}{n+9}              $, если N=9. 
\item  Создать программу, возвращающую значение $\sum\limits_{n-4<      57} \dfrac{n}{ln(n)}              $. 
\item Создать функцию, аргументом которой является массив $A$ произвольной размерностью, возвращающую значение $D_{i,j}=\begin{cases} A_{i,j}^2, & \text{если } \sqrt{A_{i,j}}<5 \\ \sqrt{A_{i,j}}, & \text{если } \sqrt{A_{i,j}}>5 \end{cases}              $ 
\item Машина трогается с места и за 11.3 минуты разгоняется с постоянным ускорением до 74.6 ${\frac{км}{ч}}$, далее машина   22 минут едет с постоянной скоростью, после чего начинает торможение с постоянным ускорением и через   5 минут останавливается. Определить путь пройденный машиной.  \item Составить программу определения суммы положительных чисел всех элементов матрицы.                                                                                                                                                                                              

\end{enumerate}
\textsc{\textbf{Вариант 6}}
\begin{enumerate}
\item  Создать программу, возвращающую значение $\sum\limits_{n=1}^{N} n-\sin(n)                     $, если N=6. 
\item  Создать программу, возвращающую значение $\sum\limits_{n^2<      41} {\dfrac{1}{n} +sin(n)}        $. 
\item Создать функцию, аргументом которой является массив $A$ произвольной размерностью, возвращающую значение $D_{i,j}=\begin{cases} i-j, & \text{если } {A_{i,j}}<i+j \\  A_{i,j}^j, & \text{если } {A_{i,j}}>i+j \end{cases}                              $ 
\item Машина трогается с места и за  7.7 минуты разгоняется с постоянным ускорением до 33.8 ${\frac{км}{ч}}$, далее машина   38 минут едет с постоянной скоростью, после чего начинает торможение с постоянным ускорением и через   3 минут останавливается. Определить путь пройденный машиной.  \item Составить программу определения произведения элементов матрицы больших 1.                                                                                                                                                                                                              

\end{enumerate}
\textsc{\textbf{Вариант 7}}
\begin{enumerate}
\item  Создать программу, возвращающую значение $\sum\limits_{n=1}^{N} \dfrac{e^(-n)}{e^{1-n}}       $, если N=5. 
\item  Создать программу, возвращающую значение $\sum\limits_{e^{n-2}<  27}  \dfrac{\ln(n)}{n^2}          $. 
\item Создать функцию, аргументом которой является массив $A$ произвольной размерностью, возвращающую значение $D_{i,j}=\begin{cases} A_{i,j}, & \text{если } A_{i,j}<0 \\ -A_{i,j}, & \text{если } A_{i,j}>0 \end{cases}                                    $ 
\item Машина трогается с места и за  6.9 минуты разгоняется с постоянным ускорением до 122.8 ${\frac{км}{ч}}$, далее машина   68 минут едет с постоянной скоростью, после чего начинает торможение с постоянным ускорением и через   2 минут останавливается. Определить путь пройденный машиной.  \item Составить программу определения количества четных чисел в произвольной матрице.                                                                                                                                                                                                  

\end{enumerate}
\textsc{\textbf{Вариант 8}}
\begin{enumerate}
\item  Создать программу, возвращающую значение $\sum\limits_{n=1}^{N} {n}^{0.9}                     $, если N=9. 
\item  Создать программу, возвращающую значение $\sum\limits_{0.3n<     38} {n (n-1)}                     $. 
\item Создать функцию, аргументом которой является массив $A$ произвольной размерностью, возвращающую значение $D_{i,j}=\begin{cases} A_{i,j}, & \text{если } {A_{i,j}}<i \\  -A_{i,j}, & \text{если } {A_{i,j}}>i \end{cases}                               $ 
\item Машина трогается с места и за  8.9 минуты разгоняется с постоянным ускорением до 44.8 ${\frac{км}{ч}}$, далее машина   47 минут едет с постоянной скоростью, после чего начинает торможение с постоянным ускорением и через   1 минут останавливается. Определить путь пройденный машиной.  \item Составить программу сравнивающую сумму элементов двух матриц, и возвращающую матрицу с большей суммой элементов.                                                                                                                                      

\end{enumerate}
\textsc{\textbf{Вариант 9}}
\begin{enumerate}
\item  Создать программу, возвращающую значение $\sum\limits_{n=1}^{N} \dfrac{n}{(n+1)^2}            $, если N=8. 
\item  Создать программу, возвращающую значение $\sum\limits_{n+4<      49} {\sin(n)}^2                   $. 
\item Создать функцию, аргументом которой является массив $A$ произвольной размерностью, возвращающую значение $D_{i,j}=\begin{cases} i-j, & \text{если } {A_{i,j}}<i+j \\  A_{i,j}^j, & \text{если } {A_{i,j}}>i+j \end{cases}                              $ 
\item Машина трогается с места и за  3.6 минуты разгоняется с постоянным ускорением до 108.5 ${\frac{км}{ч}}$, далее машина   24 минут едет с постоянной скоростью, после чего начинает торможение с постоянным ускорением и через   4 минут останавливается. Определить путь пройденный машиной.  \item Составить программу определения произведения элементов матрицы больших 1.                                                                                                                                                                                                              

\end{enumerate}
\textsc{\textbf{Вариант 10}}
\begin{enumerate}
\item  Создать программу, возвращающую значение $\sum\limits_{n=1}^{N} {n}^{0.3}                     $, если N=8. 
\item  Создать программу, возвращающую значение $\sum\limits_{n^{1.5}< 107} {\sin(n)}^2                   $. 
\item Создать функцию, аргументом которой является массив $A$ произвольной размерностью, возвращающую значение $D_{i,j}=\begin{cases} A_{i,j}, & \text{если } A_{i,j}<0 \\ -A_{i,j}, & \text{если } A_{i,j}>0 \end{cases}                                    $ 
\item Машина трогается с места и за 10.5 минуты разгоняется с постоянным ускорением до 98.4 ${\frac{км}{ч}}$, далее машина   49 минут едет с постоянной скоростью, после чего начинает торможение с постоянным ускорением и через   3 минут останавливается. Определить путь пройденный машиной.  \item Составить программу определения произведения элементов матрицы больших 1.                                                                                                                                                                                                              

\end{enumerate}
\textsc{\textbf{Вариант 11}}
\begin{enumerate}
\item  Создать программу, возвращающую значение $\sum\limits_{n=1}^{N} \dfrac{n}{(n+1)^2}            $, если N=5. 
\item  Создать программу, возвращающую значение $\sum\limits_{n/3<      92} {e^{-n}}                      $. 
\item Создать функцию, аргументом которой является массив $A$ произвольной размерностью, возвращающую значение $D_{i,j}=\begin{cases} \sin(A_{i,j}), & \text{если } A_{i,j}<\pi \\ \cos(A_{i,j}), & \text{если } A_{i,j}>\pi \end{cases}                     $ 
\item Машина трогается с места и за  7.2 минуты разгоняется с постоянным ускорением до 42.0 ${\frac{км}{ч}}$, далее машина   80 минут едет с постоянной скоростью, после чего начинает торможение с постоянным ускорением и через   4 минут останавливается. Определить путь пройденный машиной.  \item Реализовать алгоритм определения корня уравнения методом деления отрезка пополам.                                                                                                                                                                                              

\end{enumerate}
\textsc{\textbf{Вариант 12}}
\begin{enumerate}
\item  Создать программу, возвращающую значение $\sum\limits_{n=1}^{N} n \cdot sin(n)                $, если N=9. 
\item  Создать программу, возвращающую значение $\sum\limits_{e^{n-2}<  13} {e^{-n}}                      $. 
\item Создать функцию, аргументом которой является массив $A$ произвольной размерностью, возвращающую значение $D_{i,j}=\begin{cases} A_{i,j}+A_{i,j}^{i+j}, & \text{если } {A_{i,j}}<0 \\  A_{i,j}^{i+j+2}, & \text{если } {A_{i,j}}>0 \end{cases}          $ 
\item Машина трогается с места и за  3.0 минуты разгоняется с постоянным ускорением до 91.6 ${\frac{км}{ч}}$, далее машина   25 минут едет с постоянной скоростью, после чего начинает торможение с постоянным ускорением и через   4 минут останавливается. Определить путь пройденный машиной.  \item Составить программу определения произведения элементов матрицы больших 1.                                                                                                                                                                                                              

\end{enumerate}
\textsc{\textbf{Вариант 13}}
\begin{enumerate}
\item  Создать программу, возвращающую значение $\sum\limits_{n=1}^{N} {\dfrac{1}{n} +sin(n)}        $, если N=7. 
\item  Создать программу, возвращающую значение $\sum\limits_{n^{2.5}<  31} n^{2-n}                       $. 
\item Создать функцию, аргументом которой является массив $A$ произвольной размерностью, возвращающую значение $D_{i,j}=\begin{cases} 5 A_{i,j}, & \text{если } \sqrt{A_{i,j}}<7 \\ \sqrt{ 6 A_{i,j}}, & \text{если } \sqrt{A_{i,j}}>7 \end{cases}           $ 
\item Машина трогается с места и за  5.5 минуты разгоняется с постоянным ускорением до 113.5 ${\frac{км}{ч}}$, далее машина   59 минут едет с постоянной скоростью, после чего начинает торможение с постоянным ускорением и через   3 минут останавливается. Определить путь пройденный машиной.  \item Составить программу, аргументом которой является произвольная матрица, и возвращающую массив, содержащий количество элементов больших 1 и количество элментов больше 5 матрица аргумента.

\end{enumerate}
\textsc{\textbf{Вариант 14}}
\begin{enumerate}
\item  Создать программу, возвращающую значение $\sum\limits_{n=1}^{N} \dfrac{\sin(n)+1}{\cos(n)+1}  $, если N=6. 
\item  Создать программу, возвращающую значение $\sum\limits_{n^2<      79} \dfrac{n}{(n+1)^2}            $. 
\item Создать функцию, аргументом которой является массив $A$ произвольной размерностью, возвращающую значение $D_{i,j}=\begin{cases} A_{i,j}+\cos(i)+\cos(j), & \text{если } {A_{i,j}}<i \\  A_{i,j}+\sin(i)+\sin(j), & \text{если } {A_{i,j}}>i \end{cases}$ 
\item Машина трогается с места и за  9.7 минуты разгоняется с постоянным ускорением до 124.2 ${\frac{км}{ч}}$, далее машина   49 минут едет с постоянной скоростью, после чего начинает торможение с постоянным ускорением и через   4 минут останавливается. Определить путь пройденный машиной.  \item Составить программу определения количества четных чисел в произвольной матрице.                                                                                                                                                                                                  

\end{enumerate}
\textsc{\textbf{Вариант 15}}
\begin{enumerate}
\item  Создать программу, возвращающую значение $\sum\limits_{n=1}^{N} {e^{-n}}                      $, если N=8. 
\item  Создать программу, возвращающую значение $\sum\limits_{n^{2.5}<  85} {\dfrac{1}{n} }               $. 
\item Создать функцию, аргументом которой является массив $A$ произвольной размерностью, возвращающую значение $D_{i,j}=\begin{cases} A_{i,j}+4, & \text{если } {A_{i,j}}<\sin(A_{i,j}) \\  A_{i,j}-1, & \text{если } {A_{i,j}}>\sin(A_{i,j}) \end{cases}    $ 
\item Машина трогается с места и за  4.7 минуты разгоняется с постоянным ускорением до 81.4 ${\frac{км}{ч}}$, далее машина   29 минут едет с постоянной скоростью, после чего начинает торможение с постоянным ускорением и через   3 минут останавливается. Определить путь пройденный машиной.  \item Составить программу сравнивающую сумму элементов двух матриц, и возвращающую матрицу с большей суммой элементов.                                                                                                                                      

\end{enumerate}
\textsc{\textbf{Вариант 16}}
\begin{enumerate}
\item  Создать программу, возвращающую значение $\sum\limits_{n=1}^{N} \dfrac{\ln(n^2)}{n^2}         $, если N=7. 
\item  Создать программу, возвращающую значение $\sum\limits_{n^{2.5}<  40} {\dfrac{1}{n} +sin(n)}        $. 
\item Создать функцию, аргументом которой является массив $A$ произвольной размерностью, возвращающую значение $D_{i,j}=\begin{cases} A_{i,j}+A_{i,j}^{i+j}, & \text{если } {A_{i,j}}<0 \\  A_{i,j}^{i+j+2}, & \text{если } {A_{i,j}}>0 \end{cases}          $ 
\item Машина трогается с места и за  5.1 минуты разгоняется с постоянным ускорением до 119.4 ${\frac{км}{ч}}$, далее машина   37 минут едет с постоянной скоростью, после чего начинает торможение с постоянным ускорением и через   1 минут останавливается. Определить путь пройденный машиной.  \item Составить программу определения количества отрицательных чисел в произвольной матрице.                                                                                                                                                                                    

\end{enumerate}
\textsc{\textbf{Вариант 17}}
\begin{enumerate}
\item  Создать программу, возвращающую значение $\sum\limits_{n=1}^{N} n \cdot sin(n)                $, если N=4. 
\item  Создать программу, возвращающую значение $\sum\limits_{n^{1.5}<  98} n^{2-n}                       $. 
\item Создать функцию, аргументом которой является массив $A$ произвольной размерностью, возвращающую значение $D_{i,j}=\begin{cases} i-j, & \text{если } {A_{i,j}}<i+j \\  A_{i,j}^j, & \text{если } {A_{i,j}}>i+j \end{cases}                              $ 
\item Машина трогается с места и за  5.6 минуты разгоняется с постоянным ускорением до 46.5 ${\frac{км}{ч}}$, далее машина   46 минут едет с постоянной скоростью, после чего начинает торможение с постоянным ускорением и через   1 минут останавливается. Определить путь пройденный машиной.  \item Реализовать алгоритм определения корня уравнения методом деления отрезка пополам.                                                                                                                                                                                              

\end{enumerate}
\textsc{\textbf{Вариант 18}}
\begin{enumerate}
\item  Создать программу, возвращающую значение $\sum\limits_{n=1}^{N} n+\dfrac{n}{4+n}              $, если N=4. 
\item  Создать программу, возвращающую значение $\sum\limits_{e^{n-2}<  65} {n -\dfrac{1}{n}}             $. 
\item Создать функцию, аргументом которой является массив $A$ произвольной размерностью, возвращающую значение $D_{i,j}=\begin{cases} A_{i,j}+i, & \text{если } {A_{i,j}}<i \\  A_{i,j}-j, & \text{если } {A_{i,j}}>i \end{cases}                            $ 
\item Машина трогается с места и за 10.4 минуты разгоняется с постоянным ускорением до 123.6 ${\frac{км}{ч}}$, далее машина   37 минут едет с постоянной скоростью, после чего начинает торможение с постоянным ускорением и через   3 минут останавливается. Определить путь пройденный машиной.  \item Составить программу сравнивающую сумму элементов двух матриц, и возвращающую матрицу с большей суммой элементов.                                                                                                                                      

\end{enumerate}
\textsc{\textbf{Вариант 19}}
\begin{enumerate}
\item  Создать программу, возвращающую значение $\sum\limits_{n=0}^{N} \dfrac{n+3}{n+9}              $, если N=4. 
\item  Создать программу, возвращающую значение $\sum\limits_{n-3<      10} {n \sin(n)}                   $. 
\item Создать функцию, аргументом которой является массив $A$ произвольной размерностью, возвращающую значение $D_{i,j}=\begin{cases} A_{i,j}+4, & \text{если } {A_{i,j}}<\sin(A_{i,j}) \\  A_{i,j}-1, & \text{если } {A_{i,j}}>\sin(A_{i,j}) \end{cases}    $ 
\item Машина трогается с места и за 11.8 минуты разгоняется с постоянным ускорением до 74.2 ${\frac{км}{ч}}$, далее машина   50 минут едет с постоянной скоростью, после чего начинает торможение с постоянным ускорением и через   1 минут останавливается. Определить путь пройденный машиной.  \item Составить программу определения количества четных чисел в произвольной матрице.                                                                                                                                                                                                  

\end{enumerate}
\textsc{\textbf{Вариант 20}}
\begin{enumerate}
\item  Создать программу, возвращающую значение $\sum\limits_{n=1}^{N} {\dfrac{1}{n} -\dfrac{1}{n^2}}$, если N=6. 
\item  Создать программу, возвращающую значение $\sum\limits_{n<        15} {n (n-1)}                     $. 
\item Создать функцию, аргументом которой является массив $A$ произвольной размерностью, возвращающую значение $D_{i,j}=\begin{cases} A_{i,j}+A_{i,j}^{i+j}, & \text{если } {A_{i,j}}<0 \\  A_{i,j}^{i+j+2}, & \text{если } {A_{i,j}}>0 \end{cases}          $ 
\item Машина трогается с места и за  8.5 минуты разгоняется с постоянным ускорением до 42.0 ${\frac{км}{ч}}$, далее машина   56 минут едет с постоянной скоростью, после чего начинает торможение с постоянным ускорением и через   2 минут останавливается. Определить путь пройденный машиной.  \item Составить программу определения суммы положительных чисел всех элементов матрицы.                                                                                                                                                                                              

\end{enumerate}
\textsc{\textbf{Вариант 21}}
\begin{enumerate}
\item  Создать программу, возвращающую значение $\sum\limits_{n=1}^{N} \dfrac{\ln(n^2)}{n^2}         $, если N=9. 
\item  Создать программу, возвращающую значение $\sum\limits_{n<        40} {e^{-n}}                      $. 
\item Создать функцию, аргументом которой является массив $A$ произвольной размерностью, возвращающую значение $D_{i,j}=\begin{cases} i-j, & \text{если } {A_{i,j}}<i+j \\  A_{i,j}^j, & \text{если } {A_{i,j}}>i+j \end{cases}                              $ 
\item Машина трогается с места и за 10.4 минуты разгоняется с постоянным ускорением до 69.1 ${\frac{км}{ч}}$, далее машина   51 минут едет с постоянной скоростью, после чего начинает торможение с постоянным ускорением и через   5 минут останавливается. Определить путь пройденный машиной.  \item Реализовать алгоритм определения корня уравнения методом деления отрезка пополам.                                                                                                                                                                                              

\end{enumerate}
\textsc{\textbf{Вариант 22}}
\begin{enumerate}
\item  Создать программу, возвращающую значение $\sum\limits_{n=1}^{N} {n -\dfrac{1}{n}}             $, если N=8. 
\item  Создать программу, возвращающую значение $\sum\limits_{n^{2.5}<  19} {n \sin(n)}                   $. 
\item Создать функцию, аргументом которой является массив $A$ произвольной размерностью, возвращающую значение $D_{i,j}=\begin{cases} A_{1,1}, & \text{если } A_{i,j}<A_{1,1} \\ A_{i,j}, & \text{если } A_{i,j}>A_{1,1} \end{cases}                         $ 
\item Машина трогается с места и за  6.9 минуты разгоняется с постоянным ускорением до 67.5 ${\frac{км}{ч}}$, далее машина   22 минут едет с постоянной скоростью, после чего начинает торможение с постоянным ускорением и через   4 минут останавливается. Определить путь пройденный машиной.  \item Составить программу, аргументом которой является произвольная матрица, и возвращающую массив, содержащий количество элементов больших 1 и количество элментов больше 5 матрица аргумента.

\end{enumerate}
\textsc{\textbf{Вариант 23}}
\begin{enumerate}
\item  Создать программу, возвращающую значение $\sum\limits_{n=1}^{N} {\dfrac{1}{n} }               $, если N=9. 
\item  Создать программу, возвращающую значение $\sum\limits_{n-4<      77} n-\sin(n)                     $. 
\item Создать функцию, аргументом которой является массив $A$ произвольной размерностью, возвращающую значение $D_{i,j}=\begin{cases} 5 A_{i,j}, & \text{если } \sqrt{A_{i,j}}<7 \\ \sqrt{ 6 A_{i,j}}, & \text{если } \sqrt{A_{i,j}}>7 \end{cases}           $ 
\item Машина трогается с места и за  4.1 минуты разгоняется с постоянным ускорением до 92.1 ${\frac{км}{ч}}$, далее машина   73 минут едет с постоянной скоростью, после чего начинает торможение с постоянным ускорением и через   1 минут останавливается. Определить путь пройденный машиной.  \item Реализовать алгоритм определения корня уравнения методом деления отрезка пополам.                                                                                                                                                                                              

\end{enumerate}
\textsc{\textbf{Вариант 24}}
\begin{enumerate}
\item  Создать программу, возвращающую значение $\sum\limits_{n=0}^{N} \dfrac{n+3}{n+9}              $, если N=5. 
\item  Создать программу, возвращающую значение $\sum\limits_{0.3n<    107} {n}^{0.9}                     $. 
\item Создать функцию, аргументом которой является массив $A$ произвольной размерностью, возвращающую значение $D_{i,j}=\begin{cases} i+j, & \text{если } {A_{i,j}}<i+j \\  A_{i,j}+j, & \text{если } {A_{i,j}}>i+j \end{cases}                              $ 
\item Машина трогается с места и за  9.2 минуты разгоняется с постоянным ускорением до 83.3 ${\frac{км}{ч}}$, далее машина   57 минут едет с постоянной скоростью, после чего начинает торможение с постоянным ускорением и через   4 минут останавливается. Определить путь пройденный машиной.  \item Составить программу определения количества отрицательных чисел в произвольной матрице.                                                                                                                                                                                    

\end{enumerate}
\textsc{\textbf{Вариант 25}}
\begin{enumerate}
\item  Создать программу, возвращающую значение $\sum\limits_{n=1}^{N} n-\sin(n)                     $, если N=4. 
\item  Создать программу, возвращающую значение $\sum\limits_{n^{2.5}< 108} \dfrac{n}{ln(n)}              $. 
\item Создать функцию, аргументом которой является массив $A$ произвольной размерностью, возвращающую значение $D_{i,j}=\begin{cases} A_{i,j}, & \text{если } {A_{i,j}}<i \\  -A_{i,j}, & \text{если } {A_{i,j}}>i \end{cases}                               $ 
\item Машина трогается с места и за  9.2 минуты разгоняется с постоянным ускорением до 126.7 ${\frac{км}{ч}}$, далее машина   69 минут едет с постоянной скоростью, после чего начинает торможение с постоянным ускорением и через   3 минут останавливается. Определить путь пройденный машиной.  \item Реализовать алгоритм перемножения двух матриц, сравнить результат работы программы со встроенной функцией перемножения матриц.                                                                                                          

\end{enumerate}
\textsc{\textbf{Вариант 26}}
\begin{enumerate}
\item  Создать программу, возвращающую значение $\sum\limits_{n=1}^{N} {n}^{0.9}                     $, если N=9. 
\item  Создать программу, возвращающую значение $\sum\limits_{0.3n<     46} n+\dfrac{n}{4+n}              $. 
\item Создать функцию, аргументом которой является массив $A$ произвольной размерностью, возвращающую значение $D_{i,j}=\begin{cases} A_{i,j}, & \text{если } A_{i,j}<0 \\ -A_{i,j}, & \text{если } A_{i,j}>0 \end{cases}                                    $ 
\item Машина трогается с места и за 10.9 минуты разгоняется с постоянным ускорением до 117.4 ${\frac{км}{ч}}$, далее машина   37 минут едет с постоянной скоростью, после чего начинает торможение с постоянным ускорением и через   2 минут останавливается. Определить путь пройденный машиной.  \item Составить программу, аргументом которой является произвольная матрица, и возвращающую массив, содержащий количество элементов больших 1 и количество элментов больше 5 матрица аргумента.

\end{enumerate}
\textsc{\textbf{Вариант 27}}
\begin{enumerate}
\item  Создать программу, возвращающую значение $\sum\limits_{n=1}^{N} {n (n-1)}                     $, если N=9. 
\item  Создать программу, возвращающую значение $\sum\limits_{0.3n+4<   50} \dfrac{e^(-n)}{e^{1-n}}       $. 
\item Создать функцию, аргументом которой является массив $A$ произвольной размерностью, возвращающую значение $D_{i,j}=\begin{cases} A_{i,j}, & \text{если } A_{i,j}<0 \\ -A_{i,j}, & \text{если } A_{i,j}>0 \end{cases}                                    $ 
\item Машина трогается с места и за  4.0 минуты разгоняется с постоянным ускорением до 58.9 ${\frac{км}{ч}}$, далее машина   28 минут едет с постоянной скоростью, после чего начинает торможение с постоянным ускорением и через   1 минут останавливается. Определить путь пройденный машиной.  \item Составить программу определения количества отрицательных чисел в произвольной матрице.                                                                                                                                                                                    

\end{enumerate}
\textsc{\textbf{Вариант 28}}
\begin{enumerate}
\item  Создать программу, возвращающую значение $\sum\limits_{n=1}^{N} n \cdot sin(n)                $, если N=9. 
\item  Создать программу, возвращающую значение $\sum\limits_{n^2<     103} {\dfrac{1}{n} }               $. 
\item Создать функцию, аргументом которой является массив $A$ произвольной размерностью, возвращающую значение $D_{i,j}=\begin{cases} A_{i,j}+4, & \text{если } {A_{i,j}}<\sin(A_{i,j}) \\  A_{i,j}-1, & \text{если } {A_{i,j}}>\sin(A_{i,j}) \end{cases}    $ 
\item Машина трогается с места и за  3.3 минуты разгоняется с постоянным ускорением до 67.3 ${\frac{км}{ч}}$, далее машина   27 минут едет с постоянной скоростью, после чего начинает торможение с постоянным ускорением и через   3 минут останавливается. Определить путь пройденный машиной.  \item Реализовать алгоритм определения корня уравнения методом деления отрезка пополам.                                                                                                                                                                                              

\end{enumerate}
\textsc{\textbf{Вариант 29}}
\begin{enumerate}
\item  Создать программу, возвращающую значение $\sum\limits_{n=1}^{N} {e^{-n}}                      $, если N=9. 
\item  Создать программу, возвращающую значение $\sum\limits_{n^{2.5}<  33} \dfrac{e^(-n)}{e^{1-n}}       $. 
\item Создать функцию, аргументом которой является массив $A$ произвольной размерностью, возвращающую значение $D_{i,j}=\begin{cases} A_{i,j}+\cos(i)+\cos(j), & \text{если } {A_{i,j}}<i \\  A_{i,j}+\sin(i)+\sin(j), & \text{если } {A_{i,j}}>i \end{cases}$ 
\item Машина трогается с места и за 10.2 минуты разгоняется с постоянным ускорением до 47.3 ${\frac{км}{ч}}$, далее машина   29 минут едет с постоянной скоростью, после чего начинает торможение с постоянным ускорением и через   2 минут останавливается. Определить путь пройденный машиной.  \item Составить программу, аргументом которой является произвольная матрица, и возвращающую массив, содержащий количество элементов больших 1 и количество элментов больше 5 матрица аргумента.

\end{enumerate}
\textsc{\textbf{Вариант 30}}
\begin{enumerate}
\item  Создать программу, возвращающую значение $\sum\limits_{n=1}^{N} n-\sin(n)                     $, если N=9. 
\item  Создать программу, возвращающую значение $\sum\limits_{e^{n-2}<  41} {\dfrac{1}{n} e^{-n}}         $. 
\item Создать функцию, аргументом которой является массив $A$ произвольной размерностью, возвращающую значение $D_{i,j}=\begin{cases} \sin(A_{i,j}), & \text{если } A_{i,j}<\pi \\ \cos(A_{i,j}), & \text{если } A_{i,j}>\pi \end{cases}                     $ 
\item Машина трогается с места и за  8.7 минуты разгоняется с постоянным ускорением до 40.6 ${\frac{км}{ч}}$, далее машина   38 минут едет с постоянной скоростью, после чего начинает торможение с постоянным ускорением и через   4 минут останавливается. Определить путь пройденный машиной.  \item Составить программу определения суммы положительных чисел всех элементов матрицы.                                                                                                                                                                                              

\end{enumerate}
