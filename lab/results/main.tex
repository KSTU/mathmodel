
\documentclass[a4paper,12pt]{article}

%%% Работа с русским языком
\usepackage{cmap}					% поиск в PDF
\usepackage{mathtext} 				% русские буквы в формулах
\usepackage[T2A]{fontenc}			% кодировка
\usepackage[utf8]{inputenc}			% кодировка исходного текста
\usepackage[english,russian]{babel}	% локализация и переносы



\usepackage{geometry}
\geometry{
	a4paper,
	total={175mm,257mm},
	left=20mm,
	top=20mm,
}



%%% Дополнительная работа с математикой
\usepackage{amsmath,amsfonts,amssymb,amsthm,mathtools} % AMS
\usepackage{icomma} % "Умная" запятая: $0,2$ --- число, $0, 2$ --- перечисление
\usepackage{cutwin}
%% Номера формул
%\mathtoolsset{showonlyrefs=true} % Показывать номера только у тех формул, на которые есть \eqref{} в тексте.
%\usepackage{leqno} % Нумерация формул слева

%% Свои команды
\DeclareMathOperator{\sgn}{\mathop{sgn}}

%% Перенос знаков в формулах (по Львовскому)
\newcommand*{\hm}[1]{#1\nobreak\discretionary{}
{\hbox{$\mathsurround=0pt #1$}}{}}

%%% Работа с картинками
\usepackage{graphicx}  % Для вставки рисунков
%\graphicspath{{images/}{images2/}}  % папки с картинками
\setlength\fboxsep{3pt} % Отступ рамки \fbox{} от рисунка
\setlength\fboxrule{1pt} % Толщина линий рамки \fbox{}
\usepackage{wrapfig} % Обтекание рисунков текстом

%%% Работа с таблицами
\usepackage{array,tabularx,tabulary,booktabs} % Дополнительная работа с таблицами
\usepackage{longtable}  % Длинные таблицы
\usepackage{multirow} % Слияние строк в таблице

%%% Теоремы
\theoremstyle{plain} % Это стиль по умолчанию, его можно не переопределять.
\newtheorem{theorem}{Теорема}[section]
\newtheorem{proposition}[theorem]{Утверждение}
 
\theoremstyle{definition} % "Определение"
\newtheorem{corollary}{Следствие}[theorem]
\newtheorem{problem}{Задача}[section]
 
\theoremstyle{remark} % "Примечание"
\newtheorem*{nonum}{Решение}

\usepackage{fancyhdr}
\pagestyle{fancy}
\fancyhead[R]{Лабораторная работа № \arabic{nlab}}	%\rightmark
\fancyhead[L]{\input{number.tex}}
\fancyhead[C]{}
\setcounter{secnumdepth}{0}

\newcounter{nlab}
\setcounter{nlab}{0}
%\renewcommand{\subsectionmark}[1]{\markright{#1}}

%%% Программирование
\usepackage{etoolbox} % логические операторы

%\usepackage{xcolor}
\usepackage[unicode, pdftex,pdfusetitle]{hyperref} % подключаем hyperref
% % Цвета для гиперссылок
%\definecolor{linkcolor}{HTML}{799B03} % цвет ссылок
%\definecolor{urlcolor}{HTML}{799B03} % цвет гиперссылок
%
%\hypersetup{pdfstartview=FitH,  linkcolor=linkcolor,urlcolor=urlcolor, colorlinks=true}
%\usepackage{hyperref}
%\hypersetup{colorlinks=false}

%%% Заголовок
%\author{\LaTeX{} в Вышке}
%\title{Модели для описания селективности зависящей от толщины мембраны}
%\date{\today}

\begin{document}
% НАЧАЛО ТИТУЛЬНОГО ЛИСТА
\begin{center}
\hfill
\vspace{0.35\paperheight}

\Huge{Задания к лабораторным работам для группы №\input{number.tex}}\\

\normalsize дата генерации документа \today

\thispagestyle{empty} % выключаем отображение номера для этой страницы
\end{center}
% КОНЕЦ ТИТУЛЬНОГО ЛИСТА
 
\newpage
    \tableofcontents % Вывод содержания
\newpage


\textsc{\textbf{Лабораторная работа №5 <<Определение условий фазовых равновесий пар~-- жидкость>>}}

\textsc{\textbf{Вариант 1}}
\begin{enumerate}
\item По экспериментальным данным (из справочника теплофизических свойств чистых веществ) получить описание давления паров чистого компонента от температуры. Для бензола использовать уравнение Риделя $ln(p_i^0(T))=A-\frac{B}{T}+C ln(T)+DT^2$, для дихлорэтана использовать уравнение Риделя $ln(p_i^0(T))=A-\frac{B}{T}+C ln(T)+DT^2$. На основе полученных уравнений и закона Рауля, для смеси бензол-дихлорэтан построить $T-x,y$ диаграмму равновесия пар-жидкость при давлении  760.0 мм.рт.ст. . Сравнить результаты полученные по модели с экспериментальными данными и сделать вывод о применимости модели.

\item Построить $p-x,y$ диаграмму при температуре   20.0 $^\circ$C. Сравнить результаты полученные по модели с экспериментальными данными и сделать вывод о применимости модели. \item Используя экспериментальные данные определить параметеры уравнения Ван~-- Лаара для описания коэффициента активности. Используя найденные значения параметров построить $T-x,y$ и $p-x,y$ диаграммы. Сравнить с результатом, полученным по уравнению Рауля.\end{enumerate}

\textsc{\textbf{Вариант 2}}
\begin{enumerate}
\item По экспериментальным данным (из справочника теплофизических свойств чистых веществ) получить описание давления паров чистого компонента от температуры. Для бензола использовать уравнение Миллера $ln(p_i^0(T))=A-\frac{B}{T}+C T+DT^3$  , для толуола использовать уравнение Ренкина $ln(p_i^0(T))=A-\frac{B}{T}+СT^2$     . На основе полученных уравнений и закона Рауля, для смеси бензол--толуол построить $T-x,y$ диаграмму равновесия пар-жидкость при давлении  760.0 мм.рт.ст. . Сравнить результаты полученные по модели с экспериментальными данными и сделать вывод о применимости модели.

\item Построить $p-x,y$ диаграмму при температуре  240.0 $^\circ$C. Сравнить результаты полученные по модели с экспериментальными данными и сделать вывод о применимости модели. \item Используя экспериментальные данные определить параметеры уравнения NRTL                 для описания коэффициента активности. Используя найденные значения параметров построить $T-x,y$ и $p-x,y$ диаграммы. Сравнить с результатом, полученным по уравнению Рауля.\end{enumerate}

\textsc{\textbf{Вариант 3}}
\begin{enumerate}
\item По экспериментальным данным (из справочника теплофизических свойств чистых веществ) получить описание давления паров чистого компонента от температуры. Для ацетона использовать уравнение Антуана $ln(p_i^0(T))=A-\frac{B}{T+C}$         , для бензола использовать уравнение Риделя $ln(p_i^0(T))=A-\frac{B}{T}+C ln(T)+DT^2$. На основе полученных уравнений и закона Рауля, для смеси ацетон-бензол построить $T-x,y$ диаграмму равновесия пар-жидкость при давлении  732.0 мм.рт.ст. . Сравнить результаты полученные по модели с экспериментальными данными и сделать вывод о применимости модели.

\item Построить $p-x,y$ диаграмму при температуре   25.0 $^\circ$C. Сравнить результаты полученные по модели с экспериментальными данными и сделать вывод о применимости модели. \item Используя экспериментальные данные определить параметеры уравнения Вильсона     для описания коэффициента активности. Используя найденные значения параметров построить $T-x,y$ и $p-x,y$ диаграммы. Сравнить с результатом, полученным по уравнению Рауля.\end{enumerate}

\textsc{\textbf{Вариант 4}}
\begin{enumerate}
\item По экспериментальным данным (из справочника теплофизических свойств чистых веществ) получить описание давления паров чистого компонента от температуры. Для ацетона использовать уравнение Клапейрона $ln(p_i^0(T))=A-\frac{B}{T}$     , для бутанола использовать уравнение Миллера $ln(p_i^0(T))=A-\frac{B}{T}+C T+DT^3$  . На основе полученных уравнений и закона Рауля, для смеси ацетон-бутанол построить $T-x,y$ диаграмму равновесия пар-жидкость при давлении  745.0 мм.рт.ст. . Сравнить результаты полученные по модели с экспериментальными данными и сделать вывод о применимости модели.

\item Построить $p-x,y$ диаграмму при температуре   25.0 $^\circ$C. Сравнить результаты полученные по модели с экспериментальными данными и сделать вывод о применимости модели. \item Используя экспериментальные данные определить параметеры уравнения Маргулеса   для описания коэффициента активности. Используя найденные значения параметров построить $T-x,y$ и $p-x,y$ диаграммы. Сравнить с результатом, полученным по уравнению Рауля.\end{enumerate}

\textsc{\textbf{Вариант 5}}
\begin{enumerate}
\item По экспериментальным данным (из справочника теплофизических свойств чистых веществ) получить описание давления паров чистого компонента от температуры. Для этанола использовать уравнение Риделя $ln(p_i^0(T))=A-\frac{B}{T}+C ln(T)+DT^2$, для бензола использовать уравнение Антуана $ln(p_i^0(T))=A-\frac{B}{T+C}$         . На основе полученных уравнений и закона Рауля, для смеси этанол-бензол построить $T-x,y$ диаграмму равновесия пар-жидкость при давлении  300.0 мм.рт.ст. . Сравнить результаты полученные по модели с экспериментальными данными и сделать вывод о применимости модели.

\item Построить $p-x,y$ диаграмму при температуре   50.0 $^\circ$C. Сравнить результаты полученные по модели с экспериментальными данными и сделать вывод о применимости модели. \item Используя экспериментальные данные определить параметеры уравнения Ван~-- Лаара для описания коэффициента активности. Используя найденные значения параметров построить $T-x,y$ и $p-x,y$ диаграммы. Сравнить с результатом, полученным по уравнению Рауля.\end{enumerate}

\textsc{\textbf{Вариант 6}}
\begin{enumerate}
\item По экспериментальным данным (из справочника теплофизических свойств чистых веществ) получить описание давления паров чистого компонента от температуры. Для ацетона использовать уравнение Антуана $ln(p_i^0(T))=A-\frac{B}{T+C}$         , для бутанола использовать уравнение Антуана $ln(p_i^0(T))=A-\frac{B}{T+C}$         . На основе полученных уравнений и закона Рауля, для смеси ацетон-бутанол построить $T-x,y$ диаграмму равновесия пар-жидкость при давлении  745.0 мм.рт.ст. . Сравнить результаты полученные по модели с экспериментальными данными и сделать вывод о применимости модели.

\item Построить $p-x,y$ диаграмму при температуре   25.0 $^\circ$C. Сравнить результаты полученные по модели с экспериментальными данными и сделать вывод о применимости модели. \item Используя экспериментальные данные определить параметеры уравнения Вильсона     для описания коэффициента активности. Используя найденные значения параметров построить $T-x,y$ и $p-x,y$ диаграммы. Сравнить с результатом, полученным по уравнению Рауля.\end{enumerate}

\textsc{\textbf{Вариант 7}}
\begin{enumerate}
\item По экспериментальным данным (из справочника теплофизических свойств чистых веществ) получить описание давления паров чистого компонента от температуры. Для циклогексана использовать уравнение Клапейрона $ln(p_i^0(T))=A-\frac{B}{T}$     , для этанола использовать уравнение Риделя $ln(p_i^0(T))=A-\frac{B}{T}+C ln(T)+DT^2$. На основе полученных уравнений и закона Рауля, для смеси циклогексан -- этанол построить $T-x,y$ диаграмму равновесия пар-жидкость при давлении  760.0 мм.рт.ст. . Сравнить результаты полученные по модели с экспериментальными данными и сделать вывод о применимости модели.

\item Построить $p-x,y$ диаграмму при температуре   20.0 $^\circ$C. Сравнить результаты полученные по модели с экспериментальными данными и сделать вывод о применимости модели. \item Используя экспериментальные данные определить параметеры уравнения Маргулеса   для описания коэффициента активности. Используя найденные значения параметров построить $T-x,y$ и $p-x,y$ диаграммы. Сравнить с результатом, полученным по уравнению Рауля.\end{enumerate}

\textsc{\textbf{Вариант 8}}
\begin{enumerate}
\item По экспериментальным данным (из справочника теплофизических свойств чистых веществ) получить описание давления паров чистого компонента от температуры. Для ацетона использовать уравнение Миллера $ln(p_i^0(T))=A-\frac{B}{T}+C T+DT^3$  , для этанола использовать уравнение Ренкина $ln(p_i^0(T))=A-\frac{B}{T}+СT^2$     . На основе полученных уравнений и закона Рауля, для смеси ацетон--этанол построить $T-x,y$ диаграмму равновесия пар-жидкость при давлении  760.0 мм.рт.ст. . Сравнить результаты полученные по модели с экспериментальными данными и сделать вывод о применимости модели.

\item Построить $p-x,y$ диаграмму при температуре   48.0 $^\circ$C. Сравнить результаты полученные по модели с экспериментальными данными и сделать вывод о применимости модели. \item Используя экспериментальные данные определить параметеры уравнения NRTL                 для описания коэффициента активности. Используя найденные значения параметров построить $T-x,y$ и $p-x,y$ диаграммы. Сравнить с результатом, полученным по уравнению Рауля.\end{enumerate}


\end{document}
