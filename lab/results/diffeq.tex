\section{Лабораторная работа №~3 <<Решение дифференциальных уравнений>>}

 \addtocounter{nlab}{1}\textsc{\textbf{Вариант 1}}
\begin{enumerate}
\item Решить численно дифференциальное уравнение $xy\dfrac{dy}{dx}=1-x^2                $ с начальными значениями $y(     7)=     2$ на интервале от $x=     7$ до $x=    12$. Построить график функции.\item Решить численно систему дифференциальных уравнений:
 \begin{equation*}
\left\{
\begin{gathered}
\dfrac{dy}{dx}=y                     \\
\dfrac{dz}{dx}=x^{1/3}               
\end{gathered}
\right.
\end{equation*}
на интервале от $x= 6$ от $x=14$ с граничными условиями: $y( 6)=2.99$, $z( 6)=0.49$. Построить график функции. 
\item Решить численно систему дифференциальных уравнений:
 \begin{equation*}
\left\{
\begin{gathered}
\dfrac{dy}{dx}=\dfrac{x^{0.8}}{5+sin(y)}\\
\dfrac{dz}{dx}=y
\end{gathered}
\right.
\end{equation*}
на интервале от $x= 3$ от $x= 7$ с граничными условиями: $y( 3)=1.44$, $z( 7)=11.29$.  Построить график функции. 
\item  В баке находится 295.2 л раствора, содержащего 48.25 кг соли. В бак непрерывно подается вода (расход воды 0.7 л/мин), которая перемешивается с имеющимся раствором. Смесь вытекает с тем же расходом. Записать дифференциальное уравнения изменения массы соли. Построить график зависимости концентрации соли от времени. Определить какое количество соли в баке останется через  176 минут?

\item  Воронка имеет форму конуса радиуом R= 16.9 см и высотой H=21.7 см, обращенного вершиной вниз. За какое время вытечет вся вода из конуса через круглое отверстие диаметра 1.1 см, сделанное в вершине конуса. Записать дифференциальное уравнения изменения уровня жидкости (или объема) от времени и построить графики.

\end{enumerate}
\textsc{\textbf{Вариант 2}}
\begin{enumerate}
\item Решить численно дифференциальное уравнение $\dfrac{dy}{dx}=\dfrac{y}{x+y}         $ с начальными значениями $y(     4)=     4$ на интервале от $x=     4$ до $x=     9$. Построить график функции.\item Решить численно систему дифференциальных уравнений:
 \begin{equation*}
\left\{
\begin{gathered}
\dfrac{dy}{dx}=\dfrac{3+x}{x+3z}     \\
\dfrac{dz}{dx}=x-2y                  
\end{gathered}
\right.
\end{equation*}
на интервале от $x= 9$ от $x=11$ с граничными условиями: $y( 9)=4.51$, $z( 9)=1.51$. Построить график функции. 
\item Решить численно систему дифференциальных уравнений:
 \begin{equation*}
\left\{
\begin{gathered}
\dfrac{dy}{dx}=x-2y\\
\dfrac{dz}{dx}=\dfrac{x+y}{10+\sin(z)}
\end{gathered}
\right.
\end{equation*}
на интервале от $x= 1$ от $x= 4$ с граничными условиями: $y( 1)=0.68$, $z( 4)=5.14$.  Построить график функции. 
\item  В сосуд, содержащий 21.646 л воды, открытием задвижки (в начальный момент задвижка закрыта) начинают подавать раствор соли концентрацией 5.45 кг/л. Расход воды равномерно увеличивается на 0.1 л/мин. Поступающий в сосуд раствор моментально равномерно перемешивается с водой, и смесь вытекает с той же скоростью. Составить дифференциальное уравнение изменения массы соли в сосуде. Построить график изменения массы соли во времени. Сколько соли будет в сосуде через  25 минут?

\item  Сфера диаметром 17.2 см, имеющая температуру 98.9 $^\circ\mathrm{C}$, для охлаждения была опущена в сосуд объемом  105 л, заполненный водой, имеющей температуру 25.4 $^\circ\mathrm{C}$ . Плотность материала сферы --- 6212.7 \text{кг}/\text{м}$^\mathrm{3}$, теплоемкость --- 1884.8 $\frac{\text{Дж}}{\text{кг}\cdot 	ext{град}}$, плотность воды равна 1000 $\text{кг}/\text{м}^\mathrm{3}$, теплоемкость воды --- 4200 $\frac{\text{Дж}}{\text{кг}\cdot \text{град}}$, коэффициент теплоотдачи равен   440 $\frac{\text{Вт}}{\text{м}^2 \text{град.}}$. Записать систему дифференциальных уравнений изменения температуры воды и сферы при условии отсутствия теплообмена с окружающей средой. Построить график зависимости температуры воды и сферы от времени. Определить время, при котором разница между температурой воды и сферы равна 5 $^\circ\mathrm{C}$. 

\end{enumerate}
\textsc{\textbf{Вариант 3}}
\begin{enumerate}
\item Решить численно дифференциальное уравнение $\dfrac{dy}{dx}=cos(x^2-y)             $ с начальными значениями $y(     2)=     5$ на интервале от $x=     2$ до $x=     6$. Построить график функции.\item Решить численно систему дифференциальных уравнений:
 \begin{equation*}
\left\{
\begin{gathered}
\dfrac{dy}{dx}=\sqrt{x+y+z}          \\
\dfrac{dz}{dx}=y                     
\end{gathered}
\right.
\end{equation*}
на интервале от $x= 7$ от $x=11$ с граничными условиями: $y( 7)=0.37$, $z( 7)=2.68$. Построить график функции. 
\item Решить численно систему дифференциальных уравнений:
 \begin{equation*}
\left\{
\begin{gathered}
\dfrac{dy}{dx}=\sqrt{|x+y+z|}\\
\dfrac{dz}{dx}=sin(x-y+z)
\end{gathered}
\right.
\end{equation*}
на интервале от $x= 1$ от $x= 5$ с граничными условиями: $y( 1)=3.97$, $z( 5)=5.46$.  Построить график функции. 
\item  Записать дифференциальное уравнение распределения температуры вдоль стенки, материал которой имеет следующую зависимость теплопроводности от темепратуры: $\lambda=80.9+0.9\text{T}$. Построить распределение темпарутры по толщине стенки толщиной 9.3 см при температуре стенки с одной стороны равной  763 K и тепловом потоке 280.1 Вт/м. Определить температуру с другой стороны стенки.

 \item  Воронка имеет форму конуса радиуом R= 22.2 см и высотой H=28.1 см, обращенного вершиной вниз. За какое время вытечет вся вода из конуса через круглое отверстие диаметра 0.7 см, сделанное в вершине конуса. Записать дифференциальное уравнения изменения уровня жидкости (или объема) от времени и построить графики.

\end{enumerate}
\textsc{\textbf{Вариант 4}}
\begin{enumerate}
\item Решить численно дифференциальное уравнение $\dfrac{dy}{dx}=\dfrac{y}{x+y}         $ с начальными значениями $y(     3)=     2$ на интервале от $x=     3$ до $x=     9$. Построить график функции.\item Решить численно систему дифференциальных уравнений:
 \begin{equation*}
\left\{
\begin{gathered}
\dfrac{dy}{dx}=\sqrt{x-y+z}          \\
\dfrac{dz}{dx}=\sqrt{x+y+z}          
\end{gathered}
\right.
\end{equation*}
на интервале от $x= 8$ от $x=10$ с граничными условиями: $y( 8)=2.92$, $z( 8)=2.94$. Построить график функции. 
\item Решить численно систему дифференциальных уравнений:
 \begin{equation*}
\left\{
\begin{gathered}
\dfrac{dy}{dx}=sin(x+y+z)\\
\dfrac{dz}{dx}=\dfrac{y}{10+\cos(z)}
\end{gathered}
\right.
\end{equation*}
на интервале от $x= 1$ от $x= 7$ с граничными условиями: $y( 1)=4.07$, $z( 7)=3.07$.  Построить график функции. 
\item  В сосуд, содержащий 24.900 л воды, открытием задвижки (в начальный момент задвижка закрыта) начинают подавать раствор соли концентрацией 5.41 кг/л. Расход воды равномерно увеличивается на 0.2 л/мин. Поступающий в сосуд раствор моментально равномерно перемешивается с водой, и смесь вытекает с той же скоростью. Составить дифференциальное уравнение изменения массы соли в сосуде. Построить график изменения массы соли во времени. Сколько соли будет в сосуде через  26 минут?

\item  Сфера диаметром 25.7 см, имеющая температуру 110.0 $^\circ\mathrm{C}$, для охлаждения была опущена в сосуд объемом   88 л, заполненный водой, имеющей температуру 33.6 $^\circ\mathrm{C}$ . Плотность материала сферы --- 2044.8 \text{кг}/\text{м}$^\mathrm{3}$, теплоемкость --- 1230.3 $\frac{\text{Дж}}{\text{кг}\cdot 	ext{град}}$, плотность воды равна 1000 $\text{кг}/\text{м}^\mathrm{3}$, теплоемкость воды --- 4200 $\frac{\text{Дж}}{\text{кг}\cdot \text{град}}$, коэффициент теплоотдачи равен   398 $\frac{\text{Вт}}{\text{м}^2 \text{град.}}$. Записать систему дифференциальных уравнений изменения температуры воды и сферы при условии отсутствия теплообмена с окружающей средой. Построить график зависимости температуры воды и сферы от времени. Определить время, при котором разница между температурой воды и сферы равна 5 $^\circ\mathrm{C}$. 

\end{enumerate}
\textsc{\textbf{Вариант 5}}
\begin{enumerate}
\item Решить численно дифференциальное уравнение $\dfrac{dy}{dx}=cos(x^2-y)             $ с начальными значениями $y(     5)=     2$ на интервале от $x=     5$ до $x=     9$. Построить график функции.\item Решить численно систему дифференциальных уравнений:
 \begin{equation*}
\left\{
\begin{gathered}
\dfrac{dy}{dx}=\dfrac{x^2}{y}        \\
\dfrac{dz}{dx}=\sqrt{x+y+z}          
\end{gathered}
\right.
\end{equation*}
на интервале от $x= 2$ от $x=12$ с граничными условиями: $y( 2)=0.76$, $z( 2)=4.07$. Построить график функции. 
\item Решить численно систему дифференциальных уравнений:
 \begin{equation*}
\left\{
\begin{gathered}
\dfrac{dy}{dx}=sin(x+y+z)\\
\dfrac{dz}{dx}=x-2y
\end{gathered}
\right.
\end{equation*}
на интервале от $x= 3$ от $x= 6$ с граничными условиями: $y( 3)=2.53$, $z( 6)=1.53$.  Построить график функции. 
\item  Сосуд объемом 36.23 л содержит воздух (80 \% кислорода, 20 \% азота). В сосуд втекает 1.33 л азота в секунду, который моментально перемешивается, и вытекает такое же количество смеси. Записать дифференциальное уравнение изменения объема азота в сосуде. Построить график изменения объема азота по времени. Определить через какое время в сосуде будет 99 \% азота

\item  В прямоугольный бак сечением 23.7 см x 27.1 и высотой 83.7 см поступает 1.0 л в секунду. На дне имеется отверстие площадью 1.5 $\text{см}^2$. За какое время наполнится бак? Записать  дифференциальное уравнение изменения уровня жидкости (или объема воды в баке) от времени, построить график. Сравнить результат с временем заполнения этого бака без отверстия.

 \end{enumerate}
\textsc{\textbf{Вариант 6}}
\begin{enumerate}
\item Решить численно дифференциальное уравнение $dy=(x^2-1)dx                          $ с начальными значениями $y(     9)=     2$ на интервале от $x=     9$ до $x=    16$. Построить график функции.\item Решить численно систему дифференциальных уравнений:
 \begin{equation*}
\left\{
\begin{gathered}
\dfrac{dy}{dx}=sin(x-y+z)            \\
\dfrac{dz}{dx}=z                     
\end{gathered}
\right.
\end{equation*}
на интервале от $x= 9$ от $x=15$ с граничными условиями: $y( 9)=1.59$, $z( 9)=2.72$. Построить график функции. 
\item Решить численно систему дифференциальных уравнений:
 \begin{equation*}
\left\{
\begin{gathered}
\dfrac{dy}{dx}=x\\
\dfrac{dz}{dx}=\dfrac{y}{10+\cos(z)}
\end{gathered}
\right.
\end{equation*}
на интервале от $x= 3$ от $x= 8$ с граничными условиями: $y( 3)=2.87$, $z( 8)=8.80$.  Построить график функции. 
\item  В сосуд, содержащий 27.769 л воды, открытием задвижки (в начальный момент задвижка закрыта) начинают подавать раствор соли концентрацией 3.81 кг/л. Расход воды равномерно увеличивается на 0.3 л/мин. Поступающий в сосуд раствор моментально равномерно перемешивается с водой, и смесь вытекает с той же скоростью. Составить дифференциальное уравнение изменения массы соли в сосуде. Построить график изменения массы соли во времени. Сколько соли будет в сосуде через  17 минут?

\item  Сфера диаметром 30.1 см, имеющая температуру 116.9 $^\circ\mathrm{C}$, для охлаждения была опущена в сосуд объемом  129 л, заполненный водой, имеющей температуру 37.5 $^\circ\mathrm{C}$ . Плотность материала сферы --- 4049.5 \text{кг}/\text{м}$^\mathrm{3}$, теплоемкость --- 1248.9 $\frac{\text{Дж}}{\text{кг}\cdot 	ext{град}}$, плотность воды равна 1000 $\text{кг}/\text{м}^\mathrm{3}$, теплоемкость воды --- 4200 $\frac{\text{Дж}}{\text{кг}\cdot \text{град}}$, коэффициент теплоотдачи равен   315 $\frac{\text{Вт}}{\text{м}^2 \text{град.}}$. Записать систему дифференциальных уравнений изменения температуры воды и сферы при условии отсутствия теплообмена с окружающей средой. Построить график зависимости температуры воды и сферы от времени. Определить время, при котором разница между температурой воды и сферы равна 5 $^\circ\mathrm{C}$. 

\end{enumerate}
\newpage
