\section{Лабораторная работа №~5 <<Определение условий фазовых равновесий пар~-- жидкость>>}

 \addtocounter{nlab}{1}\textsc{\textbf{Вариант 1}}
\begin{enumerate}
\item По экспериментальным данным (из справочника теплофизических свойств чистых веществ) получить описание давления паров чистого компонента от температуры. Для циклогексана использовать уравнение Риделя $ln(p_i^0(T))=A-\frac{B}{T}+C ln(T)+DT^2$, для этанола использовать уравнение Миллера $ln(p_i^0(T))=A-\frac{B}{T}+C T+DT^3$  . На основе полученных уравнений и закона Рауля, для смеси циклогексан -- этанол построить $p-x,y$ и $y-x$ диаграммы равновесия пар-жидкость при температуре   20.0 $^\circ$C . Сравнить результаты полученные по модели с экспериментальными данными и сделать вывод о применимости модели.

\item Построить $T-x,y$ и $y-x$ диаграммы при давлении  760.0 мм.рт.ст.. Сравнить результаты полученные по модели с экспериментальными данными и сделать вывод о применимости модели. \item Используя экспериментальные данные определить параметеры уравнения Ван~-- Лаара для описания коэффициента активности. Используя найденные значения параметров построить $T-x,y$ и $p-x,y$ диаграммы. Сравнить с результатом, полученным по уравнению Рауля.\end{enumerate}

\textsc{\textbf{Вариант 2}}
\begin{enumerate}
\item По экспериментальным данным (из справочника теплофизических свойств чистых веществ) получить описание давления паров чистого компонента от температуры. Для ацетона использовать уравнение Антуана $ln(p_i^0(T))=A-\frac{B}{T+C}$         , для хлороформа использовать уравнение Антуана $ln(p_i^0(T))=A-\frac{B}{T+C}$         . На основе полученных уравнений и закона Рауля, для смеси ацетон-хлороформ построить $p-x,y$ и $y-x$ диаграммы равновесия пар-жидкость при температуре   45.0 $^\circ$C . Сравнить результаты полученные по модели с экспериментальными данными и сделать вывод о применимости модели.

\item Построить $T-x,y$ и $y-x$ диаграммы при давлении  735.0 мм.рт.ст.. Сравнить результаты полученные по модели с экспериментальными данными и сделать вывод о применимости модели. \item Используя экспериментальные данные определить параметеры уравнения Вильсона     для описания коэффициента активности. Используя найденные значения параметров построить $T-x,y$ и $p-x,y$ диаграммы. Сравнить с результатом, полученным по уравнению Рауля.\end{enumerate}

\textsc{\textbf{Вариант 3}}
\begin{enumerate}
\item По экспериментальным данным (из справочника теплофизических свойств чистых веществ) получить описание давления паров чистого компонента от температуры. Для ацетона использовать уравнение Антуана $ln(p_i^0(T))=A-\frac{B}{T+C}$         , для метанола использовать уравнение Клапейрона $ln(p_i^0(T))=A-\frac{B}{T}$     . На основе полученных уравнений и закона Рауля, для смеси ацетон--метанол построить $p-x,y$ и $y-x$ диаграммы равновесия пар-жидкость при температуре   20.0 $^\circ$C . Сравнить результаты полученные по модели с экспериментальными данными и сделать вывод о применимости модели.

\item Построить $T-x,y$ и $y-x$ диаграммы при давлении  760.0 мм.рт.ст.. Сравнить результаты полученные по модели с экспериментальными данными и сделать вывод о применимости модели. \item Используя экспериментальные данные определить параметеры уравнения Вильсона     для описания коэффициента активности. Используя найденные значения параметров построить $T-x,y$ и $p-x,y$ диаграммы. Сравнить с результатом, полученным по уравнению Рауля.\end{enumerate}

\textsc{\textbf{Вариант 4}}
\begin{enumerate}
\item По экспериментальным данным (из справочника теплофизических свойств чистых веществ) получить описание давления паров чистого компонента от температуры. Для ацетона использовать уравнение Клапейрона $ln(p_i^0(T))=A-\frac{B}{T}$     , для бензола использовать уравнение Кеэгоу $ln(p_i^0(T))=A+\frac{B}{T}+СT+BT^2$    . На основе полученных уравнений и закона Рауля, для смеси ацетон-бензол построить $p-x,y$ и $y-x$ диаграммы равновесия пар-жидкость при температуре   25.0 $^\circ$C . Сравнить результаты полученные по модели с экспериментальными данными и сделать вывод о применимости модели.

\item Построить $T-x,y$ и $y-x$ диаграммы при давлении  760.0 мм.рт.ст.. Сравнить результаты полученные по модели с экспериментальными данными и сделать вывод о применимости модели. \item Используя экспериментальные данные определить параметеры уравнения Маргулеса   для описания коэффициента активности. Используя найденные значения параметров построить $T-x,y$ и $p-x,y$ диаграммы. Сравнить с результатом, полученным по уравнению Рауля.\end{enumerate}

\textsc{\textbf{Вариант 5}}
\begin{enumerate}
\item По экспериментальным данным (из справочника теплофизических свойств чистых веществ) получить описание давления паров чистого компонента от температуры. Для бензола использовать уравнение Риделя $ln(p_i^0(T))=A-\frac{B}{T}+C ln(T)+DT^2$, для толуола использовать уравнение Кеэгоу $ln(p_i^0(T))=A+\frac{B}{T}+СT+BT^2$    . На основе полученных уравнений и закона Рауля, для смеси бензол--толуол построить $p-x,y$ и $y-x$ диаграммы равновесия пар-жидкость при температуре  160.0 $^\circ$C . Сравнить результаты полученные по модели с экспериментальными данными и сделать вывод о применимости модели.

\item Построить $T-x,y$ и $y-x$ диаграммы при давлении  760.0 мм.рт.ст.. Сравнить результаты полученные по модели с экспериментальными данными и сделать вывод о применимости модели. \item Используя экспериментальные данные определить параметеры уравнения Ван~-- Лаара для описания коэффициента активности. Используя найденные значения параметров построить $T-x,y$ и $p-x,y$ диаграммы. Сравнить с результатом, полученным по уравнению Рауля.\end{enumerate}

\textsc{\textbf{Вариант 6}}
\begin{enumerate}
\item По экспериментальным данным (из справочника теплофизических свойств чистых веществ) получить описание давления паров чистого компонента от температуры. Для ацетона использовать уравнение Антуана $ln(p_i^0(T))=A-\frac{B}{T+C}$         , для метанола использовать уравнение Миллера $ln(p_i^0(T))=A-\frac{B}{T}+C T+DT^3$  . На основе полученных уравнений и закона Рауля, для смеси ацетон--метанол построить $p-x,y$ и $y-x$ диаграммы равновесия пар-жидкость при температуре  100.0 $^\circ$C . Сравнить результаты полученные по модели с экспериментальными данными и сделать вывод о применимости модели.

\item Построить $T-x,y$ и $y-x$ диаграммы при давлении  760.0 мм.рт.ст.. Сравнить результаты полученные по модели с экспериментальными данными и сделать вывод о применимости модели. \item Используя экспериментальные данные определить параметеры уравнения Вильсона     для описания коэффициента активности. Используя найденные значения параметров построить $T-x,y$ и $p-x,y$ диаграммы. Сравнить с результатом, полученным по уравнению Рауля.\end{enumerate}

