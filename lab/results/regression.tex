\section{Лабораторная работа №~2 <<Регрессионный анализ, методы аппроксимации>>}

  \addtocounter{nlab}{1}\textsc{\textbf{Вариант 1}}

\textbf{Задание 1} В результате измерения зависимости переменной состояния $y$ от входного фактора $x$ были получены значения, представленные в таблице. Описать табличные данные следующими функциональными зависимостями:
 \begin{itemize} 
\item $y(x)=a x+b$
\item $y(x)=a_3 x^3 +a_2 x^2 + a_1 x +a_0$
\item $y(x)=a e^{b x}+c  $
\item $y(x)=a \cdot x^b+c$
\item $y(x)=a_0 x^{0.3}               +a_1 \dfrac{1}{x}           +a_2 x                     $
\item $y(x)=\dfrac{A x^B} {C+x}                $
\item параболический сплайн
\end{itemize}
\begin{table}[h]
\begin{tabular}{|c|c|}
\hline
x & y \\ \hline
 3.50 &     -96.37 \\ \hline 
11.60 &    -520.86 \\ \hline 
19.70 &    -886.38 \\ \hline 
27.80 &   -1769.58 \\ \hline 
35.90 &   -3610.62 \\ \hline 
44.00 &   -4051.54 \\ \hline 
52.10 &   -6301.45 \\ \hline 
60.20 &   -7681.06 \\ \hline 
68.30 &   -6968.90 \\ \hline 
76.40 &   -8068.60 \\ \hline 
\end{tabular}
\end{table}

\textbf{Задание 2}  Используя данные из справочника теплофизических свойств описать удельный объем жидкого н-октана при р = 150 атм н-октана. В качестве аппроксимирующей функции может выступать любое выражение, однако максимальное отклонение не должно превышать 10\%. Определить, при какой температуре удельный объем жидкого н-октана при р = 150 атм равна $     1.5 \cdot 10^{-3} \frac{\text{м}^3}{ \text{кг}}$.

\textbf{Задание 3} В таблице представлено изменение концентрации исходного вещества (c) от вермени ($\tau$). Определить порядок реакции и константу скорости реакции.

\begin{table}[h]
\begin{tabular}{|c|c|}
\hline
$\tau$, с & c, моль/л \\ \hline
 0.00 &      13.63 \\ \hline 
 0.69 &      10.72 \\ \hline 
 1.38 &       7.15 \\ \hline 
 2.08 &       6.40 \\ \hline 
 2.77 &       4.51 \\ \hline 
 3.46 &       3.46 \\ \hline 
 4.15 &       2.91 \\ \hline 
 4.84 &       2.09 \\ \hline 
 5.54 &       1.57 \\ \hline 
 6.23 &       1.39 \\ \hline 
 6.92 &       0.98 \\ \hline 
 7.61 &       0.78 \\ \hline 
 8.31 &       0.58 \\ \hline 
 9.00 &       0.45 \\ \hline 
 9.69 &       0.33 \\ \hline 
10.38 &       0.26 \\ \hline 
\end{tabular}
\end{table}

\newpage

\textsc{\textbf{Вариант 2}}

\textbf{Задание 1} В результате измерения зависимости переменной состояния $y$ от входного фактора $x$ были получены значения, представленные в таблице. Описать табличные данные следующими функциональными зависимостями:
 \begin{itemize} 
\item $y(x)=a x+b$
\item $y(x)=a_3 x^3 +a_2 x^2 + a_1 x +a_0$
\item $y(x)=a e^{b x}+c  $
\item $y(x)=a \cdot x^b+c$
\item $y(x)=a_0 x^{3.7}               +a_1 \dfrac{x}{1+x}         +a_2 \dfrac{x^{2.4}}{1+x^2}$
\item $y(x)=\dfrac{B+x^C}{A+x}                 $
\item параболический сплайн
\end{itemize}
\begin{table}[h]
\begin{tabular}{|c|c|}
\hline
x & y \\ \hline
 8.00 &     110.92 \\ \hline 
 8.90 &     313.38 \\ \hline 
 9.80 &     399.96 \\ \hline 
10.70 &     868.39 \\ \hline 
11.60 &    1815.88 \\ \hline 
12.50 &    2369.95 \\ \hline 
13.40 &    4104.98 \\ \hline 
14.30 &    6993.25 \\ \hline 
15.20 &    7882.51 \\ \hline 
16.10 &   10856.88 \\ \hline 
\end{tabular}
\end{table}

\textbf{Задание 2}  Используя данные из справочника теплофизических свойств описать удельный объем изобутана при р = 1 атм изобутана. В качестве аппроксимирующей функции может выступать любое выражение, однако максимальное отклонение не должно превышать 10\%. Определить, при какой температуре удельный объем изобутана при р = 1 атм равна $   734.3 \frac {\text{дм}^3}{\text{кг}}$.

\textbf{Задание 3} В таблице представлено изменение концентрации исходного вещества (c) от вермени ($\tau$). Определить порядок реакции и константу скорости реакции.

\begin{table}[h]
\begin{tabular}{|c|c|}
\hline
$\tau$, с & c, моль/л \\ \hline
 0.00 &      17.38 \\ \hline 
 0.97 &      11.21 \\ \hline 
 1.94 &       7.34 \\ \hline 
 2.91 &       5.52 \\ \hline 
 3.88 &       4.13 \\ \hline 
 4.85 &       2.91 \\ \hline 
 5.82 &       1.94 \\ \hline 
 6.79 &       1.56 \\ \hline 
 7.76 &       1.10 \\ \hline 
 8.73 &       0.84 \\ \hline 
 9.69 &       0.73 \\ \hline 
10.66 &       0.57 \\ \hline 
11.63 &       0.39 \\ \hline 
12.60 &       0.30 \\ \hline 
13.57 &       0.25 \\ \hline 
\end{tabular}
\end{table}

\newpage

\textsc{\textbf{Вариант 3}}

\textbf{Задание 1} В результате измерения зависимости переменной состояния $y$ от входного фактора $x$ были получены значения, представленные в таблице. Описать табличные данные следующими функциональными зависимостями:
 \begin{itemize} 
\item $y(x)=a x+b$
\item $y(x)=a_2 x^2 + a_1 x +a_0$
\item $y(x)=a e^{b x}+c  $
\item $y(x)=a \cdot x^b+c$
\item $y(x)=a_0 x^{1.2}               +a_1 \dfrac{1}{x}           +a_2 x^{0.3}               $
\item $y(x)=\dfrac{A x^B} {C+x}                $
\item кубический сплайн        
\end{itemize}
\begin{table}[h]
\begin{tabular}{|c|c|}
\hline
x & y \\ \hline
 9.60 &     217.09 \\ \hline 
18.40 &     413.50 \\ \hline 
27.20 &     919.53 \\ \hline 
36.00 &    1659.13 \\ \hline 
44.80 &    2136.08 \\ \hline 
53.60 &    3194.49 \\ \hline 
62.40 &    3604.30 \\ \hline 
71.20 &    4662.92 \\ \hline 
80.00 &    5761.31 \\ \hline 
88.80 &    7662.46 \\ \hline 
\end{tabular}
\end{table}

\textbf{Задание 2}  Используя данные из справочника теплофизических свойств описать теплопроводность н-гексаана при р = 40 бар н-гексана. В качестве аппроксимирующей функции может выступать любое выражение, однако максимальное отклонение не должно превышать 10\%. Определить, при какой температуре теплопроводность н-гексаана при р = 40 бар равна $   123.3 \cdot 10^{-3} \frac {\text{Вт}} {\text{м} \cdot \text{град}}$.

\textbf{Задание 3} В таблице представлено изменение концентрации исходного вещества (c) от вермени ($\tau$). Определить порядок реакции и константу скорости реакции.

\begin{table}[h]
\begin{tabular}{|c|c|}
\hline
$\tau$, с & c, моль/л \\ \hline
 0.00 &      17.06 \\ \hline 
 1.47 &      13.74 \\ \hline 
 2.94 &      11.69 \\ \hline 
 4.41 &       9.41 \\ \hline 
 5.88 &       7.70 \\ \hline 
 7.35 &       6.04 \\ \hline 
 8.82 &       4.60 \\ \hline 
10.29 &       3.65 \\ \hline 
11.76 &       2.28 \\ \hline 
13.23 &       1.48 \\ \hline 
14.70 &       0.96 \\ \hline 
16.17 &       0.52 \\ \hline 
\end{tabular}
\end{table}

\newpage

\textsc{\textbf{Вариант 4}}

\textbf{Задание 1} В результате измерения зависимости переменной состояния $y$ от входного фактора $x$ были получены значения, представленные в таблице. Описать табличные данные следующими функциональными зависимостями:
 \begin{itemize} 
\item $y(x)=a x+b$
\item $y(x)=a_3 x^3 +a_2 x^2 + a_1 x +a_0$
\item $y(x)=a e^{b x}+c  $
\item $y(x)=a \cdot x^b+c$
\item $y(x)=a_0 \sqrt{x}              +a_1 x^{3.7}                +a_2 x                     $
\item $y(x)=A \cdot e^{-\dfrac{B}{x}+C}        $
\item параболический сплайн
\end{itemize}
\begin{table}[h]
\begin{tabular}{|c|c|}
\hline
x & y \\ \hline
 2.20 &       0.50 \\ \hline 
12.00 &     -84.13 \\ \hline 
21.80 &    -193.95 \\ \hline 
31.60 &    -263.95 \\ \hline 
41.40 &    -533.30 \\ \hline 
51.20 &    -488.80 \\ \hline 
61.00 &    -255.94 \\ \hline 
70.80 &      73.61 \\ \hline 
80.60 &     609.77 \\ \hline 
90.40 &    1791.56 \\ \hline 
\end{tabular}
\end{table}

\textbf{Задание 2}  Используя данные из справочника теплофизических свойств описать вязкость жидкой фазы на линии насыщения метана. В качестве аппроксимирующей функции может выступать любое выражение, однако максимальное отклонение не должно превышать 10\%. Определить, при какой температуре вязкость жидкой фазы на линии насыщения равна $    29.0 \cdot 10^{-6} \text{Па} \cdot \text{с}$.

\textbf{Задание 3} В таблице представлено изменение концентрации исходного вещества (c) от вермени ($\tau$). Определить порядок реакции и константу скорости реакции.

\begin{table}[h]
\begin{tabular}{|c|c|}
\hline
$\tau$, с & c, моль/л \\ \hline
 0.00 &      16.11 \\ \hline 
 0.91 &       5.00 \\ \hline 
 1.81 &       2.86 \\ \hline 
 2.72 &       1.77 \\ \hline 
 3.62 &       1.27 \\ \hline 
 4.53 &       0.88 \\ \hline 
 5.43 &       0.78 \\ \hline 
 6.34 &       0.59 \\ \hline 
 7.24 &       0.47 \\ \hline 
 8.15 &       0.40 \\ \hline 
 9.05 &       0.37 \\ \hline 
 9.96 &       0.32 \\ \hline 
10.86 &       0.32 \\ \hline 
11.77 &       0.26 \\ \hline 
12.67 &       0.22 \\ \hline 
13.58 &       0.22 \\ \hline 
\end{tabular}
\end{table}

\newpage

\textsc{\textbf{Вариант 5}}

\textbf{Задание 1} В результате измерения зависимости переменной состояния $y$ от входного фактора $x$ были получены значения, представленные в таблице. Описать табличные данные следующими функциональными зависимостями:
 \begin{itemize} 
\item $y(x)=a x+b$
\item $y(x)=a_3 x^3 +a_2 x^2 + a_1 x +a_0$
\item $y(x)=a e^{b x}+c  $
\item $y(x)=a \cdot x^b+c$
\item $y(x)=a_0 x^{0.3}               +a_1 \sqrt{x}               +a_2 x^{3.7}               $
\item $y(x)=\dfrac{B+x^C}{A+x}                 $
\item кубический сплайн        
\end{itemize}
\begin{table}[h]
\begin{tabular}{|c|c|}
\hline
x & y \\ \hline
 3.60 &      71.80 \\ \hline 
 7.10 &     146.69 \\ \hline 
10.60 &     266.71 \\ \hline 
14.10 &     453.69 \\ \hline 
17.60 &     528.94 \\ \hline 
21.10 &     697.74 \\ \hline 
24.60 &     796.44 \\ \hline 
28.10 &    1162.66 \\ \hline 
31.60 &     947.00 \\ \hline 
35.10 &    1138.00 \\ \hline 
\end{tabular}
\end{table}

\textbf{Задание 2}  Используя данные из справочника теплофизических свойств описать удельный объем жидкого н-гексана при р = 90 атм н-гексана. В качестве аппроксимирующей функции может выступать любое выражение, однако максимальное отклонение не должно превышать 10\%. Определить, при какой температуре удельный объем жидкого н-гексана при р = 90 атм равна $     1.8 \cdot 10^{-3} \frac{\text{м}^3}{ \text{кг}}$.

\textbf{Задание 3} В таблице представлено изменение концентрации исходного вещества (c) от вермени ($\tau$). Определить порядок реакции и константу скорости реакции.

\begin{table}[h]
\begin{tabular}{|c|c|}
\hline
$\tau$, с & c, моль/л \\ \hline
 0.00 &      18.42 \\ \hline 
 1.72 &       1.61 \\ \hline 
 3.44 &       0.87 \\ \hline 
 5.15 &       0.77 \\ \hline 
 6.87 &       0.60 \\ \hline 
 8.59 &       0.50 \\ \hline 
10.31 &       0.49 \\ \hline 
12.02 &       0.39 \\ \hline 
13.74 &       0.35 \\ \hline 
15.46 &       0.31 \\ \hline 
17.18 &       0.28 \\ \hline 
18.90 &       0.24 \\ \hline 
20.61 &       0.25 \\ \hline 
22.33 &       0.25 \\ \hline 
24.05 &       0.21 \\ \hline 
25.77 &       0.19 \\ \hline 
\end{tabular}
\end{table}

\newpage

\textsc{\textbf{Вариант 6}}

\textbf{Задание 1} В результате измерения зависимости переменной состояния $y$ от входного фактора $x$ были получены значения, представленные в таблице. Описать табличные данные следующими функциональными зависимостями:
 \begin{itemize} 
\item $y(x)=a x+b$
\item $y(x)=a_3 x^3 +a_2 x^2 + a_1 x +a_0$
\item $y(x)=a e^{b x}+c  $
\item $y(x)=a \cdot x^b+c$
\item $y(x)=a_0 x                     +a_1 \dfrac{1}{x}           +a_2 x^{1.2}               $
\item $y(x)=\dfrac{A x^2 + B x +C}{\sqrt{x} +D}$
\item параболический сплайн
\end{itemize}
\begin{table}[h]
\begin{tabular}{|c|c|}
\hline
x & y \\ \hline
 0.80 &      -3.18 \\ \hline 
 5.70 &      -9.26 \\ \hline 
10.60 &      66.16 \\ \hline 
15.50 &     278.61 \\ \hline 
20.40 &     647.52 \\ \hline 
25.30 &    1463.70 \\ \hline 
30.20 &    2407.91 \\ \hline 
35.10 &    2995.30 \\ \hline 
40.00 &    5388.13 \\ \hline 
44.90 &    8372.87 \\ \hline 
\end{tabular}
\end{table}

\textbf{Задание 2}  Используя данные из справочника теплофизических свойств описать вязкость газообразного изобутана при р = 1 бар изобутана. В качестве аппроксимирующей функции может выступать любое выражение, однако максимальное отклонение не должно превышать 10\%. Определить, при какой температуре вязкость газообразного изобутана при р = 1 бар равна $    86.0 \cdot 10^{-7} \text{Па} \cdot \text{с}$.

\textbf{Задание 3} В таблице представлено изменение концентрации исходного вещества (c) от вермени ($\tau$). Определить порядок реакции и константу скорости реакции.

\begin{table}[h]
\begin{tabular}{|c|c|}
\hline
$\tau$, с & c, моль/л \\ \hline
 0.00 &      12.02 \\ \hline 
 3.40 &      11.44 \\ \hline 
 6.80 &       9.57 \\ \hline 
10.20 &       9.22 \\ \hline 
13.60 &       7.75 \\ \hline 
17.00 &       5.97 \\ \hline 
20.40 &       5.18 \\ \hline 
23.79 &       3.56 \\ \hline 
27.19 &       2.99 \\ \hline 
30.59 &       2.94 \\ \hline 
33.99 &       1.80 \\ \hline 
37.39 &       1.50 \\ \hline 
40.79 &       0.85 \\ \hline 
44.19 &       0.67 \\ \hline 
47.59 &       0.38 \\ \hline 
\end{tabular}
\end{table}

\newpage

\newpage
