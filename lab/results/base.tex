\section{{Лабораторная работа №~1 <<Основы математического пакета MathCad>>}}

 \addtocounter{nlab}{1}\textsc{\textbf{Вариант 1}}

\begin{enumerate}
\item Вычислить: 
\begin{equation*}\sin(4+\cos(5))                   \end{equation*}
\begin{equation*}\cos(14)                          \end{equation*}
\begin{equation*}\sin(4+\cos(5))                   \end{equation*}

\item Вычислить аналитически: 
 \begin{equation*} \int (1-e^x)^2 dx          \end{equation*}\begin{equation*} {\dfrac{\partial} {\partial x} x^2 +sin(x) -cos(x)} \end{equation*}


\item Расход воды в трубе за секунду составляет $0.73$ $\text{м}^3$. Найти скорость воды при ширине канала $2.7$ м и глубине воды $303.5$ см.

\item Построить график функции $y(x)=\dfrac{x+3}{x^2+1}    $ в диапазоне от $x=-1.4$ до $x=30.1$, определить, при каком значении $x$ $y=1.0$. На этом же графике построить функцию $z(x)=\dfrac{x+5}{x^2+1}        $. Определить координаты точки пересечения графиков. \item Решить уравнение: $x^3+4 \sqrt{x}=10    $

\item Определить сумму, произведение матриц $A=\begin{bmatrix}
8.4 &4.6 &4.2 \\
8.3 &6.8 &9.2 \\
0.2 &8.3 &9 \\
\end{bmatrix}
$ и $B=\begin{bmatrix}
1 &3 &0.9 \\
7 &0.7 &3 \\
4.7 &4 &6.6 \\
\end{bmatrix}
$. Вычислить $D_{i,j}=A_{i,j}  /  B_{i,j}$ и определитель матрицы D

\item Решить систему уравнений: \begin{equation*} \begin{cases} \sqrt{x}+2y=2        \\ x^2+y^2=2                 \end{cases} \end{equation*} 

\end{enumerate}
\textsc{\textbf{Вариант 2}}

\begin{enumerate}
\item Вычислить: 
\begin{equation*}\sin \left( \dfrac{\pi}{7} \right)\end{equation*}
\begin{equation*}\dfrac{7+1}{sin(\pi)}             \end{equation*}
\begin{equation*}\dfrac{5}{10+\ln(5)}              \end{equation*}

\item Вычислить аналитически: 
 \begin{equation*} \int \sin(7x) dx           \end{equation*}\begin{equation*} \dfrac{\partial} {\partial x} x\cos(x)              \end{equation*}


\item Расход воды в трубе за секунду составляет $1.03$ $\text{м}^3$. Найти скорость воды при ширине канала $2.7$ м и глубине воды $524.5$ см.

\item Построить график функции $y(x)=\sin(x)+\sqrt{x}      $ в диапазоне от $x=-1.8$ до $x=2.2$, определить, при каком значении $x$ $y=1.0$. На этом же графике построить функцию $z(x)=ln(x^2+1)                 $. Определить координаты точки пересечения графиков. \item Решить уравнение: $\sqrt{x}+2x=2        $

\item Определить сумму, произведение матриц $A=\begin{bmatrix}
7.9 &5 &8.9 &6 \\
2.8 &4.8 &8.6 &4.8 \\
5.4 &1.1 &0.7 &9.1 \\
4.8 &9.1 &1.9 &8.2 \\
\end{bmatrix}
$ и $B=\begin{bmatrix}
8.2 &7.3 &8.6 &5.2 \\
1.4 &1.5 &3.6 &8.8 \\
8.9 &4.9 &2.6 &4.4 \\
9.6 &4.7 &1.3 &4.7 \\
\end{bmatrix}
$. Вычислить $D_{i,j}=A_{i,j}  /  B_{i,j}$ и определитель матрицы D

\item Решить систему уравнений: \begin{equation*} \begin{cases} \sin(x)+\cos(y)=1    \\ x^2 + y^3=1               \end{cases} \end{equation*} 

\end{enumerate}
\textsc{\textbf{Вариант 3}}

\begin{enumerate}
\item Вычислить: 
\begin{equation*}\sin(4+\cos(5))                   \end{equation*}
\begin{equation*}\sin(4 e^2)                       \end{equation*}
\begin{equation*}\ln(3+sin(4))                     \end{equation*}

\item Вычислить аналитически: 
 \begin{equation*} \int (1-e^x)^2 dx          \end{equation*}\begin{equation*} \dfrac{\partial} {\partial x} x\cos(x)              \end{equation*}


\item Расход воды в трубе за секунду составляет $1.08$ $\text{м}^3$. Найти скорость воды при ширине канала $1.6$ м и глубине воды $515.6$ см.

\item Построить график функции $y(x)=\dfrac{x+4}{x^2+1}    $ в диапазоне от $x=3.9$ до $x=20.4$, определить, при каком значении $x$ $y=1.0$. На этом же графике построить функцию $z(x)=ln(x^2+1)                 $. Определить координаты точки пересечения графиков. \item Решить уравнение: $\sin(x^2)+\cos(x^2)=1$

\item Определить сумму, произведение матриц $A=\begin{bmatrix}
1.1 &9.2 &0.6 \\
1.5 &7.7 &6.2 \\
7.6 &8.8 &4.7 \\
\end{bmatrix}
$ и $B=\begin{bmatrix}
3 &7.7 &6.8 \\
7.5 &5.5 &0.6 \\
0 &4.4 &6.8 \\
\end{bmatrix}
$. Вычислить $D_{i,j}=A_{i,j}  /  B_{i,j}$ и определитель матрицы D

\item Решить систему уравнений: \begin{equation*} \begin{cases} x^3+4 \sqrt{y}=10    \\ \dfrac{x^3+10}{x^2+1} = 1 \end{cases} \end{equation*} 

\end{enumerate}
\textsc{\textbf{Вариант 4}}

\begin{enumerate}
\item Вычислить: 
\begin{equation*}\ln(3+sin(4))                     \end{equation*}
\begin{equation*}\cos(2 \pi)                       \end{equation*}
\begin{equation*}\sin(4+\cos(5))                   \end{equation*}

\item Вычислить аналитически: 
 \begin{equation*} \int \dfrac{x^4+1}{x^2} dx \end{equation*}\begin{equation*} {\dfrac{\partial} {\partial x}\sin(5 x +3)}         \end{equation*}


\item Человек стоит на полу. Масса его $76.2$ кг. Площадь подошв $634.4$ $\text{см}^2$. Какое давление оказывает человек на пол?

\item Построить график функции $y(x)=x                     $ в диапазоне от $x=3.7$ до $x=18.2$, определить, при каком значении $x$ $y=1.0$. На этом же графике построить функцию $z(x)=-0.5 x^2 + x              $. Определить координаты точки пересечения графиков. \item Решить уравнение: $\dfrac{x+4}{5}=1     $

\item Определить сумму, произведение матриц $A=\begin{bmatrix}
1.7 &0.3 &6.3 \\
7.1 &8.9 &6.9 \\
1.3 &9.5 &5.9 \\
\end{bmatrix}
$ и $B=\begin{bmatrix}
6.1 &1.6 &3.4 \\
8.6 &0.9 &7.8 \\
1 &4.9 &0.3 \\
\end{bmatrix}
$. Вычислить $D_{i,j}=A_{i,j}  +  B_{i,j}$ и определитель матрицы D

\item Решить систему уравнений: \begin{equation*} \begin{cases} x^2+y=3              \\ 5x +3y=10                 \end{cases} \end{equation*} 

\end{enumerate}
\textsc{\textbf{Вариант 5}}

\begin{enumerate}
\item Вычислить: 
\begin{equation*}\sin \left( \dfrac{\pi}{7} \right)\end{equation*}
\begin{equation*}\sin(10)+\dfrac{1}{\sin{10}}      \end{equation*}
\begin{equation*}\sin \left( \dfrac{\pi}{7} \right)\end{equation*}

\item Вычислить аналитически: 
 \begin{equation*} \int \dfrac{x^4+1}{x^2} dx \end{equation*}\begin{equation*} \dfrac{\partial} {\partial x} x^3 +x^2-10           \end{equation*}


\item Человек стоит на полу. Масса его $74.4$ кг. Площадь подошв $539.6$ $\text{см}^2$. Какое давление оказывает человек на пол?

\item Построить график функции $y(x)=\sqrt{x^2}            $ в диапазоне от $x=4.8$ до $x=46.8$, определить, при каком значении $x$ $y=1.0$. На этом же графике построить функцию $z(x)=ln(x^2+1)                 $. Определить координаты точки пересечения графиков. \item Решить уравнение: $\dfrac{x+4}{5}=1     $

\item Определить сумму, произведение матриц $A=\begin{bmatrix}
4.4 &2.3 &9.7 \\
6.7 &5.2 &0.9 \\
7.4 &0.7 &5.7 \\
\end{bmatrix}
$ и $B=\begin{bmatrix}
5.9 &6.2 &0.4 \\
6.4 &0.2 &5.3 \\
0.7 &3.6 &5.4 \\
\end{bmatrix}
$. Вычислить $D_{i,j}=A_{i,j}  -  B_{i,j}$ и определитель матрицы D

\item Решить систему уравнений: \begin{equation*} \begin{cases} x+y=6                \\ x^2+y^2=2                 \end{cases} \end{equation*} 

\end{enumerate}
\textsc{\textbf{Вариант 6}}

\begin{enumerate}
\item Вычислить: 
\begin{equation*}\dfrac{7+\sin(2)}{3}              \end{equation*}
\begin{equation*}\sin \left( \dfrac{\pi}{7} \right)\end{equation*}
\begin{equation*}\ln(3+sin(4))                     \end{equation*}

\item Вычислить аналитически: 
 \begin{equation*} \int \cos(x^2) dx          \end{equation*}\begin{equation*} {\dfrac{\partial} {\partial x}x^2 +5 x}             \end{equation*}


\item Определить плотность прямоугольного параллелепипеда. Высота --- $1.2$ м, длина --- $230.1$ см, ширина --- $192.2$ см и масса $134.9$ кг.

\item Построить график функции $y(x)=\dfrac{x+4}{x^2+1}    $ в диапазоне от $x=-3.5$ до $x=8.0$, определить, при каком значении $x$ $y=1.0$. На этом же графике построить функцию $z(x)=-0.5 x^2 + x              $. Определить координаты точки пересечения графиков. \item Решить уравнение: $x+\sin{x}=6          $

\item Определить сумму, произведение матриц $A=\begin{bmatrix}
1.8 &5.8 &1.4 \\
4 &1.3 &9.8 \\
4.2 &6.7 &1.4 \\
\end{bmatrix}
$ и $B=\begin{bmatrix}
7.4 &7.9 &0.2 \\
4 &7.1 &7.2 \\
0 &8.8 &8.4 \\
\end{bmatrix}
$. Вычислить $D_{i,j}=A_{i,j}  +  B_{i,j}$ и определитель матрицы D

\item Решить систему уравнений: \begin{equation*} \begin{cases} x^2+y=3              \\ x=7y                      \end{cases} \end{equation*} 

\end{enumerate}
\newpage
