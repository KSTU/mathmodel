\section{Лабораторная работа №~8 <<Моделирование теплообменника типа труба в трубе>>}

  \addtocounter{nlab}{1}\textsc{\textbf{Вариант 1}}

\begin{enumerate} 
\item Вычислить распределение температур теплоносителей в прямоточном теплообменнике типа <<труба в трубе>>. Использовать модель идеального вытеснения для обоих потоков. Параметры теплообменника: длина  13.5~м, диаметр внешней трубы 32.5~мм,  диаметр внутренней трубы 17.2~мм, толщина стенки $\delta_{w}$=     4~мм,  теплопроводность материала стенки $\lambda_{w}=  453~\frac{\text{Вт}}{\text{м} \cdot \text{К}}$.  Горячий теплоноситель (обозначен индексом <<h>>) направляется во внутреннюю трубу и	 имеет следующие параметры: температура $\text{T}_{h}= 262~^\circ\mathrm{C}$, теплоемкость	  $c_{p{h}}= 3251.2~\frac{\text{Дж}}{\text{кг} \cdot ^\circ\mathrm{C}}$, теплопроводность 		$\lambda_{h}= 0.446~\frac{\text{Вт}}{\text{м} \cdot ^\circ\mathrm{C}}$, плотность 		$\rho_{h}= 1392 \frac{\text{кг}}{\text{м}^3}$, коэффициент вязкости $\mu_{h}=4.767 \text{мПа} 		\cdot \text{с} $, коэффициент термического расширения $\beta_{h}=0.000737 ^\circ\mathrm{C}^{-1}$,		 расход $G_{h}= 2957.403 \frac{\text{кг}}{\text{ч}}$. Холодный теплоноситель (обозначен индексом <<c>>) 		 направляется в межтрубное пространство и имеет следующие параметры: температура $T_{c}=   25		 ~^\circ\mathrm{C}$, теплоемкость $c_{p{c}}= 2691 \frac{\text{Дж}}{\text{кг} \cdot ^\circ\mathrm{C}}$,			 теплопроводность $\lambda_{c}=0.150 \frac{\text{Вт}}{\text{м} \cdot ^\circ\mathrm{C}}$, плотность 			 $\rho_{c}=  1488~\frac{\text{кг}}{\text{м}^3}$, коэффициент вязкости $\mu_{c}=6.355~\text{мПа} \cdot \text{с} $, 			 расход $G_{c}=3211.48~\frac{\text{кг}}{\text{ч}}$. 

\item Определить распределение температуры вдоль теплообменника, если 	изменить направление движения теплоносителей на противоточное.

\item Рассчитать распределение температуры в теплообменнике структура потоков в котором описывается ячеечной моделью. Количество ячеек для обоих теплоносителей равно $m = $ 3. Коэффициент теплопередачи, свойства веществ и площадь теплообменника взять из первого задания.

\end{enumerate}

\textsc{\textbf{Вариант 2}}

\begin{enumerate} 
\item Вычислить распределение температур теплоносителей в прямоточном теплообменнике типа <<труба в трубе>>. Использовать модель идеального вытеснения для обоих потоков. Параметры теплообменника: длина  17.8~м, диаметр внешней трубы 22.0~мм,  диаметр внутренней трубы 11.2~мм, толщина стенки $\delta_{w}$=     6~мм,  теплопроводность материала стенки $\lambda_{w}=  699~\frac{\text{Вт}}{\text{м} \cdot \text{К}}$.  Горячий теплоноситель (обозначен индексом <<h>>) направляется во внутреннюю трубу и	 имеет следующие параметры: температура $\text{T}_{h}= 172~^\circ\mathrm{C}$, теплоемкость	  $c_{p{h}}= 2065.9~\frac{\text{Дж}}{\text{кг} \cdot ^\circ\mathrm{C}}$, теплопроводность 		$\lambda_{h}= 0.544~\frac{\text{Вт}}{\text{м} \cdot ^\circ\mathrm{C}}$, плотность 		$\rho_{h}=  826 \frac{\text{кг}}{\text{м}^3}$, коэффициент вязкости $\mu_{h}=10.441 \text{мПа} 		\cdot \text{с} $, коэффициент термического расширения $\beta_{h}=0.000309 ^\circ\mathrm{C}^{-1}$,		 расход $G_{h}= 1850.229 \frac{\text{кг}}{\text{ч}}$. Холодный теплоноситель (обозначен индексом <<c>>) 		 направляется в межтрубное пространство и имеет следующие параметры: температура $T_{c}=   32		 ~^\circ\mathrm{C}$, теплоемкость $c_{p{c}}= 4179 \frac{\text{Дж}}{\text{кг} \cdot ^\circ\mathrm{C}}$,			 теплопроводность $\lambda_{c}=0.153 \frac{\text{Вт}}{\text{м} \cdot ^\circ\mathrm{C}}$, плотность 			 $\rho_{c}=  1549~\frac{\text{кг}}{\text{м}^3}$, коэффициент вязкости $\mu_{c}=0.822~\text{мПа} \cdot \text{с} $, 			 расход $G_{c}=974.24~\frac{\text{кг}}{\text{ч}}$. 

\item Определить распределение температуры вдоль теплообменника, если 	изменить направление движения теплоносителей на противоточное.

\item Определить площадь теплообмена для модели идеального смешения, необходимую для достижения 	температур на выходе из теплообменника, таких же как для модели идеального вытеснения (температуры взять из предыдущих заданий).	Провести сравнение эффективности теплообменников с различной структурой потоков.

\end{enumerate}

\textsc{\textbf{Вариант 3}}

\begin{enumerate} 
\item Вычислить распределение температур теплоносителей в прямоточном теплообменнике типа <<труба в трубе>>. Использовать модель идеального вытеснения для обоих потоков. Параметры теплообменника: длина  25.1~м, диаметр внешней трубы 86.8~мм,  диаметр внутренней трубы 39.5~мм, толщина стенки $\delta_{w}$=     5~мм,  теплопроводность материала стенки $\lambda_{w}=  471~\frac{\text{Вт}}{\text{м} \cdot \text{К}}$.  Горячий теплоноситель (обозначен индексом <<h>>) направляется во внутреннюю трубу и	 имеет следующие параметры: температура $\text{T}_{h}= 194~^\circ\mathrm{C}$, теплоемкость	  $c_{p{h}}= 3973.9~\frac{\text{Дж}}{\text{кг} \cdot ^\circ\mathrm{C}}$, теплопроводность 		$\lambda_{h}= 0.456~\frac{\text{Вт}}{\text{м} \cdot ^\circ\mathrm{C}}$, плотность 		$\rho_{h}=  795 \frac{\text{кг}}{\text{м}^3}$, коэффициент вязкости $\mu_{h}=7.055 \text{мПа} 		\cdot \text{с} $, коэффициент термического расширения $\beta_{h}=0.000357 ^\circ\mathrm{C}^{-1}$,		 расход $G_{h}= 1163.570 \frac{\text{кг}}{\text{ч}}$. Холодный теплоноситель (обозначен индексом <<c>>) 		 направляется в межтрубное пространство и имеет следующие параметры: температура $T_{c}=   36		 ~^\circ\mathrm{C}$, теплоемкость $c_{p{c}}= 3901 \frac{\text{Дж}}{\text{кг} \cdot ^\circ\mathrm{C}}$,			 теплопроводность $\lambda_{c}=0.343 \frac{\text{Вт}}{\text{м} \cdot ^\circ\mathrm{C}}$, плотность 			 $\rho_{c}=  1562~\frac{\text{кг}}{\text{м}^3}$, коэффициент вязкости $\mu_{c}=5.540~\text{мПа} \cdot \text{с} $, 			 расход $G_{c}=1611.33~\frac{\text{кг}}{\text{ч}}$. 

\item Определить распределение температуры вдоль теплообменника, если 	изменить направление движения теплоносителей на противоточное.

\item Рассчитать распределение температуры в теплообменнике структура потоков в котором описывается диффузионной моделью.Коэффициент обратного перемешивания $D_L = $0.090. Коэффициент теплопередачи, свойства веществ и размеры теплообменника взять из первого задания. 

\end{enumerate}

\textsc{\textbf{Вариант 4}}

\begin{enumerate} 
\item Вычислить распределение температур теплоносителей в прямоточном теплообменнике типа <<труба в трубе>>. Использовать модель идеального вытеснения для обоих потоков. Параметры теплообменника: длина  29.7~м, диаметр внешней трубы 87.0~мм,  диаметр внутренней трубы 37.1~мм, толщина стенки $\delta_{w}$=     7~мм,  теплопроводность материала стенки $\lambda_{w}=  658~\frac{\text{Вт}}{\text{м} \cdot \text{К}}$.  Горячий теплоноситель (обозначен индексом <<h>>) направляется во внутреннюю трубу и	 имеет следующие параметры: температура $\text{T}_{h}= 106~^\circ\mathrm{C}$, теплоемкость	  $c_{p{h}}= 3025.7~\frac{\text{Дж}}{\text{кг} \cdot ^\circ\mathrm{C}}$, теплопроводность 		$\lambda_{h}= 0.212~\frac{\text{Вт}}{\text{м} \cdot ^\circ\mathrm{C}}$, плотность 		$\rho_{h}=  764 \frac{\text{кг}}{\text{м}^3}$, коэффициент вязкости $\mu_{h}=4.719 \text{мПа} 		\cdot \text{с} $, коэффициент термического расширения $\beta_{h}=0.000823 ^\circ\mathrm{C}^{-1}$,		 расход $G_{h}= 466.946 \frac{\text{кг}}{\text{ч}}$. Холодный теплоноситель (обозначен индексом <<c>>) 		 направляется в межтрубное пространство и имеет следующие параметры: температура $T_{c}=   23		 ~^\circ\mathrm{C}$, теплоемкость $c_{p{c}}= 1302 \frac{\text{Дж}}{\text{кг} \cdot ^\circ\mathrm{C}}$,			 теплопроводность $\lambda_{c}=0.104 \frac{\text{Вт}}{\text{м} \cdot ^\circ\mathrm{C}}$, плотность 			 $\rho_{c}=  1197~\frac{\text{кг}}{\text{м}^3}$, коэффициент вязкости $\mu_{c}=0.878~\text{мПа} \cdot \text{с} $, 			 расход $G_{c}=575.04~\frac{\text{кг}}{\text{ч}}$. 

\item Определить распределение температуры вдоль теплообменника, если 	изменить направление движения теплоносителей на противоточное.

\item Рассчитать распределение температуры в теплообменнике структура потоков в котором описывается диффузионной моделью.Коэффициент обратного перемешивания $D_L = $0.086. Коэффициент теплопередачи, свойства веществ и размеры теплообменника взять из первого задания. 

\end{enumerate}

\textsc{\textbf{Вариант 5}}

\begin{enumerate} 
\item Вычислить распределение температур теплоносителей в прямоточном теплообменнике типа <<труба в трубе>>. Использовать модель идеального вытеснения для обоих потоков. Параметры теплообменника: длина  16.2~м, диаметр внешней трубы 37.3~мм,  диаметр внутренней трубы 16.3~мм, толщина стенки $\delta_{w}$=     6~мм,  теплопроводность материала стенки $\lambda_{w}=  395~\frac{\text{Вт}}{\text{м} \cdot \text{К}}$.  Горячий теплоноситель (обозначен индексом <<h>>) направляется во внутреннюю трубу и	 имеет следующие параметры: температура $\text{T}_{h}= 170~^\circ\mathrm{C}$, теплоемкость	  $c_{p{h}}= 3657.3~\frac{\text{Дж}}{\text{кг} \cdot ^\circ\mathrm{C}}$, теплопроводность 		$\lambda_{h}= 0.335~\frac{\text{Вт}}{\text{м} \cdot ^\circ\mathrm{C}}$, плотность 		$\rho_{h}= 1082 \frac{\text{кг}}{\text{м}^3}$, коэффициент вязкости $\mu_{h}=1.595 \text{мПа} 		\cdot \text{с} $, коэффициент термического расширения $\beta_{h}=0.000337 ^\circ\mathrm{C}^{-1}$,		 расход $G_{h}= 38.921 \frac{\text{кг}}{\text{ч}}$. Холодный теплоноситель (обозначен индексом <<c>>) 		 направляется в межтрубное пространство и имеет следующие параметры: температура $T_{c}=   39		 ~^\circ\mathrm{C}$, теплоемкость $c_{p{c}}= 1307 \frac{\text{Дж}}{\text{кг} \cdot ^\circ\mathrm{C}}$,			 теплопроводность $\lambda_{c}=0.214 \frac{\text{Вт}}{\text{м} \cdot ^\circ\mathrm{C}}$, плотность 			 $\rho_{c}=  1155~\frac{\text{кг}}{\text{м}^3}$, коэффициент вязкости $\mu_{c}=3.995~\text{мПа} \cdot \text{с} $, 			 расход $G_{c}=59.27~\frac{\text{кг}}{\text{ч}}$. 

\item Определить распределение температуры вдоль теплообменника, если 	изменить направление движения теплоносителей на противоточное.

\item Определить площадь теплообмена для модели идеального смешения, необходимую для достижения 	температур на выходе из теплообменника, таких же как для модели идеального вытеснения (температуры взять из предыдущих заданий).	Провести сравнение эффективности теплообменников с различной структурой потоков.

\end{enumerate}

\textsc{\textbf{Вариант 6}}

\begin{enumerate} 
\item Вычислить распределение температур теплоносителей в прямоточном теплообменнике типа <<труба в трубе>>. Использовать модель идеального вытеснения для обоих потоков. Параметры теплообменника: длина  17.7~м, диаметр внешней трубы 23.4~мм,  диаметр внутренней трубы 16.2~мм, толщина стенки $\delta_{w}$=     2~мм,  теплопроводность материала стенки $\lambda_{w}=  680~\frac{\text{Вт}}{\text{м} \cdot \text{К}}$.  Горячий теплоноситель (обозначен индексом <<h>>) направляется во внутреннюю трубу и	 имеет следующие параметры: температура $\text{T}_{h}= 130~^\circ\mathrm{C}$, теплоемкость	  $c_{p{h}}= 3589.1~\frac{\text{Дж}}{\text{кг} \cdot ^\circ\mathrm{C}}$, теплопроводность 		$\lambda_{h}= 0.323~\frac{\text{Вт}}{\text{м} \cdot ^\circ\mathrm{C}}$, плотность 		$\rho_{h}=  865 \frac{\text{кг}}{\text{м}^3}$, коэффициент вязкости $\mu_{h}=6.054 \text{мПа} 		\cdot \text{с} $, коэффициент термического расширения $\beta_{h}=0.000290 ^\circ\mathrm{C}^{-1}$,		 расход $G_{h}= 288.638 \frac{\text{кг}}{\text{ч}}$. Холодный теплоноситель (обозначен индексом <<c>>) 		 направляется в межтрубное пространство и имеет следующие параметры: температура $T_{c}=   26		 ~^\circ\mathrm{C}$, теплоемкость $c_{p{c}}= 1566 \frac{\text{Дж}}{\text{кг} \cdot ^\circ\mathrm{C}}$,			 теплопроводность $\lambda_{c}=0.369 \frac{\text{Вт}}{\text{м} \cdot ^\circ\mathrm{C}}$, плотность 			 $\rho_{c}=  1365~\frac{\text{кг}}{\text{м}^3}$, коэффициент вязкости $\mu_{c}=6.774~\text{мПа} \cdot \text{с} $, 			 расход $G_{c}=410.30~\frac{\text{кг}}{\text{ч}}$. 

\item Определить распределение температуры вдоль теплообменника, если 	изменить направление движения теплоносителей на противоточное.

\item Определить площадь теплообмена для модели идеального смешения, необходимую для достижения 	температур на выходе из теплообменника, таких же как для модели идеального вытеснения (температуры взять из предыдущих заданий).	Провести сравнение эффективности теплообменников с различной структурой потоков.

\end{enumerate}

\textsc{\textbf{Вариант 7}}

\begin{enumerate} 
\item Вычислить распределение температур теплоносителей в прямоточном теплообменнике типа <<труба в трубе>>. Использовать модель идеального вытеснения для обоих потоков. Параметры теплообменника: длина  16.7~м, диаметр внешней трубы 32.6~мм,  диаметр внутренней трубы 13.7~мм, толщина стенки $\delta_{w}$=     4~мм,  теплопроводность материала стенки $\lambda_{w}=  521~\frac{\text{Вт}}{\text{м} \cdot \text{К}}$.  Горячий теплоноситель (обозначен индексом <<h>>) направляется во внутреннюю трубу и	 имеет следующие параметры: температура $\text{T}_{h}= 256~^\circ\mathrm{C}$, теплоемкость	  $c_{p{h}}= 1347.1~\frac{\text{Дж}}{\text{кг} \cdot ^\circ\mathrm{C}}$, теплопроводность 		$\lambda_{h}= 0.426~\frac{\text{Вт}}{\text{м} \cdot ^\circ\mathrm{C}}$, плотность 		$\rho_{h}=  726 \frac{\text{кг}}{\text{м}^3}$, коэффициент вязкости $\mu_{h}=1.945 \text{мПа} 		\cdot \text{с} $, коэффициент термического расширения $\beta_{h}=0.000538 ^\circ\mathrm{C}^{-1}$,		 расход $G_{h}= 1012.446 \frac{\text{кг}}{\text{ч}}$. Холодный теплоноситель (обозначен индексом <<c>>) 		 направляется в межтрубное пространство и имеет следующие параметры: температура $T_{c}=   34		 ~^\circ\mathrm{C}$, теплоемкость $c_{p{c}}= 2592 \frac{\text{Дж}}{\text{кг} \cdot ^\circ\mathrm{C}}$,			 теплопроводность $\lambda_{c}=0.363 \frac{\text{Вт}}{\text{м} \cdot ^\circ\mathrm{C}}$, плотность 			 $\rho_{c}=  1283~\frac{\text{кг}}{\text{м}^3}$, коэффициент вязкости $\mu_{c}=4.730~\text{мПа} \cdot \text{с} $, 			 расход $G_{c}=970.86~\frac{\text{кг}}{\text{ч}}$. 

\item Определить распределение температуры вдоль теплообменника, если 	изменить направление движения теплоносителей на противоточное.

\item Определить площадь теплообмена для модели идеального смешения, необходимую для достижения 	температур на выходе из теплообменника, таких же как для модели идеального вытеснения (температуры взять из предыдущих заданий).	Провести сравнение эффективности теплообменников с различной структурой потоков.

\end{enumerate}

\textsc{\textbf{Вариант 8}}

\begin{enumerate} 
\item Вычислить распределение температур теплоносителей в прямоточном теплообменнике типа <<труба в трубе>>. Использовать модель идеального вытеснения для обоих потоков. Параметры теплообменника: длина  24.3~м, диаметр внешней трубы 45.9~мм,  диаметр внутренней трубы 20.3~мм, толщина стенки $\delta_{w}$=     4~мм,  теплопроводность материала стенки $\lambda_{w}=  565~\frac{\text{Вт}}{\text{м} \cdot \text{К}}$.  Горячий теплоноситель (обозначен индексом <<h>>) направляется во внутреннюю трубу и	 имеет следующие параметры: температура $\text{T}_{h}=  90~^\circ\mathrm{C}$, теплоемкость	  $c_{p{h}}= 3292.4~\frac{\text{Дж}}{\text{кг} \cdot ^\circ\mathrm{C}}$, теплопроводность 		$\lambda_{h}= 0.165~\frac{\text{Вт}}{\text{м} \cdot ^\circ\mathrm{C}}$, плотность 		$\rho_{h}=  795 \frac{\text{кг}}{\text{м}^3}$, коэффициент вязкости $\mu_{h}=7.531 \text{мПа} 		\cdot \text{с} $, коэффициент термического расширения $\beta_{h}=0.000920 ^\circ\mathrm{C}^{-1}$,		 расход $G_{h}= 216.325 \frac{\text{кг}}{\text{ч}}$. Холодный теплоноситель (обозначен индексом <<c>>) 		 направляется в межтрубное пространство и имеет следующие параметры: температура $T_{c}=   30		 ~^\circ\mathrm{C}$, теплоемкость $c_{p{c}}= 3670 \frac{\text{Дж}}{\text{кг} \cdot ^\circ\mathrm{C}}$,			 теплопроводность $\lambda_{c}=0.588 \frac{\text{Вт}}{\text{м} \cdot ^\circ\mathrm{C}}$, плотность 			 $\rho_{c}=   885~\frac{\text{кг}}{\text{м}^3}$, коэффициент вязкости $\mu_{c}=5.777~\text{мПа} \cdot \text{с} $, 			 расход $G_{c}=328.05~\frac{\text{кг}}{\text{ч}}$. 

\item Определить распределение температуры вдоль теплообменника, если 	изменить направление движения теплоносителей на противоточное.

\item Рассчитать распределение температуры в теплообменнике структура потоков в котором описывается диффузионной моделью.Коэффициент обратного перемешивания $D_L = $0.070. Коэффициент теплопередачи, свойства веществ и размеры теплообменника взять из первого задания. 

\end{enumerate}

\textsc{\textbf{Вариант 9}}

\begin{enumerate} 
\item Вычислить распределение температур теплоносителей в прямоточном теплообменнике типа <<труба в трубе>>. Использовать модель идеального вытеснения для обоих потоков. Параметры теплообменника: длина  23.2~м, диаметр внешней трубы 75.8~мм,  диаметр внутренней трубы 34.8~мм, толщина стенки $\delta_{w}$=     5~мм,  теплопроводность материала стенки $\lambda_{w}=  528~\frac{\text{Вт}}{\text{м} \cdot \text{К}}$.  Горячий теплоноситель (обозначен индексом <<h>>) направляется во внутреннюю трубу и	 имеет следующие параметры: температура $\text{T}_{h}=  78~^\circ\mathrm{C}$, теплоемкость	  $c_{p{h}}= 2557.6~\frac{\text{Дж}}{\text{кг} \cdot ^\circ\mathrm{C}}$, теплопроводность 		$\lambda_{h}= 0.171~\frac{\text{Вт}}{\text{м} \cdot ^\circ\mathrm{C}}$, плотность 		$\rho_{h}= 1331 \frac{\text{кг}}{\text{м}^3}$, коэффициент вязкости $\mu_{h}=6.548 \text{мПа} 		\cdot \text{с} $, коэффициент термического расширения $\beta_{h}=0.000409 ^\circ\mathrm{C}^{-1}$,		 расход $G_{h}= 4603.444 \frac{\text{кг}}{\text{ч}}$. Холодный теплоноситель (обозначен индексом <<c>>) 		 направляется в межтрубное пространство и имеет следующие параметры: температура $T_{c}=   27		 ~^\circ\mathrm{C}$, теплоемкость $c_{p{c}}= 3395 \frac{\text{Дж}}{\text{кг} \cdot ^\circ\mathrm{C}}$,			 теплопроводность $\lambda_{c}=0.359 \frac{\text{Вт}}{\text{м} \cdot ^\circ\mathrm{C}}$, плотность 			 $\rho_{c}=   959~\frac{\text{кг}}{\text{м}^3}$, коэффициент вязкости $\mu_{c}=8.082~\text{мПа} \cdot \text{с} $, 			 расход $G_{c}=6105.22~\frac{\text{кг}}{\text{ч}}$. 

\item Определить распределение температуры вдоль теплообменника, если 	изменить направление движения теплоносителей на противоточное.

\item Рассчитать распределение температуры в теплообменнике структура потоков в котором описывается ячеечной моделью. Количество ячеек для обоих теплоносителей равно $m = $ 2. Коэффициент теплопередачи, свойства веществ и площадь теплообменника взять из первого задания.

\end{enumerate}

\textsc{\textbf{Вариант 10}}

\begin{enumerate} 
\item Вычислить распределение температур теплоносителей в прямоточном теплообменнике типа <<труба в трубе>>. Использовать модель идеального вытеснения для обоих потоков. Параметры теплообменника: длина  18.4~м, диаметр внешней трубы 30.0~мм,  диаметр внутренней трубы 16.6~мм, толщина стенки $\delta_{w}$=     7~мм,  теплопроводность материала стенки $\lambda_{w}=  359~\frac{\text{Вт}}{\text{м} \cdot \text{К}}$.  Горячий теплоноситель (обозначен индексом <<h>>) направляется во внутреннюю трубу и	 имеет следующие параметры: температура $\text{T}_{h}= 262~^\circ\mathrm{C}$, теплоемкость	  $c_{p{h}}= 3872.6~\frac{\text{Дж}}{\text{кг} \cdot ^\circ\mathrm{C}}$, теплопроводность 		$\lambda_{h}= 0.457~\frac{\text{Вт}}{\text{м} \cdot ^\circ\mathrm{C}}$, плотность 		$\rho_{h}= 1408 \frac{\text{кг}}{\text{м}^3}$, коэффициент вязкости $\mu_{h}=10.181 \text{мПа} 		\cdot \text{с} $, коэффициент термического расширения $\beta_{h}=0.000555 ^\circ\mathrm{C}^{-1}$,		 расход $G_{h}= 8534.967 \frac{\text{кг}}{\text{ч}}$. Холодный теплоноситель (обозначен индексом <<c>>) 		 направляется в межтрубное пространство и имеет следующие параметры: температура $T_{c}=   32		 ~^\circ\mathrm{C}$, теплоемкость $c_{p{c}}= 3721 \frac{\text{Дж}}{\text{кг} \cdot ^\circ\mathrm{C}}$,			 теплопроводность $\lambda_{c}=0.559 \frac{\text{Вт}}{\text{м} \cdot ^\circ\mathrm{C}}$, плотность 			 $\rho_{c}=  1904~\frac{\text{кг}}{\text{м}^3}$, коэффициент вязкости $\mu_{c}=4.627~\text{мПа} \cdot \text{с} $, 			 расход $G_{c}=14453.90~\frac{\text{кг}}{\text{ч}}$. 

\item Определить распределение температуры вдоль теплообменника, если 	изменить направление движения теплоносителей на противоточное.

\item Рассчитать распределение температуры в теплообменнике структура потоков в котором описывается ячеечной моделью. Количество ячеек для обоих теплоносителей равно $m = $ 2. Коэффициент теплопередачи, свойства веществ и площадь теплообменника взять из первого задания.

\end{enumerate}

\textsc{\textbf{Вариант 11}}

\begin{enumerate} 
\item Вычислить распределение температур теплоносителей в прямоточном теплообменнике типа <<труба в трубе>>. Использовать модель идеального вытеснения для обоих потоков. Параметры теплообменника: длина  27.1~м, диаметр внешней трубы 24.3~мм,  диаметр внутренней трубы 11.5~мм, толщина стенки $\delta_{w}$=     3~мм,  теплопроводность материала стенки $\lambda_{w}=  567~\frac{\text{Вт}}{\text{м} \cdot \text{К}}$.  Горячий теплоноситель (обозначен индексом <<h>>) направляется во внутреннюю трубу и	 имеет следующие параметры: температура $\text{T}_{h}= 149~^\circ\mathrm{C}$, теплоемкость	  $c_{p{h}}= 2061.4~\frac{\text{Дж}}{\text{кг} \cdot ^\circ\mathrm{C}}$, теплопроводность 		$\lambda_{h}= 0.594~\frac{\text{Вт}}{\text{м} \cdot ^\circ\mathrm{C}}$, плотность 		$\rho_{h}= 1439 \frac{\text{кг}}{\text{м}^3}$, коэффициент вязкости $\mu_{h}=9.163 \text{мПа} 		\cdot \text{с} $, коэффициент термического расширения $\beta_{h}=0.000975 ^\circ\mathrm{C}^{-1}$,		 расход $G_{h}= 1950.614 \frac{\text{кг}}{\text{ч}}$. Холодный теплоноситель (обозначен индексом <<c>>) 		 направляется в межтрубное пространство и имеет следующие параметры: температура $T_{c}=   35		 ~^\circ\mathrm{C}$, теплоемкость $c_{p{c}}= 1410 \frac{\text{Дж}}{\text{кг} \cdot ^\circ\mathrm{C}}$,			 теплопроводность $\lambda_{c}=0.288 \frac{\text{Вт}}{\text{м} \cdot ^\circ\mathrm{C}}$, плотность 			 $\rho_{c}=  1007~\frac{\text{кг}}{\text{м}^3}$, коэффициент вязкости $\mu_{c}=5.963~\text{мПа} \cdot \text{с} $, 			 расход $G_{c}=2394.05~\frac{\text{кг}}{\text{ч}}$. 

\item Определить распределение температуры вдоль теплообменника, если 	изменить направление движения теплоносителей на противоточное.

\item Определить площадь теплообмена для модели идеального смешения, необходимую для достижения 	температур на выходе из теплообменника, таких же как для модели идеального вытеснения (температуры взять из предыдущих заданий).	Провести сравнение эффективности теплообменников с различной структурой потоков.

\end{enumerate}

\textsc{\textbf{Вариант 12}}

\begin{enumerate} 
\item Вычислить распределение температур теплоносителей в прямоточном теплообменнике типа <<труба в трубе>>. Использовать модель идеального вытеснения для обоих потоков. Параметры теплообменника: длина  14.7~м, диаметр внешней трубы 33.9~мм,  диаметр внутренней трубы 14.5~мм, толщина стенки $\delta_{w}$=     4~мм,  теплопроводность материала стенки $\lambda_{w}=  345~\frac{\text{Вт}}{\text{м} \cdot \text{К}}$.  Горячий теплоноситель (обозначен индексом <<h>>) направляется во внутреннюю трубу и	 имеет следующие параметры: температура $\text{T}_{h}= 182~^\circ\mathrm{C}$, теплоемкость	  $c_{p{h}}= 3493.4~\frac{\text{Дж}}{\text{кг} \cdot ^\circ\mathrm{C}}$, теплопроводность 		$\lambda_{h}= 0.166~\frac{\text{Вт}}{\text{м} \cdot ^\circ\mathrm{C}}$, плотность 		$\rho_{h}=  788 \frac{\text{кг}}{\text{м}^3}$, коэффициент вязкости $\mu_{h}=3.423 \text{мПа} 		\cdot \text{с} $, коэффициент термического расширения $\beta_{h}=0.000489 ^\circ\mathrm{C}^{-1}$,		 расход $G_{h}= 71.401 \frac{\text{кг}}{\text{ч}}$. Холодный теплоноситель (обозначен индексом <<c>>) 		 направляется в межтрубное пространство и имеет следующие параметры: температура $T_{c}=   35		 ~^\circ\mathrm{C}$, теплоемкость $c_{p{c}}= 3542 \frac{\text{Дж}}{\text{кг} \cdot ^\circ\mathrm{C}}$,			 теплопроводность $\lambda_{c}=0.491 \frac{\text{Вт}}{\text{м} \cdot ^\circ\mathrm{C}}$, плотность 			 $\rho_{c}=  1486~\frac{\text{кг}}{\text{м}^3}$, коэффициент вязкости $\mu_{c}=0.500~\text{мПа} \cdot \text{с} $, 			 расход $G_{c}=37.32~\frac{\text{кг}}{\text{ч}}$. 

\item Определить распределение температуры вдоль теплообменника, если 	изменить направление движения теплоносителей на противоточное.

\item Рассчитать распределение температуры в теплообменнике структура потоков в котором описывается ячеечной моделью. Количество ячеек для обоих теплоносителей равно $m = $ 3. Коэффициент теплопередачи, свойства веществ и площадь теплообменника взять из первого задания.

\end{enumerate}

\textsc{\textbf{Вариант 13}}

\begin{enumerate} 
\item Вычислить распределение температур теплоносителей в прямоточном теплообменнике типа <<труба в трубе>>. Использовать модель идеального вытеснения для обоих потоков. Параметры теплообменника: длина  18.6~м, диаметр внешней трубы 44.8~мм,  диаметр внутренней трубы 20.2~мм, толщина стенки $\delta_{w}$=     4~мм,  теплопроводность материала стенки $\lambda_{w}=  634~\frac{\text{Вт}}{\text{м} \cdot \text{К}}$.  Горячий теплоноситель (обозначен индексом <<h>>) направляется во внутреннюю трубу и	 имеет следующие параметры: температура $\text{T}_{h}= 220~^\circ\mathrm{C}$, теплоемкость	  $c_{p{h}}= 2173.5~\frac{\text{Дж}}{\text{кг} \cdot ^\circ\mathrm{C}}$, теплопроводность 		$\lambda_{h}= 0.321~\frac{\text{Вт}}{\text{м} \cdot ^\circ\mathrm{C}}$, плотность 		$\rho_{h}= 1219 \frac{\text{кг}}{\text{м}^3}$, коэффициент вязкости $\mu_{h}=1.943 \text{мПа} 		\cdot \text{с} $, коэффициент термического расширения $\beta_{h}=0.000430 ^\circ\mathrm{C}^{-1}$,		 расход $G_{h}= 789.289 \frac{\text{кг}}{\text{ч}}$. Холодный теплоноситель (обозначен индексом <<c>>) 		 направляется в межтрубное пространство и имеет следующие параметры: температура $T_{c}=   22		 ~^\circ\mathrm{C}$, теплоемкость $c_{p{c}}= 3043 \frac{\text{Дж}}{\text{кг} \cdot ^\circ\mathrm{C}}$,			 теплопроводность $\lambda_{c}=0.544 \frac{\text{Вт}}{\text{м} \cdot ^\circ\mathrm{C}}$, плотность 			 $\rho_{c}=  1632~\frac{\text{кг}}{\text{м}^3}$, коэффициент вязкости $\mu_{c}=4.807~\text{мПа} \cdot \text{с} $, 			 расход $G_{c}=465.98~\frac{\text{кг}}{\text{ч}}$. 

\item Определить распределение температуры вдоль теплообменника, если 	изменить направление движения теплоносителей на противоточное.

\item Рассчитать распределение температуры в теплообменнике структура потоков в котором описывается ячеечной моделью. Количество ячеек для обоих теплоносителей равно $m = $ 3. Коэффициент теплопередачи, свойства веществ и площадь теплообменника взять из первого задания.

\end{enumerate}

\textsc{\textbf{Вариант 14}}

\begin{enumerate} 
\item Вычислить распределение температур теплоносителей в прямоточном теплообменнике типа <<труба в трубе>>. Использовать модель идеального вытеснения для обоих потоков. Параметры теплообменника: длина  11.1~м, диаметр внешней трубы 57.4~мм,  диаметр внутренней трубы 28.7~мм, толщина стенки $\delta_{w}$=     4~мм,  теплопроводность материала стенки $\lambda_{w}=  343~\frac{\text{Вт}}{\text{м} \cdot \text{К}}$.  Горячий теплоноситель (обозначен индексом <<h>>) направляется во внутреннюю трубу и	 имеет следующие параметры: температура $\text{T}_{h}= 252~^\circ\mathrm{C}$, теплоемкость	  $c_{p{h}}= 3811.1~\frac{\text{Дж}}{\text{кг} \cdot ^\circ\mathrm{C}}$, теплопроводность 		$\lambda_{h}= 0.533~\frac{\text{Вт}}{\text{м} \cdot ^\circ\mathrm{C}}$, плотность 		$\rho_{h}=  970 \frac{\text{кг}}{\text{м}^3}$, коэффициент вязкости $\mu_{h}=7.030 \text{мПа} 		\cdot \text{с} $, коэффициент термического расширения $\beta_{h}=0.000432 ^\circ\mathrm{C}^{-1}$,		 расход $G_{h}= 11927.411 \frac{\text{кг}}{\text{ч}}$. Холодный теплоноситель (обозначен индексом <<c>>) 		 направляется в межтрубное пространство и имеет следующие параметры: температура $T_{c}=   24		 ~^\circ\mathrm{C}$, теплоемкость $c_{p{c}}= 3443 \frac{\text{Дж}}{\text{кг} \cdot ^\circ\mathrm{C}}$,			 теплопроводность $\lambda_{c}=0.559 \frac{\text{Вт}}{\text{м} \cdot ^\circ\mathrm{C}}$, плотность 			 $\rho_{c}=  1797~\frac{\text{кг}}{\text{м}^3}$, коэффициент вязкости $\mu_{c}=1.081~\text{мПа} \cdot \text{с} $, 			 расход $G_{c}=8770.59~\frac{\text{кг}}{\text{ч}}$. 

\item Определить распределение температуры вдоль теплообменника, если 	изменить направление движения теплоносителей на противоточное.

\item Рассчитать распределение температуры в теплообменнике структура потоков в котором описывается ячеечной моделью. Количество ячеек для обоих теплоносителей равно $m = $ 2. Коэффициент теплопередачи, свойства веществ и площадь теплообменника взять из первого задания.

\end{enumerate}

\textsc{\textbf{Вариант 15}}

\begin{enumerate} 
\item Вычислить распределение температур теплоносителей в прямоточном теплообменнике типа <<труба в трубе>>. Использовать модель идеального вытеснения для обоих потоков. Параметры теплообменника: длина  28.7~м, диаметр внешней трубы 72.2~мм,  диаметр внутренней трубы 38.2~мм, толщина стенки $\delta_{w}$=     7~мм,  теплопроводность материала стенки $\lambda_{w}=  622~\frac{\text{Вт}}{\text{м} \cdot \text{К}}$.  Горячий теплоноситель (обозначен индексом <<h>>) направляется во внутреннюю трубу и	 имеет следующие параметры: температура $\text{T}_{h}= 126~^\circ\mathrm{C}$, теплоемкость	  $c_{p{h}}= 3625.7~\frac{\text{Дж}}{\text{кг} \cdot ^\circ\mathrm{C}}$, теплопроводность 		$\lambda_{h}= 0.120~\frac{\text{Вт}}{\text{м} \cdot ^\circ\mathrm{C}}$, плотность 		$\rho_{h}=  801 \frac{\text{кг}}{\text{м}^3}$, коэффициент вязкости $\mu_{h}=3.421 \text{мПа} 		\cdot \text{с} $, коэффициент термического расширения $\beta_{h}=0.000859 ^\circ\mathrm{C}^{-1}$,		 расход $G_{h}= 240.007 \frac{\text{кг}}{\text{ч}}$. Холодный теплоноситель (обозначен индексом <<c>>) 		 направляется в межтрубное пространство и имеет следующие параметры: температура $T_{c}=   27		 ~^\circ\mathrm{C}$, теплоемкость $c_{p{c}}= 3639 \frac{\text{Дж}}{\text{кг} \cdot ^\circ\mathrm{C}}$,			 теплопроводность $\lambda_{c}=0.237 \frac{\text{Вт}}{\text{м} \cdot ^\circ\mathrm{C}}$, плотность 			 $\rho_{c}=  1177~\frac{\text{кг}}{\text{м}^3}$, коэффициент вязкости $\mu_{c}=0.888~\text{мПа} \cdot \text{с} $, 			 расход $G_{c}=260.49~\frac{\text{кг}}{\text{ч}}$. 

\item Определить распределение температуры вдоль теплообменника, если 	изменить направление движения теплоносителей на противоточное.

\item Рассчитать распределение температуры в теплообменнике структура потоков в котором описывается диффузионной моделью.Коэффициент обратного перемешивания $D_L = $0.120. Коэффициент теплопередачи, свойства веществ и размеры теплообменника взять из первого задания. 

\end{enumerate}

\textsc{\textbf{Вариант 16}}

\begin{enumerate} 
\item Вычислить распределение температур теплоносителей в прямоточном теплообменнике типа <<труба в трубе>>. Использовать модель идеального вытеснения для обоих потоков. Параметры теплообменника: длина  27.1~м, диаметр внешней трубы 20.6~мм,  диаметр внутренней трубы 12.7~мм, толщина стенки $\delta_{w}$=     3~мм,  теплопроводность материала стенки $\lambda_{w}=  308~\frac{\text{Вт}}{\text{м} \cdot \text{К}}$.  Горячий теплоноситель (обозначен индексом <<h>>) направляется во внутреннюю трубу и	 имеет следующие параметры: температура $\text{T}_{h}= 178~^\circ\mathrm{C}$, теплоемкость	  $c_{p{h}}= 3795.6~\frac{\text{Дж}}{\text{кг} \cdot ^\circ\mathrm{C}}$, теплопроводность 		$\lambda_{h}= 0.374~\frac{\text{Вт}}{\text{м} \cdot ^\circ\mathrm{C}}$, плотность 		$\rho_{h}=  699 \frac{\text{кг}}{\text{м}^3}$, коэффициент вязкости $\mu_{h}=8.848 \text{мПа} 		\cdot \text{с} $, коэффициент термического расширения $\beta_{h}=0.000871 ^\circ\mathrm{C}^{-1}$,		 расход $G_{h}= 172.088 \frac{\text{кг}}{\text{ч}}$. Холодный теплоноситель (обозначен индексом <<c>>) 		 направляется в межтрубное пространство и имеет следующие параметры: температура $T_{c}=   25		 ~^\circ\mathrm{C}$, теплоемкость $c_{p{c}}= 1978 \frac{\text{Дж}}{\text{кг} \cdot ^\circ\mathrm{C}}$,			 теплопроводность $\lambda_{c}=0.239 \frac{\text{Вт}}{\text{м} \cdot ^\circ\mathrm{C}}$, плотность 			 $\rho_{c}=  1450~\frac{\text{кг}}{\text{м}^3}$, коэффициент вязкости $\mu_{c}=1.037~\text{мПа} \cdot \text{с} $, 			 расход $G_{c}=143.90~\frac{\text{кг}}{\text{ч}}$. 

\item Определить распределение температуры вдоль теплообменника, если 	изменить направление движения теплоносителей на противоточное.

\item Определить площадь теплообмена для модели идеального смешения, необходимую для достижения 	температур на выходе из теплообменника, таких же как для модели идеального вытеснения (температуры взять из предыдущих заданий).	Провести сравнение эффективности теплообменников с различной структурой потоков.

\end{enumerate}

\textsc{\textbf{Вариант 17}}

\begin{enumerate} 
\item Вычислить распределение температур теплоносителей в прямоточном теплообменнике типа <<труба в трубе>>. Использовать модель идеального вытеснения для обоих потоков. Параметры теплообменника: длина  29.1~м, диаметр внешней трубы 40.5~мм,  диаметр внутренней трубы 19.1~мм, толщина стенки $\delta_{w}$=     5~мм,  теплопроводность материала стенки $\lambda_{w}=  475~\frac{\text{Вт}}{\text{м} \cdot \text{К}}$.  Горячий теплоноситель (обозначен индексом <<h>>) направляется во внутреннюю трубу и	 имеет следующие параметры: температура $\text{T}_{h}= 216~^\circ\mathrm{C}$, теплоемкость	  $c_{p{h}}= 3006.7~\frac{\text{Дж}}{\text{кг} \cdot ^\circ\mathrm{C}}$, теплопроводность 		$\lambda_{h}= 0.224~\frac{\text{Вт}}{\text{м} \cdot ^\circ\mathrm{C}}$, плотность 		$\rho_{h}= 1433 \frac{\text{кг}}{\text{м}^3}$, коэффициент вязкости $\mu_{h}=4.625 \text{мПа} 		\cdot \text{с} $, коэффициент термического расширения $\beta_{h}=0.000960 ^\circ\mathrm{C}^{-1}$,		 расход $G_{h}= 3083.214 \frac{\text{кг}}{\text{ч}}$. Холодный теплоноситель (обозначен индексом <<c>>) 		 направляется в межтрубное пространство и имеет следующие параметры: температура $T_{c}=   34		 ~^\circ\mathrm{C}$, теплоемкость $c_{p{c}}= 3045 \frac{\text{Дж}}{\text{кг} \cdot ^\circ\mathrm{C}}$,			 теплопроводность $\lambda_{c}=0.535 \frac{\text{Вт}}{\text{м} \cdot ^\circ\mathrm{C}}$, плотность 			 $\rho_{c}=  1975~\frac{\text{кг}}{\text{м}^3}$, коэффициент вязкости $\mu_{c}=4.239~\text{мПа} \cdot \text{с} $, 			 расход $G_{c}=4369.79~\frac{\text{кг}}{\text{ч}}$. 

\item Определить распределение температуры вдоль теплообменника, если 	изменить направление движения теплоносителей на противоточное.

\item Определить площадь теплообмена для модели идеального смешения, необходимую для достижения 	температур на выходе из теплообменника, таких же как для модели идеального вытеснения (температуры взять из предыдущих заданий).	Провести сравнение эффективности теплообменников с различной структурой потоков.

\end{enumerate}

\textsc{\textbf{Вариант 18}}

\begin{enumerate} 
\item Вычислить распределение температур теплоносителей в прямоточном теплообменнике типа <<труба в трубе>>. Использовать модель идеального вытеснения для обоих потоков. Параметры теплообменника: длина  25.2~м, диаметр внешней трубы 82.2~мм,  диаметр внутренней трубы 39.2~мм, толщина стенки $\delta_{w}$=     3~мм,  теплопроводность материала стенки $\lambda_{w}=  328~\frac{\text{Вт}}{\text{м} \cdot \text{К}}$.  Горячий теплоноситель (обозначен индексом <<h>>) направляется во внутреннюю трубу и	 имеет следующие параметры: температура $\text{T}_{h}= 126~^\circ\mathrm{C}$, теплоемкость	  $c_{p{h}}= 4022.1~\frac{\text{Дж}}{\text{кг} \cdot ^\circ\mathrm{C}}$, теплопроводность 		$\lambda_{h}= 0.130~\frac{\text{Вт}}{\text{м} \cdot ^\circ\mathrm{C}}$, плотность 		$\rho_{h}=  874 \frac{\text{кг}}{\text{м}^3}$, коэффициент вязкости $\mu_{h}=10.130 \text{мПа} 		\cdot \text{с} $, коэффициент термического расширения $\beta_{h}=0.000789 ^\circ\mathrm{C}^{-1}$,		 расход $G_{h}= 19269.866 \frac{\text{кг}}{\text{ч}}$. Холодный теплоноситель (обозначен индексом <<c>>) 		 направляется в межтрубное пространство и имеет следующие параметры: температура $T_{c}=   27		 ~^\circ\mathrm{C}$, теплоемкость $c_{p{c}}= 2529 \frac{\text{Дж}}{\text{кг} \cdot ^\circ\mathrm{C}}$,			 теплопроводность $\lambda_{c}=0.134 \frac{\text{Вт}}{\text{м} \cdot ^\circ\mathrm{C}}$, плотность 			 $\rho_{c}=  1282~\frac{\text{кг}}{\text{м}^3}$, коэффициент вязкости $\mu_{c}=7.844~\text{мПа} \cdot \text{с} $, 			 расход $G_{c}=11916.70~\frac{\text{кг}}{\text{ч}}$. 

\item Определить распределение температуры вдоль теплообменника, если 	изменить направление движения теплоносителей на противоточное.

\item Определить площадь теплообмена для модели идеального смешения, необходимую для достижения 	температур на выходе из теплообменника, таких же как для модели идеального вытеснения (температуры взять из предыдущих заданий).	Провести сравнение эффективности теплообменников с различной структурой потоков.

\end{enumerate}

\textsc{\textbf{Вариант 19}}

\begin{enumerate} 
\item Вычислить распределение температур теплоносителей в прямоточном теплообменнике типа <<труба в трубе>>. Использовать модель идеального вытеснения для обоих потоков. Параметры теплообменника: длина  24.8~м, диаметр внешней трубы 65.1~мм,  диаметр внутренней трубы 35.2~мм, толщина стенки $\delta_{w}$=     5~мм,  теплопроводность материала стенки $\lambda_{w}=  570~\frac{\text{Вт}}{\text{м} \cdot \text{К}}$.  Горячий теплоноситель (обозначен индексом <<h>>) направляется во внутреннюю трубу и	 имеет следующие параметры: температура $\text{T}_{h}= 232~^\circ\mathrm{C}$, теплоемкость	  $c_{p{h}}= 1811.3~\frac{\text{Дж}}{\text{кг} \cdot ^\circ\mathrm{C}}$, теплопроводность 		$\lambda_{h}= 0.393~\frac{\text{Вт}}{\text{м} \cdot ^\circ\mathrm{C}}$, плотность 		$\rho_{h}= 1500 \frac{\text{кг}}{\text{м}^3}$, коэффициент вязкости $\mu_{h}=5.586 \text{мПа} 		\cdot \text{с} $, коэффициент термического расширения $\beta_{h}=0.000622 ^\circ\mathrm{C}^{-1}$,		 расход $G_{h}= 3344.018 \frac{\text{кг}}{\text{ч}}$. Холодный теплоноситель (обозначен индексом <<c>>) 		 направляется в межтрубное пространство и имеет следующие параметры: температура $T_{c}=   23		 ~^\circ\mathrm{C}$, теплоемкость $c_{p{c}}= 3250 \frac{\text{Дж}}{\text{кг} \cdot ^\circ\mathrm{C}}$,			 теплопроводность $\lambda_{c}=0.399 \frac{\text{Вт}}{\text{м} \cdot ^\circ\mathrm{C}}$, плотность 			 $\rho_{c}=  1617~\frac{\text{кг}}{\text{м}^3}$, коэффициент вязкости $\mu_{c}=0.145~\text{мПа} \cdot \text{с} $, 			 расход $G_{c}=4213.95~\frac{\text{кг}}{\text{ч}}$. 

\item Определить распределение температуры вдоль теплообменника, если 	изменить направление движения теплоносителей на противоточное.

\item Определить площадь теплообмена для модели идеального смешения, необходимую для достижения 	температур на выходе из теплообменника, таких же как для модели идеального вытеснения (температуры взять из предыдущих заданий).	Провести сравнение эффективности теплообменников с различной структурой потоков.

\end{enumerate}

\textsc{\textbf{Вариант 20}}

\begin{enumerate} 
\item Вычислить распределение температур теплоносителей в прямоточном теплообменнике типа <<труба в трубе>>. Использовать модель идеального вытеснения для обоих потоков. Параметры теплообменника: длина  28.2~м, диаметр внешней трубы 21.1~мм,  диаметр внутренней трубы 12.6~мм, толщина стенки $\delta_{w}$=     3~мм,  теплопроводность материала стенки $\lambda_{w}=  591~\frac{\text{Вт}}{\text{м} \cdot \text{К}}$.  Горячий теплоноситель (обозначен индексом <<h>>) направляется во внутреннюю трубу и	 имеет следующие параметры: температура $\text{T}_{h}= 107~^\circ\mathrm{C}$, теплоемкость	  $c_{p{h}}= 1422.5~\frac{\text{Дж}}{\text{кг} \cdot ^\circ\mathrm{C}}$, теплопроводность 		$\lambda_{h}= 0.143~\frac{\text{Вт}}{\text{м} \cdot ^\circ\mathrm{C}}$, плотность 		$\rho_{h}=  766 \frac{\text{кг}}{\text{м}^3}$, коэффициент вязкости $\mu_{h}=7.001 \text{мПа} 		\cdot \text{с} $, коэффициент термического расширения $\beta_{h}=0.000721 ^\circ\mathrm{C}^{-1}$,		 расход $G_{h}= 147.683 \frac{\text{кг}}{\text{ч}}$. Холодный теплоноситель (обозначен индексом <<c>>) 		 направляется в межтрубное пространство и имеет следующие параметры: температура $T_{c}=   22		 ~^\circ\mathrm{C}$, теплоемкость $c_{p{c}}= 4145 \frac{\text{Дж}}{\text{кг} \cdot ^\circ\mathrm{C}}$,			 теплопроводность $\lambda_{c}=0.370 \frac{\text{Вт}}{\text{м} \cdot ^\circ\mathrm{C}}$, плотность 			 $\rho_{c}=  1840~\frac{\text{кг}}{\text{м}^3}$, коэффициент вязкости $\mu_{c}=0.560~\text{мПа} \cdot \text{с} $, 			 расход $G_{c}=164.12~\frac{\text{кг}}{\text{ч}}$. 

\item Определить распределение температуры вдоль теплообменника, если 	изменить направление движения теплоносителей на противоточное.

\item Рассчитать распределение температуры в теплообменнике структура потоков в котором описывается диффузионной моделью.Коэффициент обратного перемешивания $D_L = $0.090. Коэффициент теплопередачи, свойства веществ и размеры теплообменника взять из первого задания. 

\end{enumerate}

\textsc{\textbf{Вариант 21}}

\begin{enumerate} 
\item Вычислить распределение температур теплоносителей в прямоточном теплообменнике типа <<труба в трубе>>. Использовать модель идеального вытеснения для обоих потоков. Параметры теплообменника: длина  13.7~м, диаметр внешней трубы 20.3~мм,  диаметр внутренней трубы 13.1~мм, толщина стенки $\delta_{w}$=     7~мм,  теплопроводность материала стенки $\lambda_{w}=  697~\frac{\text{Вт}}{\text{м} \cdot \text{К}}$.  Горячий теплоноситель (обозначен индексом <<h>>) направляется во внутреннюю трубу и	 имеет следующие параметры: температура $\text{T}_{h}= 274~^\circ\mathrm{C}$, теплоемкость	  $c_{p{h}}= 3194.3~\frac{\text{Дж}}{\text{кг} \cdot ^\circ\mathrm{C}}$, теплопроводность 		$\lambda_{h}= 0.593~\frac{\text{Вт}}{\text{м} \cdot ^\circ\mathrm{C}}$, плотность 		$\rho_{h}= 1338 \frac{\text{кг}}{\text{м}^3}$, коэффициент вязкости $\mu_{h}=9.348 \text{мПа} 		\cdot \text{с} $, коэффициент термического расширения $\beta_{h}=0.000538 ^\circ\mathrm{C}^{-1}$,		 расход $G_{h}= 4900.570 \frac{\text{кг}}{\text{ч}}$. Холодный теплоноситель (обозначен индексом <<c>>) 		 направляется в межтрубное пространство и имеет следующие параметры: температура $T_{c}=   31		 ~^\circ\mathrm{C}$, теплоемкость $c_{p{c}}= 2014 \frac{\text{Дж}}{\text{кг} \cdot ^\circ\mathrm{C}}$,			 теплопроводность $\lambda_{c}=0.201 \frac{\text{Вт}}{\text{м} \cdot ^\circ\mathrm{C}}$, плотность 			 $\rho_{c}=   925~\frac{\text{кг}}{\text{м}^3}$, коэффициент вязкости $\mu_{c}=0.660~\text{мПа} \cdot \text{с} $, 			 расход $G_{c}=5952.55~\frac{\text{кг}}{\text{ч}}$. 

\item Определить распределение температуры вдоль теплообменника, если 	изменить направление движения теплоносителей на противоточное.

\item Рассчитать распределение температуры в теплообменнике структура потоков в котором описывается ячеечной моделью. Количество ячеек для обоих теплоносителей равно $m = $ 2. Коэффициент теплопередачи, свойства веществ и площадь теплообменника взять из первого задания.

\end{enumerate}

\textsc{\textbf{Вариант 22}}

\begin{enumerate} 
\item Вычислить распределение температур теплоносителей в прямоточном теплообменнике типа <<труба в трубе>>. Использовать модель идеального вытеснения для обоих потоков. Параметры теплообменника: длина  12.1~м, диаметр внешней трубы 66.6~мм,  диаметр внутренней трубы 35.9~мм, толщина стенки $\delta_{w}$=     3~мм,  теплопроводность материала стенки $\lambda_{w}=  608~\frac{\text{Вт}}{\text{м} \cdot \text{К}}$.  Горячий теплоноситель (обозначен индексом <<h>>) направляется во внутреннюю трубу и	 имеет следующие параметры: температура $\text{T}_{h}= 145~^\circ\mathrm{C}$, теплоемкость	  $c_{p{h}}= 1683.4~\frac{\text{Дж}}{\text{кг} \cdot ^\circ\mathrm{C}}$, теплопроводность 		$\lambda_{h}= 0.467~\frac{\text{Вт}}{\text{м} \cdot ^\circ\mathrm{C}}$, плотность 		$\rho_{h}= 1369 \frac{\text{кг}}{\text{м}^3}$, коэффициент вязкости $\mu_{h}=1.301 \text{мПа} 		\cdot \text{с} $, коэффициент термического расширения $\beta_{h}=0.000970 ^\circ\mathrm{C}^{-1}$,		 расход $G_{h}= 161.909 \frac{\text{кг}}{\text{ч}}$. Холодный теплоноситель (обозначен индексом <<c>>) 		 направляется в межтрубное пространство и имеет следующие параметры: температура $T_{c}=   27		 ~^\circ\mathrm{C}$, теплоемкость $c_{p{c}}= 3494 \frac{\text{Дж}}{\text{кг} \cdot ^\circ\mathrm{C}}$,			 теплопроводность $\lambda_{c}=0.381 \frac{\text{Вт}}{\text{м} \cdot ^\circ\mathrm{C}}$, плотность 			 $\rho_{c}=  1266~\frac{\text{кг}}{\text{м}^3}$, коэффициент вязкости $\mu_{c}=2.358~\text{мПа} \cdot \text{с} $, 			 расход $G_{c}=210.01~\frac{\text{кг}}{\text{ч}}$. 

\item Определить распределение температуры вдоль теплообменника, если 	изменить направление движения теплоносителей на противоточное.

\item Рассчитать распределение температуры в теплообменнике структура потоков в котором описывается диффузионной моделью.Коэффициент обратного перемешивания $D_L = $0.085. Коэффициент теплопередачи, свойства веществ и размеры теплообменника взять из первого задания. 

\end{enumerate}

\textsc{\textbf{Вариант 23}}

\begin{enumerate} 
\item Вычислить распределение температур теплоносителей в прямоточном теплообменнике типа <<труба в трубе>>. Использовать модель идеального вытеснения для обоих потоков. Параметры теплообменника: длина  28.5~м, диаметр внешней трубы 41.1~мм,  диаметр внутренней трубы 25.8~мм, толщина стенки $\delta_{w}$=     5~мм,  теплопроводность материала стенки $\lambda_{w}=  669~\frac{\text{Вт}}{\text{м} \cdot \text{К}}$.  Горячий теплоноситель (обозначен индексом <<h>>) направляется во внутреннюю трубу и	 имеет следующие параметры: температура $\text{T}_{h}= 214~^\circ\mathrm{C}$, теплоемкость	  $c_{p{h}}= 2392.1~\frac{\text{Дж}}{\text{кг} \cdot ^\circ\mathrm{C}}$, теплопроводность 		$\lambda_{h}= 0.156~\frac{\text{Вт}}{\text{м} \cdot ^\circ\mathrm{C}}$, плотность 		$\rho_{h}=  633 \frac{\text{кг}}{\text{м}^3}$, коэффициент вязкости $\mu_{h}=3.779 \text{мПа} 		\cdot \text{с} $, коэффициент термического расширения $\beta_{h}=0.000492 ^\circ\mathrm{C}^{-1}$,		 расход $G_{h}= 3363.117 \frac{\text{кг}}{\text{ч}}$. Холодный теплоноситель (обозначен индексом <<c>>) 		 направляется в межтрубное пространство и имеет следующие параметры: температура $T_{c}=   38		 ~^\circ\mathrm{C}$, теплоемкость $c_{p{c}}= 3475 \frac{\text{Дж}}{\text{кг} \cdot ^\circ\mathrm{C}}$,			 теплопроводность $\lambda_{c}=0.144 \frac{\text{Вт}}{\text{м} \cdot ^\circ\mathrm{C}}$, плотность 			 $\rho_{c}=   925~\frac{\text{кг}}{\text{м}^3}$, коэффициент вязкости $\mu_{c}=5.545~\text{мПа} \cdot \text{с} $, 			 расход $G_{c}=5009.26~\frac{\text{кг}}{\text{ч}}$. 

\item Определить распределение температуры вдоль теплообменника, если 	изменить направление движения теплоносителей на противоточное.

\item Определить площадь теплообмена для модели идеального смешения, необходимую для достижения 	температур на выходе из теплообменника, таких же как для модели идеального вытеснения (температуры взять из предыдущих заданий).	Провести сравнение эффективности теплообменников с различной структурой потоков.

\end{enumerate}

\textsc{\textbf{Вариант 24}}

\begin{enumerate} 
\item Вычислить распределение температур теплоносителей в прямоточном теплообменнике типа <<труба в трубе>>. Использовать модель идеального вытеснения для обоих потоков. Параметры теплообменника: длина  13.6~м, диаметр внешней трубы 47.0~мм,  диаметр внутренней трубы 19.9~мм, толщина стенки $\delta_{w}$=     4~мм,  теплопроводность материала стенки $\lambda_{w}=  564~\frac{\text{Вт}}{\text{м} \cdot \text{К}}$.  Горячий теплоноситель (обозначен индексом <<h>>) направляется во внутреннюю трубу и	 имеет следующие параметры: температура $\text{T}_{h}= 237~^\circ\mathrm{C}$, теплоемкость	  $c_{p{h}}= 2570.3~\frac{\text{Дж}}{\text{кг} \cdot ^\circ\mathrm{C}}$, теплопроводность 		$\lambda_{h}= 0.536~\frac{\text{Вт}}{\text{м} \cdot ^\circ\mathrm{C}}$, плотность 		$\rho_{h}= 1486 \frac{\text{кг}}{\text{м}^3}$, коэффициент вязкости $\mu_{h}=0.597 \text{мПа} 		\cdot \text{с} $, коэффициент термического расширения $\beta_{h}=0.000357 ^\circ\mathrm{C}^{-1}$,		 расход $G_{h}= 524.599 \frac{\text{кг}}{\text{ч}}$. Холодный теплоноситель (обозначен индексом <<c>>) 		 направляется в межтрубное пространство и имеет следующие параметры: температура $T_{c}=   23		 ~^\circ\mathrm{C}$, теплоемкость $c_{p{c}}= 2273 \frac{\text{Дж}}{\text{кг} \cdot ^\circ\mathrm{C}}$,			 теплопроводность $\lambda_{c}=0.306 \frac{\text{Вт}}{\text{м} \cdot ^\circ\mathrm{C}}$, плотность 			 $\rho_{c}=  1256~\frac{\text{кг}}{\text{м}^3}$, коэффициент вязкости $\mu_{c}=4.752~\text{мПа} \cdot \text{с} $, 			 расход $G_{c}=859.22~\frac{\text{кг}}{\text{ч}}$. 

\item Определить распределение температуры вдоль теплообменника, если 	изменить направление движения теплоносителей на противоточное.

\item Определить площадь теплообмена для модели идеального смешения, необходимую для достижения 	температур на выходе из теплообменника, таких же как для модели идеального вытеснения (температуры взять из предыдущих заданий).	Провести сравнение эффективности теплообменников с различной структурой потоков.

\end{enumerate}

\textsc{\textbf{Вариант 25}}

\begin{enumerate} 
\item Вычислить распределение температур теплоносителей в прямоточном теплообменнике типа <<труба в трубе>>. Использовать модель идеального вытеснения для обоих потоков. Параметры теплообменника: длина  25.1~м, диаметр внешней трубы 28.2~мм,  диаметр внутренней трубы 13.1~мм, толщина стенки $\delta_{w}$=     4~мм,  теплопроводность материала стенки $\lambda_{w}=  381~\frac{\text{Вт}}{\text{м} \cdot \text{К}}$.  Горячий теплоноситель (обозначен индексом <<h>>) направляется во внутреннюю трубу и	 имеет следующие параметры: температура $\text{T}_{h}= 175~^\circ\mathrm{C}$, теплоемкость	  $c_{p{h}}= 4125.0~\frac{\text{Дж}}{\text{кг} \cdot ^\circ\mathrm{C}}$, теплопроводность 		$\lambda_{h}= 0.510~\frac{\text{Вт}}{\text{м} \cdot ^\circ\mathrm{C}}$, плотность 		$\rho_{h}= 1456 \frac{\text{кг}}{\text{м}^3}$, коэффициент вязкости $\mu_{h}=1.123 \text{мПа} 		\cdot \text{с} $, коэффициент термического расширения $\beta_{h}=0.000248 ^\circ\mathrm{C}^{-1}$,		 расход $G_{h}= 55.631 \frac{\text{кг}}{\text{ч}}$. Холодный теплоноситель (обозначен индексом <<c>>) 		 направляется в межтрубное пространство и имеет следующие параметры: температура $T_{c}=   25		 ~^\circ\mathrm{C}$, теплоемкость $c_{p{c}}= 1691 \frac{\text{Дж}}{\text{кг} \cdot ^\circ\mathrm{C}}$,			 теплопроводность $\lambda_{c}=0.465 \frac{\text{Вт}}{\text{м} \cdot ^\circ\mathrm{C}}$, плотность 			 $\rho_{c}=   839~\frac{\text{кг}}{\text{м}^3}$, коэффициент вязкости $\mu_{c}=5.287~\text{мПа} \cdot \text{с} $, 			 расход $G_{c}=96.14~\frac{\text{кг}}{\text{ч}}$. 

\item Определить распределение температуры вдоль теплообменника, если 	изменить направление движения теплоносителей на противоточное.

\item Рассчитать распределение температуры в теплообменнике структура потоков в котором описывается ячеечной моделью. Количество ячеек для обоих теплоносителей равно $m = $ 3. Коэффициент теплопередачи, свойства веществ и площадь теплообменника взять из первого задания.

\end{enumerate}

\textsc{\textbf{Вариант 26}}

\begin{enumerate} 
\item Вычислить распределение температур теплоносителей в прямоточном теплообменнике типа <<труба в трубе>>. Использовать модель идеального вытеснения для обоих потоков. Параметры теплообменника: длина  11.3~м, диаметр внешней трубы 65.8~мм,  диаметр внутренней трубы 36.4~мм, толщина стенки $\delta_{w}$=     6~мм,  теплопроводность материала стенки $\lambda_{w}=  337~\frac{\text{Вт}}{\text{м} \cdot \text{К}}$.  Горячий теплоноситель (обозначен индексом <<h>>) направляется во внутреннюю трубу и	 имеет следующие параметры: температура $\text{T}_{h}= 198~^\circ\mathrm{C}$, теплоемкость	  $c_{p{h}}= 1653.0~\frac{\text{Дж}}{\text{кг} \cdot ^\circ\mathrm{C}}$, теплопроводность 		$\lambda_{h}= 0.205~\frac{\text{Вт}}{\text{м} \cdot ^\circ\mathrm{C}}$, плотность 		$\rho_{h}= 1082 \frac{\text{кг}}{\text{м}^3}$, коэффициент вязкости $\mu_{h}=8.389 \text{мПа} 		\cdot \text{с} $, коэффициент термического расширения $\beta_{h}=0.000606 ^\circ\mathrm{C}^{-1}$,		 расход $G_{h}= 3860.580 \frac{\text{кг}}{\text{ч}}$. Холодный теплоноситель (обозначен индексом <<c>>) 		 направляется в межтрубное пространство и имеет следующие параметры: температура $T_{c}=   29		 ~^\circ\mathrm{C}$, теплоемкость $c_{p{c}}= 3284 \frac{\text{Дж}}{\text{кг} \cdot ^\circ\mathrm{C}}$,			 теплопроводность $\lambda_{c}=0.394 \frac{\text{Вт}}{\text{м} \cdot ^\circ\mathrm{C}}$, плотность 			 $\rho_{c}=  1290~\frac{\text{кг}}{\text{м}^3}$, коэффициент вязкости $\mu_{c}=5.566~\text{мПа} \cdot \text{с} $, 			 расход $G_{c}=2982.84~\frac{\text{кг}}{\text{ч}}$. 

\item Определить распределение температуры вдоль теплообменника, если 	изменить направление движения теплоносителей на противоточное.

\item Определить площадь теплообмена для модели идеального смешения, необходимую для достижения 	температур на выходе из теплообменника, таких же как для модели идеального вытеснения (температуры взять из предыдущих заданий).	Провести сравнение эффективности теплообменников с различной структурой потоков.

\end{enumerate}

\textsc{\textbf{Вариант 27}}

\begin{enumerate} 
\item Вычислить распределение температур теплоносителей в прямоточном теплообменнике типа <<труба в трубе>>. Использовать модель идеального вытеснения для обоих потоков. Параметры теплообменника: длина  20.9~м, диаметр внешней трубы 38.7~мм,  диаметр внутренней трубы 20.1~мм, толщина стенки $\delta_{w}$=     3~мм,  теплопроводность материала стенки $\lambda_{w}=  561~\frac{\text{Вт}}{\text{м} \cdot \text{К}}$.  Горячий теплоноситель (обозначен индексом <<h>>) направляется во внутреннюю трубу и	 имеет следующие параметры: температура $\text{T}_{h}= 104~^\circ\mathrm{C}$, теплоемкость	  $c_{p{h}}= 3586.3~\frac{\text{Дж}}{\text{кг} \cdot ^\circ\mathrm{C}}$, теплопроводность 		$\lambda_{h}= 0.313~\frac{\text{Вт}}{\text{м} \cdot ^\circ\mathrm{C}}$, плотность 		$\rho_{h}=  638 \frac{\text{кг}}{\text{м}^3}$, коэффициент вязкости $\mu_{h}=6.339 \text{мПа} 		\cdot \text{с} $, коэффициент термического расширения $\beta_{h}=0.000377 ^\circ\mathrm{C}^{-1}$,		 расход $G_{h}= 7008.518 \frac{\text{кг}}{\text{ч}}$. Холодный теплоноситель (обозначен индексом <<c>>) 		 направляется в межтрубное пространство и имеет следующие параметры: температура $T_{c}=   28		 ~^\circ\mathrm{C}$, теплоемкость $c_{p{c}}= 2844 \frac{\text{Дж}}{\text{кг} \cdot ^\circ\mathrm{C}}$,			 теплопроводность $\lambda_{c}=0.427 \frac{\text{Вт}}{\text{м} \cdot ^\circ\mathrm{C}}$, плотность 			 $\rho_{c}=  1457~\frac{\text{кг}}{\text{м}^3}$, коэффициент вязкости $\mu_{c}=4.849~\text{мПа} \cdot \text{с} $, 			 расход $G_{c}=11674.79~\frac{\text{кг}}{\text{ч}}$. 

\item Определить распределение температуры вдоль теплообменника, если 	изменить направление движения теплоносителей на противоточное.

\item Рассчитать распределение температуры в теплообменнике структура потоков в котором описывается диффузионной моделью.Коэффициент обратного перемешивания $D_L = $0.021. Коэффициент теплопередачи, свойства веществ и размеры теплообменника взять из первого задания. 

\end{enumerate}

\textsc{\textbf{Вариант 28}}

\begin{enumerate} 
\item Вычислить распределение температур теплоносителей в прямоточном теплообменнике типа <<труба в трубе>>. Использовать модель идеального вытеснения для обоих потоков. Параметры теплообменника: длина  20.5~м, диаметр внешней трубы 34.5~мм,  диаметр внутренней трубы 20.1~мм, толщина стенки $\delta_{w}$=     4~мм,  теплопроводность материала стенки $\lambda_{w}=  499~\frac{\text{Вт}}{\text{м} \cdot \text{К}}$.  Горячий теплоноситель (обозначен индексом <<h>>) направляется во внутреннюю трубу и	 имеет следующие параметры: температура $\text{T}_{h}= 106~^\circ\mathrm{C}$, теплоемкость	  $c_{p{h}}= 3953.7~\frac{\text{Дж}}{\text{кг} \cdot ^\circ\mathrm{C}}$, теплопроводность 		$\lambda_{h}= 0.183~\frac{\text{Вт}}{\text{м} \cdot ^\circ\mathrm{C}}$, плотность 		$\rho_{h}=  937 \frac{\text{кг}}{\text{м}^3}$, коэффициент вязкости $\mu_{h}=3.034 \text{мПа} 		\cdot \text{с} $, коэффициент термического расширения $\beta_{h}=0.000733 ^\circ\mathrm{C}^{-1}$,		 расход $G_{h}= 105.283 \frac{\text{кг}}{\text{ч}}$. Холодный теплоноситель (обозначен индексом <<c>>) 		 направляется в межтрубное пространство и имеет следующие параметры: температура $T_{c}=   37		 ~^\circ\mathrm{C}$, теплоемкость $c_{p{c}}= 3984 \frac{\text{Дж}}{\text{кг} \cdot ^\circ\mathrm{C}}$,			 теплопроводность $\lambda_{c}=0.449 \frac{\text{Вт}}{\text{м} \cdot ^\circ\mathrm{C}}$, плотность 			 $\rho_{c}=   933~\frac{\text{кг}}{\text{м}^3}$, коэффициент вязкости $\mu_{c}=1.081~\text{мПа} \cdot \text{с} $, 			 расход $G_{c}=52.71~\frac{\text{кг}}{\text{ч}}$. 

\item Определить распределение температуры вдоль теплообменника, если 	изменить направление движения теплоносителей на противоточное.

\item Рассчитать распределение температуры в теплообменнике структура потоков в котором описывается диффузионной моделью.Коэффициент обратного перемешивания $D_L = $0.069. Коэффициент теплопередачи, свойства веществ и размеры теплообменника взять из первого задания. 

\end{enumerate}

\textsc{\textbf{Вариант 29}}

\begin{enumerate} 
\item Вычислить распределение температур теплоносителей в прямоточном теплообменнике типа <<труба в трубе>>. Использовать модель идеального вытеснения для обоих потоков. Параметры теплообменника: длина  26.9~м, диаметр внешней трубы 34.7~мм,  диаметр внутренней трубы 21.2~мм, толщина стенки $\delta_{w}$=     7~мм,  теплопроводность материала стенки $\lambda_{w}=  382~\frac{\text{Вт}}{\text{м} \cdot \text{К}}$.  Горячий теплоноситель (обозначен индексом <<h>>) направляется во внутреннюю трубу и	 имеет следующие параметры: температура $\text{T}_{h}= 209~^\circ\mathrm{C}$, теплоемкость	  $c_{p{h}}= 2084.2~\frac{\text{Дж}}{\text{кг} \cdot ^\circ\mathrm{C}}$, теплопроводность 		$\lambda_{h}= 0.314~\frac{\text{Вт}}{\text{м} \cdot ^\circ\mathrm{C}}$, плотность 		$\rho_{h}= 1327 \frac{\text{кг}}{\text{м}^3}$, коэффициент вязкости $\mu_{h}=7.772 \text{мПа} 		\cdot \text{с} $, коэффициент термического расширения $\beta_{h}=0.000644 ^\circ\mathrm{C}^{-1}$,		 расход $G_{h}= 1868.328 \frac{\text{кг}}{\text{ч}}$. Холодный теплоноситель (обозначен индексом <<c>>) 		 направляется в межтрубное пространство и имеет следующие параметры: температура $T_{c}=   38		 ~^\circ\mathrm{C}$, теплоемкость $c_{p{c}}= 3772 \frac{\text{Дж}}{\text{кг} \cdot ^\circ\mathrm{C}}$,			 теплопроводность $\lambda_{c}=0.183 \frac{\text{Вт}}{\text{м} \cdot ^\circ\mathrm{C}}$, плотность 			 $\rho_{c}=   956~\frac{\text{кг}}{\text{м}^3}$, коэффициент вязкости $\mu_{c}=4.710~\text{мПа} \cdot \text{с} $, 			 расход $G_{c}=1228.78~\frac{\text{кг}}{\text{ч}}$. 

\item Определить распределение температуры вдоль теплообменника, если 	изменить направление движения теплоносителей на противоточное.

\item Рассчитать распределение температуры в теплообменнике структура потоков в котором описывается диффузионной моделью.Коэффициент обратного перемешивания $D_L = $0.119. Коэффициент теплопередачи, свойства веществ и размеры теплообменника взять из первого задания. 

\end{enumerate}

\textsc{\textbf{Вариант 30}}

\begin{enumerate} 
\item Вычислить распределение температур теплоносителей в прямоточном теплообменнике типа <<труба в трубе>>. Использовать модель идеального вытеснения для обоих потоков. Параметры теплообменника: длина  21.7~м, диаметр внешней трубы 56.1~мм,  диаметр внутренней трубы 24.4~мм, толщина стенки $\delta_{w}$=     3~мм,  теплопроводность материала стенки $\lambda_{w}=  465~\frac{\text{Вт}}{\text{м} \cdot \text{К}}$.  Горячий теплоноситель (обозначен индексом <<h>>) направляется во внутреннюю трубу и	 имеет следующие параметры: температура $\text{T}_{h}= 227~^\circ\mathrm{C}$, теплоемкость	  $c_{p{h}}= 3956.6~\frac{\text{Дж}}{\text{кг} \cdot ^\circ\mathrm{C}}$, теплопроводность 		$\lambda_{h}= 0.405~\frac{\text{Вт}}{\text{м} \cdot ^\circ\mathrm{C}}$, плотность 		$\rho_{h}= 1319 \frac{\text{кг}}{\text{м}^3}$, коэффициент вязкости $\mu_{h}=9.166 \text{мПа} 		\cdot \text{с} $, коэффициент термического расширения $\beta_{h}=0.000261 ^\circ\mathrm{C}^{-1}$,		 расход $G_{h}= 620.965 \frac{\text{кг}}{\text{ч}}$. Холодный теплоноситель (обозначен индексом <<c>>) 		 направляется в межтрубное пространство и имеет следующие параметры: температура $T_{c}=   39		 ~^\circ\mathrm{C}$, теплоемкость $c_{p{c}}= 1920 \frac{\text{Дж}}{\text{кг} \cdot ^\circ\mathrm{C}}$,			 теплопроводность $\lambda_{c}=0.480 \frac{\text{Вт}}{\text{м} \cdot ^\circ\mathrm{C}}$, плотность 			 $\rho_{c}=  1992~\frac{\text{кг}}{\text{м}^3}$, коэффициент вязкости $\mu_{c}=7.266~\text{мПа} \cdot \text{с} $, 			 расход $G_{c}=329.65~\frac{\text{кг}}{\text{ч}}$. 

\item Определить распределение температуры вдоль теплообменника, если 	изменить направление движения теплоносителей на противоточное.

\item Определить площадь теплообмена для модели идеального смешения, необходимую для достижения 	температур на выходе из теплообменника, таких же как для модели идеального вытеснения (температуры взять из предыдущих заданий).	Провести сравнение эффективности теплообменников с различной структурой потоков.

\end{enumerate}

