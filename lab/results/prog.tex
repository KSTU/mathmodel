\section {Лабораторная работа №~4 <<Основы программирования>>}

 \addtocounter{nlab}{1}\textsc{\textbf{Вариант 1}}
\begin{enumerate}
\item  Создать программу, возвращающую значение $\sum\limits_{n=0}^{N} \dfrac{n+3}{n+9}              $, если N=5. 
\item  Создать программу, возвращающую значение $\sum\limits_{n+4<      92} {n -\dfrac{1}{n}}             $. 
\item Создать функцию, аргументом которой является массив $A$ произвольной размерностью, возвращающую значение $D_{i,j}=\begin{cases} A_{i,j}^{i-j}, & \text{если } {A_{i,j}}<0 \\  A_{i,j}^{i+-j+1}, & \text{если } {A_{i,j}}>0 \end{cases}                 $ 
\item Машина трогается с места и за  6.8 минуты разгоняется с постоянным ускорением до 114.9 ${\frac{км}{ч}}$, далее машина   61 минут едет с постоянной скоростью, после чего начинает торможение с постоянным ускорением и через   3 минут останавливается. Построить график изменения скорости от времени. Решить дифференциальное уравнение для изменения координаты машины во времени и определить путь пройденный машиной.  \item Реализовать алгоритм перемножения двух матриц, сравнить результат работы программы со встроенной функцией перемножения матриц.                                                                        

\end{enumerate}
\textsc{\textbf{Вариант 2}}
\begin{enumerate}
\item  Создать программу, возвращающую значение $\sum\limits_{n=1}^{N} \dfrac{\ln(n)+e^{-n}}{e^{-n}} $, если N=8. 
\item  Создать программу, возвращающую значение $\sum\limits_{n^{2.5}<  57} {e^{-n}}                      $. 
\item Создать функцию, аргументом которой является массив $A$ произвольной размерностью, возвращающую значение $D_{i,j}=\begin{cases} A_{i,j}^{i-j}, & \text{если } {A_{i,j}}<0 \\  A_{i,j}^{i+-j+1}, & \text{если } {A_{i,j}}>0 \end{cases}                 $ 
\item Машина трогается с места и за  8.8 минуты разгоняется с постоянным ускорением до 48.1 ${\frac{км}{ч}}$, далее машина   72 минут едет с постоянной скоростью, после чего начинает торможение с постоянным ускорением и через   5 минут останавливается. Построить график изменения скорости от времени. Решить дифференциальное уравнение для изменения координаты машины во времени и определить путь пройденный машиной.  \item Составить программу определения произведения элементов матрицы больших 1.                                                                                                                                                                            

\end{enumerate}
\textsc{\textbf{Вариант 3}}
\begin{enumerate}
\item  Создать программу, возвращающую значение $\sum\limits_{n=1}^{N} n+\dfrac{n}{4+n}              $, если N=9. 
\item  Создать программу, возвращающую значение $\sum\limits_{n^{2.5}<  38} {n -\dfrac{1}{n}}             $. 
\item Создать функцию, аргументом которой является массив $A$ произвольной размерностью, возвращающую значение $D_{i,j}=\begin{cases} A_{i,j}^2, & \text{если } \sqrt{A_{i,j}}<5 \\ \sqrt{A_{i,j}}, & \text{если } \sqrt{A_{i,j}}>5 \end{cases}              $ 
\item Машина трогается с места и за 11.7 минуты разгоняется с постоянным ускорением до 114.0 ${\frac{км}{ч}}$, далее машина   53 минут едет с постоянной скоростью, после чего начинает торможение с постоянным ускорением и через   4 минут останавливается. Построить график изменения скорости от времени. Решить дифференциальное уравнение для изменения координаты машины во времени и определить путь пройденный машиной.  \item Составить программу сравнивающую сумму элементов двух матриц, и возвращающую матрицу с большей суммой элементов.                                                                                                    

\end{enumerate}
\textsc{\textbf{Вариант 4}}
\begin{enumerate}
\item  Создать программу, возвращающую значение $\sum\limits_{n=1}^{N} {n \sin(n)}                   $, если N=4. 
\item  Создать программу, возвращающую значение $\sum\limits_{n+4<      13} {n}^{0.3}                     $. 
\item Создать функцию, аргументом которой является массив $A$ произвольной размерностью, возвращающую значение $D_{i,j}=\begin{cases} i+j, & \text{если } {A_{i,j}}<i+j \\  A_{i,j}+j, & \text{если } {A_{i,j}}>i+j \end{cases}                              $ 
\item Машина трогается с места и за  5.0 минуты разгоняется с постоянным ускорением до 99.8 ${\frac{км}{ч}}$, далее машина   70 минут едет с постоянной скоростью, после чего начинает торможение с постоянным ускорением и через   3 минут останавливается. Построить график изменения скорости от времени. Решить дифференциальное уравнение для изменения координаты машины во времени и определить путь пройденный машиной.  \item Составить программу, аргументом которой является произвольная матрица, и возвращающую массив, содержащий количество элементов больших 1 и количество элментов больше 5.

\end{enumerate}
\textsc{\textbf{Вариант 5}}
\begin{enumerate}
\item  Создать программу, возвращающую значение $\sum\limits_{n=1}^{N} \dfrac{\sin(n)+1}{\cos(n)+1}  $, если N=4. 
\item  Создать программу, возвращающую значение $\sum\limits_{n+4<      30}  \dfrac{n+3}{n+9}             $. 
\item Создать функцию, аргументом которой является массив $A$ произвольной размерностью, возвращающую значение $D_{i,j}=\begin{cases} A_{i,j}+i, & \text{если } {A_{i,j}}<i \\  A_{i,j}-j, & \text{если } {A_{i,j}}>i \end{cases}                            $ 
\item Машина трогается с места и за  4.6 минуты разгоняется с постоянным ускорением до 75.3 ${\frac{км}{ч}}$, далее машина   55 минут едет с постоянной скоростью, после чего начинает торможение с постоянным ускорением и через   2 минут останавливается. Построить график изменения скорости от времени. Решить дифференциальное уравнение для изменения координаты машины во времени и определить путь пройденный машиной.  \item Реализовать алгоритм перемножения двух матриц, сравнить результат работы программы со встроенной функцией перемножения матриц.                                                                        

\end{enumerate}
\textsc{\textbf{Вариант 6}}
\begin{enumerate}
\item  Создать программу, возвращающую значение $\sum\limits_{n=1}^{N} \dfrac{\ln(n)+e^{-n}}{e^{-n}} $, если N=6. 
\item  Создать программу, возвращающую значение $\sum\limits_{n^2<      70} {\dfrac{1}{n} -\dfrac{1}{n^2}}$. 
\item Создать функцию, аргументом которой является массив $A$ произвольной размерностью, возвращающую значение $D_{i,j}=\begin{cases} A_{i,j}+\cos(i)+\cos(j), & \text{если } {A_{i,j}}<i \\  A_{i,j}+\sin(i)+\sin(j), & \text{если } {A_{i,j}}>i \end{cases}$ 
\item Машина трогается с места и за 12.8 минуты разгоняется с постоянным ускорением до 106.0 ${\frac{км}{ч}}$, далее машина   77 минут едет с постоянной скоростью, после чего начинает торможение с постоянным ускорением и через   5 минут останавливается. Построить график изменения скорости от времени. Решить дифференциальное уравнение для изменения координаты машины во времени и определить путь пройденный машиной.  \item Составить программу определения количества отрицательных чисел в произвольной матрице.                                                                                                                                                  

\end{enumerate}
\textsc{\textbf{Вариант 7}}
\begin{enumerate}
\item  Создать программу, возвращающую значение $\sum\limits_{n=1}^{N} {n (n+1)}                     $, если N=8. 
\item  Создать программу, возвращающую значение $\sum\limits_{0.3n+4<   48} {\dfrac{1}{n} e^{-n}}         $. 
\item Создать функцию, аргументом которой является массив $A$ произвольной размерностью, возвращающую значение $D_{i,j}=\begin{cases} A_{i,j}, & \text{если } {A_{i,j}}<i \\  -A_{i,j}, & \text{если } {A_{i,j}}>i \end{cases}                               $ 
\item Машина трогается с места и за 12.9 минуты разгоняется с постоянным ускорением до 125.7 ${\frac{км}{ч}}$, далее машина   80 минут едет с постоянной скоростью, после чего начинает торможение с постоянным ускорением и через   3 минут останавливается. Построить график изменения скорости от времени. Решить дифференциальное уравнение для изменения координаты машины во времени и определить путь пройденный машиной.  \item Составить программу определения количества четных чисел в произвольной матрице.                                                                                                                                                                

\end{enumerate}
\textsc{\textbf{Вариант 8}}
\begin{enumerate}
\item  Создать программу, возвращающую значение $\sum\limits_{n=1}^{N} n+\dfrac{n}{4+n}              $, если N=8. 
\item  Создать программу, возвращающую значение $\sum\limits_{n<        69} {n \sin(n)}                   $. 
\item Создать функцию, аргументом которой является массив $A$ произвольной размерностью, возвращающую значение $D_{i,j}=\begin{cases} i+j, & \text{если } {A_{i,j}}<i+j \\  A_{i,j}+j, & \text{если } {A_{i,j}}>i+j \end{cases}                              $ 
\item Машина трогается с места и за  3.0 минуты разгоняется с постоянным ускорением до 84.7 ${\frac{км}{ч}}$, далее машина   70 минут едет с постоянной скоростью, после чего начинает торможение с постоянным ускорением и через   3 минут останавливается. Построить график изменения скорости от времени. Решить дифференциальное уравнение для изменения координаты машины во времени и определить путь пройденный машиной.  \item Составить программу сравнивающую сумму элементов двух матриц, и возвращающую матрицу с большей суммой элементов.                                                                                                    

\end{enumerate}
\textsc{\textbf{Вариант 9}}
\begin{enumerate}
\item  Создать программу, возвращающую значение $\sum\limits_{n=1}^{N} {\dfrac{1}{n} +sin(n)}        $, если N=8. 
\item  Создать программу, возвращающую значение $\sum\limits_{e^{n-2}<  54} \dfrac{\ln(n+1)}{\ln(n)}      $. 
\item Создать функцию, аргументом которой является массив $A$ произвольной размерностью, возвращающую значение $D_{i,j}=\begin{cases} 5 A_{i,j}, & \text{если } \sqrt{A_{i,j}}<7 \\ \sqrt{ 6 A_{i,j}}, & \text{если } \sqrt{A_{i,j}}>7 \end{cases}           $ 
\item Машина трогается с места и за  9.6 минуты разгоняется с постоянным ускорением до 68.1 ${\frac{км}{ч}}$, далее машина   34 минут едет с постоянной скоростью, после чего начинает торможение с постоянным ускорением и через   3 минут останавливается. Построить график изменения скорости от времени. Решить дифференциальное уравнение для изменения координаты машины во времени и определить путь пройденный машиной.  \item Составить программу определения произведения элементов матрицы больших 1.                                                                                                                                                                            

\end{enumerate}
\textsc{\textbf{Вариант 10}}
\begin{enumerate}
\item  Создать программу, возвращающую значение $\sum\limits_{n=1}^{N} {\dfrac{1}{n} -\dfrac{1}{n^2}}$, если N=9. 
\item  Создать программу, возвращающую значение $\sum\limits_{n+4<      92} \dfrac{\sin(n)+1}{\cos(n)+1}  $. 
\item Создать функцию, аргументом которой является массив $A$ произвольной размерностью, возвращающую значение $D_{i,j}=\begin{cases} A_{i,j}+\cos(i)+\cos(j), & \text{если } {A_{i,j}}<i \\  A_{i,j}+\sin(i)+\sin(j), & \text{если } {A_{i,j}}>i \end{cases}$ 
\item Машина трогается с места и за  4.1 минуты разгоняется с постоянным ускорением до 77.5 ${\frac{км}{ч}}$, далее машина   25 минут едет с постоянной скоростью, после чего начинает торможение с постоянным ускорением и через   2 минут останавливается. Построить график изменения скорости от времени. Решить дифференциальное уравнение для изменения координаты машины во времени и определить путь пройденный машиной.  \item Составить программу определения количества четных чисел в произвольной матрице.                                                                                                                                                                

\end{enumerate}
