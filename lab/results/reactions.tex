\section {Лабораторная работа №~9 <<Моделирование реакций в реакторах с различной структурой потоков>>}

  \addtocounter{nlab}{1}\textsc{\textbf{Вариант 1}}

 В реакторе идеального вытеснения протекает реакция: \begin{equation*} \begin{aligned} A \xleftrightarrow[k_2]{k_1} B + \Delta H_1 \\ B \xrightarrow{k_3} C + \Delta H_2 \end{aligned} \end{equation*}                              На вход  реактор подается смесь при температуре $ T_{н} =  259 K$, теплоемкость смеси $c_p= 3979 \frac{Дж}{моль \cdot K}$, состав подаваемой смеси: $c_A=25.4 \frac{моль}{л}$, $c_B=0.4 \frac{моль}{л}$. Параметры реакций: энергии активации $E_{a1}=10.5 \frac{кДж}{моль}$, $E_{a2}=25.9  \frac{кДж}{моль}$, $E_{a3}=23.2  \frac{кДж}{моль}$, предэкспоненциальный множитель $k_{01}=        14$,$k_{02}=      5166$,$k_{03}=      1857$, тепловой эффект $\Delta H_1= -8.8  \frac{кДж}{моль}$, $\Delta H_2= 5.1 \frac{кДж}{моль}$.\begin{itemize} \item Составить математическую модель изотермического реактора. Определить распределение концентрации компонентов по времени. Определить изменение конверсии по компоненту A, селективности и выхода по компоненту B. \end{itemize}

\textsc{\textbf{Вариант 2}}

 В реакторе идеального вытеснения протекает реакция: \begin{equation*} \begin{aligned} A+B \xleftrightarrow[k_2]{k_1} C + \Delta H_1 \\ B + C \xrightarrow{k_3} D + \Delta H_2 \end{aligned} \end{equation*}                        На вход  реактор подается смесь при температуре $ T_{н} =  209 K$, теплоемкость смеси $c_p= 2054 \frac{Дж}{моль \cdot K}$, состав подаваемой смеси: $c_A=27.5 \frac{моль}{л}$, $c_B=0.2 \frac{моль}{л}$. Параметры реакций: энергии активации $E_{a1}= 7.4 \frac{кДж}{моль}$, $E_{a2}=18.2  \frac{кДж}{моль}$, $E_{a3}=13.9  \frac{кДж}{моль}$, предэкспоненциальный множитель $k_{01}=        10$,$k_{02}=       909$,$k_{03}=       254$, тепловой эффект $\Delta H_1= 37.4  \frac{кДж}{моль}$, $\Delta H_2=38.8 \frac{кДж}{моль}$.\begin{itemize} \item Составить математическую модель изотермического реактора. Определить распределение концентрации компонентов по времени. Определить изменение конверсии по компоненту A, селективности и выхода по компоненту B. \end{itemize}

\textsc{\textbf{Вариант 3}}

 В реакторе идеального вытеснения протекает реакция: \begin{equation*} \begin{aligned} A \xleftrightarrow[k_2]{k_1} B + \Delta H_1 \\ B \xrightarrow{k_3} C + \Delta H_2 \end{aligned} \end{equation*}                              На вход  реактор подается смесь при температуре $ T_{н} =  259 K$, теплоемкость смеси $c_p= 2089 \frac{Дж}{моль \cdot K}$, состав подаваемой смеси: $c_A=30.4 \frac{моль}{л}$, $c_B=0.2 \frac{моль}{л}$. Параметры реакций: энергии активации $E_{a1}=15.9 \frac{кДж}{моль}$, $E_{a2}=28.7  \frac{кДж}{моль}$, $E_{a3}=19.4  \frac{кДж}{моль}$, предэкспоненциальный множитель $k_{01}=       274$,$k_{02}=     21289$,$k_{03}=       693$, тепловой эффект $\Delta H_1=  6.1  \frac{кДж}{моль}$, $\Delta H_2=-16.1 \frac{кДж}{моль}$.\begin{itemize} \item Составить математическую модель изотермического реактора. Определить распределение концентрации компонентов по времени. Определить изменение конверсии по компоненту A, селективности и выхода по компоненту B. \end{itemize}

\textsc{\textbf{Вариант 4}}

 В реакторе идеального вытеснения протекает реакция: \begin{equation*} \begin{aligned} A \xrightarrow{k_1} B + \Delta H_1 \\ A \xrightarrow{k_2} C + \Delta H_2 \\ A \xrightarrow{k_3} D + \Delta H_3 \end{aligned} \end{equation*} На вход  реактор подается смесь при температуре $ T_{н} =  264 K$, теплоемкость смеси $c_p= 2023 \frac{Дж}{моль \cdot K}$, состав подаваемой смеси: $c_A=21.7 \frac{моль}{л}$, $c_B=0.3 \frac{моль}{л}$. Параметры реакций: энергии активации $E_{a1}=11.0 \frac{кДж}{моль}$, $E_{a2}=24.5  \frac{кДж}{моль}$, $E_{a3}=16.3  \frac{кДж}{моль}$, предэкспоненциальный множитель $k_{01}=        22$,$k_{02}=      3679$,$k_{03}=       197$, тепловой эффект $\Delta H_1= 13.0 \frac{кДж}{моль}$, $\Delta H_2=-22.6 \frac{кДж}{моль}$, $\Delta H_3 = 31.3 \frac{кДж}{моль}$\begin{itemize} \item Составить математическую модель изотермического реактора. Определить распределение концентрации компонентов по времени. Определить изменение конверсии по компоненту A, селективности и выхода по компоненту B. \end{itemize}

\textsc{\textbf{Вариант 5}}

 В реакторе идеального вытеснения протекает реакция: \begin{equation*} \begin{aligned} A \xleftrightarrow[k_2]{k_1} B + \Delta H_1 \\ B \xrightarrow{k_3} C + \Delta H_2 \end{aligned} \end{equation*}                              На вход  реактор подается смесь при температуре $ T_{н} =  208 K$, теплоемкость смеси $c_p= 2963 \frac{Дж}{моль \cdot K}$, состав подаваемой смеси: $c_A=18.7 \frac{моль}{л}$, $c_B=0.2 \frac{моль}{л}$. Параметры реакций: энергии активации $E_{a1}= 7.7 \frac{кДж}{моль}$, $E_{a2}=13.5  \frac{кДж}{моль}$, $E_{a3}=13.5  \frac{кДж}{моль}$, предэкспоненциальный множитель $k_{01}=        14$,$k_{02}=       141$,$k_{03}=       154$, тепловой эффект $\Delta H_1= 33.2  \frac{кДж}{моль}$, $\Delta H_2= 6.2 \frac{кДж}{моль}$.\begin{itemize} \item Составить математическую модель изотермического реактора. Определить распределение концентрации компонентов по времени. Определить изменение конверсии по компоненту A, селективности и выхода по компоненту B. \end{itemize}

\textsc{\textbf{Вариант 6}}

 В реакторе идеального вытеснения протекает реакция: \begin{equation*} \begin{aligned} A \xleftrightarrow[k_2]{k_1} C + \Delta H_1 \\ A \xrightarrow{k_3} B + \Delta H_2 \end{aligned} \end{equation*}                              На вход  реактор подается смесь при температуре $ T_{н} =  326 K$, теплоемкость смеси $c_p= 3352 \frac{Дж}{моль \cdot K}$, состав подаваемой смеси: $c_A=21.7 \frac{моль}{л}$, $c_B=0.3 \frac{моль}{л}$. Параметры реакций: энергии активации $E_{a1}=15.7 \frac{кДж}{моль}$, $E_{a2}=38.5  \frac{кДж}{моль}$, $E_{a3}=34.5  \frac{кДж}{моль}$, предэкспоненциальный множитель $k_{01}=        54$,$k_{02}=     58186$,$k_{03}=     17408$, тепловой эффект $\Delta H_1= -20.3  \frac{кДж}{моль}$, $\Delta H_2=21.6 \frac{кДж}{моль}$.\begin{itemize} \item Составить математическую модель изотермического реактора. Определить распределение концентрации компонентов по времени. Определить изменение конверсии по компоненту A, селективности и выхода по компоненту B. \end{itemize}

\textsc{\textbf{Вариант 7}}

 В реакторе идеального вытеснения протекает реакция: \begin{equation*} \begin{aligned} A \xleftrightarrow[k_2]{k_1} B + \Delta H_1 \\ B \xrightarrow{k_3} C + \Delta H_2 \end{aligned} \end{equation*}                              На вход  реактор подается смесь при температуре $ T_{н} =  266 K$, теплоемкость смеси $c_p= 3337 \frac{Дж}{моль \cdot K}$, состав подаваемой смеси: $c_A=32.2 \frac{моль}{л}$, $c_B=0.4 \frac{моль}{л}$. Параметры реакций: энергии активации $E_{a1}=14.1 \frac{кДж}{моль}$, $E_{a2}=21.9  \frac{кДж}{моль}$, $E_{a3}=23.2  \frac{кДж}{моль}$, предэкспоненциальный множитель $k_{01}=        83$,$k_{02}=       838$,$k_{03}=      2222$, тепловой эффект $\Delta H_1= -36.3  \frac{кДж}{моль}$, $\Delta H_2=-28.1 \frac{кДж}{моль}$.\begin{itemize} \item Составить математическую модель изотермического реактора. Определить распределение концентрации компонентов по времени. Определить изменение конверсии по компоненту A, селективности и выхода по компоненту B. \end{itemize}

\textsc{\textbf{Вариант 8}}

 В реакторе идеального вытеснения протекает реакция: \begin{equation*} \begin{aligned} A \xrightarrow{k_1} B + \Delta H_1 \\ A \xrightarrow{k_2} C + \Delta H_2 \\ A \xrightarrow{k_3} D + \Delta H_3 \end{aligned} \end{equation*} На вход  реактор подается смесь при температуре $ T_{н} =  202 K$, теплоемкость смеси $c_p= 3574 \frac{Дж}{моль \cdot K}$, состав подаваемой смеси: $c_A=27.9 \frac{моль}{л}$, $c_B=0.2 \frac{моль}{л}$. Параметры реакций: энергии активации $E_{a1}=10.5 \frac{кДж}{моль}$, $E_{a2}=15.8  \frac{кДж}{моль}$, $E_{a3}= 9.9  \frac{кДж}{моль}$, предэкспоненциальный множитель $k_{01}=        70$,$k_{02}=       575$,$k_{03}=        35$, тепловой эффект $\Delta H_1= -41.7 \frac{кДж}{моль}$, $\Delta H_2=-23.6 \frac{кДж}{моль}$, $\Delta H_3 = -44.8 \frac{кДж}{моль}$\begin{itemize} \item Составить математическую модель изотермического реактора. Определить распределение концентрации компонентов по времени. Определить изменение конверсии по компоненту A, селективности и выхода по компоненту B. \end{itemize}

\textsc{\textbf{Вариант 9}}

 В реакторе идеального вытеснения протекает реакция: \begin{equation*} \begin{aligned} A \xleftrightarrow[k_2]{k_1} C + \Delta H_1 \\ A \xrightarrow{k_3} B + \Delta H_2 \end{aligned} \end{equation*}                              На вход  реактор подается смесь при температуре $ T_{н} =  236 K$, теплоемкость смеси $c_p= 2093 \frac{Дж}{моль \cdot K}$, состав подаваемой смеси: $c_A=28.1 \frac{моль}{л}$, $c_B=0.3 \frac{моль}{л}$. Параметры реакций: энергии активации $E_{a1}=11.2 \frac{кДж}{моль}$, $E_{a2}=25.8  \frac{кДж}{моль}$, $E_{a3}=19.6  \frac{кДж}{моль}$, предэкспоненциальный множитель $k_{01}=        45$,$k_{02}=     10531$,$k_{03}=       749$, тепловой эффект $\Delta H_1= -11.9  \frac{кДж}{моль}$, $\Delta H_2=23.4 \frac{кДж}{моль}$.\begin{itemize} \item Составить математическую модель изотермического реактора. Определить распределение концентрации компонентов по времени. Определить изменение конверсии по компоненту A, селективности и выхода по компоненту B. \end{itemize}

\textsc{\textbf{Вариант 10}}

 В реакторе идеального вытеснения протекает реакция: \begin{equation*} \begin{aligned} A \xrightarrow{k_1} B + \Delta H_1 \\ A \xrightarrow{k_2} C + \Delta H_2 \\ A \xrightarrow{k_3} D + \Delta H_3 \end{aligned} \end{equation*} На вход  реактор подается смесь при температуре $ T_{н} =  204 K$, теплоемкость смеси $c_p= 2656 \frac{Дж}{моль \cdot K}$, состав подаваемой смеси: $c_A=26.7 \frac{моль}{л}$, $c_B=0.2 \frac{моль}{л}$. Параметры реакций: энергии активации $E_{a1}= 8.7 \frac{кДж}{моль}$, $E_{a2}=11.9  \frac{кДж}{моль}$, $E_{a3}=11.4  \frac{кДж}{моль}$, предэкспоненциальный множитель $k_{01}=        19$,$k_{02}=        66$,$k_{03}=        85$, тепловой эффект $\Delta H_1= -38.0 \frac{кДж}{моль}$, $\Delta H_2=40.3 \frac{кДж}{моль}$, $\Delta H_3 = 17.6 \frac{кДж}{моль}$\begin{itemize} \item Составить математическую модель изотермического реактора. Определить распределение концентрации компонентов по времени. Определить изменение конверсии по компоненту A, селективности и выхода по компоненту B. \end{itemize}

\textsc{\textbf{Вариант 11}}

 В реакторе идеального вытеснения протекает реакция: \begin{equation*} \begin{aligned} A+B \xleftrightarrow[k_2]{k_1} C + \Delta H_1 \\ B + C \xrightarrow{k_3} D + \Delta H_2 \end{aligned} \end{equation*}                        На вход  реактор подается смесь при температуре $ T_{н} =  313 K$, теплоемкость смеси $c_p= 3719 \frac{Дж}{моль \cdot K}$, состав подаваемой смеси: $c_A=34.4 \frac{моль}{л}$, $c_B=0.4 \frac{моль}{л}$. Параметры реакций: энергии активации $E_{a1}=17.7 \frac{кДж}{моль}$, $E_{a2}=34.3  \frac{кДж}{моль}$, $E_{a3}=21.5  \frac{кДж}{моль}$, предэкспоненциальный множитель $k_{01}=        80$,$k_{02}=     11434$,$k_{03}=       382$, тепловой эффект $\Delta H_1= -9.1  \frac{кДж}{моль}$, $\Delta H_2=-27.8 \frac{кДж}{моль}$.\begin{itemize} \item Составить математическую модель изотермического реактора. Определить распределение концентрации компонентов по времени. Определить изменение конверсии по компоненту A, селективности и выхода по компоненту B. \end{itemize}

\textsc{\textbf{Вариант 12}}

 В реакторе идеального вытеснения протекает реакция: \begin{equation*} \begin{aligned} A+B \xleftrightarrow[k_2]{k_1} C +\Delta H_1 \\ A \xrightarrow{k_3} B + \Delta H_2 \end{aligned} \end{equation*}                             На вход  реактор подается смесь при температуре $ T_{н} =  253 K$, теплоемкость смеси $c_p= 2991 \frac{Дж}{моль \cdot K}$, состав подаваемой смеси: $c_A=31.8 \frac{моль}{л}$, $c_B=0.3 \frac{моль}{л}$. Параметры реакций: энергии активации $E_{a1}=13.3 \frac{кДж}{моль}$, $E_{a2}=22.1  \frac{кДж}{моль}$, $E_{a3}=20.8  \frac{кДж}{моль}$, предэкспоненциальный множитель $k_{01}=        83$,$k_{02}=      2124$,$k_{03}=      1075$, тепловой эффект $\Delta H_1= -8.5  \frac{кДж}{моль}$, $\Delta H_2=-21.7 \frac{кДж}{моль}$.\begin{itemize} \item Составить математическую модель изотермического реактора. Определить распределение концентрации компонентов по времени. Определить изменение конверсии по компоненту A, селективности и выхода по компоненту B. \end{itemize}

\textsc{\textbf{Вариант 13}}

 В реакторе идеального вытеснения протекает реакция: \begin{equation*} \begin{aligned} A \xleftrightarrow[k_2]{k_1} B + \Delta H_1 \\ B \xrightarrow{k_3} C + \Delta H_2 \end{aligned} \end{equation*}                              На вход  реактор подается смесь при температуре $ T_{н} =  236 K$, теплоемкость смеси $c_p= 2179 \frac{Дж}{моль \cdot K}$, состав подаваемой смеси: $c_A=23.9 \frac{моль}{л}$, $c_B=0.3 \frac{моль}{л}$. Параметры реакций: энергии активации $E_{a1}=13.9 \frac{кДж}{моль}$, $E_{a2}=23.9  \frac{кДж}{моль}$, $E_{a3}=13.2  \frac{кДж}{моль}$, предэкспоненциальный множитель $k_{01}=       159$,$k_{02}=      4466$,$k_{03}=        85$, тепловой эффект $\Delta H_1= -6.7  \frac{кДж}{моль}$, $\Delta H_2=-16.6 \frac{кДж}{моль}$.\begin{itemize} \item Составить математическую модель изотермического реактора. Определить распределение концентрации компонентов по времени. Определить изменение конверсии по компоненту A, селективности и выхода по компоненту B. \end{itemize}

\textsc{\textbf{Вариант 14}}

 В реакторе идеального вытеснения протекает реакция: \begin{equation*} \begin{aligned} A \xleftrightarrow[k_2]{k_1} C + \Delta H_1 \\ A \xrightarrow{k_3} B + \Delta H_2 \end{aligned} \end{equation*}                              На вход  реактор подается смесь при температуре $ T_{н} =  226 K$, теплоемкость смеси $c_p= 3289 \frac{Дж}{моль \cdot K}$, состав подаваемой смеси: $c_A=28.4 \frac{моль}{л}$, $c_B=0.3 \frac{моль}{л}$. Параметры реакций: энергии активации $E_{a1}= 7.8 \frac{кДж}{моль}$, $E_{a2}=13.1  \frac{кДж}{моль}$, $E_{a3}=12.0  \frac{кДж}{моль}$, предэкспоненциальный множитель $k_{01}=         5$,$k_{02}=        41$,$k_{03}=        42$, тепловой эффект $\Delta H_1= 41.8  \frac{кДж}{моль}$, $\Delta H_2=-29.3 \frac{кДж}{моль}$.\begin{itemize} \item Составить математическую модель изотермического реактора. Определить распределение концентрации компонентов по времени. Определить изменение конверсии по компоненту A, селективности и выхода по компоненту B. \end{itemize}

\textsc{\textbf{Вариант 15}}

 В реакторе идеального вытеснения протекает реакция: \begin{equation*} \begin{aligned} A \xleftrightarrow[k_2]{k_1} B + \Delta H_1 \\ B \xrightarrow{k_3} C + \Delta H_2 \end{aligned} \end{equation*}                              На вход  реактор подается смесь при температуре $ T_{н} =  207 K$, теплоемкость смеси $c_p= 2827 \frac{Дж}{моль \cdot K}$, состав подаваемой смеси: $c_A=15.5 \frac{моль}{л}$, $c_B=0.2 \frac{моль}{л}$. Параметры реакций: энергии активации $E_{a1}= 9.3 \frac{кДж}{моль}$, $E_{a2}=16.3  \frac{кДж}{моль}$, $E_{a3}=13.8  \frac{кДж}{моль}$, предэкспоненциальный множитель $k_{01}=        33$,$k_{02}=       681$,$k_{03}=       205$, тепловой эффект $\Delta H_1= 24.0  \frac{кДж}{моль}$, $\Delta H_2=-18.8 \frac{кДж}{моль}$.\begin{itemize} \item Составить математическую модель изотермического реактора. Определить распределение концентрации компонентов по времени. Определить изменение конверсии по компоненту A, селективности и выхода по компоненту B. \end{itemize}

\textsc{\textbf{Вариант 16}}

 В реакторе идеального вытеснения протекает реакция: \begin{equation*} \begin{aligned} A \xleftrightarrow[k_2]{k_1} B + \Delta H_1 \\ B \xrightarrow{k_3} C + \Delta H_2 \end{aligned} \end{equation*}                              На вход  реактор подается смесь при температуре $ T_{н} =  279 K$, теплоемкость смеси $c_p= 3295 \frac{Дж}{моль \cdot K}$, состав подаваемой смеси: $c_A=22.6 \frac{моль}{л}$, $c_B=0.2 \frac{моль}{л}$. Параметры реакций: энергии активации $E_{a1}=11.1 \frac{кДж}{моль}$, $E_{a2}=28.9  \frac{кДж}{моль}$, $E_{a3}= 9.6  \frac{кДж}{моль}$, предэкспоненциальный множитель $k_{01}=         9$,$k_{02}=      5061$,$k_{03}=         6$, тепловой эффект $\Delta H_1= -28.3  \frac{кДж}{моль}$, $\Delta H_2=-12.2 \frac{кДж}{моль}$.\begin{itemize} \item Составить математическую модель изотермического реактора. Определить распределение концентрации компонентов по времени. Определить изменение конверсии по компоненту A, селективности и выхода по компоненту B. \end{itemize}

\textsc{\textbf{Вариант 17}}

 В реакторе идеального вытеснения протекает реакция: \begin{equation*} \begin{aligned} A \xleftrightarrow[k_2]{k_1} C + \Delta H_1 \\ A \xrightarrow{k_3} B + \Delta H_2 \end{aligned} \end{equation*}                              На вход  реактор подается смесь при температуре $ T_{н} =  323 K$, теплоемкость смеси $c_p= 3844 \frac{Дж}{моль \cdot K}$, состав подаваемой смеси: $c_A=16.1 \frac{моль}{л}$, $c_B=0.2 \frac{моль}{л}$. Параметры реакций: энергии активации $E_{a1}=23.5 \frac{кДж}{моль}$, $E_{a2}=32.6  \frac{кДж}{моль}$, $E_{a3}=18.5  \frac{кДж}{моль}$, предэкспоненциальный множитель $k_{01}=       608$,$k_{02}=      7186$,$k_{03}=        88$, тепловой эффект $\Delta H_1= -15.2  \frac{кДж}{моль}$, $\Delta H_2=34.6 \frac{кДж}{моль}$.\begin{itemize} \item Составить математическую модель изотермического реактора. Определить распределение концентрации компонентов по времени. Определить изменение конверсии по компоненту A, селективности и выхода по компоненту B. \end{itemize}

\textsc{\textbf{Вариант 18}}

 В реакторе идеального вытеснения протекает реакция: \begin{equation*} \begin{aligned} A+B \xleftrightarrow[k_2]{k_1} C + \Delta H_1 \\ B + C \xrightarrow{k_3} D + \Delta H_2 \end{aligned} \end{equation*}                        На вход  реактор подается смесь при температуре $ T_{н} =  238 K$, теплоемкость смеси $c_p= 3292 \frac{Дж}{моль \cdot K}$, состав подаваемой смеси: $c_A=30.7 \frac{моль}{л}$, $c_B=0.4 \frac{моль}{л}$. Параметры реакций: энергии активации $E_{a1}=13.2 \frac{кДж}{моль}$, $E_{a2}=22.8  \frac{кДж}{моль}$, $E_{a3}=13.8  \frac{кДж}{моль}$, предэкспоненциальный множитель $k_{01}=       124$,$k_{02}=      2583$,$k_{03}=       104$, тепловой эффект $\Delta H_1= 13.2  \frac{кДж}{моль}$, $\Delta H_2=-23.5 \frac{кДж}{моль}$.\begin{itemize} \item Составить математическую модель изотермического реактора. Определить распределение концентрации компонентов по времени. Определить изменение конверсии по компоненту A, селективности и выхода по компоненту B. \end{itemize}

\textsc{\textbf{Вариант 19}}

 В реакторе идеального вытеснения протекает реакция: \begin{equation*} \begin{aligned} A+B \xleftrightarrow[k_2]{k_1} C +\Delta H_1 \\ A \xrightarrow{k_3} B + \Delta H_2 \end{aligned} \end{equation*}                             На вход  реактор подается смесь при температуре $ T_{н} =  318 K$, теплоемкость смеси $c_p= 2916 \frac{Дж}{моль \cdot K}$, состав подаваемой смеси: $c_A=20.1 \frac{моль}{л}$, $c_B=0.3 \frac{моль}{л}$. Параметры реакций: энергии активации $E_{a1}=15.7 \frac{кДж}{моль}$, $E_{a2}=24.5  \frac{кДж}{моль}$, $E_{a3}=11.9  \frac{кДж}{моль}$, предэкспоненциальный множитель $k_{01}=        25$,$k_{02}=       384$,$k_{03}=        10$, тепловой эффект $\Delta H_1= -12.2  \frac{кДж}{моль}$, $\Delta H_2=22.1 \frac{кДж}{моль}$.\begin{itemize} \item Составить математическую модель изотермического реактора. Определить распределение концентрации компонентов по времени. Определить изменение конверсии по компоненту A, селективности и выхода по компоненту B. \end{itemize}

\textsc{\textbf{Вариант 20}}

 В реакторе идеального вытеснения протекает реакция: \begin{equation*} \begin{aligned} A+B \xleftrightarrow[k_2]{k_1} C + \Delta H_1 \\ B + C \xrightarrow{k_3} D + \Delta H_2 \end{aligned} \end{equation*}                        На вход  реактор подается смесь при температуре $ T_{н} =  358 K$, теплоемкость смеси $c_p= 3612 \frac{Дж}{моль \cdot K}$, состав подаваемой смеси: $c_A=33.8 \frac{моль}{л}$, $c_B=0.3 \frac{моль}{л}$. Параметры реакций: энергии активации $E_{a1}=28.7 \frac{кДж}{моль}$, $E_{a2}=39.6  \frac{кДж}{моль}$, $E_{a3}=18.6  \frac{кДж}{моль}$, предэкспоненциальный множитель $k_{01}=      1699$,$k_{02}=     18883$,$k_{03}=        58$, тепловой эффект $\Delta H_1= -38.7  \frac{кДж}{моль}$, $\Delta H_2=12.1 \frac{кДж}{моль}$.\begin{itemize} \item Составить математическую модель изотермического реактора. Определить распределение концентрации компонентов по времени. Определить изменение конверсии по компоненту A, селективности и выхода по компоненту B. \end{itemize}

\textsc{\textbf{Вариант 21}}

 В реакторе идеального вытеснения протекает реакция: \begin{equation*} \begin{aligned} A \xrightarrow{k_1} B + \Delta H_1 \\ A \xrightarrow{k_2} C + \Delta H_2 \\ A \xrightarrow{k_3} D + \Delta H_3 \end{aligned} \end{equation*} На вход  реактор подается смесь при температуре $ T_{н} =  202 K$, теплоемкость смеси $c_p= 3231 \frac{Дж}{моль \cdot K}$, состав подаваемой смеси: $c_A=19.7 \frac{моль}{л}$, $c_B=0.4 \frac{моль}{л}$. Параметры реакций: энергии активации $E_{a1}=10.4 \frac{кДж}{моль}$, $E_{a2}=12.9  \frac{кДж}{моль}$, $E_{a3}= 7.4  \frac{кДж}{моль}$, предэкспоненциальный множитель $k_{01}=        53$,$k_{02}=        85$,$k_{03}=         9$, тепловой эффект $\Delta H_1= 43.7 \frac{кДж}{моль}$, $\Delta H_2=33.5 \frac{кДж}{моль}$, $\Delta H_3 = 16.5 \frac{кДж}{моль}$\begin{itemize} \item Составить математическую модель изотермического реактора. Определить распределение концентрации компонентов по времени. Определить изменение конверсии по компоненту A, селективности и выхода по компоненту B. \end{itemize}

\textsc{\textbf{Вариант 22}}

 В реакторе идеального вытеснения протекает реакция: \begin{equation*} \begin{aligned} A+B \xleftrightarrow[k_2]{k_1} C +\Delta H_1 \\ A \xrightarrow{k_3} B + \Delta H_2 \end{aligned} \end{equation*}                             На вход  реактор подается смесь при температуре $ T_{н} =  331 K$, теплоемкость смеси $c_p= 2352 \frac{Дж}{моль \cdot K}$, состав подаваемой смеси: $c_A=34.7 \frac{моль}{л}$, $c_B=0.3 \frac{моль}{л}$. Параметры реакций: энергии активации $E_{a1}=17.7 \frac{кДж}{моль}$, $E_{a2}=26.2  \frac{кДж}{моль}$, $E_{a3}=17.4  \frac{кДж}{моль}$, предэкспоненциальный множитель $k_{01}=        55$,$k_{02}=       544$,$k_{03}=        57$, тепловой эффект $\Delta H_1= 11.9  \frac{кДж}{моль}$, $\Delta H_2= 8.6 \frac{кДж}{моль}$.\begin{itemize} \item Составить математическую модель изотермического реактора. Определить распределение концентрации компонентов по времени. Определить изменение конверсии по компоненту A, селективности и выхода по компоненту B. \end{itemize}

\textsc{\textbf{Вариант 23}}

 В реакторе идеального вытеснения протекает реакция: \begin{equation*} \begin{aligned} A+B \xleftrightarrow[k_2]{k_1} C + \Delta H_1 \\ B + C \xrightarrow{k_3} D + \Delta H_2 \end{aligned} \end{equation*}                        На вход  реактор подается смесь при температуре $ T_{н} =  335 K$, теплоемкость смеси $c_p= 2891 \frac{Дж}{моль \cdot K}$, состав подаваемой смеси: $c_A=18.1 \frac{моль}{л}$, $c_B=0.2 \frac{моль}{л}$. Параметры реакций: энергии активации $E_{a1}=21.0 \frac{кДж}{моль}$, $E_{a2}=38.1  \frac{кДж}{моль}$, $E_{a3}=35.8  \frac{кДж}{моль}$, предэкспоненциальный множитель $k_{01}=       254$,$k_{02}=     35138$,$k_{03}=     22363$, тепловой эффект $\Delta H_1= 42.5  \frac{кДж}{моль}$, $\Delta H_2=-44.7 \frac{кДж}{моль}$.\begin{itemize} \item Составить математическую модель изотермического реактора. Определить распределение концентрации компонентов по времени. Определить изменение конверсии по компоненту A, селективности и выхода по компоненту B. \end{itemize}

\textsc{\textbf{Вариант 24}}

 В реакторе идеального вытеснения протекает реакция: \begin{equation*} \begin{aligned} A \xleftrightarrow[k_2]{k_1} B + \Delta H_1 \\ B \xrightarrow{k_3} C + \Delta H_2 \end{aligned} \end{equation*}                              На вход  реактор подается смесь при температуре $ T_{н} =  342 K$, теплоемкость смеси $c_p= 3010 \frac{Дж}{моль \cdot K}$, состав подаваемой смеси: $c_A=19.3 \frac{моль}{л}$, $c_B=0.3 \frac{моль}{л}$. Параметры реакций: энергии активации $E_{a1}=24.4 \frac{кДж}{моль}$, $E_{a2}=27.8  \frac{кДж}{моль}$, $E_{a3}=26.2  \frac{кДж}{моль}$, предэкспоненциальный множитель $k_{01}=       314$,$k_{02}=       723$,$k_{03}=       484$, тепловой эффект $\Delta H_1= -28.8  \frac{кДж}{моль}$, $\Delta H_2=-9.6 \frac{кДж}{моль}$.\begin{itemize} \item Составить математическую модель изотермического реактора. Определить распределение концентрации компонентов по времени. Определить изменение конверсии по компоненту A, селективности и выхода по компоненту B. \end{itemize}

\textsc{\textbf{Вариант 25}}

 В реакторе идеального вытеснения протекает реакция: \begin{equation*} \begin{aligned} A+B \xleftrightarrow[k_2]{k_1} C +\Delta H_1 \\ A \xrightarrow{k_3} B + \Delta H_2 \end{aligned} \end{equation*}                             На вход  реактор подается смесь при температуре $ T_{н} =  360 K$, теплоемкость смеси $c_p= 3610 \frac{Дж}{моль \cdot K}$, состав подаваемой смеси: $c_A=27.4 \frac{моль}{л}$, $c_B=0.2 \frac{моль}{л}$. Параметры реакций: энергии активации $E_{a1}=29.0 \frac{кДж}{моль}$, $E_{a2}=49.7  \frac{кДж}{моль}$, $E_{a3}=42.6  \frac{кДж}{моль}$, предэкспоненциальный множитель $k_{01}=      2447$,$k_{02}=    559504$,$k_{03}=     71440$, тепловой эффект $\Delta H_1= -26.0  \frac{кДж}{моль}$, $\Delta H_2=28.8 \frac{кДж}{моль}$.\begin{itemize} \item Составить математическую модель изотермического реактора. Определить распределение концентрации компонентов по времени. Определить изменение конверсии по компоненту A, селективности и выхода по компоненту B. \end{itemize}

\textsc{\textbf{Вариант 26}}

 В реакторе идеального вытеснения протекает реакция: \begin{equation*} \begin{aligned} A \xleftrightarrow[k_2]{k_1} C + \Delta H_1 \\ A \xrightarrow{k_3} B + \Delta H_2 \end{aligned} \end{equation*}                              На вход  реактор подается смесь при температуре $ T_{н} =  219 K$, теплоемкость смеси $c_p= 2303 \frac{Дж}{моль \cdot K}$, состав подаваемой смеси: $c_A=30.7 \frac{моль}{л}$, $c_B=0.3 \frac{моль}{л}$. Параметры реакций: энергии активации $E_{a1}=10.1 \frac{кДж}{моль}$, $E_{a2}=18.1  \frac{кДж}{моль}$, $E_{a3}=11.3  \frac{кДж}{моль}$, предэкспоненциальный множитель $k_{01}=        43$,$k_{02}=       813$,$k_{03}=        48$, тепловой эффект $\Delta H_1= -16.7  \frac{кДж}{моль}$, $\Delta H_2=-12.3 \frac{кДж}{моль}$.\begin{itemize} \item Составить математическую модель изотермического реактора. Определить распределение концентрации компонентов по времени. Определить изменение конверсии по компоненту A, селективности и выхода по компоненту B. \end{itemize}

\textsc{\textbf{Вариант 27}}

 В реакторе идеального вытеснения протекает реакция: \begin{equation*} \begin{aligned} A \xleftrightarrow[k_2]{k_1} C + \Delta H_1 \\ A \xrightarrow{k_3} B + \Delta H_2 \end{aligned} \end{equation*}                              На вход  реактор подается смесь при температуре $ T_{н} =  355 K$, теплоемкость смеси $c_p= 3120 \frac{Дж}{моль \cdot K}$, состав подаваемой смеси: $c_A=19.1 \frac{моль}{л}$, $c_B=0.3 \frac{моль}{л}$. Параметры реакций: энергии активации $E_{a1}=28.0 \frac{кДж}{моль}$, $E_{a2}=36.2  \frac{кДж}{моль}$, $E_{a3}=34.4  \frac{кДж}{моль}$, предэкспоненциальный множитель $k_{01}=      2044$,$k_{02}=      8253$,$k_{03}=      5195$, тепловой эффект $\Delta H_1= 29.3  \frac{кДж}{моль}$, $\Delta H_2=-21.1 \frac{кДж}{моль}$.\begin{itemize} \item Составить математическую модель изотермического реактора. Определить распределение концентрации компонентов по времени. Определить изменение конверсии по компоненту A, селективности и выхода по компоненту B. \end{itemize}

\textsc{\textbf{Вариант 28}}

 В реакторе идеального вытеснения протекает реакция: \begin{equation*} \begin{aligned} A \xrightarrow{k_1} B + \Delta H_1 \\ A \xrightarrow{k_2} C + \Delta H_2 \\ A \xrightarrow{k_3} D + \Delta H_3 \end{aligned} \end{equation*} На вход  реактор подается смесь при температуре $ T_{н} =  291 K$, теплоемкость смеси $c_p= 2519 \frac{Дж}{моль \cdot K}$, состав подаваемой смеси: $c_A=26.1 \frac{моль}{л}$, $c_B=0.2 \frac{моль}{л}$. Параметры реакций: энергии активации $E_{a1}=18.5 \frac{кДж}{моль}$, $E_{a2}=31.8  \frac{кДж}{моль}$, $E_{a3}=21.1  \frac{кДж}{моль}$, предэкспоненциальный множитель $k_{01}=       337$,$k_{02}=     13592$,$k_{03}=       430$, тепловой эффект $\Delta H_1= -15.7 \frac{кДж}{моль}$, $\Delta H_2=43.1 \frac{кДж}{моль}$, $\Delta H_3 = 43.5 \frac{кДж}{моль}$\begin{itemize} \item Составить математическую модель изотермического реактора. Определить распределение концентрации компонентов по времени. Определить изменение конверсии по компоненту A, селективности и выхода по компоненту B. \end{itemize}

\textsc{\textbf{Вариант 29}}

 В реакторе идеального вытеснения протекает реакция: \begin{equation*} \begin{aligned} A \xleftrightarrow[k_2]{k_1} B + \Delta H_1 \\ B \xrightarrow{k_3} C + \Delta H_2 \end{aligned} \end{equation*}                              На вход  реактор подается смесь при температуре $ T_{н} =  347 K$, теплоемкость смеси $c_p= 2537 \frac{Дж}{моль \cdot K}$, состав подаваемой смеси: $c_A=33.0 \frac{моль}{л}$, $c_B=0.3 \frac{моль}{л}$. Параметры реакций: энергии активации $E_{a1}=25.6 \frac{кДж}{моль}$, $E_{a2}=37.8  \frac{кДж}{моль}$, $E_{a3}=27.3  \frac{кДж}{моль}$, предэкспоненциальный множитель $k_{01}=       753$,$k_{02}=     20313$,$k_{03}=       907$, тепловой эффект $\Delta H_1= -29.6  \frac{кДж}{моль}$, $\Delta H_2=-32.9 \frac{кДж}{моль}$.\begin{itemize} \item Составить математическую модель изотермического реактора. Определить распределение концентрации компонентов по времени. Определить изменение конверсии по компоненту A, селективности и выхода по компоненту B. \end{itemize}

\textsc{\textbf{Вариант 30}}

 В реакторе идеального вытеснения протекает реакция: \begin{equation*} \begin{aligned} A+B \xleftrightarrow[k_2]{k_1} C + \Delta H_1 \\ B + C \xrightarrow{k_3} D + \Delta H_2 \end{aligned} \end{equation*}                        На вход  реактор подается смесь при температуре $ T_{н} =  302 K$, теплоемкость смеси $c_p= 3334 \frac{Дж}{моль \cdot K}$, состав подаваемой смеси: $c_A=28.9 \frac{моль}{л}$, $c_B=0.2 \frac{моль}{л}$. Параметры реакций: энергии активации $E_{a1}=14.7 \frac{кДж}{моль}$, $E_{a2}=19.6  \frac{кДж}{моль}$, $E_{a3}=21.5  \frac{кДж}{моль}$, предэкспоненциальный множитель $k_{01}=        25$,$k_{02}=       110$,$k_{03}=       322$, тепловой эффект $\Delta H_1= 36.9  \frac{кДж}{моль}$, $\Delta H_2=-11.8 \frac{кДж}{моль}$.\begin{itemize} \item Составить математическую модель изотермического реактора. Определить распределение концентрации компонентов по времени. Определить изменение конверсии по компоненту A, селективности и выхода по компоненту B. \end{itemize}

\textsc{\textbf{Вариант 31}}

 В реакторе идеального вытеснения протекает реакция: \begin{equation*} \begin{aligned} A \xleftrightarrow[k_2]{k_1} C + \Delta H_1 \\ A \xrightarrow{k_3} B + \Delta H_2 \end{aligned} \end{equation*}                              На вход  реактор подается смесь при температуре $ T_{н} =  219 K$, теплоемкость смеси $c_p= 2192 \frac{Дж}{моль \cdot K}$, состав подаваемой смеси: $c_A=28.0 \frac{моль}{л}$, $c_B=0.2 \frac{моль}{л}$. Параметры реакций: энергии активации $E_{a1}=12.0 \frac{кДж}{моль}$, $E_{a2}=16.0  \frac{кДж}{моль}$, $E_{a3}=12.7  \frac{кДж}{моль}$, предэкспоненциальный множитель $k_{01}=        67$,$k_{02}=       256$,$k_{03}=        70$, тепловой эффект $\Delta H_1= -27.4  \frac{кДж}{моль}$, $\Delta H_2=-39.8 \frac{кДж}{моль}$.\begin{itemize} \item Составить математическую модель изотермического реактора. Определить распределение концентрации компонентов по времени. Определить изменение конверсии по компоненту A, селективности и выхода по компоненту B. \end{itemize}

\textsc{\textbf{Вариант 32}}

 В реакторе идеального вытеснения протекает реакция: \begin{equation*} \begin{aligned} A \xrightarrow{k_1} B + \Delta H_1 \\ A \xrightarrow{k_2} C + \Delta H_2 \\ A \xrightarrow{k_3} D + \Delta H_3 \end{aligned} \end{equation*} На вход  реактор подается смесь при температуре $ T_{н} =  320 K$, теплоемкость смеси $c_p= 3473 \frac{Дж}{моль \cdot K}$, состав подаваемой смеси: $c_A=17.1 \frac{моль}{л}$, $c_B=0.4 \frac{моль}{л}$. Параметры реакций: энергии активации $E_{a1}=15.5 \frac{кДж}{моль}$, $E_{a2}=28.8  \frac{кДж}{моль}$, $E_{a3}=26.6  \frac{кДж}{моль}$, предэкспоненциальный множитель $k_{01}=        35$,$k_{02}=      2720$,$k_{03}=       808$, тепловой эффект $\Delta H_1= 26.3 \frac{кДж}{моль}$, $\Delta H_2=-43.5 \frac{кДж}{моль}$, $\Delta H_3 = -26.8 \frac{кДж}{моль}$\begin{itemize} \item Составить математическую модель изотермического реактора. Определить распределение концентрации компонентов по времени. Определить изменение конверсии по компоненту A, селективности и выхода по компоненту B. \end{itemize}

\textsc{\textbf{Вариант 33}}

 В реакторе идеального вытеснения протекает реакция: \begin{equation*} \begin{aligned} A \xleftrightarrow[k_2]{k_1} B + \Delta H_1 \\ B \xrightarrow{k_3} C + \Delta H_2 \end{aligned} \end{equation*}                              На вход  реактор подается смесь при температуре $ T_{н} =  309 K$, теплоемкость смеси $c_p= 3443 \frac{Дж}{моль \cdot K}$, состав подаваемой смеси: $c_A=28.6 \frac{моль}{л}$, $c_B=0.3 \frac{моль}{л}$. Параметры реакций: энергии активации $E_{a1}=21.5 \frac{кДж}{моль}$, $E_{a2}=39.5  \frac{кДж}{моль}$, $E_{a3}=24.7  \frac{кДж}{моль}$, предэкспоненциальный множитель $k_{01}=       705$,$k_{02}=    156572$,$k_{03}=       956$, тепловой эффект $\Delta H_1= 27.9  \frac{кДж}{моль}$, $\Delta H_2=-18.5 \frac{кДж}{моль}$.\begin{itemize} \item Составить математическую модель изотермического реактора. Определить распределение концентрации компонентов по времени. Определить изменение конверсии по компоненту A, селективности и выхода по компоненту B. \end{itemize}

\textsc{\textbf{Вариант 34}}

 В реакторе идеального вытеснения протекает реакция: \begin{equation*} \begin{aligned} A \xrightarrow{k_1} B + \Delta H_1 \\ A \xrightarrow{k_2} C + \Delta H_2 \\ A \xrightarrow{k_3} D + \Delta H_3 \end{aligned} \end{equation*} На вход  реактор подается смесь при температуре $ T_{н} =  377 K$, теплоемкость смеси $c_p= 3778 \frac{Дж}{моль \cdot K}$, состав подаваемой смеси: $c_A=15.2 \frac{моль}{л}$, $c_B=0.4 \frac{моль}{л}$. Параметры реакций: энергии активации $E_{a1}=18.9 \frac{кДж}{моль}$, $E_{a2}=53.1  \frac{кДж}{моль}$, $E_{a3}=27.8  \frac{кДж}{моль}$, предэкспоненциальный множитель $k_{01}=        60$,$k_{02}=    536824$,$k_{03}=       575$, тепловой эффект $\Delta H_1= -42.4 \frac{кДж}{моль}$, $\Delta H_2=-40.1 \frac{кДж}{моль}$, $\Delta H_3 = -36.9 \frac{кДж}{моль}$\begin{itemize} \item Составить математическую модель изотермического реактора. Определить распределение концентрации компонентов по времени. Определить изменение конверсии по компоненту A, селективности и выхода по компоненту B. \end{itemize}

\textsc{\textbf{Вариант 35}}

 В реакторе идеального вытеснения протекает реакция: \begin{equation*} \begin{aligned} A+B \xleftrightarrow[k_2]{k_1} C +\Delta H_1 \\ A \xrightarrow{k_3} B + \Delta H_2 \end{aligned} \end{equation*}                             На вход  реактор подается смесь при температуре $ T_{н} =  360 K$, теплоемкость смеси $c_p= 3069 \frac{Дж}{моль \cdot K}$, состав подаваемой смеси: $c_A=28.5 \frac{моль}{л}$, $c_B=0.4 \frac{моль}{л}$. Параметры реакций: энергии активации $E_{a1}=24.4 \frac{кДж}{моль}$, $E_{a2}=38.0  \frac{кДж}{моль}$, $E_{a3}=23.5  \frac{кДж}{моль}$, предэкспоненциальный множитель $k_{01}=       251$,$k_{02}=      9975$,$k_{03}=       267$, тепловой эффект $\Delta H_1= -25.7  \frac{кДж}{моль}$, $\Delta H_2=-16.4 \frac{кДж}{моль}$.\begin{itemize} \item Составить математическую модель изотермического реактора. Определить распределение концентрации компонентов по времени. Определить изменение конверсии по компоненту A, селективности и выхода по компоненту B. \end{itemize}

\textsc{\textbf{Вариант 36}}

 В реакторе идеального вытеснения протекает реакция: \begin{equation*} \begin{aligned} A+B \xleftrightarrow[k_2]{k_1} C +\Delta H_1 \\ A \xrightarrow{k_3} B + \Delta H_2 \end{aligned} \end{equation*}                             На вход  реактор подается смесь при температуре $ T_{н} =  344 K$, теплоемкость смеси $c_p= 3458 \frac{Дж}{моль \cdot K}$, состав подаваемой смеси: $c_A=25.1 \frac{моль}{л}$, $c_B=0.4 \frac{моль}{л}$. Параметры реакций: энергии активации $E_{a1}=25.3 \frac{кДж}{моль}$, $E_{a2}=37.0  \frac{кДж}{моль}$, $E_{a3}=36.4  \frac{кДж}{моль}$, предэкспоненциальный множитель $k_{01}=      1039$,$k_{02}=     15969$,$k_{03}=     18043$, тепловой эффект $\Delta H_1= 24.5  \frac{кДж}{моль}$, $\Delta H_2=-23.8 \frac{кДж}{моль}$.\begin{itemize} \item Составить математическую модель изотермического реактора. Определить распределение концентрации компонентов по времени. Определить изменение конверсии по компоненту A, селективности и выхода по компоненту B. \end{itemize}

\textsc{\textbf{Вариант 37}}

 В реакторе идеального вытеснения протекает реакция: \begin{equation*} \begin{aligned} A \xleftrightarrow[k_2]{k_1} B + \Delta H_1 \\ B \xrightarrow{k_3} C + \Delta H_2 \end{aligned} \end{equation*}                              На вход  реактор подается смесь при температуре $ T_{н} =  283 K$, теплоемкость смеси $c_p= 2562 \frac{Дж}{моль \cdot K}$, состав подаваемой смеси: $c_A=27.8 \frac{моль}{л}$, $c_B=0.3 \frac{моль}{л}$. Параметры реакций: энергии активации $E_{a1}=13.7 \frac{кДж}{моль}$, $E_{a2}=25.7  \frac{кДж}{моль}$, $E_{a3}=12.1  \frac{кДж}{моль}$, предэкспоненциальный множитель $k_{01}=        32$,$k_{02}=      2471$,$k_{03}=        18$, тепловой эффект $\Delta H_1= 30.9  \frac{кДж}{моль}$, $\Delta H_2=24.5 \frac{кДж}{моль}$.\begin{itemize} \item Составить математическую модель изотермического реактора. Определить распределение концентрации компонентов по времени. Определить изменение конверсии по компоненту A, селективности и выхода по компоненту B. \end{itemize}

\textsc{\textbf{Вариант 38}}

 В реакторе идеального вытеснения протекает реакция: \begin{equation*} \begin{aligned} A \xleftrightarrow[k_2]{k_1} B + \Delta H_1 \\ B \xrightarrow{k_3} C + \Delta H_2 \end{aligned} \end{equation*}                              На вход  реактор подается смесь при температуре $ T_{н} =  364 K$, теплоемкость смеси $c_p= 3060 \frac{Дж}{моль \cdot K}$, состав подаваемой смеси: $c_A=30.4 \frac{моль}{л}$, $c_B=0.3 \frac{моль}{л}$. Параметры реакций: энергии активации $E_{a1}=21.6 \frac{кДж}{моль}$, $E_{a2}=48.0  \frac{кДж}{моль}$, $E_{a3}=42.6  \frac{кДж}{моль}$, предэкспоненциальный множитель $k_{01}=       212$,$k_{02}=    295324$,$k_{03}=     84701$, тепловой эффект $\Delta H_1= -20.4  \frac{кДж}{моль}$, $\Delta H_2=-7.6 \frac{кДж}{моль}$.\begin{itemize} \item Составить математическую модель изотермического реактора. Определить распределение концентрации компонентов по времени. Определить изменение конверсии по компоненту A, селективности и выхода по компоненту B. \end{itemize}

\textsc{\textbf{Вариант 39}}

 В реакторе идеального вытеснения протекает реакция: \begin{equation*} \begin{aligned} A+B \xleftrightarrow[k_2]{k_1} C + \Delta H_1 \\ B + C \xrightarrow{k_3} D + \Delta H_2 \end{aligned} \end{equation*}                        На вход  реактор подается смесь при температуре $ T_{н} =  235 K$, теплоемкость смеси $c_p= 3223 \frac{Дж}{моль \cdot K}$, состав подаваемой смеси: $c_A=22.5 \frac{моль}{л}$, $c_B=0.2 \frac{моль}{л}$. Параметры реакций: энергии активации $E_{a1}=13.2 \frac{кДж}{моль}$, $E_{a2}=17.0  \frac{кДж}{моль}$, $E_{a3}=19.1  \frac{кДж}{моль}$, предэкспоненциальный множитель $k_{01}=        80$,$k_{02}=       215$,$k_{03}=       748$, тепловой эффект $\Delta H_1= 32.1  \frac{кДж}{моль}$, $\Delta H_2=41.0 \frac{кДж}{моль}$.\begin{itemize} \item Составить математическую модель изотермического реактора. Определить распределение концентрации компонентов по времени. Определить изменение конверсии по компоненту A, селективности и выхода по компоненту B. \end{itemize}

\textsc{\textbf{Вариант 40}}

 В реакторе идеального вытеснения протекает реакция: \begin{equation*} \begin{aligned} A \xleftrightarrow[k_2]{k_1} C + \Delta H_1 \\ A \xrightarrow{k_3} B + \Delta H_2 \end{aligned} \end{equation*}                              На вход  реактор подается смесь при температуре $ T_{н} =  365 K$, теплоемкость смеси $c_p= 3220 \frac{Дж}{моль \cdot K}$, состав подаваемой смеси: $c_A=30.9 \frac{моль}{л}$, $c_B=0.3 \frac{моль}{л}$. Параметры реакций: энергии активации $E_{a1}=20.2 \frac{кДж}{моль}$, $E_{a2}=30.8  \frac{кДж}{моль}$, $E_{a3}=19.2  \frac{кДж}{моль}$, предэкспоненциальный множитель $k_{01}=        47$,$k_{02}=       936$,$k_{03}=        61$, тепловой эффект $\Delta H_1= -22.6  \frac{кДж}{моль}$, $\Delta H_2=26.0 \frac{кДж}{моль}$.\begin{itemize} \item Составить математическую модель изотермического реактора. Определить распределение концентрации компонентов по времени. Определить изменение конверсии по компоненту A, селективности и выхода по компоненту B. \end{itemize}

