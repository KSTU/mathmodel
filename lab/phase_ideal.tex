\textsc{\textbf{Лабораторная работа №5 <<Определение условий фазовых равновесий пар~-- жидкость>>}}

\textsc{\textbf{Вариант 1}}
\begin{enumerate}
\item По экспериментальным данным (из справочника теплофизических свойств чистых веществ) получить описание давления паров чистого компонента от температуры. Для бензола использовать уравнение Риделя $ln(p_i^0(T))=A-\frac{B}{T}+C ln(T)+DT^2$, для дихлорэтана использовать уравнение Риделя $ln(p_i^0(T))=A-\frac{B}{T}+C ln(T)+DT^2$. На основе полученных уравнений и закона Рауля, для смеси бензол-дихлорэтан построить $T-x,y$ диаграмму равновесия пар-жидкость при давлении  760.0 мм.рт.ст. . Сравнить результаты полученные по модели с экспериментальными данными и сделать вывод о применимости модели.

\item Построить $p-x,y$ диаграмму при температуре   20.0 $^\circ$C. Сравнить результаты полученные по модели с экспериментальными данными и сделать вывод о применимости модели. \item Используя экспериментальные данные определить параметеры уравнения Ван~-- Лаара для описания коэффициента активности. Используя найденные значения параметров построить $T-x,y$ и $p-x,y$ диаграммы. Сравнить с результатом, полученным по уравнению Рауля.\end{enumerate}

\textsc{\textbf{Вариант 2}}
\begin{enumerate}
\item По экспериментальным данным (из справочника теплофизических свойств чистых веществ) получить описание давления паров чистого компонента от температуры. Для бензола использовать уравнение Миллера $ln(p_i^0(T))=A-\frac{B}{T}+C T+DT^3$  , для толуола использовать уравнение Ренкина $ln(p_i^0(T))=A-\frac{B}{T}+СT^2$     . На основе полученных уравнений и закона Рауля, для смеси бензол--толуол построить $T-x,y$ диаграмму равновесия пар-жидкость при давлении  760.0 мм.рт.ст. . Сравнить результаты полученные по модели с экспериментальными данными и сделать вывод о применимости модели.

\item Построить $p-x,y$ диаграмму при температуре  240.0 $^\circ$C. Сравнить результаты полученные по модели с экспериментальными данными и сделать вывод о применимости модели. \item Используя экспериментальные данные определить параметеры уравнения NRTL                 для описания коэффициента активности. Используя найденные значения параметров построить $T-x,y$ и $p-x,y$ диаграммы. Сравнить с результатом, полученным по уравнению Рауля.\end{enumerate}

\textsc{\textbf{Вариант 3}}
\begin{enumerate}
\item По экспериментальным данным (из справочника теплофизических свойств чистых веществ) получить описание давления паров чистого компонента от температуры. Для ацетона использовать уравнение Антуана $ln(p_i^0(T))=A-\frac{B}{T+C}$         , для бензола использовать уравнение Риделя $ln(p_i^0(T))=A-\frac{B}{T}+C ln(T)+DT^2$. На основе полученных уравнений и закона Рауля, для смеси ацетон-бензол построить $T-x,y$ диаграмму равновесия пар-жидкость при давлении  732.0 мм.рт.ст. . Сравнить результаты полученные по модели с экспериментальными данными и сделать вывод о применимости модели.

\item Построить $p-x,y$ диаграмму при температуре   25.0 $^\circ$C. Сравнить результаты полученные по модели с экспериментальными данными и сделать вывод о применимости модели. \item Используя экспериментальные данные определить параметеры уравнения Вильсона     для описания коэффициента активности. Используя найденные значения параметров построить $T-x,y$ и $p-x,y$ диаграммы. Сравнить с результатом, полученным по уравнению Рауля.\end{enumerate}

\textsc{\textbf{Вариант 4}}
\begin{enumerate}
\item По экспериментальным данным (из справочника теплофизических свойств чистых веществ) получить описание давления паров чистого компонента от температуры. Для ацетона использовать уравнение Клапейрона $ln(p_i^0(T))=A-\frac{B}{T}$     , для бутанола использовать уравнение Миллера $ln(p_i^0(T))=A-\frac{B}{T}+C T+DT^3$  . На основе полученных уравнений и закона Рауля, для смеси ацетон-бутанол построить $T-x,y$ диаграмму равновесия пар-жидкость при давлении  745.0 мм.рт.ст. . Сравнить результаты полученные по модели с экспериментальными данными и сделать вывод о применимости модели.

\item Построить $p-x,y$ диаграмму при температуре   25.0 $^\circ$C. Сравнить результаты полученные по модели с экспериментальными данными и сделать вывод о применимости модели. \item Используя экспериментальные данные определить параметеры уравнения Маргулеса   для описания коэффициента активности. Используя найденные значения параметров построить $T-x,y$ и $p-x,y$ диаграммы. Сравнить с результатом, полученным по уравнению Рауля.\end{enumerate}

\textsc{\textbf{Вариант 5}}
\begin{enumerate}
\item По экспериментальным данным (из справочника теплофизических свойств чистых веществ) получить описание давления паров чистого компонента от температуры. Для этанола использовать уравнение Риделя $ln(p_i^0(T))=A-\frac{B}{T}+C ln(T)+DT^2$, для бензола использовать уравнение Антуана $ln(p_i^0(T))=A-\frac{B}{T+C}$         . На основе полученных уравнений и закона Рауля, для смеси этанол-бензол построить $T-x,y$ диаграмму равновесия пар-жидкость при давлении  300.0 мм.рт.ст. . Сравнить результаты полученные по модели с экспериментальными данными и сделать вывод о применимости модели.

\item Построить $p-x,y$ диаграмму при температуре   50.0 $^\circ$C. Сравнить результаты полученные по модели с экспериментальными данными и сделать вывод о применимости модели. \item Используя экспериментальные данные определить параметеры уравнения Ван~-- Лаара для описания коэффициента активности. Используя найденные значения параметров построить $T-x,y$ и $p-x,y$ диаграммы. Сравнить с результатом, полученным по уравнению Рауля.\end{enumerate}

\textsc{\textbf{Вариант 6}}
\begin{enumerate}
\item По экспериментальным данным (из справочника теплофизических свойств чистых веществ) получить описание давления паров чистого компонента от температуры. Для ацетона использовать уравнение Антуана $ln(p_i^0(T))=A-\frac{B}{T+C}$         , для бутанола использовать уравнение Антуана $ln(p_i^0(T))=A-\frac{B}{T+C}$         . На основе полученных уравнений и закона Рауля, для смеси ацетон-бутанол построить $T-x,y$ диаграмму равновесия пар-жидкость при давлении  745.0 мм.рт.ст. . Сравнить результаты полученные по модели с экспериментальными данными и сделать вывод о применимости модели.

\item Построить $p-x,y$ диаграмму при температуре   25.0 $^\circ$C. Сравнить результаты полученные по модели с экспериментальными данными и сделать вывод о применимости модели. \item Используя экспериментальные данные определить параметеры уравнения Вильсона     для описания коэффициента активности. Используя найденные значения параметров построить $T-x,y$ и $p-x,y$ диаграммы. Сравнить с результатом, полученным по уравнению Рауля.\end{enumerate}

\textsc{\textbf{Вариант 7}}
\begin{enumerate}
\item По экспериментальным данным (из справочника теплофизических свойств чистых веществ) получить описание давления паров чистого компонента от температуры. Для циклогексана использовать уравнение Клапейрона $ln(p_i^0(T))=A-\frac{B}{T}$     , для этанола использовать уравнение Риделя $ln(p_i^0(T))=A-\frac{B}{T}+C ln(T)+DT^2$. На основе полученных уравнений и закона Рауля, для смеси циклогексан -- этанол построить $T-x,y$ диаграмму равновесия пар-жидкость при давлении  760.0 мм.рт.ст. . Сравнить результаты полученные по модели с экспериментальными данными и сделать вывод о применимости модели.

\item Построить $p-x,y$ диаграмму при температуре   20.0 $^\circ$C. Сравнить результаты полученные по модели с экспериментальными данными и сделать вывод о применимости модели. \item Используя экспериментальные данные определить параметеры уравнения Маргулеса   для описания коэффициента активности. Используя найденные значения параметров построить $T-x,y$ и $p-x,y$ диаграммы. Сравнить с результатом, полученным по уравнению Рауля.\end{enumerate}

\textsc{\textbf{Вариант 8}}
\begin{enumerate}
\item По экспериментальным данным (из справочника теплофизических свойств чистых веществ) получить описание давления паров чистого компонента от температуры. Для ацетона использовать уравнение Миллера $ln(p_i^0(T))=A-\frac{B}{T}+C T+DT^3$  , для этанола использовать уравнение Ренкина $ln(p_i^0(T))=A-\frac{B}{T}+СT^2$     . На основе полученных уравнений и закона Рауля, для смеси ацетон--этанол построить $T-x,y$ диаграмму равновесия пар-жидкость при давлении  760.0 мм.рт.ст. . Сравнить результаты полученные по модели с экспериментальными данными и сделать вывод о применимости модели.

\item Построить $p-x,y$ диаграмму при температуре   48.0 $^\circ$C. Сравнить результаты полученные по модели с экспериментальными данными и сделать вывод о применимости модели. \item Используя экспериментальные данные определить параметеры уравнения NRTL                 для описания коэффициента активности. Используя найденные значения параметров построить $T-x,y$ и $p-x,y$ диаграммы. Сравнить с результатом, полученным по уравнению Рауля.\end{enumerate}

