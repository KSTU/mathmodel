\textsc{\textbf{Лабораторная работа №1 }}

\textsc{\textbf{Вариант 1}}

\begin{enumerate}
\item Вычислить: 
\begin{equation*}\sin \left( \dfrac{\pi}{7} \right)\end{equation*}
\begin{equation*}\ln(3+sin(4))                     \end{equation*}
\begin{equation*}\sin \left( \dfrac{\pi}{7} \right)\end{equation*}

\item Вычислить аналитически: 
 \begin{equation*} \int \cos(x^2)          \end{equation*}\begin{equation*} {\dfrac{\partial \sin(5 x +3)}{\partial x}} \end{equation*}
\item Построить график функции y в диапазоне от $x=0.1$ до $x=9.1$, определить, при каком значениии $x$ $y=1.0$. На этом же графике построить функцию $z $. Определить координаты точки пересечения графиков. \item Определить сумму, произведение матриц $A=\begin{bmatrix}
3.3 &0.7 &5 \\
0.7 &0.9 &8.1 \\
1.5 &4.4 &2.9 \\
\end{bmatrix}
$ и $B=\begin{bmatrix}
0.5 &5.4 &6.1 \\
1.9 &1.8 &2.1 \\
7.7 &2.8 &5 \\
\end{bmatrix}
$. Вычислить $D_{i,j}=A_{i,j} + B_{i,j}$ и определитель матрицы D
\end{enumerate}
\textsc{\textbf{Вариант 2}}

\begin{enumerate}
\item Вычислить: 
\begin{equation*}\cos(14)                          \end{equation*}
\begin{equation*}\sin(4 e^2)                       \end{equation*}
\begin{equation*}\dfrac{7+\sin(2)}{3}              \end{equation*}

\item Вычислить аналитически: 
 \begin{equation*} \int \cos(x^2)          \end{equation*}\begin{equation*} {\dfrac{\partial \sin(5 x +3)}{\partial x}} \end{equation*}
\item Построить график функции y в диапазоне от $x=-1.7$ до $x=2.8$, определить, при каком значениии $x$ $y=1.0$. На этом же графике построить функцию $z $. Определить координаты точки пересечения графиков. \item Определить сумму, произведение матриц $A=\begin{bmatrix}
4.7 &5 &0.4 \\
0.5 &3.3 &6.9 \\
2.7 &5.1 &7.6 \\
\end{bmatrix}
$ и $B=\begin{bmatrix}
0.7 &0.5 &1.6 \\
7.4 &4.9 &8.4 \\
5 &5.3 &0.8 \\
\end{bmatrix}
$. Вычислить $D_{i,j}=A_{i,j} + B_{i,j}$ и определитель матрицы D
\end{enumerate}
\textsc{\textbf{Вариант 3}}

\begin{enumerate}
\item Вычислить: 
\begin{equation*}\dfrac{7+\sin(2)}{3}              \end{equation*}
\begin{equation*}\ln(3+sin(4))                     \end{equation*}
\begin{equation*}\dfrac{7+\sin(2)}{3}              \end{equation*}

\item Вычислить аналитически: 
 \begin{equation*} \int \dfrac{x^4+1}{x^2} \end{equation*}\begin{equation*} {\dfrac{\partial \sin(5 x +3)}{\partial x}} \end{equation*}
\item Построить график функции y в диапазоне от $x=-4.4$ до $x=29.1$, определить, при каком значениии $x$ $y=1.0$. На этом же графике построить функцию $z $. Определить координаты точки пересечения графиков. \item Определить сумму, произведение матриц $A=\begin{bmatrix}
8.9 &5.1 &3.2 &5.7 \\
5.3 &1.6 &1.2 &7.2 \\
1.1 &4.2 &7.3 &5.7 \\
7.9 &9.4 &6 &2.1 \\
\end{bmatrix}
$ и $B=\begin{bmatrix}
3.3 &2.4 &8.1 &7.8 \\
1.8 &1 &1.9 &3.9 \\
5.7 &8 &4.1 &2.8 \\
9 &7.6 &9.8 &8 \\
\end{bmatrix}
$. Вычислить $D_{i,j}=A_{i,j} + B_{i,j}$ и определитель матрицы D
\end{enumerate}
\textsc{\textbf{Вариант 4}}

\begin{enumerate}
\item Вычислить: 
\begin{equation*}\dfrac{7+\sin(2)}{3}              \end{equation*}
\begin{equation*}\sin \left( \dfrac{\pi}{7} \right)\end{equation*}
\begin{equation*}\ln(3+sin(4))                     \end{equation*}

\item Вычислить аналитически: 
 \begin{equation*} \int \cos(x^2)          \end{equation*}\begin{equation*} {\dfrac{\partial \sin(5 x +3)}{\partial x}} \end{equation*}
\item Построить график функции y в диапазоне от $x=0.5$ до $x=22.0$, определить, при каком значениии $x$ $y=1.0$. На этом же графике построить функцию $z $. Определить координаты точки пересечения графиков. \item Определить сумму, произведение матриц $A=\begin{bmatrix}
5.1 &6.4 &5.7 \\
1.7 &1.9 &5.1 \\
1.6 &1.2 &8.5 \\
\end{bmatrix}
$ и $B=\begin{bmatrix}
2.7 &9.9 &0.5 \\
8.9 &3.2 &3.6 \\
4.8 &7 &7.7 \\
\end{bmatrix}
$. Вычислить $D_{i,j}=A_{i,j} + B_{i,j}$ и определитель матрицы D
\end{enumerate}
\textsc{\textbf{Вариант 5}}

\begin{enumerate}
\item Вычислить: 
\begin{equation*}\ln(3+sin(4))                     \end{equation*}
\begin{equation*}\sin \left( \dfrac{\pi}{7} \right)\end{equation*}
\begin{equation*}\ln(3+sin(4))                     \end{equation*}

\item Вычислить аналитически: 
 \begin{equation*} \int \sin(7x)           \end{equation*}\begin{equation*} {\dfrac{\partial \sin(5 x +3)}{\partial x}} \end{equation*}
\item Построить график функции y в диапазоне от $x=2.9$ до $x=21.4$, определить, при каком значениии $x$ $y=1.0$. На этом же графике построить функцию $z $. Определить координаты точки пересечения графиков. \item Определить сумму, произведение матриц $A=\begin{bmatrix}
4.5 &5.2 &6 \\
1.5 &0 &6.9 \\
0.6 &7.6 &5.7 \\
\end{bmatrix}
$ и $B=\begin{bmatrix}
8.3 &3 &0.7 \\
6.9 &5.8 &7.3 \\
7.2 &7.6 &1 \\
\end{bmatrix}
$. Вычислить $D_{i,j}=A_{i,j} + B_{i,j}$ и определитель матрицы D
\end{enumerate}
\textsc{\textbf{Вариант 6}}

\begin{enumerate}
\item Вычислить: 
\begin{equation*}\dfrac{7+\sin(2)}{3}              \end{equation*}
\begin{equation*}\sin(4 e^2)                       \end{equation*}
\begin{equation*}\ln(3+sin(4))                     \end{equation*}

\item Вычислить аналитически: 
 \begin{equation*} \int \dfrac{x^4+1}{x^2} \end{equation*}\begin{equation*} {\dfrac{\partial \sin(5 x +3)}{\partial x}} \end{equation*}
\item Построить график функции y в диапазоне от $x=-4.9$ до $x=14.6$, определить, при каком значениии $x$ $y=1.0$. На этом же графике построить функцию $z $. Определить координаты точки пересечения графиков. \item Определить сумму, произведение матриц $A=\begin{bmatrix}
7.5 &3.6 &5.2 \\
0.9 &6.2 &3.5 \\
1.9 &7.2 &5.9 \\
\end{bmatrix}
$ и $B=\begin{bmatrix}
6.3 &4.7 &2.4 \\
0.2 &5.9 &6.4 \\
5.8 &2.8 &3.9 \\
\end{bmatrix}
$. Вычислить $D_{i,j}=A_{i,j} + B_{i,j}$ и определитель матрицы D
\end{enumerate}
\textsc{\textbf{Вариант 7}}

\begin{enumerate}
\item Вычислить: 
\begin{equation*}\sin \left( \dfrac{\pi}{7} \right)\end{equation*}
\begin{equation*}\dfrac{7+\sin(2)}{3}              \end{equation*}
\begin{equation*}\sin \left( \dfrac{\pi}{7} \right)\end{equation*}

\item Вычислить аналитически: 
 \begin{equation*} \int \cos(x^2)          \end{equation*}\begin{equation*} {\dfrac{\partial \sin(5 x +3)}{\partial x}} \end{equation*}
\item Построить график функции y в диапазоне от $x=2.0$ до $x=6.0$, определить, при каком значениии $x$ $y=1.0$. На этом же графике построить функцию $z $. Определить координаты точки пересечения графиков. \item Определить сумму, произведение матриц $A=\begin{bmatrix}
6 &4.9 &4.3 \\
2.4 &8.4 &1.4 \\
0.5 &4.5 &7.6 \\
\end{bmatrix}
$ и $B=\begin{bmatrix}
1.6 &3.5 &2.7 \\
1 &3.9 &5.1 \\
7.6 &6 &1.4 \\
\end{bmatrix}
$. Вычислить $D_{i,j}=A_{i,j} + B_{i,j}$ и определитель матрицы D
\end{enumerate}
\textsc{\textbf{Вариант 8}}

\begin{enumerate}
\item Вычислить: 
\begin{equation*}\sin(4 e^2)                       \end{equation*}
\begin{equation*}\cos(14)                          \end{equation*}
\begin{equation*}\ln(3+sin(4))                     \end{equation*}

\item Вычислить аналитически: 
 \begin{equation*} \int \cos(x^2)          \end{equation*}\begin{equation*} {\dfrac{\partial \sin(5 x +3)}{\partial x}} \end{equation*}
\item Построить график функции y в диапазоне от $x=-2.7$ до $x=30.8$, определить, при каком значениии $x$ $y=1.0$. На этом же графике построить функцию $z $. Определить координаты точки пересечения графиков. \item Определить сумму, произведение матриц $A=\begin{bmatrix}
5.9 &9.6 &1.6 &2.1 \\
2.1 &6.7 &7 &0.9 \\
7.9 &2.8 &3.1 &2.5 \\
6.2 &0.7 &0.9 &7.4 \\
\end{bmatrix}
$ и $B=\begin{bmatrix}
1.3 &2.8 &9.3 &6.8 \\
8.1 &1.9 &8 &9.1 \\
5.6 &0.2 &5.6 &3.5 \\
5.1 &6.5 &0.4 &2.2 \\
\end{bmatrix}
$. Вычислить $D_{i,j}=A_{i,j} + B_{i,j}$ и определитель матрицы D
\end{enumerate}
\textsc{\textbf{Вариант 9}}

\begin{enumerate}
\item Вычислить: 
\begin{equation*}\dfrac{7+\sin(2)}{3}              \end{equation*}
\begin{equation*}\sin \left( \dfrac{\pi}{7} \right)\end{equation*}
\begin{equation*}\dfrac{7+\sin(2)}{3}              \end{equation*}

\item Вычислить аналитически: 
 \begin{equation*} \int \sin(7x)           \end{equation*}\begin{equation*} {\dfrac{\partial \sin(5 x +3)}{\partial x}} \end{equation*}
\item Построить график функции y в диапазоне от $x=0.5$ до $x=46.0$, определить, при каком значениии $x$ $y=1.0$. На этом же графике построить функцию $z $. Определить координаты точки пересечения графиков. \item Определить сумму, произведение матриц $A=\begin{bmatrix}
6.1 &7.3 &0.2 \\
8.9 &2.7 &0 \\
6.8 &8.1 &7.5 \\
\end{bmatrix}
$ и $B=\begin{bmatrix}
0.3 &1.3 &0.2 \\
0.5 &1 &0.9 \\
7 &8.5 &3.2 \\
\end{bmatrix}
$. Вычислить $D_{i,j}=A_{i,j} + B_{i,j}$ и определитель матрицы D
\end{enumerate}
\textsc{\textbf{Вариант 10}}

\begin{enumerate}
\item Вычислить: 
\begin{equation*}\cos(14)                          \end{equation*}
\begin{equation*}\sin(4 e^2)                       \end{equation*}
\begin{equation*}\cos(14)                          \end{equation*}

\item Вычислить аналитически: 
 \begin{equation*} \int \dfrac{x^4+1}{x^2} \end{equation*}\begin{equation*} {\dfrac{\partial \sin(5 x +3)}{\partial x}} \end{equation*}
\item Построить график функции y в диапазоне от $x=-2.0$ до $x=34.5$, определить, при каком значениии $x$ $y=1.0$. На этом же графике построить функцию $z $. Определить координаты точки пересечения графиков. \item Определить сумму, произведение матриц $A=\begin{bmatrix}
5 &2 &5.5 \\
1.2 &4.8 &0.6 \\
7.5 &0.9 &7.6 \\
\end{bmatrix}
$ и $B=\begin{bmatrix}
0.3 &0.8 &7.9 \\
8.9 &0.9 &2.6 \\
5.2 &6.4 &1.6 \\
\end{bmatrix}
$. Вычислить $D_{i,j}=A_{i,j} + B_{i,j}$ и определитель матрицы D
\end{enumerate}
\textsc{\textbf{Вариант 11}}

\begin{enumerate}
\item Вычислить: 
\begin{equation*}\dfrac{7+\sin(2)}{3}              \end{equation*}
\begin{equation*}\sin(4 e^2)                       \end{equation*}
\begin{equation*}\dfrac{7+\sin(2)}{3}              \end{equation*}

\item Вычислить аналитически: 
 \begin{equation*} \int \cos(x^2)          \end{equation*}\begin{equation*} {\dfrac{\partial \sin(5 x +3)}{\partial x}} \end{equation*}
\item Построить график функции y в диапазоне от $x=0.3$ до $x=34.3$, определить, при каком значениии $x$ $y=1.0$. На этом же графике построить функцию $z $. Определить координаты точки пересечения графиков. \item Определить сумму, произведение матриц $A=\begin{bmatrix}
3 &6.5 &3.5 \\
0.8 &7.8 &6 \\
0.4 &3.9 &6.8 \\
\end{bmatrix}
$ и $B=\begin{bmatrix}
2.1 &2.6 &3.9 \\
3.6 &7.9 &1.8 \\
5.2 &1.8 &6.7 \\
\end{bmatrix}
$. Вычислить $D_{i,j}=A_{i,j} + B_{i,j}$ и определитель матрицы D
\end{enumerate}
\textsc{\textbf{Вариант 12}}

\begin{enumerate}
\item Вычислить: 
\begin{equation*}\dfrac{7+\sin(2)}{3}              \end{equation*}
\begin{equation*}\sin \left( \dfrac{\pi}{7} \right)\end{equation*}
\begin{equation*}\ln(3+sin(4))                     \end{equation*}

\item Вычислить аналитически: 
 \begin{equation*} \int \cos(x^2)          \end{equation*}\begin{equation*} {\dfrac{\partial \sin(5 x +3)}{\partial x}} \end{equation*}
\item Построить график функции y в диапазоне от $x=-0.7$ до $x=9.8$, определить, при каком значениии $x$ $y=1.0$. На этом же графике построить функцию $z $. Определить координаты точки пересечения графиков. \item Определить сумму, произведение матриц $A=\begin{bmatrix}
2 &0.1 &8 \\
4.8 &8.4 &1.3 \\
6.6 &8.8 &6.1 \\
\end{bmatrix}
$ и $B=\begin{bmatrix}
5.4 &5.7 &2 \\
2.7 &4.5 &6.6 \\
7.3 &6.1 &5 \\
\end{bmatrix}
$. Вычислить $D_{i,j}=A_{i,j} + B_{i,j}$ и определитель матрицы D
\end{enumerate}
\textsc{\textbf{Вариант 13}}

\begin{enumerate}
\item Вычислить: 
\begin{equation*}\dfrac{7+\sin(2)}{3}              \end{equation*}
\begin{equation*}\sin \left( \dfrac{\pi}{7} \right)\end{equation*}
\begin{equation*}\dfrac{7+\sin(2)}{3}              \end{equation*}

\item Вычислить аналитически: 
 \begin{equation*} \int \cos(x^2)          \end{equation*}\begin{equation*} {\dfrac{\partial \sin(5 x +3)}{\partial x}} \end{equation*}
\item Построить график функции y в диапазоне от $x=-1.6$ до $x=10.4$, определить, при каком значениии $x$ $y=1.0$. На этом же графике построить функцию $z $. Определить координаты точки пересечения графиков. \item Определить сумму, произведение матриц $A=\begin{bmatrix}
8.7 &3.4 &3.1 \\
3 &6.7 &5.4 \\
8.3 &3 &7.5 \\
\end{bmatrix}
$ и $B=\begin{bmatrix}
9.5 &2.3 &3 \\
5.5 &9.7 &1.1 \\
6.2 &0.8 &1.1 \\
\end{bmatrix}
$. Вычислить $D_{i,j}=A_{i,j} + B_{i,j}$ и определитель матрицы D
\end{enumerate}
\textsc{\textbf{Вариант 14}}

\begin{enumerate}
\item Вычислить: 
\begin{equation*}\ln(3+sin(4))                     \end{equation*}
\begin{equation*}\cos(14)                          \end{equation*}
\begin{equation*}\sin \left( \dfrac{\pi}{7} \right)\end{equation*}

\item Вычислить аналитически: 
 \begin{equation*} \int \sin(7x)           \end{equation*}\begin{equation*} {\dfrac{\partial \sin(5 x +3)}{\partial x}} \end{equation*}
\item Построить график функции y в диапазоне от $x=1.2$ до $x=39.2$, определить, при каком значениии $x$ $y=1.0$. На этом же графике построить функцию $z $. Определить координаты точки пересечения графиков. \item Определить сумму, произведение матриц $A=\begin{bmatrix}
8.8 &8 &2.6 \\
6 &7.3 &2.5 \\
5.5 &8.8 &6.6 \\
\end{bmatrix}
$ и $B=\begin{bmatrix}
7.4 &8.2 &2.1 \\
6.3 &8.9 &3.5 \\
7.5 &1.6 &3.5 \\
\end{bmatrix}
$. Вычислить $D_{i,j}=A_{i,j} + B_{i,j}$ и определитель матрицы D
\end{enumerate}
\textsc{\textbf{Вариант 15}}

\begin{enumerate}
\item Вычислить: 
\begin{equation*}\cos(14)                          \end{equation*}
\begin{equation*}\ln(3+sin(4))                     \end{equation*}
\begin{equation*}\sin(4 e^2)                       \end{equation*}

\item Вычислить аналитически: 
 \begin{equation*} \int \cos(x^2)          \end{equation*}\begin{equation*} {\dfrac{\partial \sin(5 x +3)}{\partial x}} \end{equation*}
\item Построить график функции y в диапазоне от $x=3.6$ до $x=38.6$, определить, при каком значениии $x$ $y=1.0$. На этом же графике построить функцию $z $. Определить координаты точки пересечения графиков. \item Определить сумму, произведение матриц $A=\begin{bmatrix}
9.8 &4 &5.7 \\
3.5 &2.9 &3.5 \\
5.6 &7.3 &9 \\
\end{bmatrix}
$ и $B=\begin{bmatrix}
4.3 &1.3 &4.8 \\
3.4 &9.8 &2.8 \\
4.7 &8.5 &5.4 \\
\end{bmatrix}
$. Вычислить $D_{i,j}=A_{i,j} + B_{i,j}$ и определитель матрицы D
\end{enumerate}
