\textsc{\textbf{Лабораторная работа №1 }}

\textsc{\textbf{Вариант 1}}

\begin{enumerate}
\item Вычислить: 
\begin{equation*}\cos(14)                          \end{equation*}
\begin{equation*}\dfrac{7+\sin(2)}{3}              \end{equation*}
\begin{equation*}\sin(4 e^2)                       \end{equation*}

\item Вычислить аналитически: 
 \begin{equation*} \int \sin(7x)           \end{equation*}\begin{equation*} {\dfrac{\partial \sin(5 x +3)}{\partial x}} \end{equation*}
\item Построить график функции $y(x)=sin(x)+\sqrt{x}  $ в диапазоне от $x=-2.6$ до $x=33.9$, определить, при каком значении $x$ $y=1.0$. На этом же графике построить функцию $z(x)=\dfrac{x}{3}+\dfrac{1}{3} $. Определить координаты точки пересечения графиков. \item Решить уравнение: $x^3+4 \sqrt{x}=10    $

\item Определить сумму, произведение матриц $A=\begin{bmatrix}
7 &8.7 &7.5 \\
8.7 &8.6 &1.3 \\
2.4 &8.8 &3.7 \\
\end{bmatrix}
$ и $B=\begin{bmatrix}
9.6 &1.6 &2.8 \\
3.9 &8.1 &7.9 \\
1.1 &7.7 &9.2 \\
\end{bmatrix}
$. Вычислить $D_{i,j}=A_{i,j}  \cdot  B_{i,j}$ и определитель матрицы D

\item Решить систему уравнений: \begin{equation*} \begin{cases} x^2+y=3              \\ x=7y                  \end{cases} \end{equation*} 

\end{enumerate}
\textsc{\textbf{Вариант 2}}

\begin{enumerate}
\item Вычислить: 
\begin{equation*}\cos(14)                          \end{equation*}
\begin{equation*}\ln(3+sin(4))                     \end{equation*}
\begin{equation*}\sin(4 e^2)                       \end{equation*}

\item Вычислить аналитически: 
 \begin{equation*} \int \dfrac{x^4+1}{x^2} \end{equation*}\begin{equation*} {\dfrac{\partial \sin(5 x +3)}{\partial x}} \end{equation*}
\item Построить график функции $y(x)=dfrac{x+4}{x^2+1}$ в диапазоне от $x=4.1$ до $x=47.6$, определить, при каком значении $x$ $y=1.0$. На этом же графике построить функцию $z(x)=\dfrac{x^3+10}{x^2+1}     $. Определить координаты точки пересечения графиков. \item Решить уравнение: $x+\sin{x}=6          $

\item Определить сумму, произведение матриц $A=\begin{bmatrix}
9 &9.1 &0.2 \\
5.1 &7.7 &6.8 \\
8.5 &0.4 &4.7 \\
\end{bmatrix}
$ и $B=\begin{bmatrix}
0.7 &0.8 &6.4 \\
8.2 &8.4 &1.4 \\
5.2 &7.8 &1.6 \\
\end{bmatrix}
$. Вычислить $D_{i,j}=A_{i,j}  /  B_{i,j}$ и определитель матрицы D

\item Решить систему уравнений: \begin{equation*} \begin{cases} \dfrac{x+4}{y^2+1}=1 \\ 5x +3y=10             \end{cases} \end{equation*} 

\end{enumerate}
\textsc{\textbf{Вариант 3}}

\begin{enumerate}
\item Вычислить: 
\begin{equation*}\cos(14)                          \end{equation*}
\begin{equation*}\dfrac{7+\sin(2)}{3}              \end{equation*}
\begin{equation*}\ln(3+sin(4))                     \end{equation*}

\item Вычислить аналитически: 
 \begin{equation*} \int \cos(x^2)          \end{equation*}\begin{equation*} {\dfrac{\partial \sin(5 x +3)}{\partial x}} \end{equation*}
\item Построить график функции $y(x)=dfrac{x+4}{x^2+1}$ в диапазоне от $x=4.2$ до $x=23.2$, определить, при каком значении $x$ $y=1.0$. На этом же графике построить функцию $z(x)=-0.5 x^2 + x              $. Определить координаты точки пересечения графиков. \item Решить уравнение: $x^3+4 \sqrt{x}=10    $

\item Определить сумму, произведение матриц $A=\begin{bmatrix}
6.1 &1.4 &0.7 \\
1.3 &7 &2.8 \\
3.1 &4.7 &2.7 \\
\end{bmatrix}
$ и $B=\begin{bmatrix}
2.7 &5.5 &9.9 \\
7.4 &3.3 &1.6 \\
8 &2 &4.7 \\
\end{bmatrix}
$. Вычислить $D_{i,j}=A_{i,j}  /  B_{i,j}$ и определитель матрицы D

\item Решить систему уравнений: \begin{equation*} \begin{cases} \sqrt{x}+2y=2        \\ 5x +3y=10             \end{cases} \end{equation*} 

\end{enumerate}
\textsc{\textbf{Вариант 4}}

\begin{enumerate}
\item Вычислить: 
\begin{equation*}\sin(4 e^2)                       \end{equation*}
\begin{equation*}\ln(3+sin(4))                     \end{equation*}
\begin{equation*}\cos(14)                          \end{equation*}

\item Вычислить аналитически: 
 \begin{equation*} \int \cos(x^2)          \end{equation*}\begin{equation*} {\dfrac{\partial \sin(5 x +3)}{\partial x}} \end{equation*}
\item Построить график функции $y(x)=x^3+4 \sqrt(x)   $ в диапазоне от $x=4.6$ до $x=31.6$, определить, при каком значении $x$ $y=1.0$. На этом же графике построить функцию $z(x)=\dfrac{x^3+10}{x^2+1}     $. Определить координаты точки пересечения графиков. \item Решить уравнение: $x+\sin{x}=6          $

\item Определить сумму, произведение матриц $A=\begin{bmatrix}
1.3 &4.7 &4.7 \\
8.8 &8.7 &7 \\
4.3 &6.3 &0.5 \\
\end{bmatrix}
$ и $B=\begin{bmatrix}
5 &0.6 &4 \\
9.7 &0.5 &8.8 \\
2.4 &4.9 &2.2 \\
\end{bmatrix}
$. Вычислить $D_{i,j}=A_{i,j}  -  B_{i,j}$ и определитель матрицы D

\item Решить систему уравнений: \begin{equation*} \begin{cases} \dfrac{x+4}{y^2+1}=1 \\ \dfrac{x^3+10}{x^2+1} \end{cases} \end{equation*} 

\end{enumerate}
\textsc{\textbf{Вариант 5}}

\begin{enumerate}
\item Вычислить: 
\begin{equation*}\ln(3+sin(4))                     \end{equation*}
\begin{equation*}\cos(14)                          \end{equation*}
\begin{equation*}\ln(3+sin(4))                     \end{equation*}

\item Вычислить аналитически: 
 \begin{equation*} \int \dfrac{x^4+1}{x^2} \end{equation*}\begin{equation*} {\dfrac{\partial \sin(5 x +3)}{\partial x}} \end{equation*}
\item Построить график функции $y(x)=x^2+4            $ в диапазоне от $x=-2.7$ до $x=37.3$, определить, при каком значении $x$ $y=1.0$. На этом же графике построить функцию $z(x)=\dfrac{x}{3}+\dfrac{1}{3} $. Определить координаты точки пересечения графиков. \item Решить уравнение: $\dfrac{x+4}{5}=1     $

\item Определить сумму, произведение матриц $A=\begin{bmatrix}
0.6 &3.5 &2.5 \\
7.3 &8.2 &1.3 \\
5.7 &8.2 &3 \\
\end{bmatrix}
$ и $B=\begin{bmatrix}
3 &5.8 &3.5 \\
7.4 &9.7 &1.4 \\
4.1 &1.4 &8.3 \\
\end{bmatrix}
$. Вычислить $D_{i,j}=A_{i,j}  -  B_{i,j}$ и определитель матрицы D

\item Решить систему уравнений: \begin{equation*} \begin{cases} x^3+4 \sqrt{y}=10    \\ x=7y                  \end{cases} \end{equation*} 

\end{enumerate}
\textsc{\textbf{Вариант 6}}

\begin{enumerate}
\item Вычислить: 
\begin{equation*}\dfrac{7+\sin(2)}{3}              \end{equation*}
\begin{equation*}\ln(3+sin(4))                     \end{equation*}
\begin{equation*}\sin \left( \dfrac{\pi}{7} \right)\end{equation*}

\item Вычислить аналитически: 
 \begin{equation*} \int \dfrac{x^4+1}{x^2} \end{equation*}\begin{equation*} {\dfrac{\partial \sin(5 x +3)}{\partial x}} \end{equation*}
\item Построить график функции $y(x)=x                $ в диапазоне от $x=4.1$ до $x=45.1$, определить, при каком значении $x$ $y=1.0$. На этом же графике построить функцию $z(x)=\dfrac{x^3+10}{x^2+1}     $. Определить координаты точки пересечения графиков. \item Решить уравнение: $\sqrt{x}+2x=2        $

\item Определить сумму, произведение матриц $A=\begin{bmatrix}
3.8 &1.2 &5.5 \\
4.4 &8 &6 \\
6.7 &4.4 &4.4 \\
\end{bmatrix}
$ и $B=\begin{bmatrix}
3.5 &2.4 &6.8 \\
0.5 &3.6 &8.9 \\
3.8 &9.4 &7.8 \\
\end{bmatrix}
$. Вычислить $D_{i,j}=A_{i,j}  +  B_{i,j}$ и определитель матрицы D

\item Решить систему уравнений: \begin{equation*} \begin{cases} \sin(x)+\cos(y)=1    \\ x=\sqrt{y+1}          \end{cases} \end{equation*} 

\end{enumerate}
\textsc{\textbf{Вариант 7}}

\begin{enumerate}
\item Вычислить: 
\begin{equation*}\sin \left( \dfrac{\pi}{7} \right)\end{equation*}
\begin{equation*}\sin(4 e^2)                       \end{equation*}
\begin{equation*}\dfrac{7+\sin(2)}{3}              \end{equation*}

\item Вычислить аналитически: 
 \begin{equation*} \int \sin(7x)           \end{equation*}\begin{equation*} {\dfrac{\partial \sin(5 x +3)}{\partial x}} \end{equation*}
\item Построить график функции $y(x)=x^3+4 \sqrt(x)   $ в диапазоне от $x=2.9$ до $x=8.9$, определить, при каком значении $x$ $y=1.0$. На этом же графике построить функцию $z(x)=ln(x^2+1)                 $. Определить координаты точки пересечения графиков. \item Решить уравнение: $\dfrac{x+4}{5}=1     $

\item Определить сумму, произведение матриц $A=\begin{bmatrix}
5.1 &5.8 &7.6 \\
2.3 &0.6 &4.8 \\
0.7 &2.5 &6.5 \\
\end{bmatrix}
$ и $B=\begin{bmatrix}
2.3 &1.8 &0.8 \\
6.6 &7.8 &6.8 \\
1.8 &2.1 &7.4 \\
\end{bmatrix}
$. Вычислить $D_{i,j}=A_{i,j}  /  B_{i,j}$ и определитель матрицы D

\item Решить систему уравнений: \begin{equation*} \begin{cases} x^3+4 \sqrt{y}=10    \\ \dfrac{x^3+10}{x^2+1} \end{cases} \end{equation*} 

\end{enumerate}
\textsc{\textbf{Вариант 8}}

\begin{enumerate}
\item Вычислить: 
\begin{equation*}\dfrac{7+\sin(2)}{3}              \end{equation*}
\begin{equation*}\sin \left( \dfrac{\pi}{7} \right)\end{equation*}
\begin{equation*}\dfrac{7+\sin(2)}{3}              \end{equation*}

\item Вычислить аналитически: 
 \begin{equation*} \int \dfrac{x^4+1}{x^2} \end{equation*}\begin{equation*} {\dfrac{\partial \sin(5 x +3)}{\partial x}} \end{equation*}
\item Построить график функции $y(x)=\sqrt{x}         $ в диапазоне от $x=-1.1$ до $x=17.4$, определить, при каком значении $x$ $y=1.0$. На этом же графике построить функцию $z(x)=\dfrac{x^3+10}{x^2+1}     $. Определить координаты точки пересечения графиков. \item Решить уравнение: $\sqrt{x}+2x=2        $

\item Определить сумму, произведение матриц $A=\begin{bmatrix}
0.7 &7.8 &2.9 \\
5.2 &6.9 &9.1 \\
2.7 &3.8 &0.6 \\
\end{bmatrix}
$ и $B=\begin{bmatrix}
4.8 &2.9 &1.9 \\
4.6 &8 &9.1 \\
2.3 &4.4 &1.5 \\
\end{bmatrix}
$. Вычислить $D_{i,j}=A_{i,j}  /  B_{i,j}$ и определитель матрицы D

\item Решить систему уравнений: \begin{equation*} \begin{cases} \dfrac{x+4}{y^2+1}=1 \\ 5x +3y=10             \end{cases} \end{equation*} 

\end{enumerate}
\textsc{\textbf{Вариант 9}}

\begin{enumerate}
\item Вычислить: 
\begin{equation*}\cos(14)                          \end{equation*}
\begin{equation*}\ln(3+sin(4))                     \end{equation*}
\begin{equation*}\sin(4 e^2)                       \end{equation*}

\item Вычислить аналитически: 
 \begin{equation*} \int \sin(7x)           \end{equation*}\begin{equation*} {\dfrac{\partial \sin(5 x +3)}{\partial x}} \end{equation*}
\item Построить график функции $y(x)=x                $ в диапазоне от $x=-3.5$ до $x=38.5$, определить, при каком значении $x$ $y=1.0$. На этом же графике построить функцию $z(x)=ln(x^2+1)                 $. Определить координаты точки пересечения графиков. \item Решить уравнение: $\dfrac{x+4}{5}=1     $

\item Определить сумму, произведение матриц $A=\begin{bmatrix}
4.2 &6.3 &3.4 \\
7.3 &1.6 &3.5 \\
4 &5.9 &9.9 \\
\end{bmatrix}
$ и $B=\begin{bmatrix}
7.2 &4.9 &4.8 \\
1.8 &5.1 &9.6 \\
6.6 &9.4 &7.4 \\
\end{bmatrix}
$. Вычислить $D_{i,j}=A_{i,j}  -  B_{i,j}$ и определитель матрицы D

\item Решить систему уравнений: \begin{equation*} \begin{cases} x^3+4 \sqrt{y}=10    \\ ln(x^2+y)=2           \end{cases} \end{equation*} 

\end{enumerate}
\textsc{\textbf{Вариант 10}}

\begin{enumerate}
\item Вычислить: 
\begin{equation*}\ln(3+sin(4))                     \end{equation*}
\begin{equation*}\dfrac{7+\sin(2)}{3}              \end{equation*}
\begin{equation*}\ln(3+sin(4))                     \end{equation*}

\item Вычислить аналитически: 
 \begin{equation*} \int \dfrac{x^4+1}{x^2} \end{equation*}\begin{equation*} {\dfrac{\partial \sin(5 x +3)}{\partial x}} \end{equation*}
\item Построить график функции $y(x)=sin(x)+\sqrt{x}  $ в диапазоне от $x=0.7$ до $x=29.7$, определить, при каком значении $x$ $y=1.0$. На этом же графике построить функцию $z(x)=10-x^2                    $. Определить координаты точки пересечения графиков. \item Решить уравнение: $\sqrt{x}+2x=2        $

\item Определить сумму, произведение матриц $A=\begin{bmatrix}
2.6 &3.7 &7.4 &3.6 \\
5.5 &8.8 &5.8 &9.9 \\
9.1 &0.9 &2.2 &2 \\
9.4 &3.2 &6.6 &3.1 \\
\end{bmatrix}
$ и $B=\begin{bmatrix}
0.8 &6.9 &9.8 &7.6 \\
5.4 &6.2 &6.7 &7.2 \\
0.1 &8.9 &9.7 &4.9 \\
4.9 &1.1 &5.1 &7.2 \\
\end{bmatrix}
$. Вычислить $D_{i,j}=A_{i,j}  -  B_{i,j}$ и определитель матрицы D

\item Решить систему уравнений: \begin{equation*} \begin{cases} \sqrt{x}+2y=2        \\ x^2 + y^3=1           \end{cases} \end{equation*} 

\end{enumerate}
