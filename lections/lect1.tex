\lecture{Лекция 1}{lec1}
\section{Введение}

\begin{frame}
\frametitle{Цели освоения дисциплины}
	\begin{itemize}
		\item освоение методов построения математических моделей основных тепло- и массообменных процессов, а также сопряженных и совмещенных процессов; 
		\item изучение алгоритмов идентификации параметров математических моделей и способов проверки их адекватности; 
		\item освоение специализированных программно-вычислительных комплексов, позволяющих решать задачи моделирования химико-технологических процессов.
	\end{itemize}
\end{frame}

\begin{frame}
\frametitle{Сопутствующие дисциплины}
	\begin{itemize}
		\item Математика
		\item Физика
		\item Физическая химия
		\item Процессы и аппараты химической технологии
		\item Программирование
	\end{itemize}
	\begin{figure}[h]
		\includegraphics[width=5cm]{puzzle.jpg}
	\end{figure}
\end{frame}

\begin{frame}
	\frametitle{Знания. Умения. Навыки}
	 Знать:
			\begin{itemize}
				\item методологические основы построения математических моделей процессов химической технологии;
				\item основные математические модели описания гидродинамической структуры потоков в аппаратах и физико-химических свойств рабочих агентов;
				\item методы идентификации параметров моделей и проверки их адекватности.
			\end{itemize}
	 
\end{frame}

\begin{frame}
	\frametitle{Знания. Умения. Навыки}
	Уметь:
	\begin{itemize}
		\item составлять математические модели процессов тепломассопереноса, осложненных химической реакцией;
		\item проводить идентификацию параметров модели и оценивать ее адекватность.
	\end{itemize}
	Владеть:
	\begin{itemize}
		\item навыками работы в программных продуктах, позволяющих решать задачи моделирования химико-технологических процессов.
	\end{itemize}
\end{frame}

\begin{frame}
	\begin{figure}[h]
		\includegraphics[width=0.8\paperwidth]{funny1.png}
	\end{figure}
\end{frame}

\begin{frame}
	\frametitle{MathCad}
	\begin{itemize}
		\item WYSIWYG (What You See Is What You Get — «что видишь, то и получаешь»)
		\item Простота для конечного пользователя
		\item Проприетарный
		\item Нет мультиплатформенности
	\end{itemize}
	
\end{frame}

\begin{frame}
	\frametitle{Литература}
	\begin{itemize}
		\item Клинов, А.В. Математическое моделирование химико-технологических процессов: учебное пособие / А.В. Клинов, А.Г. Мухаметзянова. – Казань: Изд-во КГТУ, 2009. – 136 с.
		\item Клинов, А.В. Лабораторный практикум по математическому моделированию химико-технологических процессов: учебное пособие / А.В. Клинов, А.В. Малыгин. – Казань: Изд-во КГТУ, 2011. – 100 с.
		\item Закгейм, А.Ю. Общая химическая технология: введение в моделирование химико-технологических процессов: учебное пособие / А.Ю. Закгейм. – М.: Логос, 2012. – 304 с.
	\end{itemize}
\end{frame}

\begin{frame}
	\frametitle{Литература}
	\begin{itemize}
		\item Кафаров, В.В. Математическое моделирование основных процессов химических производств: учебное пособие для ВУЗов / В.В. Кафаров, М.Б. Глебов. – М.: Высш.шк, 1991. – 400 с.
	\end{itemize}
\end{frame}

\begin{frame}
	\frametitle{Модель}
	Условный образ объекта исследования, конструируемый исследователем так, чтобы отобразить \textbf{основные} характеристики и существенные особенности его поведения.
\end{frame}

\begin{frame}
	\frametitle{Виды моделей}
	Концептуальные модели: 	совокупность уже известных фактов или представлений относительно исследуемого объекта или системы истолковывается с помощью некоторых специальных знаков, символов, операций над ними или с помощью естественного или искусственного языков;
	
	
	Физические модели: 	модель и моделируемый объект представляют собой реальные объекты или процессы единой или различной физической природы, причем между процессами в объекте-оригинале и в модели выполняются некоторые соотношения подобия, вытекающие из схожести физических явлений;
\end{frame}

\begin{frame}
	\frametitle{Физические модели}
	\begin{figure}
		\centering
		\begin{minipage}{0.45\textwidth}
			\centering
			\includegraphics[width=0.9\textwidth]{mgu1.png} % first figure itself
			\caption{Оригинал}
		\end{minipage}\hfill
		\begin{minipage}{0.45\textwidth}
			\centering
			\includegraphics[width=0.9\textwidth]{mgu3.jpg} % second figure itself
			\caption{Модель}
		\end{minipage}
	\end{figure}
\end{frame}


\begin{frame}
	\frametitle{Гидравлический компьютер Лукьянова}
	\begin{figure}
		\centering
		\begin{minipage}{0.45\textwidth}
			Дифференциальные уравнения
		\end{minipage}\hfill
		\begin{minipage}{0.45\textwidth}
			\centering
			\includegraphics[width=0.9\textwidth]{gcpu.jpg} % second figure itself
			\caption{Модель}
		\end{minipage}
	\end{figure}
\end{frame}

\begin{frame}
	\frametitle{Физические модели}
	\begin{figure}
		\centering
		\begin{minipage}{0.45\textwidth}
			\centering
			\includegraphics[width=0.9\textwidth]{mesh2.png} % first figure itself
			\caption{Оригинал}
		\end{minipage}\hfill
		\begin{minipage}{0.45\textwidth}
			\centering
			\includegraphics[width=0.9\textwidth]{mesh1.png} % second figure itself
			\caption{Модель}
		\end{minipage}
	\end{figure}
\end{frame}

\begin{frame}
	\frametitle{Структурно-функциональные модели}
	моделями являются схемы (блок-схемы), графики, чертежи, диаграммы, таблицы, рисунки, дополненные специальными правилами их объединения и преобразования
	\begin{figure}[h]
		\includegraphics[width=6cm]{schem1.png}
	\end{figure}
\end{frame}

\begin{frame}
	\frametitle{Структурно-функциональные модели}
	\begin{figure}[h]
		\includegraphics[width=0.9\textwidth]{schem2.png}
	\end{figure}
\end{frame}

\begin{frame}
	\frametitle{Математические и иммитационные модели}
	\begin{itemize}
		\item математические модели – построение модели осуществляется средствами математики и логики;
		\item имитационные (программные) (simulation) модели - логико-математическая модель исследуемого объекта представляет собой алгоритм функционирования объекта, реализованный в виде программного комплекса для компьютера.
	\end{itemize}
	
\end{frame}


\begin{frame}
	\frametitle{Имитационные модели}
	\begin{itemize}
	\item Метод молекулярной динамики
	\item Метод Монте-Карло
	\item CFD (computation fluid dynamic) методы вычислительной гидродинамики
	\item Вычисления методом конечных объемов
	\item Квантово-химические методы
	\end{itemize}
\end{frame}

\begin{frame}
	\frametitle{Химико-технологические процессы (ХТП)}
	технологические процессы, связанные с физико-химической и химической переработкой реагентов в конечные продукты. ХТП – это сложная физико-химическая система, которая имеет двойственную детерминированную и стохастическую (случайную) структуру, переменную, как во времени, так и в пространстве.
\end{frame}

\begin{frame}
	\frametitle{«Элементарные» процессы протекающие в ХТП}
\begin{itemize}
	\item Гидромеханические
	\item Тепловые
	\item Массообменные
	\item Механические
	\item Химические
\end{itemize}
\end{frame}

\begin{frame}
\frametitle{Основные этапы моделирования ХТП}
	\begin{itemize}
	\item Постановка цели моделирования --- выявление основных исследуемых характеристик, постановка задачи и вид результатов моделирования
	\item Построение модели --- установление протекающих процессов, упрощение модели путем сокращения описания процессов, выведение уравнений/систем равнений.
	\item Идентификация модели --- определение параметров модели и вычисление их значений на основании экспериментальных данных
\end{itemize}
\end{frame}

\begin{frame}
	\frametitle{Основные этапы моделирования ХТП}
	\begin{itemize}
		\item Проверка адекватности модели на основании сравнения с экспериментальными данными, проведение корректировки модели или параметров моделирования.
		\item Использование модели для исследования свойств и поведения объекта моделирования
	\end{itemize}
\end{frame}

\begin{frame}
	\frametitle{Задачи моделирования ХТП}
	\begin{itemize}
		\item исследование новых процессов;
		\item проектирование производств;
		\item оптимизация отдельных аппаратов и технологических схем;
		\item выявление резервов мощности и отыскание наиболее
		\item эффективных способов модернизации действующих
		производств; 
		\item оптимальное планирование производств; 
		\item 	разработка автоматизированных систем управления
		\item проектируемыми производствами.
	\end{itemize}
	
\end{frame}

\begin{frame}
	\frametitle{Преимущества математического моделирования}
	\begin{itemize}
		\item позволяет осуществить с помощью одного устройства (ЭВМ) решение целого класса задач, имеющих одинаковое математическое описание;
		\item обеспечивает простоту перехода от одной задачи к другой, позволяет вводить переменные параметры, возмущения и различные условия однозначности (моделировать различные по размерам объекты и различные свойства);
		\item дает возможность проводить моделирование по частям ("элементарным процессам"), что особенно существенно при исследовании сложных объектов химической технологии;
		\item экономичнее метода физического моделирования.
	\end{itemize}
	
\end{frame}

\begin{frame}
	\frametitle{Классификация математических моделей}
	\begin{itemize}
		\item Статические – инвариантны ко времени; Пример: аппарат полного смешения некоторого объема в установившемся режиме работы, в который непрерывно подаются реагенты А и В и отводится продукт реакции Р.
		\item Динамические – являются функцией времени; Пример: аппарат полного смешения некоторого объема в неустановившемся режиме работы.
	\end{itemize}
\end{frame}

\begin{frame}
	\frametitle{Классификация математических моделей}
	\begin{itemize}
		\item С сосредоточенными параметрами – постоянство переменных в пространстве (алгебраические, либо дифференциальные уравнения первого порядка для нестационарных процессов). Пример: аппарат с полным перемешиванием потока, концентрация во всех точках одинакова;
		\item С распределенными параметрами – переменные изменяются в пространстве (дифференциальные уравнения с частными производными); Пример: трубчатый аппарат с большим отношением длины к диаметру и значительной скоростью движения реагентов.
	\end{itemize}
	
\end{frame}

\begin{frame}
	\frametitle{Основные методы построения математических моделей}
	\begin{itemize}
		\item Эмпирический (экспериментально-статистический, метод «черного ящика») --- не рассматривается природа происходящих процессов. Все явления происходящие в рамках химико-технологического процесса рассматриваются как «черный ящик»
		\item Теоретический (структурный) --- химико-теологический процесс разбивается на элементарные явления. Каждые из явлений описываются отдельными уравнениями.
	\end{itemize}
	
\end{frame}

\begin{frame}
	\frametitle{Блок-схема построения математической модели}
	\begin{tikzpicture}[node distance=1.5cm,
	every node/.style={fill=white, font=\sffamily}, align=center]
	% Specification of nodes (position, etc.)
	\node (object)     [process]          {Объект исследования};
	\node (think)      [process, below of=object]   {Мысленная модель};
	\node (math)      [process, below of=think]   {Математичская модель};
	\node (data)    [process, right of=object, xshift=4cm]{Экспериментальные данные};
	\node (results)      [process, below of=math]   {Решение на компьютере};
	\node (comp)      [process, right of=results, xshift=4cm]   {Сравнение};
	
	\draw[->]             (object) -- (data);
	\draw[->]             (object) -- (think);
	\draw[->]             (think) -- (math);
	\draw[->]             (math) -- (results);
	\draw[->]             (data) -- (comp);
	\draw[->]             (results) -- (comp);
	%\draw[->]             (comp) -- (think);
	%\draw[->]             (comp) -- (math);
	\draw[->] (comp.south) -- ++(0,-1) -- ++(-8.5,0) -- ++(0,2) -- ++(0,1) --                
	node[xshift=3.2cm,yshift=-2.5cm, text width=2.5cm]
	{плохое}(math.west);
	\draw[->] (comp.south) -- ++(0,-1) -- ++(-8.5,0) -- ++(0,2) -- ++(0,2.5) -- (think.west);
	\draw[->] (comp.east) -- ++(2,0) node[xshift=-1cm,yshift=0.4cm]{хорошее} -- ++(0,5.8) -- ++(-8,0) node[xshift=3.1cm,yshift=0.4cm]{управление, оптимизация, контроль} -- (object.north);
	
	\end{tikzpicture}
\end{frame}



\begin{frame}
	\frametitle{Пространственно временной масштаб моделей}
	\begin{figure}[h]
		\includegraphics[width=0.9\textwidth]{total.png}
	\end{figure}
\end{frame}

