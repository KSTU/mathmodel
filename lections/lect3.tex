\lecture{Лекция 3}{lec3}
	\section{Теоретический метод построения математических моделей}
	
\begin{frame}
	\frametitle{Область применения}
	\begin{itemize}
	\item Законы сохранения массы, импульса, энергии (уравнения балансов). (статика)
	\item Законы переноса массы, энергии, импульса и законы химической кинетики. (кинетика)
	\item Законы термодинамики. Связывают воздействие на вещество с его свойствами. Термодинамические свойства, условия равновесия, кинетические (теплофизические свойства)
	\end{itemize}
\end{frame}

\begin{frame}
\frametitle{Законы сохранения}
\begin{itemize}
	\item Интегральная форма (применительно ко всей системе)
	\item Дифференциальная/локальная форма (применительно к элементарным объемам)
\end{itemize}
\end{frame}


\begin{frame}
\frametitle{Закон сохранения массы}
Интегральная форма:
\begin{equation}
	m_i^{вх} - m_i^{вых} =r_i
\end{equation}
Если нет химической реакции:
\begin{equation}
m_i^{вх} = m_i^{вых} 
\end{equation}

Локальная форма:
\begin{equation}
	\dfrac{\partial m} {\partial t} =r
\end{equation}
Уравнение неразрывности:
\begin{equation}
\dfrac{\partial \rho} {\partial t} + \nabla(\rho \upsilon) =r
\end{equation}
где $\rho$ --- плотность, $\upsilon$ --- скорость движения среды.


\end{frame}

\begin{frame}
\frametitle{Закон сохранения энергии}
Интегральная форма:
\begin{equation}
 \dfrac{d E} {d t} =0
\end{equation}
$E$ --- полная энергия.

Первый закон термодинамики:
\begin{equation}
	d U = \delta Q -p dV
\end{equation}
$U $ --- внутренняя энергия системы.
Теплообмен при постоянном объеме:
\begin{equation}
	d U = \delta Q = c_v dT
\end{equation}
$c_V$ --- изохорная теплоемкость

\end{frame}

\begin{frame}
\frametitle{Закон сохранения энергии}
При теплообмен при постоянном давлении удобно использовать энтальпию:
\begin{equation*}
	H =U +pV
\end{equation*}
\begin{equation*}
	d H = d U +p d V +V d p
\end{equation*}
\begin{equation*}
	d H = \delta Q +V d p
\end{equation*}
\begin{equation*}
	\delta Q = d H = c_p dT
\end{equation*}
\begin{equation*}
  Q = H_1 - H_2 = c_p (T_1 - T_2)
\end{equation*}
$c_p$ --- изобарная теплоемкость

\end{frame}

\begin{frame}
	\begin{figure}[h]
		\includegraphics[width=6cm]{l3-t1.png}
	\end{figure}	
	
	\begin{equation*}
		G_1 c_{p1}(T_{1н}-T_{1k})=G_2 c_{p2}(T_{2н}-T_{2k})
	\end{equation*}
	$G$ --- массовый расход
	\begin{equation*}
	G_1 c_{p1}(T_{1н}-T_{1k})=G_2 c_{p2}(T_{2н}-T_{2k})+Q_r
	\end{equation*}
	$Q_r$ --- внутренний источник/сток тепла
	
	Уравнение Бернулли:
	\begin{equation*}
		\dfrac{\rho \upsilon}{2} + \rho g h + z = const
	\end{equation*}
\end{frame}

\begin{frame}
	\frametitle{Закон сохранения импульса}
	\begin{equation}
	\dfrac{d \vec{p}}{d t }=0
	\end{equation}
	Уравнение Навье-Стокса:
	\begin{equation}
	\dfrac{\partial \vec{\upsilon}}{\partial t} = - (\vec{\upsilon} \nabla) \vec{\upsilon} +\nu \Delta \vec{\upsilon} - \dfrac{1}{\rho} \nabla p = \vec{f}
	\end{equation}
	$\upsilon$ --- поле скорости, $\nu$ --- кинематическая вязкость, $f$ --- поле внешних сил
\end{frame}


\begin{frame}
	\frametitle{Законы переноса}
	\begin{itemize}
	\item Молекулярный перенос субстанции (энергии, массы, импульса) – перенос за счет теплового движения молекул
	\item Конвективный перенос – перенос за счет движения среды как единого целого
	\item Турбулентный перенос – перенос за счет турбулентных пульсаций (вихрей)
	\end{itemize}
\end{frame}

\begin{frame}
	\frametitle{Молекулярный перенос}
	Поток массы:
	\begin{equation}
		\vec{j_i} = -D_{ij}\nabla c_i
	\end{equation}
	$D$ --- коэффициент диффузии, $с$ --- концентрация компонента $i$
	
	Поток тепла:
	\begin{equation}
	\vec{q} = -\lambda\nabla T
	\end{equation}
	$\lambda$ --- коэффициент теплопроводности
	
	Поток импульса:
	\begin{equation}
	\tau_{xy} = -\mu \dfrac{d \upsilon_x}{d y}
	\end{equation}
	$\mu$ --- коэффициент вязкости
	
\end{frame}

\begin{frame}
	\frametitle{Конвективный механизм}
	Поток массы:
	\begin{equation}
		\vec{j_i} = c_i \vec{\upsilon}
	\end{equation}
	Поток тепла:
	\begin{equation}
		\vec{q} = \rho c_p T \vec{\upsilon}
	\end{equation}
	Поток импульса:
	\begin{equation}
		\tau_{xy} = \rho  \upsilon_x \upsilon_y
	\end{equation}
	
\end{frame}

\begin{frame}
	\frametitle{Турбулентный перенос}
	Поток массы:
	\begin{equation}
	\vec{j_i} = -D_{T}\nabla c_i
	\end{equation}
	Поток тепла:
	\begin{equation}
	\vec{q} = -\lambda_T \nabla T
	\end{equation}
	Поток импульса:
	\begin{equation}
	\tau_{xy} = -\mu_T \dfrac{d \upsilon_x}{d y}
	\end{equation}
	$D=f(T,p,x)$ $D_T=f(T,p,x,\upsilon)$
	
\end{frame}


\begin{frame}
\frametitle{Скорость химической реакции}
$v_A A + v_B B \rightarrow v_C C$

$2 H_2 + O_2 \rightarrow  2 H_2 O$

Скорость гомогенной реакции:
	\begin{equation}
		r_A = \dfrac{1}{V} \dfrac{d N_A}{d t}
	\end{equation}
Скорость гетерогенной реакции:
	\begin{equation}
		r_A = \dfrac{1}{F} \dfrac{d N_A}{d t}
	\end{equation}
	
	$r_A \ne r_B \ne r_C$
\end{frame}

\begin{frame}
\frametitle{Уравнения формальной кинетики}
Скорость химической реакции:
\begin{equation}
r_A = k c_A^{n_A}
\end{equation}
$k$ --- константа скорости, $n_i$ --- порядок реакции по компоненту $i$

Уравнение Аррениуса:
\begin{equation}
k=k_0 \exp \left( \dfrac{-E_A}{k_B T} \right)
\end{equation}

$k_0$ --- предэкспоненциальный множитель, $E_A$ --- энергия активации
\end{frame}

\begin{frame}
\frametitle{Исчерпывающее описание}
\begin{itemize}
\item Уравнения сохранения
\item Уравнения переноса
\item Начальные условия (в начальный момент времени или на входе в аппарат)
\item Граничные условия (условия на стенках аппарата и т. д.)
\item Форма аппарата
\item Свойства веществ (более 70\% исходных данных) 
\end{itemize}

\end{frame}

\begin{frame}
\frametitle{Законы термодинамики и термодинамического равновесия}

\begin{equation}
d U = TdS - pdV + \sum_{i=1}^{n} \mu_i d N_i
\end{equation}
$S$ --- энтропия, $\mu$ --- химический потенциал, $N$ --- количество вещества (молей)
\begin{equation}
U = TS - pV + \sum_{i=1}^{n} \mu_i N_i
\end{equation}

\begin{equation}
d U = T dS + S dT - p dV - V dp + \sum_{i=1}^{n} \mu_i d N_i + \sum_{i=1}^{n}  N_i d \mu_i
\end{equation}



\end{frame}

\begin{frame}
Соотношение Гиббса-Дюгема:
\begin{equation}
S dT -V dp + \sum_{i=1}^{n} N_i d \mu_i =0
\end{equation}
При $p=const$, $T=const$:
\begin{equation}
\sum_{i=1}^{n} N_i d \mu_i =0
\end{equation}
\begin{equation}
\sum_{i=1}^{n} x_i d \mu_i =0
\end{equation}


\end{frame}


\begin{frame}

V,T,N
\begin{equation}
F=U-TS=\sum \mu N -pV
\end{equation}
\begin{equation*}
d F=-S dT -p dV + \sum \mu dN 
\end{equation*}
P,T,N
\begin{equation}
G=U-TS+pV=\sum \mu N 
\end{equation}
\begin{equation*}
d G=-S dT -V dp + \sum \mu dN 
\end{equation*}
P,S,N
\begin{equation}
H=U+pV=TS +\sum \mu N 
\end{equation}
\begin{equation*}
d H=T dS +V dp + \sum \mu dN 
\end{equation*}

\end{frame}

\begin{frame}
\frametitle{Соотношения Максвелла}

\begin{equation}
\mu_i=\left.\dfrac{\partial U}{\partial N_i} \right|_{S,V,N} = \left.\dfrac{\partial F}{\partial N_i} \right|_{V,T,N} = \left.\dfrac{\partial G}{\partial N_i} \right|_{p,T,N}=\left.\dfrac{\partial H}{\partial N_i} \right|_{S,p,N}
\end{equation}

\begin{equation}
	p=-\left.\dfrac{\partial F}{\partial V}\right|_{T,N}=-\left.\dfrac{\partial U}{\partial V}\right|_{S,N}
\end{equation}
Уравнение состояния:
\begin{equation}
p=f(T,V,N_1,N_2,..., N_n)
\end{equation}
\begin{equation}
p=f(T,V,x_1,x_2,..., x_{n-1})
\end{equation}

\end{frame}

\begin{frame}
\frametitle{Уравнения состояния}
Идеального газа:
\begin{equation}
	pV=NRT
\end{equation}
\begin{figure}[h]
	\includegraphics[width=7cm]{l3-phase.pdf}
\end{figure}

\end{frame}


\begin{frame}
\frametitle{Уравнения состояния}
Уравнение Ван-дер-Ваальса:
\begin{equation}
p=\dfrac{RT}{V_M -b}-\dfrac{a}{V_M^2}
\end{equation}
$V_M$ --- молярный объем; $a$ --- параметр, учитывающий притяжение молекул; $b$ --- параметр, учитывающий размер молекул.

\begin{figure}[h]
	\includegraphics[width=4cm]{l3-vdw.png}
\end{figure}

\end{frame}

\begin{frame}
\frametitle{Условия поведения в критической точке}
\begin{equation*}
	\dfrac{\partial  p(T,V_M)}{\partial V_M}=0
\end{equation*}
\begin{equation*}
\dfrac{\partial^2  p(T,V_M)}{\partial V_M^2}=0
\end{equation*}


Критические свойства по уравнению Ван-дер-Ваальса
\begin{equation*}
	V_c=3b
\end{equation*}
\begin{equation*}
	p_c = \dfrac{a}{27 b^2}
\end{equation*}
\begin{equation*}
	T_c=\dfrac{8 a}{27 R b}
\end{equation*}


\end{frame}

\begin{frame}
\frametitle{Теория соответственных состояний}
\begin{figure}[h]
	\includegraphics[width=8cm]{l3-phase2.png}
\end{figure}
$p^*=\dfrac{p}{p_c}$;
$T^*=\dfrac{T}{T_c}$;
$V_M^*=\dfrac{V_M}{V_c}$;
$p^*=\dfrac{8T^*}{3V_M^*-1}-\dfrac{3}{V_M^{*2}}$


\end{frame}


\begin{frame}
Уравнение состояния Редлиха-Квонга:
\begin{equation}
p=\dfrac{RT}{V_M -b}-\dfrac{a}{T^{0.5} V_M (V_M+b)}
\end{equation}

Уравнение Бенедикта-Вэбба-Руббина:
\begin{equation}
	p= RT \rho + \left( B_0 RT - A_0 - \dfrac{C_0} {T^2} \right) \rho^2 - (b RT -a )\rho^3+a \alpha \rho^6 + 
	\dfrac { c \rho^3 }  {T^2} ( 1+\gamma \rho^2 ) exp ( - \gamma \rho^2 )
\end{equation}

\end{frame}

\begin{frame}
\frametitle{Уравнение состояния воды}
\begin{figure}[h]
	\includegraphics[width=10cm]{l3-wagner.png}
\end{figure}

\end{frame}

\begin{frame}
\frametitle{Вириальное уравнение состояния:}
\begin{equation}
	p=\dfrac{RT}{V_M}+ B_2(T) \dfrac{1}{V_M^2}+B_3(T) \dfrac{1}{V_M^3} + ...
\end{equation}

B – вириальный коэффициент
\end{frame}

\begin{frame}
\frametitle{Параметры смешения}
$a_m=f_1(x_1,x_2,x_3,...,x_{n-1})$ \quad
$b_m=f_2(x_1,x_2,x_3,...,x_{n-1})$

\begin{equation}
	a_m=\sum_{i=1}^{n} x_i a_i^k
\end{equation}

<<геометрические>> параметры
\begin{equation}
	a_{ij}=\eta \dfrac{a_i+ a_j}{2}
\end{equation}

<<энергетические>> параметры
\begin{equation}
	a_{ij}=\zeta \sqrt{a_i a_j}
\end{equation}

\end{frame}

\begin{frame}
\frametitle{Условия равновесия фаз}
$T'=T''$

$p'=p''$

$\mu_1'=\mu_1''$

$\mu_2'=\mu_2''$

Правило фаз Гиббса:
\begin{equation}
	C=n-\Phi+2
\end{equation}

$C$ --- число степеней свободы,$ n$ --- количество компонентов, $\Phi$ --- количество фаз
\end{frame}

\begin{frame}
$\mu=\dfrac{G}{N}$

$\dfrac{\partial G}{\partial p} = V$

$\int_{\mu_0}^{\mu_1} d \mu = \dfrac{1}{N} \int_{p_0}{p} V dp $

Для идеального газа:
$\mu( T,p ) -\mu_0( T,p_0 )=RT \int_{p_0}^p \dfrac {1}{p} dp = RT ln( p / p_0 )$

Химический потенциал идеального газа:
$\mu( T ) =\mu_0( T )+ RT ln( p )$
Химический потенциал смеси идеальных газов:
$\mu_i( T,P,x ) =\mu_{i0}( T )+ RT ln( p_i )=\mu_{i0}( T )+ RT ln( p x_i )$
\end{frame}

\begin{frame}
\frametitle{Реальный газ}
фугитивность (летучесть) чистого вещества:
$f=p \gamma_f$

фугитивность смеси:
$f_i= p \gamma_i$
,где $\gamma$ --- коэффициент фугитивности

$\mu( T, p ) =\mu^0( T )+ RT \ln( f )$

химический потенциал газовой фазы:
$\mu_i( T, p,x ) =\mu^0_i( T )+ RT \ln( f_i )$

химический потенциал жидкой фазы:
$\mu_i( T, p,x ) =\mu^0_i( T,P )+ RT \ln( a_i )=\mu^0_i( T,P )+ RT \ln( \gamma_i x )$
\end{frame}

\begin{frame}
	$F=\mu N -pV$
	
	$\ln( f )=\ln( p \gamma_f ) = \dfrac { \mu -\mu^0 } {RT} = \dfrac{ F-F^0 } { NRT } + \dfrac{ pV- (pV)^0 } {NRT}$
	
	$pV=Z_V N R T$, где $Z_V =f( T, V )$ --- фактор сжимаемости
	
	$p= - \left. \dfrac{ \partial F } { \partial V }  \right|  _{ T,N_i }$
	
	$F-F^0=-NRT \int_{V_0}^V \dfrac{Z_V}{V} dV=-NRT \left( \int_\infty^V {\dfrac{Z_V-1}{V} dV} +{\int_{V_0}^V {\dfrac{1} {V} dV}}   \right)$
	
	$F-F^0=-NRT \int_{V_0}^V \dfrac{Z_V}{V} dV=-NRT \left( \int_\infty^V {\dfrac{Z_V-1}{V} dV} +\ln \dfrac{Z_V}{p}  \right)$
	
		
\end{frame}

\begin{frame}

$\ln( \gamma_f ) = ( Z_V -1 ) - \int_\infty^V { \dfrac{Z_V -1} {V} dV } -ln( Z_V )$

$Z_V = \dfrac{p(T,V) V} { N R T}$

для многокомпонентной смеси:
$\ln( \gamma_{f_i} ) = -\int_\infty^ V \left( \dfrac{1-x_i}{RNT} \left. \dfrac{\partial p}{\partial x_i}  \right|_{V,T} - \dfrac{1} {V}   \right ) dV - \ln( Z_V )$

$\underbrace{\mu^0_i(T)+RT \ln(f_i)}_ {vapor} = \underbrace{\mu^0_i(T,P)+RT \ln(a_i)}_{liquid} $


$y_i=\dfrac{ f_i^0 \gamma_i x_i } { \gamma_{f_i} p  }$

$x$ --- доля компонента в жидкости, $y$ --- доля компонента в паре.

$y_i=m x_i$

\end{frame}


\begin{frame}

	\begin{equation}
	\left\lbrace 
	\begin{gathered} 
	p^{I}(T,x_1,x_2,...,x_{n-1})=p^{II}(T,y_1,y_2,...,y_n-1),\\
	y_i=\dfrac{ f_i^0 \gamma_i x_i } { \gamma_{f_i} p  } \quad i=1..n \\
	\sum y =1
	\end{gathered} 
	\right.
	\end{equation}
Определение фазового состава:	
	\begin{equation}
	\left\lbrace 
	\begin{gathered} 
	y_i=\dfrac{ f_i^0 \gamma_i x_i } { \gamma_{f_i} p  } \quad i=1..n \\
	\sum y =1
	\end{gathered} 
	\right.
	\end{equation}
\end{frame}

\begin{frame}
	\frametitle{Критерии моделей расчета физико-химических свойств}
	\begin{itemize}
		\item Выдавать надежные результаты для чистых веществ и их смесей во всей области состояний
		\item Адекватно воспроизводить фазовое состояние
		\item Требовать минимальное количество экспериментальных данных
		\item Обладать невысокой вычислительной трудоемкостью
	\end{itemize}
\end{frame}


\begin{frame}
	\frametitle{Расчет термодинамических свойств на основе избыточных функций}
	Функция смешения:
	\begin{equation}
		A^M ( p, T, x_1 , x_2 , ... , x_{n-1} )=A( p,T, x_1 , x_2 , ... , x_{n-1} )- \sum_{i=1}^n x_i A_i^0 ( T,p )
	\end{equation}
		
	Энергия Гиббса смешения:
	\begin{equation}
		\dfrac{G^M} {NRT} = \dfrac{ \sum x_i \mu_i -\sum x_i \mu_i^0 } { RT} = \sum x_i \ln( x_i \gamma_i )
	\end{equation}
	
	
	Избыточная функция:
	\begin{equation}
		A^E ( p, T, x_1 , x_2 , ... , x_{n-1} )=A^M( p,T, x_1 , x_2 , ... , x_{n-1} )- A^M_{id}( p,T, x_1 , x_2 , ... , x_{n-1} )
	\end{equation}
	
\end{frame}

\begin{frame}
	Избыточная энергия Гиббса:
\begin{equation}
\dfrac {G^E}{NRT} = { \sum x_i ln(x_i \gamma_i) - \sum x_i }  = \sum x_i \ln( \gamma_i )
\end{equation}
Требования для модели избыточной энергии Гиббса:
\begin{itemize}
\item Асимптотика поведения $G^E=0$ при $x_1=0$ и $x_2=0$
\item Соблюдение уравнения Гиббса-Дюгема $S dT - V dp + \sum_{ i=1}^n N_i d \mu_i =0$


\end{itemize}
\end{frame}

\begin{frame}
	\begin{figure}[h]
		\includegraphics[width=9cm]{gibbsE.jpg}
	\end{figure}
	
	Excess Gibbs’ Free energy  with mole fraction for 1-iodobutane + benzene ($\square$), 1-iodobutane + toluene ($\circ$), 1-iodobutane + o-xylene ($\triangle$), 1-iodobutane + m-xylene ($\nabla$), 1-iodobutane + p-xylene ($\diamond$), and 1-iodobutane + mesitylene  at 308.15 K.
\end{frame}

\begin{frame}
	\frametitle{Модели активности}
	Модель Маргулиса:
	
	$\dfrac{G^E}{NRT} = A_{12} x_1 x_2$
	$\ln( \gamma_1 )=A_{12} x_2^2$
	$\ln( \gamma_2 )=A_{12} x_1^2$
	
	Модель Вильсона:
	
	$\dfrac{G^E} {NRT} = -\sum_{i=1}^k x_i \ln( \sum_{j=1}^k x_i \lambda_{ij} ) $
	$\lambda_{ij} = \dfrac {V_j}{V_i} e^{ \dfrac{-C_{ij}} {RT} }$
	$\ln( \gamma_i )=1-\ln \left( \sum_{j=1}^k x_j \lambda_{ij} \right) - \sum_{m=1}^k \left(  \dfrac{x_m \lambda_{mi}}{\sum_{j=1}^k x_j \lambda_{mj} } \right)$
	
\end{frame}

\begin{frame}
	\frametitle{Unifac}
	UNIQUAC Functional-group Activity Coefficients
	
	\begin{figure}[h]
		\includegraphics[width=9cm]{l3-unifaq.jpg}
	\end{figure}
\end{frame}

\begin{frame}
	\frametitle{Аппроксимация давления насыщенных паров}
	Соотношение Клапйерона-Клаузиуса:
	
	$\dfrac{\partial p^0( T )} { \partial T } = \dfrac{ \Delta H } { T \Delta V }$
	$ \Delta H = H_V -H_L$
	$ \Delta V = V_V -V_L$
	
	$\dfrac{\partial \ln( p^0( T ))} { \partial T } = \dfrac{ \Delta H } { R \Delta Z_V }$
	
	Уравнение Клапейрона:
	 $\ln( p^0( T ))  = A + \dfrac{B} {T}$
	 
	 Уравнение Антуана:
	 $ \ln( p^0( T ))  = A - \dfrac{B} {C+T}$
	 
	 Уравнение Риделя:
	 $ \ln( p^0( T ))  = A - \dfrac{B} {T} + C \ln( T ) + D T^E$
\end{frame}


\begin{frame}
	\frametitle{Известно $x_1$-$x_n,$ T }
	\begin{itemize}
		\item Для заданной температуры рассчитываются давления паров чистых компонентов $p^0_i$ , например по уравнению Антуана
		\item Для заданного состава и температуры рассчитываются коэффициенты активности по выбранной модели. Несмотря на то, что коэффициенты активности зависят от состава, температуры и давления, на практике используются модели не учитывающие влияние давления
		\item Рассчитывают парциальные давления компонентов смеси $p_i = p_i^0( T ) /\gamma_i x_i$. И полное давление в системе: $p= \sum p_i$
		\item Определяют равновесные концентрации в паровой фазе: $y_i = \dfrac{p_i} {p}$
	\end{itemize}
\end{frame}

\begin{frame}
	\frametitle{Известно $x_1$-$x_n$, $p_{cur}$ }
	\begin{itemize}
		\item Задается начальное приближение по температуре
		\item Проводятся расчеты по первым трем пунктам предыдущего алгоритма до выполнения условия $p=p_{cur}$
		\item При невыполнении условия изменяют приближение температуры и делают новую итерацию
		\item Определяют равновесные концентрации в паровой фазе: $y_i = \dfrac {p_i} {p}$
		
	\end{itemize}
	
\end{frame}

\begin{frame}
	\frametitle{Химическое равновесие}
	$v_A A +v_B B \leftrightarrow v_C C$
	
	Химическое сродство:
	$C=v_A \mu_A +v_B \mu_B -v_c \mu_C $
	
	$v_A (\mu^0_A(T) +RT \ln(a_A)) v_B (\mu^0_B(T) + RT \ln(a_B)) -v_C (\mu^0_C(T) +RT \ln(a_C))  =0$
	
	$\dfrac { a_C^{v_C} } { a_A^{v_A} a_B^{v_B} } = exp( v_A \mu_A^0 +v_B \mu_B^0 -v_c \mu_C^0 )=K( T )$
	
	Вариант 2
	
	$r_1=k_1 c_A c_B$
	
	$r_2=k_2 c_C$
	
	$r_1=r_2=k_2 c_C=k_1 c_A c_B$
	
	$K( T )=\dfrac {k_1} {k_2}$
\end{frame}
