\lecture{Лекция 5}{lec5}
\section{Моделирование теплообменных процессов}

\begin{frame}
	\frametitle{Теплообмен}
	Задача: определение поле температуры теплоносителей $T(\tau, x, y, z)$
	
	Теплообменник типа труба в трубе:
	\begin{figure}[h]
		\includegraphics[width=8cm]{l5-tepl1.png}
	\end{figure}
\end{frame}

\begin{frame}
\frametitle{Уравнение теплового баланса}
Уравнение теплового баланса для первого горячего теплоносителя:
\begin{equation}
	\upsilon_1 T_1 \rho_1 c_{p1} S_1 - \upsilon_1 (T_1-dT_1) \rho_1 c_{p1} S_1 -q_{12} dF =0
\end{equation}
Элементарный объем: $dV_1=S_1 dx$

Тепловой поток межфазной поверхности:
\begin{equation}
	q_{12}=K ( T_1 - T_2 )
\end{equation}
K --- коэффициент теплопередачи

\begin{equation}
	\upsilon_1  \dfrac{ dT_1 }  { dx } = - \dfrac { K (T_1 -T_2)  }  { \rho_1 c_{p1} } \dfrac { d F } { S_1 dx }
\end{equation}
$ \dfrac {dF} { d x } = \dfrac { P L } {L} = P$
, где $P$ --- периметр теплообмена
\end{frame}

\begin{frame}
Система уравнений для двух теплоносителей:

\begin{equation}
\left\lbrace 
\begin{gathered} 
\dfrac {dT_1} {dx} = \dfrac {-K (T_1-T_2)} {G_1 c_{p1}} P \\
\dfrac {dT_2} {dx} = \dfrac {K (T_1-T_2)} {G_2 c_{p2}} P
\end{gathered} 
\right.
\end{equation}
где $G$ --- массовый расход
Граничные условия:
\begin{itemize}
\item Прямоток (задача Коши)
$T_1( 0 ) =T_{1 in}$,  $T_2( 0 ) =T_{2 in}$
\item Противоток (краевая задача)
$T_1( 0 ) =T_{1 in}$,  $T_1( L ) =T_{1 in}$

\end{itemize}
\end{frame}

\begin{frame}
	\frametitle{Противоток}
	Тепловой баланс второго потока в случае противотока:
	\begin{equation}
	-\upsilon_2 T_2 \rho_2 c_{p2} S_2 + \upsilon_2 (T_2-d T_2) \rho_2 c_{p2} S_2 +q_{12} dF =0
	\end{equation}
	
	\begin{equation}
	\left\lbrace 
	\begin{gathered} 
	\dfrac {dT_1} {dx} = \dfrac {-K (T_1-T_2)} {G_1 c_{p1}} P \\
	\dfrac {dT_2} {dx} = \dfrac {-K (T_1-T_2)} {G_2 c_{p2}} P
	\end{gathered} 
	\right.
	\end{equation}
	
\end{frame}

\begin{frame}
\frametitle{Диффузионная модель}
Уравнение теплового баланса для диффузионной модели:
%\begin{equation}
	\begin{multline}
	\upsilon_1 T_1 \rho_1 c_{p1} S_1 -D_L \dfrac {dT} {dx} S_1 \rho_1 c_{p1} - \upsilon_1 (T_1-dT_1) \rho_1 c_{p1} S_1 + \\   +D_L \dfrac {d(T_1+dT_1)} {dx} S_1 \rho_1 c_{p1}-q_{12} dF =0
	\end{multline}
%\end{equation}

\begin{equation}
\left\lbrace 
\begin{gathered} 
\upsilon_1 \dfrac {dT_1} {dx} = D_L1 \dfrac {d^2 T_1} {dx^2} - \dfrac {K(T_1-T_2) P} {\rho_1 c_{p1} S_1} \\
\upsilon_2 \dfrac {dT_1} {dx} = D_L2 \dfrac {d^2 T_2} {dx^2} + \dfrac {K(T_1-T_2) P} {\rho_2 c_{p2} S_2} 
\end{gathered} 
\right.
\end{equation}

\end{frame}

\begin{frame}
	\frametitle{Граничные условия}
Для решения системы записываются 2 граничных условия для каждого уравнения:
 
$x=0$ ; \quad $\upsilon_1 ( T_{in} - T_1 ) = -D_{L1} \left.  \dfrac { d T_1 } { dx }  \right| _{ x=0 }$

$x=L$ \quad $ \left. \dfrac { d T_1 } { dx }  \right|_{ x=L } =0$


\begin{figure}[h]
	\includegraphics[width=8cm]{l5-gr1.png}
\end{figure}
\end{frame}

\begin{frame}
	\frametitle{Противоток}
	\begin{equation}
	\left\lbrace 
	\begin{gathered} 
	\upsilon_1 \dfrac {dT_1} {dx} = D_L1 \dfrac {d^2 T_1} {dx^2} - \dfrac {K(T_1-T_2) P} {\rho_1 c_{p1} S_1} \\
	\upsilon_2 \dfrac {dT_1} {dx} = - D_L2 \dfrac {d^2 T_2} {dx^2} - \dfrac {K(T_1-T_2) P} {\rho_2 c_{p2} S_2} 
	\end{gathered} 
	\right.
	\end{equation}
	Для решения системы записываются 2 граничных условия для каждого уравнения:
	
	$x=0$ ; \quad $\upsilon_1 ( T_{in1} - T_1 ) = -D_{L1} \left.  \dfrac { d T_1 } { dx }  \right| _{ x=0 }$
	
	$x=L$ \quad $ \left. \dfrac { d T_1 } { dx }  \right|_{ x=L } =0$
	
	$x=L$ ; \quad $\upsilon_2 ( T_{in2} - T_2 ) = -D_{L2} \left.  \dfrac { d T_2 } { dx }  \right| _{ x=L }$
	
	$x=0$ \quad $ \left. \dfrac { d T_2 } { dx }  \right|_{ x=0 } =0$
	
\end{frame}

\begin{frame}
	\frametitle{МИС}
	Уравнение теплового баланса для МИС:
	\begin{equation}
		T_1-T_{in 1}=\dfrac { K(T_1 - T_{P} )F} { G_1 c_{p1} }
	\end{equation}
	
	\begin{figure}[h]
		\includegraphics[width=5cm]{l5-mis.png}
	\end{figure}
\end{frame}

\begin{frame}
	\frametitle{Ячеечная модель}
	Уравнение теплового баланса для ячеечной модели:
	\begin{equation}
		G c_p ( T_{1,j} - T_{1,j-1} ) = K_j \dfrac {F} {m} ( T_P - T_{1,j} ) \quad j=1,m
	\end{equation}
\end{frame}


\begin{frame}
	\frametitle{Сравнение различных моделей}
	\begin{figure}[h]
	\includegraphics[width=10cm]{l5-models.png}
	\end{figure}
\end{frame}

\begin{frame}
	\frametitle{Коэффициент теплопередачи}
	\begin{equation}
		K = \dfrac {q}  { F \Delta T }
	\end{equation}
	\begin{equation}
		K = \dfrac {1} { \dfrac{1} {\alpha_1} + \sum{\dfrac {\delta} {\lambda}} + \dfrac {1} {\alpha_2} }
	\end{equation}
	, где $\alpha$ --- коэффициент теплоотдачи
	
	Коэффициент теплоотдачи зависит от:
	\begin{itemize}
		\item Режима течения $Re$, (скорость, вязкость, форма аппарата)
		\item Теплофизических свойств $Pr$ (теплоемкость, теплопроводность …)
		\item Геометрии аппарата
	\end{itemize}
	
\end{frame}

\begin{frame}
	Коэффициент теплоотдачи:
	\begin{equation}
		Nu = \dfrac { \alpha l } {\lambda}
	\end{equation}
	Если движение в трубе (канале) носит характер переходного режима, т.е. $Re = 2300 - 10000$, то критерий Нуссельта:
	
	Для ламинарного движения ( Re < 2300):
	
	$Nu = 0.08 Re^{0.9} Pr^{0.43}$
	
	Для ламинарного движения ( Re < 2300):
	$Nu =a ( Re Pr )^{0.2} ( Gr Pr )^{0.1}$
	где $a$ --- множитель (для горизонтальных труб d = 0,74; для вертикальных труб a= 0,85)
\end{frame}