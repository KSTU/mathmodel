% Этот шаблон документа разработан в 2014 году
% Данилом Фёдоровых (danil@fedorovykh.ru) 
% для использования в курсе 
% <<Документы и презентации в \LaTeX>>, записанном НИУ ВШЭ
% для Coursera.org: http://coursera.org/course/latex .
% Исходная версия шаблона --- 
% https://www.writelatex.com/coursera/latex/5.1

\documentclass[t]{beamer}  % [t], [c], или [b] --- вертикальное выравнивание на слайдах (верх, центр, низ)

%usetheme{Rochester} % Тема оформления
\usetheme{metropolis}
\usecolortheme{default} % Цветовая схема

%%% Работа с русским языком
\usepackage{cmap}					% поиск в PDF
\usepackage{mathtext} 				% русские буквы в формулах
\usepackage[T2A]{fontenc}			% кодировка
\usepackage[utf8]{inputenc}			% кодировка исходного текста
\usepackage[english,russian]{babel}	% локализация и переносы

%% Beamer по-русски\texttt{}
\newtheorem{rtheorem}{Теорема}
\newtheorem{rproof}{Доказательство}
\newtheorem{rexample}{Пример}

%%% Дополнительная работа с математикой
\usepackage{amsmath,amsfonts,amssymb,amsthm,mathtools} % AMS
\usepackage{icomma} % "Умная" запятая: $0,2$ --- число, $0, 2$ --- перечисление

%% Свои команды
\DeclareMathOperator{\sgn}{\mathop{sgn}}

%% Перенос знаков в формулах (по Львовскому)
\newcommand*{\hm}[1]{#1\nobreak\discretionary{}
{\hbox{$\mathsurround=0pt #1$}}{}}

%%% Работа с картинками
\usepackage{graphicx}  % Для вставки рисунков
\graphicspath{{figures/lect1/}{figures/lect2/}{figures/lect3/}{figures/lect4/}{figures/lect5/}}  % папки с картинками
\setlength\fboxsep{3pt} % Отступ рамки \fbox{} от рисунка
\setlength\fboxrule{1pt} % Толщина линий рамки \fbox{}
\usepackage{wrapfig} % Обтекание рисунков текстом

%%% Работа с таблицами
\usepackage{array,tabularx,tabulary,booktabs} % Дополнительная работа с таблицами
\usepackage{longtable}  % Длинные таблицы
\usepackage{multirow} % Слияние строк в таблице

%%% Программирование
\usepackage{etoolbox} % логические операторы

%%% Другие пакеты
\usepackage{lastpage} % Узнать, сколько всего страниц в документе.
\usepackage{soul} % Модификаторы начертания
\usepackage{csquotes} % Еще инструменты для ссылок
%\usepackage[style=authoryear,maxcitenames=2,backend=biber,sorting=nty]{biblatex}
\usepackage{multicol} % Несколько колонок

%%% Картинки
\usepackage{tikz} % Работа с графикой
\usepackage{pgfplots}
\usepackage{pgfplotstable}
\usetikzlibrary{arrows.meta}
\tikzset{%
	>={Latex[width=2mm,length=2mm]},
	% Specifications for style of nodes:
	base/.style = {rectangle, rounded corners, draw=black,
		minimum width=4cm, minimum height=1cm,
		text centered, font=\sffamily},
	activityStarts/.style = {base, fill=blue!30},
	startstop/.style = {base, fill=red!30},
	activityRuns/.style = {base, fill=green!30},
	process/.style = {base, minimum width=2.5cm, fill=orange!15,
		font=\ttfamily},
}

\usetikzlibrary{arrows,positioning} %
%\usepackage[active,tightpage]{preview}%
%\PreviewEnvironment{tikzpicture}%
%\setlength\PreviewBorder{1em}%

\title{Моделирование Химико-Технологических Процессов}
\author{к.т.н., доцент Анашкин Иван Петрович}
\date{\today}

\begin{document}

\frame[plain]{\titlepage}	% Титульный слайд

\lecture{Лекция 1}{lec1}
\section{Введение}

\begin{frame}
\frametitle{Цели освоения дисциплины}
	\begin{itemize}
		\item освоение методов построения математических моделей основных тепло- и массообменных процессов, а также сопряженных и совмещенных процессов; 
		\item изучение алгоритмов идентификации параметров математических моделей и способов проверки их адекватности; 
		\item освоение специализированных программно-вычислительных комплексов, позволяющих решать задачи моделирования химико-технологических процессов.
	\end{itemize}
\end{frame}

\begin{frame}
\frametitle{Сопутствующие дисциплины}
	\begin{itemize}
		\item Математика
		\item Физика
		\item Физическая химия
		\item Процессы и аппараты химической технологии
		\item Программирование
	\end{itemize}
	\begin{figure}[h]
		\includegraphics[width=5cm]{puzzle.jpg}
	\end{figure}
\end{frame}

\begin{frame}
	\frametitle{Знания. Умения. Навыки}
	 Знать:
			\begin{itemize}
				\item методологические основы построения математических моделей процессов химической технологии;
				\item основные математические модели описания гидродинамической структуры потоков в аппаратах и физико-химических свойств рабочих агентов;
				\item методы идентификации параметров моделей и проверки их адекватности.
			\end{itemize}
	 
\end{frame}

\begin{frame}
	\frametitle{Знания. Умения. Навыки}
	Уметь:
	\begin{itemize}
		\item составлять математические модели процессов тепломассопереноса, осложненных химической реакцией;
		\item проводить идентификацию параметров модели и оценивать ее адекватность.
	\end{itemize}
	Владеть:
	\begin{itemize}
		\item навыками работы в программных продуктах, позволяющих решать задачи моделирования химико-технологических процессов.
	\end{itemize}
\end{frame}

\begin{frame}
	\begin{figure}[h]
		\includegraphics[width=0.8\paperwidth]{funny1.png}
	\end{figure}
\end{frame}

\begin{frame}
	\frametitle{MathCad}
	\begin{itemize}
		\item WYSIWYG (What You See Is What You Get — «что видишь, то и получаешь»)
		\item Простота для конечного пользователя
		\item Проприетарный
		\item Нет мультиплатформенности
	\end{itemize}
	
\end{frame}

\begin{frame}
	\frametitle{Литература}
	\begin{itemize}
		\item Клинов, А.В. Математическое моделирование химико-технологических процессов: учебное пособие / А.В. Клинов, А.Г. Мухаметзянова. – Казань: Изд-во КГТУ, 2009. – 136 с.
		\item Клинов, А.В. Лабораторный практикум по математическому моделированию химико-технологических процессов: учебное пособие / А.В. Клинов, А.В. Малыгин. – Казань: Изд-во КГТУ, 2011. – 100 с.
		\item Закгейм, А.Ю. Общая химическая технология: введение в моделирование химико-технологических процессов: учебное пособие / А.Ю. Закгейм. – М.: Логос, 2012. – 304 с.
	\end{itemize}
\end{frame}

\begin{frame}
	\frametitle{Литература}
	\begin{itemize}
		\item Кафаров, В.В. Математическое моделирование основных процессов химических производств: учебное пособие для ВУЗов / В.В. Кафаров, М.Б. Глебов. – М.: Высш.шк, 1991. – 400 с.
	\end{itemize}
\end{frame}

\begin{frame}
	\frametitle{Модель}
	Условный образ объекта исследования, конструируемый исследователем так, чтобы отобразить \textbf{основные} характеристики и существенные особенности его поведения.
\end{frame}

\begin{frame}
	\frametitle{Виды моделей}
	Концептуальные модели: 	совокупность уже известных фактов или представлений относительно исследуемого объекта или системы истолковывается с помощью некоторых специальных знаков, символов, операций над ними или с помощью естественного или искусственного языков;
	
	
	Физические модели: 	модель и моделируемый объект представляют собой реальные объекты или процессы единой или различной физической природы, причем между процессами в объекте-оригинале и в модели выполняются некоторые соотношения подобия, вытекающие из схожести физических явлений;
\end{frame}

\begin{frame}
	\frametitle{Физические модели}
	\begin{figure}
		\centering
		\begin{minipage}{0.45\textwidth}
			\centering
			\includegraphics[width=0.9\textwidth]{mgu1.png} % first figure itself
			\caption{Оригинал}
		\end{minipage}\hfill
		\begin{minipage}{0.45\textwidth}
			\centering
			\includegraphics[width=0.9\textwidth]{mgu3.jpg} % second figure itself
			\caption{Модель}
		\end{minipage}
	\end{figure}
\end{frame}


\begin{frame}
	\frametitle{Гидравлический компьютер Лукьянова}
	\begin{figure}
		\centering
		\begin{minipage}{0.45\textwidth}
			Дифференциальные уравнения
		\end{minipage}\hfill
		\begin{minipage}{0.45\textwidth}
			\centering
			\includegraphics[width=0.9\textwidth]{gcpu.jpg} % second figure itself
			\caption{Модель}
		\end{minipage}
	\end{figure}
\end{frame}

\begin{frame}
	\frametitle{Физические модели}
	\begin{figure}
		\centering
		\begin{minipage}{0.45\textwidth}
			\centering
			\includegraphics[width=0.9\textwidth]{mesh2.png} % first figure itself
			\caption{Оригинал}
		\end{minipage}\hfill
		\begin{minipage}{0.45\textwidth}
			\centering
			\includegraphics[width=0.9\textwidth]{mesh1.png} % second figure itself
			\caption{Модель}
		\end{minipage}
	\end{figure}
\end{frame}

\begin{frame}
	\frametitle{Структурно-функциональные модели}
	моделями являются схемы (блок-схемы), графики, чертежи, диаграммы, таблицы, рисунки, дополненные специальными правилами их объединения и преобразования
	\begin{figure}[h]
		\includegraphics[width=6cm]{schem1.png}
	\end{figure}
\end{frame}

\begin{frame}
	\frametitle{Структурно-функциональные модели}
	\begin{figure}[h]
		\includegraphics[width=0.9\textwidth]{schem2.png}
	\end{figure}
\end{frame}

\begin{frame}
	\frametitle{Математические и иммитационные модели}
	\begin{itemize}
		\item математические модели – построение модели осуществляется средствами математики и логики;
		\item имитационные (программные) (simulation) модели - логико-математическая модель исследуемого объекта представляет собой алгоритм функционирования объекта, реализованный в виде программного комплекса для компьютера.
	\end{itemize}
	
\end{frame}


\begin{frame}
	\frametitle{Имитационные модели}
	\begin{itemize}
	\item Метод молекулярной динамики
	\item Метод Монте-Карло
	\item CFD (computation fluid dynamic) методы вычислительной гидродинамики
	\item Вычисления методом конечных объемов
	\item Квантово-химические методы
	\end{itemize}
\end{frame}

\begin{frame}
	\frametitle{Химико-технологические процессы (ХТП)}
	технологические процессы, связанные с физико-химической и химической переработкой реагентов в конечные продукты. ХТП – это сложная физико-химическая система, которая имеет двойственную детерминированную и стохастическую (случайную) структуру, переменную, как во времени, так и в пространстве.
\end{frame}

\begin{frame}
	\frametitle{«Элементарные» процессы протекающие в ХТП}
\begin{itemize}
	\item Гидромеханические
	\item Тепловые
	\item Массообменные
	\item Механические
	\item Химические
\end{itemize}
\end{frame}

\begin{frame}
\frametitle{Основные этапы моделирования ХТП}
	\begin{itemize}
	\item Постановка цели моделирования --- выявление основных исследуемых характеристик, постановка задачи и вид результатов моделирования
	\item Построение модели --- установление протекающих процессов, упрощение модели путем сокращения описания процессов, выведение уравнений/систем равнений.
	\item Идентификация модели --- определение параметров модели и вычисление их значений на основании экспериментальных данных
\end{itemize}
\end{frame}

\begin{frame}
	\frametitle{Основные этапы моделирования ХТП}
	\begin{itemize}
		\item Проверка адекватности модели на основании сравнения с экспериментальными данными, проведение корректировки модели или параметров моделирования.
		\item Использование модели для исследования свойств и поведения объекта моделирования
	\end{itemize}
\end{frame}

\begin{frame}
	\frametitle{Задачи моделирования ХТП}
	\begin{itemize}
		\item исследование новых процессов;
		\item проектирование производств;
		\item оптимизация отдельных аппаратов и технологических схем;
		\item выявление резервов мощности и отыскание наиболее
		\item эффективных способов модернизации действующих
		производств; 
		\item оптимальное планирование производств; 
		\item 	разработка автоматизированных систем управления
		\item проектируемыми производствами.
	\end{itemize}
	
\end{frame}

\begin{frame}
	\frametitle{Преимущества математического моделирования}
	\begin{itemize}
		\item позволяет осуществить с помощью одного устройства (ЭВМ) решение целого класса задач, имеющих одинаковое математическое описание;
		\item обеспечивает простоту перехода от одной задачи к другой, позволяет вводить переменные параметры, возмущения и различные условия однозначности (моделировать различные по размерам объекты и различные свойства);
		\item дает возможность проводить моделирование по частям ("элементарным процессам"), что особенно существенно при исследовании сложных объектов химической технологии;
		\item экономичнее метода физического моделирования.
	\end{itemize}
	
\end{frame}

\begin{frame}
	\frametitle{Классификация математических моделей}
	\begin{itemize}
		\item Статические – инвариантны ко времени; Пример: аппарат полного смешения некоторого объема в установившемся режиме работы, в который непрерывно подаются реагенты А и В и отводится продукт реакции Р.
		\item Динамические – являются функцией времени; Пример: аппарат полного смешения некоторого объема в неустановившемся режиме работы.
	\end{itemize}
\end{frame}

\begin{frame}
	\frametitle{Классификация математических моделей}
	\begin{itemize}
		\item С сосредоточенными параметрами – постоянство переменных в пространстве (алгебраические, либо дифференциальные уравнения первого порядка для нестационарных процессов). Пример: аппарат с полным перемешиванием потока, концентрация во всех точках одинакова;
		\item С распределенными параметрами – переменные изменяются в пространстве (дифференциальные уравнения с частными производными); Пример: трубчатый аппарат с большим отношением длины к диаметру и значительной скоростью движения реагентов.
	\end{itemize}
	
\end{frame}

\begin{frame}
	\frametitle{Основные методы построения математических моделей}
	\begin{itemize}
		\item Эмпирический (экспериментально-статистический, метод «черного ящика») --- не рассматривается природа происходящих процессов. Все явления происходящие в рамках химико-технологического процесса рассматриваются как «черный ящик»
		\item Теоретический (структурный) --- химико-теологический процесс разбивается на элементарные явления. Каждые из явлений описываются отдельными уравнениями.
	\end{itemize}
	
\end{frame}

\begin{frame}
	\frametitle{Блок-схема построения математической модели}
	\begin{tikzpicture}[node distance=1.5cm,
	every node/.style={fill=white, font=\sffamily}, align=center]
	% Specification of nodes (position, etc.)
	\node (object)     [process]          {Объект исследования};
	\node (think)      [process, below of=object]   {Мысленная модель};
	\node (math)      [process, below of=think]   {Математичская модель};
	\node (data)    [process, right of=object, xshift=4cm]{Экспериментальные данные};
	\node (results)      [process, below of=math]   {Решение на компьютере};
	\node (comp)      [process, right of=results, xshift=4cm]   {Сравнение};
	
	\draw[->]             (object) -- (data);
	\draw[->]             (object) -- (think);
	\draw[->]             (think) -- (math);
	\draw[->]             (math) -- (results);
	\draw[->]             (data) -- (comp);
	\draw[->]             (results) -- (comp);
	%\draw[->]             (comp) -- (think);
	%\draw[->]             (comp) -- (math);
	\draw[->] (comp.south) -- ++(0,-1) -- ++(-8.5,0) -- ++(0,2) -- ++(0,1) --                
	node[xshift=3.2cm,yshift=-2.5cm, text width=2.5cm]
	{плохое}(math.west);
	\draw[->] (comp.south) -- ++(0,-1) -- ++(-8.5,0) -- ++(0,2) -- ++(0,2.5) -- (think.west);
	\draw[->] (comp.east) -- ++(2,0) node[xshift=-1cm,yshift=0.4cm]{хорошее} -- ++(0,5.8) -- ++(-8,0) node[xshift=3.1cm,yshift=0.4cm]{управление, оптимизация, контроль} -- (object.north);
	
	\end{tikzpicture}
\end{frame}



\begin{frame}
	\frametitle{Пространственно временной масштаб моделей}
	\begin{figure}[h]
		\includegraphics[width=0.9\textwidth]{total.png}
	\end{figure}
\end{frame}


\lecture{Лекция 2}{lec2}
	\section{Эмпирический метод построения математического описания (метод черного ящика)}
	
\begin{frame}
	\frametitle{Область применения}
	\begin{itemize}
	\item Объект исследования малоизучен
	\item Природа объекта исследования не известна
	\item Действующий (функционирующий) объект
	\end{itemize}
\end{frame}

\begin{frame}
	\frametitle{Этапы составления  модели}
	\begin{itemize}
		\item Формирование цели, выбор факторов и переменных состояния, планирование эксперимента
		\item Проведение экспериментов, изучение реакции объекта на различные возмущения
		\item Статистическая обработка результатов
		\item Проведение исследований на основе полученной модели
	\end{itemize}
	
\end{frame}

\begin{frame}
	\frametitle{Схема внешних связей}
	\begin{tikzpicture}
	[auto,
	block/.style ={rectangle, draw=black, thick, fill=black!85, text width=8em, text centered, text=white, rounded corners, minimum height=10em},
	line/.style ={draw, thick, -latex}]
	
	
	
	\node(in1){X 1};
	\foreach\innum[count=\inlag] in {2,...,3}
	\node(in\innum)[below=of in\inlag]{X \innum};
	
	\node(out1)[right=7cm of in1]{Y 1};
	\foreach\outnum[count=\outlag] in {2,...,3}
	\node(out\outnum)[below=of out\outlag]{Y \outnum};
	
	\node(ik1)[above right = 1cm and 1.3cm of in1]{U 1};
	\foreach\outnum[count=\outlag] in {2,...,3}
	\node(ik\outnum)[right=of ik\outlag]{U \outnum};
	
	\node(iz1)[below right = 1cm and 1.3cm of in3]{Z 1};
	\foreach\outnum[count=\outlag] in {2,...,3}
	\node(iz\outnum)[right=of iz\outlag]{Z \outnum};
	
	\node[block] (bb) at (current bounding box.center) {Черный
		
		 ящик};
	
	\begin{scope}[every path/.style=line]
	\path (in1.east) -- ([yshift=3em]bb.west);
	\path (in2.east) -- ([yshift=0em]bb.west);
	\path (in3.east) -- ([yshift=-3em]bb.west);
	%\path (in2.east) -- ([yshift=0em]bb.west);
	%\path (in3.east) -- ([yshift=-1em]bb.west);
	%\path (in5.east) -- ([yshift=-2em]bb.west);
	%\path (in6.east) -- ([yshift=-3em]bb.west);
	
	\path ([yshift=3em]bb.east) -- (out1.west);
	\path ([yshift=0em]bb.east) -- (out2.west);
	\path ([yshift=-3em]bb.east) -- (out3.west);
	%\path ([yshift=-1em]bb.east) -- (out4.west);
	%\path ([yshift=-2em]bb.east) -- (out5.west);
	%\path ([yshift=-3em]bb.east) -- (out6.west);
	
	\path (ik1.south) -- ([xshift=-3em]bb.north);
	\path (ik2.south) -- ([xshift=0em]bb.north);
	\path (ik3.south) -- ([xshift=3em]bb.north);
	
	\path (iz1.north) -- ([xshift=-3em]bb.south);
	\path (iz2.north) -- ([xshift=0em]bb.south);
	\path (iz3.north) -- ([xshift=3em]bb.south);
	
	\end{scope}
	
	\end{tikzpicture}
\end{frame}

\begin{frame}
	$X$ --- факторы, которые можно контролировать и регулировать
	

	$U$ --- факторы, которые можно контролировать, но нельзя регулировать
	
	$Z$ --- факторы, которые нельзя контролировать и регулировать
	
	$Y$ --- выходные факторы, переменные состояния, функции отклика.
\end{frame}

\begin{frame}
	\frametitle{Переменные состояния}
	Экономические:
	\begin{itemize}
		\item Производительность
		\item Себестоимость
	\end{itemize}
	Технологические:
	\begin{itemize}
		\item Качество продукта
		\item Выход целевого продукта
	\end{itemize}
	
\end{frame}

\begin{frame}
	\frametitle{Требования при выборе переменной состояния}

	\begin{itemize}
		\item Должна иметь количественную характеристику
		\item Должна однозначно измерять эффективность объекта
		\item Должна быть статистически эффективной (обладать меньшей дисперсией)
		\item Должны иметь области определения заданное технологическими и принципиальными ограничениями
		\item Между факторами и переменными состояния должно существовать однозначное соответствие
	\end{itemize}
	
\end{frame}

\begin{frame}
	\frametitle{Уравнение математического описания}
	\begin{equation}
		\bar{Y}=F(U,X,Z)
	\end{equation}
	где $\bar{Y}=\dfrac{1}{n} \sum_{i=1}^n Y_1$, $n$ --- количество параллельных опытов
	\begin{equation}
	\bar{Y}=F(U,X)+f(Z)
	\end{equation}
	где $f()$ --- шум от внешних факторов
	\begin{equation}
	\bar{Y}=F(A,X)
	\end{equation}
	где $A=[a_1,a_2,a_3...a_m]$, $m$ --- количество параметров модели, $X=[x_1,x_2,x_3,..., x_k]$, k ---  количество входящих параметров
\end{frame}
\begin{frame}
	\frametitle{Функции линейные по параметрам}
	Линейные:
	\begin{equation}
	\bar{Y}=\sum_{i=1}^m a_i \phi_i(x)
	\end{equation}
	$\bar{Y}=a_1 +a_2 x$; $\bar{Y}=a_1 +a_2 \sin(\sqrt{x})$
	
	Нелинейные:
	\begin{equation}
	\bar{Y}=\sum_{i=1}^m \phi_i( a_i x)
	\end{equation}
	$\bar{Y}=a_1 +\sin(a_2 \sqrt{x})$; $\bar{Y}=\sin(a_1\sqrt{a_2 x})$;
	$\bar{Y}=\exp(a_1+a_2 x)$
\end{frame}

\begin{frame}
	\frametitle{Планирование и проведение эксперимента}
	\textbf{Пассивный эксперимент} проводится сбор и анализ информации об объекте без специального изменения входных параметров
	\begin{itemize}
		\item Отсутствуют затраты не эксперимент
		\item Небольшие колебания сходных характеристик
		\item Необходимо большое количество экспериментальных данных
	\end{itemize}
	
\end{frame}


\begin{frame}
	\frametitle{Планирование эксперимента. Активный эксперимент}
	\textbf{Активный эксперимент} состоит в целенаправленном изменении входных параметров технологического процесса
	\begin{itemize}
		\item Наглядность
		\item Простота интерпретации результатов
	\end{itemize}
	
\end{frame}

\begin{frame}
	\frametitle{Определение реакции на стандартные возмущения}
	При определении реакции на стандартные возмущения на вход подается какой-либо стандартный сигнал
	
	\includegraphics[width=0.9\textwidth]{l2_inp.png}
\end{frame}


\begin{frame}
	\frametitle{Статистическая обработка результатов}
	\begin{tabular}{ |p{1.2cm}|c|c|c|c|c|p{1.8cm}|p{1.7cm}| }
		\hline
		Номер серии опытов & \multicolumn{5}{ p{3cm}| }{Результаты параллельных 			 измерений} &Среднее значение & Дисперсия \\ \hline
		1 & $Y_{11}$ & $Y_{12}$ & $Y_{13}$ & $...$ & $Y_{1n}$ & $\bar{Y_1}$ & $\sigma_1^2$\\
		2 & $Y_{21}$ & $Y_{22}$ & $Y_{23}$ & $...$ & $Y_{2n}$ & $\bar{Y_2}$ & $\sigma_2^2$\\
		3 & $Y_{31}$ & $Y_{32}$ & $Y_{33}$ & $...$ & $Y_{3n}$ & $\bar{Y_3}$ & $\sigma_3^2$\\
		4 & $Y_{41}$ & $Y_{42}$ & $Y_{43}$ & $...$ & $Y_{4n}$ & $\bar{Y_4}$ & $\sigma_4^2$\\
		$...$ &   &   &   &   &   &   &  \\
		$m$ & $Y_{m1}$ & $Y_{m2}$ & $Y_{m3}$ & $...$ & $Y_{mn}$ & $\bar{Y_m}$ & $\sigma_m^2$\\ \hline
	\end{tabular}
\end{frame}

\begin{frame}
	\frametitle{Оценка воспроизводимости}
	Определяется среднее арифметическое:
	\begin{equation}
		\bar{Y_i}=\dfrac{1}{m} \sum_{j=1}^{m}Y_{ij}
	\end{equation}
	Определяется дисперсия каждой величины:
	\begin{equation}
		\sigma_i^2=\dfrac{1}{n-1}\sum_{j=1}^{m} (Y_{ij} - \bar{Y_i})^2
	\end{equation}
	Определяется значение критерия Кохрена:
	\begin{equation}
		G=\dfrac{max(\sigma_i^2)}{\sum_{j=1}^{m}\sigma_i^2}
	\end{equation}
\end{frame}


\begin{frame}
	\frametitle{Структурно-регрессионный анализ}
	Корреляция (от лат. correlatio «соотношение, взаимосвязь») или корреляционная зависимость --- это статистическая взаимосвязь двух или более случайных величин (либо величин, которые можно с некоторой допустимой степенью точности считать таковыми). При этом изменения значений одной или нескольких из этих величин сопутствуют систематическому изменению значений другой или других величин.
\end{frame}

\begin{frame}
	Ковариационный момент:
	\begin{equation}
		K_{xy}=\overline{(x-\overline{x})(y-\overline{y})}
	\end{equation}
	Коэффициент корреляции:
	\begin{equation}
		r_{xy}=\dfrac{K_{xy}}{\sigma_x \sigma_y}
	\end{equation}
\end{frame}

\begin{frame}
	\frametitle{Коэффициент корреляции}
	\begin{figure}
		\centering
		\begin{minipage}{0.45\textwidth}
			\centering
			\includegraphics[width=0.9\textwidth]{l2-corr1.pdf} % first figure itself
			%\caption{Оригинал}
		\end{minipage}\hfill
		\begin{minipage}{0.45\textwidth}
			\centering
			\includegraphics[width=0.9\textwidth]{l2-corr2.pdf} % second figure itself
			%\caption{Модель}
		\end{minipage}
	\end{figure}
\end{frame}

\begin{frame}
	\frametitle{Коэффициент корреляции}
		\begin{figure}
			\centering
			\begin{minipage}{0.45\textwidth}
				\centering
				\includegraphics[width=0.9\textwidth]{l2-corr3.pdf} % first figure itself
				%\caption{Оригинал}
			\end{minipage}\hfill
			\begin{minipage}{0.45\textwidth}
				\centering
				\includegraphics[width=0.9\textwidth]{l2-corr4.pdf} % second figure itself
				%\caption{Модель}
			\end{minipage}
		\end{figure}
\end{frame}

\begin{frame}
	\frametitle{Коэффициент корреляции}
	\begin{figure}
		\centering
			\includegraphics[width=0.9\textwidth]{l2-corr5.pdf} % first figure 
	\end{figure}
\end{frame}


\begin{frame}
	\frametitle{Коэффициент корреляции}
	\begin{figure}
		\centering
		\includegraphics[width=0.9\textwidth]{600px-Correlationexamples.png} % first figure 
	\end{figure}
\end{frame}

\begin{frame}
	\frametitle{Расчет коэффициентов регрессии}
	Метод наименьших квадратов:
	\begin{equation}
		F=\sum_{i=1}^{m} (y_i^э- y_i^р)^2 \rightarrow min
	\end{equation}
	
	\begin{equation}
	\left\lbrace 
	\begin{gathered} 
	\dfrac{\partial F}{\partial a_1}=0 \\
	\dfrac{\partial F}{\partial a_2}=0 \\
	...\\
	\dfrac{\partial F}{\partial a_N}=0 \\
	\end{gathered} 
	\right.
	\end{equation}
\end{frame}

\begin{frame}
	\frametitle{Расчет коэффициентов регрессии}
	\begin{equation}
		\dfrac{\partial F}{\partial a_k} = \dfrac{\sum_{i=1}^{m} \left(
			y_i^э -\sum_{j=1}^{N} a_j \phi_i(x_i)
			 \right)^2}{\partial a_k} = 0
	\end{equation}
	Дифференцируя:
	\begin{equation}
		2 \sum_{i=1}^{m} \left(
		y_i^э -\sum_{j=1}^{N} a_j \phi_i(x_i)
		\right) \phi_k(x_i)=0
	\end{equation}
	Перегруппировка членов
	\begin{equation}\label{eq:l2eq}
		\sum_{j=1}^{N} a_j \sum_{i=1}^{m}\phi_j(x_i)\phi_k(x_i)=\sum_{i=1}^{m} y_i^э \phi_k(x_i)
	\end{equation}
	
\end{frame}

\begin{frame}
	\frametitle{Расчет коэффициентов регрессии}
	Матрица входных переменных:
	\begin{equation}
	\Phi = \begin{bmatrix}
	\phi_1(x_1) & \phi_2(x_1) & \phi_3(x_1) & \dots  & \phi_N(x_1) \\
	\phi_1(x_2) & \phi_2(x_2) & \phi_3(x_2) & \dots  & \phi_N(x_2) \\
	\vdots & \vdots & \vdots & \ddots & \vdots \\
	\phi_1(x_m) & \phi_2(x_m) & \phi_3(x_m) & \dots  & \phi_N(x_m) \\
	\end{bmatrix}
	\end{equation}
	Информационная матрица:
	\begin{equation}
	I_{jk} = \sum_{i=1}^{m} \phi_j (x_i) \phi_k (x_i)
	\end{equation}
	\begin{equation}
	I = \Phi^T \Phi
	\end{equation}
	
\end{frame}

\begin{frame}
	\frametitle{Расчет коэффициентов регрессии}
	Матрица параметров:
	\begin{equation}
	A = \begin{bmatrix}
	a_1  \\
	a_2 \\
	a_3
	\vdots \\
	a_N \\
	\end{bmatrix}
	\end{equation}
	Матрица экспериментальны значений:
	\begin{equation}
	Y^э = \begin{bmatrix}
	y^э_1  \\
	y^э_2 \\
	y^э_3
	\vdots \\
	y^э_N \\
	\end{bmatrix}
	\end{equation}
	
	\begin{equation}
		B=\Phi^T Y^э
	\end{equation}
\end{frame}

\begin{frame}
	\frametitle{Расчет коэффициентов регрессии}
	Тогда искомую систему уравнений \ref{eq:l2eq} можно представить в виде:
	\begin{equation}
		I A = B
	\end{equation}
	Вектор значений параметров определяется согласно уравнению:
	\begin{equation}
		A =I^{-1} B = (\Phi^T \Phi)^{-1} \Phi Y^э
	\end{equation}
	Оценка значимости критериев:
	
\end{frame}

\begin{frame}
	\frametitle{Проверка на адекватность}
	Дисперсия адекватности:
	\begin{equation}
		S_{ад}^2= \dfrac{1}{m-B}\sum_{i=1}{m}(y_i^э-y_i^р)^2
	\end{equation}
	где $B$ --- число значимых коэффициентов регрессии.
	Критерий Фишера:
	\begin{equation}
		F=\dfrac{max(S_{ад}^2, \sigma^2)}{min(S_{ад}^2, \sigma^2)}
	\end{equation}
	Оценка значимости критериев:
	\begin{equation}
		t_i=\dfrac{|a_i|}{\sigma_a}
	\end{equation}
	где  $\sigma_a =\sqrt{\sum_{j=1}^{m} \dfrac{\partial a_i}{\partial y{i}} \sigma^2}$
\end{frame}

\begin{frame}
\frametitle{Схема построения математической модели}
\begin{tikzpicture}[node distance=1.5cm,
every node/.style={fill=white, font=\sffamily}, align=center]
% Specification of nodes (position, etc.)
\node (exp)     [process]          {Эксперимент};
\node (str)      [process, below of=exp]   {Структурная идентифакация};
\node (param)      [process, below of=str]   {Параметрическая идентификация};
\node (zn)      [process, below of=param]   {Проверка значимости параметров};
\node (ad)      [process, below of=zn]   {Проверка адекватности описания};
\node (pr)      [process, below of=ad]   {Применение модели};

\draw[->]             (exp) -- (str);
\draw[->]             (str) -- (param);
\draw[->]             (param) -- (zn);
\draw[->]             (zn) -- (ad);
\draw[->]             (ad) -- (pr);
%\draw[->]             (results) -- (comp);
%\draw[->]             (comp) -- (think);
%\draw[->]             (comp) -- (math);
\draw[->] (ad.west) -- ++(-1.5,0) -- ++(0,4.5) -- (str.west);
\draw[->] (zn.east) -- ++(1.5,0) -- ++(0,3.0) -- (str.east);
%\draw[->] (comp.south) -- ++(0,-1) -- ++(-8.5,0) -- ++(0,2) -- ++(0,2.5) -- (think.west);
%\draw[->] (comp.east) -- ++(2,0) node[xshift=-1cm,yshift=0.4cm]{хорошее} -- ++(0,5.8) -- ++(-8,0) %node[xshift=3.1cm,yshift=0.4cm]{управление, оптимизация, контроль} -- (object.north);

\end{tikzpicture}
\end{frame}

\begin{frame}
	\frametitle{Достоинства и недостатки}
	Достоинства:
	\begin{itemize}
		\item Простота описания
		\item Доступность получения моделей
		\item Возможность изучения при отсутствии теории
	\end{itemize}
	Недостатки:
	\begin{itemize}
		\item Слабые возможности экстраполяции
		\item Необходимость большого количества экспериментов
		\item Отсутствие физического смысла у параметров
		\item Непереносимость на другой аппарат
	\end{itemize}
\end{frame}
\lecture{Лекция 3}{lec3}
	\section{Теоретический метод построения математических моделей}
	
\begin{frame}
	\frametitle{Область применения}
	\begin{itemize}
	\item Законы сохранения массы, импульса, энергии (уравнения балансов). (статика)
	\item Законы переноса массы, энергии, импульса и законы химической кинетики. (кинетика)
	\item Законы термодинамики. Связывают воздействие на вещество с его свойствами. Термодинамические свойства, условия равновесия, кинетические (теплофизические свойства)
	\end{itemize}
\end{frame}

\begin{frame}
\frametitle{Законы сохранения}
\begin{itemize}
	\item Интегральная форма (применительно ко всей системе)
	\item Дифференциальная/локальная форма (применительно к элементарным объемам)
\end{itemize}
\end{frame}


\begin{frame}
\frametitle{Закон сохранения массы}
Интегральная форма:
\begin{equation}
	m_i^{вх} - m_i^{вых} =r_i
\end{equation}
Если нет химической реакции:
\begin{equation}
m_i^{вх} = m_i^{вых} 
\end{equation}

Локальная форма:
\begin{equation}
	\dfrac{\partial m} {\partial t} =r
\end{equation}
Уравнение неразрывности:
\begin{equation}
\dfrac{\partial \rho} {\partial t} + \nabla(\rho \upsilon) =r
\end{equation}
где $\rho$ --- плотность, $\upsilon$ --- скорость движения среды.


\end{frame}

\begin{frame}
\frametitle{Закон сохранения энергии}
Интегральная форма:
\begin{equation}
 \dfrac{d E} {d t} =0
\end{equation}
$E$ --- полная энергия.

Первый закон термодинамики:
\begin{equation}
	d U = \delta Q -p dV
\end{equation}
$U $ --- внутренняя энергия системы.
Теплообмен при постоянном объеме:
\begin{equation}
	d U = \delta Q = c_v dT
\end{equation}
$c_V$ --- изохорная теплоемкость

\end{frame}

\begin{frame}
\frametitle{Закон сохранения энергии}
При теплообмен при постоянном давлении удобно использовать энтальпию:
\begin{equation*}
	H =U +pV
\end{equation*}
\begin{equation*}
	d H = d U +p d V +V d p
\end{equation*}
\begin{equation*}
	d H = \delta Q +V d p
\end{equation*}
\begin{equation*}
	\delta Q = d H = c_p dT
\end{equation*}
\begin{equation*}
  Q = H_1 - H_2 = c_p (T_1 - T_2)
\end{equation*}
$c_p$ --- изобарная теплоемкость

\end{frame}

\begin{frame}
	\begin{figure}[h]
		\includegraphics[width=6cm]{l3-t1.png}
	\end{figure}	
	
	\begin{equation*}
		G_1 c_{p1}(T_{1н}-T_{1k})=G_2 c_{p2}(T_{2н}-T_{2k})
	\end{equation*}
	$G$ --- массовый расход
	\begin{equation*}
	G_1 c_{p1}(T_{1н}-T_{1k})=G_2 c_{p2}(T_{2н}-T_{2k})+Q_r
	\end{equation*}
	$Q_r$ --- внутренний источник/сток тепла
	
	Уравнение Бернулли:
	\begin{equation*}
		\dfrac{\rho \upsilon}{2} + \rho g h + z = const
	\end{equation*}
\end{frame}

\begin{frame}
	\frametitle{Закон сохранения импульса}
	\begin{equation}
	\dfrac{d \vec{p}}{d t }=0
	\end{equation}
	Уравнение Навье-Стокса:
	\begin{equation}
	\dfrac{\partial \vec{\upsilon}}{\partial t} = - (\vec{\upsilon} \nabla) \vec{\upsilon} +\nu \Delta \vec{\upsilon} - \dfrac{1}{\rho} \nabla p = \vec{f}
	\end{equation}
	$\upsilon$ --- поле скорости, $\nu$ --- кинематическая вязкость, $f$ --- поле внешних сил
\end{frame}


\begin{frame}
	\frametitle{Законы переноса}
	\begin{itemize}
	\item Молекулярный перенос субстанции (энергии, массы, импульса) – перенос за счет теплового движения молекул
	\item Конвективный перенос – перенос за счет движения среды как единого целого
	\item Турбулентный перенос – перенос за счет турбулентных пульсаций (вихрей)
	\end{itemize}
\end{frame}

\begin{frame}
	\frametitle{Молекулярный перенос}
	Поток массы:
	\begin{equation}
		\vec{j_i} = -D_{ij}\nabla c_i
	\end{equation}
	$D$ --- коэффициент диффузии, $с$ --- концентрация компонента $i$
	
	Поток тепла:
	\begin{equation}
	\vec{q} = -\lambda\nabla T
	\end{equation}
	$\lambda$ --- коэффициент теплопроводности
	
	Поток импульса:
	\begin{equation}
	\tau_{xy} = -\mu \dfrac{d \upsilon_x}{d y}
	\end{equation}
	$\mu$ --- коэффициент вязкости
	
\end{frame}

\begin{frame}
	\frametitle{Конвективный механизм}
	Поток массы:
	\begin{equation}
		\vec{j_i} = c_i \vec{\upsilon}
	\end{equation}
	Поток тепла:
	\begin{equation}
		\vec{q} = \rho c_p T \vec{\upsilon}
	\end{equation}
	Поток импульса:
	\begin{equation}
		\tau_{xy} = \rho  \upsilon_x \upsilon_y
	\end{equation}
	
\end{frame}

\begin{frame}
	\frametitle{Турбулентный перенос}
	Поток массы:
	\begin{equation}
	\vec{j_i} = -D_{T}\nabla c_i
	\end{equation}
	Поток тепла:
	\begin{equation}
	\vec{q} = -\lambda_T \nabla T
	\end{equation}
	Поток импульса:
	\begin{equation}
	\tau_{xy} = -\mu_T \dfrac{d \upsilon_x}{d y}
	\end{equation}
	$D=f(T,p,x)$ $D_T=f(T,p,x,\upsilon)$
	
\end{frame}


\begin{frame}
\frametitle{Скорость химической реакции}
$v_A A + v_B B \rightarrow v_C C$

$2 H_2 + O_2 \rightarrow  2 H_2 O$

Скорость гомогенной реакции:
	\begin{equation}
		r_A = \dfrac{1}{V} \dfrac{d N_A}{d t}
	\end{equation}
Скорость гетерогенной реакции:
	\begin{equation}
		r_A = \dfrac{1}{F} \dfrac{d N_A}{d t}
	\end{equation}
	
	$r_A \ne r_B \ne r_C$
\end{frame}

\begin{frame}
\frametitle{Уравнения формальной кинетики}
Скорость химической реакции:
\begin{equation}
r_A = k c_A^{n_A}
\end{equation}
$k$ --- константа скорости, $n_i$ --- порядок реакции по компоненту $i$

Уравнение Аррениуса:
\begin{equation}
k=k_0 \exp \left( \dfrac{-E_A}{k_B T} \right)
\end{equation}

$k_0$ --- предэкспоненциальный множитель, $E_A$ --- энергия активации
\end{frame}

\begin{frame}
\frametitle{Исчерпывающее описание}
\begin{itemize}
\item Уравнения сохранения
\item Уравнения переноса
\item Начальные условия (в начальный момент времени или на входе в аппарат)
\item Граничные условия (условия на стенках аппарата и т. д.)
\item Форма аппарата
\item Свойства веществ (более 70\% исходных данных) 
\end{itemize}

\end{frame}

\begin{frame}
\frametitle{Законы термодинамики и термодинамического равновесия}

\begin{equation}
d U = TdS - pdV + \sum_{i=1}^{n} \mu_i d N_i
\end{equation}
$S$ --- энтропия, $\mu$ --- химический потенциал, $N$ --- количество вещества (молей)
\begin{equation}
U = TS - pV + \sum_{i=1}^{n} \mu_i N_i
\end{equation}

\begin{equation}
d U = T dS + S dT - p dV - V dp + \sum_{i=1}^{n} \mu_i d N_i + \sum_{i=1}^{n}  N_i d \mu_i
\end{equation}



\end{frame}

\begin{frame}
Соотношение Гиббса-Дюгема:
\begin{equation}
S dT -V dp + \sum_{i=1}^{n} N_i d \mu_i =0
\end{equation}
При $p=const$, $T=const$:
\begin{equation}
\sum_{i=1}^{n} N_i d \mu_i =0
\end{equation}
\begin{equation}
\sum_{i=1}^{n} x_i d \mu_i =0
\end{equation}


\end{frame}


\begin{frame}

V,T,N
\begin{equation}
F=U-TS=\sum \mu N -pV
\end{equation}
\begin{equation*}
d F=-S dT -p dV + \sum \mu dN 
\end{equation*}
P,T,N
\begin{equation}
G=U-TS+pV=\sum \mu N 
\end{equation}
\begin{equation*}
d G=-S dT -V dp + \sum \mu dN 
\end{equation*}
P,S,N
\begin{equation}
H=U+pV=TS +\sum \mu N 
\end{equation}
\begin{equation*}
d H=T dS +V dp + \sum \mu dN 
\end{equation*}

\end{frame}

\begin{frame}
\frametitle{Соотношения Максвелла}

\begin{equation}
\mu_i=\left.\dfrac{\partial U}{\partial N_i} \right|_{S,V,N} = \left.\dfrac{\partial F}{\partial N_i} \right|_{V,T,N} = \left.\dfrac{\partial G}{\partial N_i} \right|_{p,T,N}=\left.\dfrac{\partial H}{\partial N_i} \right|_{S,p,N}
\end{equation}

\begin{equation}
	p=-\left.\dfrac{\partial F}{\partial V}\right|_{T,N}=-\left.\dfrac{\partial U}{\partial V}\right|_{S,N}
\end{equation}
Уравнение состояния:
\begin{equation}
p=f(T,V,N_1,N_2,..., N_n)
\end{equation}
\begin{equation}
p=f(T,V,x_1,x_2,..., x_{n-1})
\end{equation}

\end{frame}

\begin{frame}
\frametitle{Уравнения состояния}
Идеального газа:
\begin{equation}
	pV=NRT
\end{equation}
\begin{figure}[h]
	\includegraphics[width=7cm]{l3-phase.pdf}
\end{figure}

\end{frame}


\begin{frame}
\frametitle{Уравнения состояния}
Уравнение Ван-дер-Ваальса:
\begin{equation}
p=\dfrac{RT}{V_M -b}-\dfrac{a}{V_M^2}
\end{equation}
$V_M$ --- молярный объем; $a$ --- параметр, учитывающий притяжение молекул; $b$ --- параметр, учитывающий размер молекул.

\begin{figure}[h]
	\includegraphics[width=4cm]{l3-vdw.png}
\end{figure}

\end{frame}

\begin{frame}
\frametitle{Условия поведения в критической точке}
\begin{equation*}
	\dfrac{\partial  p(T,V_M)}{\partial V_M}=0
\end{equation*}
\begin{equation*}
\dfrac{\partial^2  p(T,V_M)}{\partial V_M^2}=0
\end{equation*}


Критические свойства по уравнению Ван-дер-Ваальса
\begin{equation*}
	V_c=3b
\end{equation*}
\begin{equation*}
	p_c = \dfrac{a}{27 b^2}
\end{equation*}
\begin{equation*}
	T_c=\dfrac{8 a}{27 R b}
\end{equation*}


\end{frame}

\begin{frame}
\frametitle{Теория соответственных состояний}
\begin{figure}[h]
	\includegraphics[width=8cm]{l3-phase2.png}
\end{figure}
$p^*=\dfrac{p}{p_c}$;
$T^*=\dfrac{T}{T_c}$;
$V_M^*=\dfrac{V_M}{V_c}$;
$p^*=\dfrac{8T^*}{3V_M^*-1}-\dfrac{3}{V_M^{*2}}$


\end{frame}


\begin{frame}
Уравнение состояния Редлиха-Квонга:
\begin{equation}
p=\dfrac{RT}{V_M -b}-\dfrac{a}{T^{0.5} V_M (V_M+b)}
\end{equation}

Уравнение Бенедикта-Вэбба-Руббина:
\begin{equation}
	p= RT \rho + \left( B_0 RT - A_0 - \dfrac{C_0} {T^2} \right) \rho^2 - (b RT -a )\rho^3+a \alpha \rho^6 + 
	\dfrac { c \rho^3 }  {T^2} ( 1+\gamma \rho^2 ) exp ( - \gamma \rho^2 )
\end{equation}

\end{frame}

\begin{frame}
\frametitle{Уравнение состояния воды}
\begin{figure}[h]
	\includegraphics[width=10cm]{l3-wagner.png}
\end{figure}

\end{frame}

\begin{frame}
\frametitle{Вириальное уравнение состояния:}
\begin{equation}
	p=\dfrac{RT}{V_M}+ B_2(T) \dfrac{1}{V_M^2}+B_3(T) \dfrac{1}{V_M^3} + ...
\end{equation}

B – вириальный коэффициент
\end{frame}

\begin{frame}
\frametitle{Параметры смешения}
$a_m=f_1(x_1,x_2,x_3,...,x_{n-1})$ \quad
$b_m=f_2(x_1,x_2,x_3,...,x_{n-1})$

\begin{equation}
	a_m=\sum_{i=1}^{n} x_i a_i^k
\end{equation}

<<геометрические>> параметры
\begin{equation}
	a_{ij}=\eta \dfrac{a_i+ a_j}{2}
\end{equation}

<<энергетические>> параметры
\begin{equation}
	a_{ij}=\zeta \sqrt{a_i a_j}
\end{equation}

\end{frame}

\begin{frame}
\frametitle{Условия равновесия фаз}
$T'=T''$

$p'=p''$

$\mu_1'=\mu_1''$

$\mu_2'=\mu_2''$

Правило фаз Гиббса:
\begin{equation}
	C=n-\Phi+2
\end{equation}

$C$ --- число степеней свободы,$ n$ --- количество компонентов, $\Phi$ --- количество фаз
\end{frame}

\begin{frame}
$\mu=\dfrac{G}{N}$

$\dfrac{\partial G}{\partial p} = V$

$\int_{\mu_0}^{\mu_1} d \mu = \dfrac{1}{N} \int_{p_0}{p} V dp $

Для идеального газа:
$\mu( T,p ) -\mu_0( T,p_0 )=RT \int_{p_0}^p \dfrac {1}{p} dp = RT ln( p / p_0 )$

Химический потенциал идеального газа:
$\mu( T ) =\mu_0( T )+ RT ln( p )$
Химический потенциал смеси идеальных газов:
$\mu_i( T,P,x ) =\mu_{i0}( T )+ RT ln( p_i )=\mu_{i0}( T )+ RT ln( p x_i )$
\end{frame}

\begin{frame}
\frametitle{Реальный газ}
фугитивность (летучесть) чистого вещества:
$f=p \gamma_f$

фугитивность смеси:
$f_i= p \gamma_i$
,где $\gamma$ --- коэффициент фугитивности

$\mu( T, p ) =\mu^0( T )+ RT \ln( f )$

химический потенциал газовой фазы:
$\mu_i( T, p,x ) =\mu^0_i( T )+ RT \ln( f_i )$

химический потенциал жидкой фазы:
$\mu_i( T, p,x ) =\mu^0_i( T,P )+ RT \ln( a_i )=\mu^0_i( T,P )+ RT \ln( \gamma_i x )$
\end{frame}

\begin{frame}
	$F=\mu N -pV$
	
	$\ln( f )=\ln( p \gamma_f ) = \dfrac { \mu -\mu^0 } {RT} = \dfrac{ F-F^0 } { NRT } + \dfrac{ pV- (pV)^0 } {NRT}$
	
	$pV=Z_V N R T$, где $Z_V =f( T, V )$ --- фактор сжимаемости
	
	$p= - \left. \dfrac{ \partial F } { \partial V }  \right|  _{ T,N_i }$
	
	$F-F^0=-NRT \int_{V_0}^V \dfrac{Z_V}{V} dV=-NRT \left( \int_\infty^V {\dfrac{Z_V-1}{V} dV} +{\int_{V_0}^V {\dfrac{1} {V} dV}}   \right)$
	
	$F-F^0=-NRT \int_{V_0}^V \dfrac{Z_V}{V} dV=-NRT \left( \int_\infty^V {\dfrac{Z_V-1}{V} dV} +\ln \dfrac{Z_V}{p}  \right)$
	
		
\end{frame}

\begin{frame}

$\ln( \gamma_f ) = ( Z_V -1 ) - \int_\infty^V { \dfrac{Z_V -1} {V} dV } -ln( Z_V )$

$Z_V = \dfrac{p(T,V) V} { N R T}$

для многокомпонентной смеси:
$\ln( \gamma_{f_i} ) = -\int_\infty^ V \left( \dfrac{1-x_i}{RNT} \left. \dfrac{\partial p}{\partial x_i}  \right|_{V,T} - \dfrac{1} {V}   \right ) dV - \ln( Z_V )$

$\underbrace{\mu^0_i(T)+RT \ln(f_i)}_ {vapor} = \underbrace{\mu^0_i(T,P)+RT \ln(a_i)}_{liquid} $


$y_i=\dfrac{ f_i^0 \gamma_i x_i } { \gamma_{f_i} p  }$

$x$ --- доля компонента в жидкости, $y$ --- доля компонента в паре.

$y_i=m x_i$

\end{frame}


\begin{frame}

	\begin{equation}
	\left\lbrace 
	\begin{gathered} 
	p^{I}(T,x_1,x_2,...,x_{n-1})=p^{II}(T,y_1,y_2,...,y_n-1),\\
	y_i=\dfrac{ f_i^0 \gamma_i x_i } { \gamma_{f_i} p  } \quad i=1..n \\
	\sum y =1
	\end{gathered} 
	\right.
	\end{equation}
Определение фазового состава:	
	\begin{equation}
	\left\lbrace 
	\begin{gathered} 
	y_i=\dfrac{ f_i^0 \gamma_i x_i } { \gamma_{f_i} p  } \quad i=1..n \\
	\sum y =1
	\end{gathered} 
	\right.
	\end{equation}
\end{frame}

\begin{frame}
	\frametitle{Критерии моделей расчета физико-химических свойств}
	\begin{itemize}
		\item Выдавать надежные результаты для чистых веществ и их смесей во всей области состояний
		\item Адекватно воспроизводить фазовое состояние
		\item Требовать минимальное количество экспериментальных данных
		\item Обладать невысокой вычислительной трудоемкостью
	\end{itemize}
\end{frame}


\begin{frame}
	\frametitle{Расчет термодинамических свойств на основе избыточных функций}
	Функция смешения:
	\begin{equation}
		A^M ( p, T, x_1 , x_2 , ... , x_{n-1} )=A( p,T, x_1 , x_2 , ... , x_{n-1} )- \sum_{i=1}^n x_i A_i^0 ( T,p )
	\end{equation}
		
	Энергия Гиббса смешения:
	\begin{equation}
		\dfrac{G^M} {NRT} = \dfrac{ \sum x_i \mu_i -\sum x_i \mu_i^0 } { RT} = \sum x_i \ln( x_i \gamma_i )
	\end{equation}
	
	
	Избыточная функция:
	\begin{equation}
		A^E ( p, T, x_1 , x_2 , ... , x_{n-1} )=A^M( p,T, x_1 , x_2 , ... , x_{n-1} )- A^M_{id}( p,T, x_1 , x_2 , ... , x_{n-1} )
	\end{equation}
	
\end{frame}

\begin{frame}
	Избыточная энергия Гиббса:
\begin{equation}
\dfrac {G^E}{NRT} = { \sum x_i ln(x_i \gamma_i) - \sum x_i }  = \sum x_i \ln( \gamma_i )
\end{equation}
Требования для модели избыточной энергии Гиббса:
\begin{itemize}
\item Асимптотика поведения $G^E=0$ при $x_1=0$ и $x_2=0$
\item Соблюдение уравнения Гиббса-Дюгема $S dT - V dp + \sum_{ i=1}^n N_i d \mu_i =0$


\end{itemize}
\end{frame}

\begin{frame}
	\begin{figure}[h]
		\includegraphics[width=9cm]{gibbsE.jpg}
	\end{figure}
	
	Excess Gibbs’ Free energy  with mole fraction for 1-iodobutane + benzene ($\square$), 1-iodobutane + toluene ($\circ$), 1-iodobutane + o-xylene ($\triangle$), 1-iodobutane + m-xylene ($\nabla$), 1-iodobutane + p-xylene ($\diamond$), and 1-iodobutane + mesitylene  at 308.15 K.
\end{frame}

\begin{frame}
	\frametitle{Модели активности}
	Модель Маргулиса:
	
	$\dfrac{G^E}{NRT} = A_{12} x_1 x_2$
	$\ln( \gamma_1 )=A_{12} x_2^2$
	$\ln( \gamma_2 )=A_{12} x_1^2$
	
	Модель Вильсона:
	
	$\dfrac{G^E} {NRT} = -\sum_{i=1}^k x_i \ln( \sum_{j=1}^k x_i \lambda_{ij} ) $
	$\lambda_{ij} = \dfrac {V_j}{V_i} e^{ \dfrac{-C_{ij}} {RT} }$
	$\ln( \gamma_i )=1-\ln \left( \sum_{j=1}^k x_j \lambda_{ij} \right) - \sum_{m=1}^k \left(  \dfrac{x_m \lambda_{mi}}{\sum_{j=1}^k x_j \lambda_{mj} } \right)$
	
\end{frame}

\begin{frame}
	\frametitle{Unifac}
	UNIQUAC Functional-group Activity Coefficients
	
	\begin{figure}[h]
		\includegraphics[width=9cm]{l3-unifaq.jpg}
	\end{figure}
\end{frame}

\begin{frame}
	\frametitle{Аппроксимация давления насыщенных паров}
	Соотношение Клапйерона-Клаузиуса:
	
	$\dfrac{\partial p^0( T )} { \partial T } = \dfrac{ \Delta H } { T \Delta V }$
	
	$ \Delta H = H_V -H_L$
	
	$ \Delta V = V_V -V_L$
	
	$\dfrac{\partial \ln( p^0( T ))} { \partial T } = \dfrac{ \Delta H } { R \Delta Z_V }$
	
	Уравнение Клапейрона:
	 $\ln( p^0( T ))  = A + \dfrac{B} {T}$
	 
	 Уравнение Антуана:
	 $ \ln( p^0( T ))  = A - \dfrac{B} {C+T}$
	 
	 Уравнение Риделя:
	 $ \ln( p^0( T ))  = A - \dfrac{B} {T} + C \ln( T ) + D T^E$
\end{frame}


\begin{frame}
	\frametitle{Известно $x_1$-$x_n,$ T }
	\begin{itemize}
		\item Для заданной температуры рассчитываются давления паров чистых компонентов $p^0_i$ , например по уравнению Антуана
		\item Для заданного состава и температуры рассчитываются коэффициенты активности по выбранной модели. Несмотря на то, что коэффициенты активности зависят от состава, температуры и давления, на практике используются модели не учитывающие влияние давления
		\item Рассчитывают парциальные давления компонентов смеси $p_i = p_i^0( T ) \gamma_i x_i$. И полное давление в системе: $p= \sum p_i$
		\item Определяют равновесные концентрации в паровой фазе: $y_i = \dfrac{p_i} {p}$
	\end{itemize}
\end{frame}

\begin{frame}
	\frametitle{Известно $x_1$-$x_n$, $p_{cur}$ }
	\begin{itemize}
		\item Задается начальное приближение по температуре
		\item Проводятся расчеты по первым трем пунктам предыдущего алгоритма до выполнения условия $p=p_{cur}$
		\item При невыполнении условия изменяют приближение температуры и делают новую итерацию
		\item Определяют равновесные концентрации в паровой фазе: $y_i = \dfrac {p_i} {p}$
		
	\end{itemize}
	
\end{frame}

\begin{frame}
	\frametitle{Химическое равновесие}
	$v_A A +v_B B \leftrightarrow v_C C$
	
	Химическое сродство:
	$C=v_A \mu_A +v_B \mu_B -v_c \mu_C $
	
	$v_A (\mu^0_A(T) +RT \ln(a_A)) v_B (\mu^0_B(T) + RT \ln(a_B)) -v_C (\mu^0_C(T) +RT \ln(a_C))  =0$
	
	$\dfrac { a_C^{v_C} } { a_A^{v_A} a_B^{v_B} } = exp( v_A \mu_A^0 +v_B \mu_B^0 -v_c \mu_C^0 )=K( T )$
	
	Вариант 2
	
	$r_1=k_1 c_A c_B$
	
	$r_2=k_2 c_C$
	
	$r_1=r_2=k_2 c_C=k_1 c_A c_B$
	
	$K( T )=\dfrac {k_1} {k_2}$
\end{frame}

\lecture{Лекция 4}{lec4}
	\section{Теоретический метод построения математических моделей}
	
\begin{frame}
	\frametitle{Поле скорости}
	
	\begin{figure}[h]
		\includegraphics[width=9cm]{l4-struct.png}
	\end{figure}
\end{frame}	

\begin{frame}
	\frametitle{Элементарная ячейка}
	
	\begin{figure}[h]
		\includegraphics[width=9cm]{l4-box.png}
	\end{figure}
\end{frame}	



\begin{frame}
	\frametitle{Материальный баланс}
	Уравнение материального баланса для выделенного объема:
	\begin{equation}
		j dS +j_{12} dS -( j+d j )dS -r_i dV = \dfrac { \partial C_i }  { \partial \tau } dV
	\end{equation}
	где $r_i=k \sum_i C_i^{n_i}$ --- скорость химической реакции
	\begin{equation}
		\dfrac { \partial C_i } { \partial \tau } + \dfrac { \partial j_i } { \partial x } = j_{12} \dfrac { dS } { dV  } - r_i
	\end{equation}
	
	Поток массы при конвективном и молекулярном механизме запишется в виде:
	\begin{equation}
		j_i = \upsilon C_i -D \dfrac{ d C_i } { d x }
	\end{equation}
	
	Межфазный поток запишется в виде:
	\begin{equation}
		j_{12} = K ( C_i - C_i^* )
	\end{equation}
\end{frame}	

\begin{frame}
	\begin{equation}
	\dfrac { \partial C_i } { \partial \tau } + \upsilon \dfrac { \partial C_i }  { \partial x } = D \dfrac { \partial ^2 C_i } {\partial { x^2 }} + K a ( C_i - C_i^* ) - k \sum C_i^n
	\end{equation}
	где $a=\frac{d S}{d V}$ --- удельный объем, $i = K a ( C_i - C_i^* ) -k C_i^n$ --- источник/сток массы.
	
	Диффузионная модель:
	\begin{equation}
	\dfrac { \partial C_i } { \partial \tau } + \upsilon \dfrac { \partial C_i } { \partial x } = D_L \dfrac{ \partial ^2 C_i } {\partial { x^2 }}+ i_i
	\end{equation}
	где $D_L$ --- коэффициент обратного перемешивания.
	
	
\end{frame}	

\begin{frame}
	В стационарных условиях:
	\begin{equation}
		\upsilon \dfrac{ \partial C_i } { \partial x } = D_L \dfrac { \partial ^2 C_i } {\partial { x^2 }}+ i_i
	\end{equation}
	
	Начальные и граничные условия:
	$C_{0i} = C_i( \tau_0 , x )$
	
	$\upsilon ( C_i -C_{in} ) = -D_L \left. \dfrac {d C_i} {dx} \right| _ { x=0 }$
	
	$\left. \dfrac {dC_i} {dx} \right| _ { x=L } =0$
	
	
	\begin{figure}[h]
		\includegraphics[width=5cm]{l4-bound.png}
	\end{figure}
\end{frame}	

\begin{frame}
	Безразмерные переменные
	$X= \dfrac {x} {L}$
	
	$\bar \tau = \dfrac {V_ap} {G} = \dfrac { S L }  { S \upsilon } = \dfrac {L} {\upsilon}$
	
	$\theta= \dfrac { \tau } { \bar{\tau} }$
	
	\begin{equation}
		\dfrac {\partial C_i} {\partial \theta } + \dfrac{ \partial C_i } { \partial X } = \dfrac {1}  { Pe_L } \dfrac { \partial^2 C_i }  { \partial X } + i \bar{\tau}
	\end{equation}
	
		
		
	Двухпараметрическая диффузионная модель
	\begin{equation}
		\dfrac{ \partial C_i } { \partial \tau } + \upsilon \dfrac{ \partial C_i } { \partial x } = D \dfrac { \partial ^2 C_i } {\partial { x^2 }} D_R \left( \dfrac{\partial ^2 C_i}{\partial r_i^2}+ \dfrac {1} {r} \dfrac {\partial C_i} {\partial r} \right)+i
	\end{equation}
	
\end{frame}

\begin{frame}
	Начальные и граничные условия:
	$\upsilon ( C_i -C_{in} ) = -D_L \left. \dfrac {d C_i} {dx} \right| _ { x=0 }$
	
	$\left. \dfrac {dC_i}{dr} \right| _ { x=0 } =0$
	
	$\left. \dfrac { dC_i} {dx} \right| _ { x=L } =0$
	
	$\left. \dfrac {dC_i} {dr} \right| _ { x=L } =0$
	
	
	\begin{figure}[h]
		\includegraphics[width=5cm]{l4-bound2.png}
	\end{figure}
\end{frame}

\begin{frame}
	\frametitle{Модель идеального вытеснения (МИВ)}
	\begin{equation}
		\dfrac{ \partial C_i } { \partial \tau } + \upsilon \dfrac { \partial C_i } { \partial x } = i_i
	\end{equation}
	Начальные и граничные условия:
	$C_{0i} = C_i( \tau_0 , x )$, $C=C_{in}$
	
	
\end{frame}

\begin{frame}
	\frametitle{Модель идеального смешения (МИС)}

	\begin{equation}
	\dfrac { \partial C_i }  { \partial \tau } + \dfrac {C_i - C_{in}} { \bar{\tau} } = i_i
	\end{equation}
	
	Стационарный режим:
	\begin{equation}
		\dfrac {C_i - C_{in}} { \bar{\tau} } = i_i
	\end{equation}
		
\end{frame}

\begin{frame}
	\frametitle{Ячеечная модель}
	\begin{figure}[h]
		\includegraphics[width=8cm]{l4-cells.png}
	\end{figure}
	
	\begin{itemize}
		\item В каждой ячейке структура идеального смешения
		\item Перемешивание между ячейками отсутствует
		\item Объемный расход не изменяется
		\item Объемы ячеек одинаковые
		\item Сумма объемов ячеек равна объему аппарата
		\item Среднее время пребывания в ячейке $\bar{\tau}_j = \dfrac{\bar{\tau}} {m}  $
		\item Среднее пребывание в системе $\bar{\tau} =\dfrac{ V_ап } { \dot{V} }$
	\end{itemize}
	
\end{frame}

\begin{frame}
	\begin{equation}
	\left\lbrace 
	\begin{gathered} 
	\dfrac { \partial C_1 } {\partial \tau} + \dfrac{C_{in} - C_1} {\bar{\tau_1}} =i_1  \\
	\dfrac { \partial C_2 } {\partial \tau} + \dfrac{C_{in} - C_2} {\bar{\tau_2}} =i_2 \\
	... \\
	\dfrac { \partial C_n } {\partial \tau} + \dfrac{C_{in} - C_n} {\bar{\tau_n}} =i_n
	\end{gathered} 
	\right.
	\end{equation}
\end{frame}

\begin{frame}
	\frametitle{Импульсный ввод индикатора для определения параметров типовых и комбинаторных моделей}
	
	
	\begin{figure}[h]
		\includegraphics[width=9cm]{l4-cout.png}
	\end{figure}
\end{frame}

\begin{frame}
	Функция распределения частиц по времени пребывания:
	\begin{equation}
		f( \tau ) = \dfrac{ dN (\tau) } { N dt } = \dfrac {C( \tau )} { \int_0^\infty C(\tau) d \tau }
	\end{equation}
	$dN(\tau)$ --- количество элементов потока, время пребывания которых составляет от $\tau$ до $\tau + d \tau$; $N $ --- общее количество элементов.
	
	$\int_0^\infty f( \tau ) d \tau = 1$

	$\int_0^\infty C( \tau ) d \tau = m$
\end{frame}

\begin{frame}
	Общий вид начальных моментов:
	\begin{equation}
		M_n = \int_0^\infty \tau^n f( \tau ) d \tau
	\end{equation}
	Общий вид центральных моментов:
	\begin{equation}
		\mu_n=\int_0^\infty ( \tau - \bar{\tau} )^n f( \tau)d\tau
	\end{equation}
	\begin{equation}
		\sigma^2= \mu_2=\int_0^\infty (\tau - \bar{\tau} )^2 f(\tau)d\tau
	\end{equation}	
		
	Безразмерный вид:
	$\theta = \dfrac{\tau} {\bar{\tau}}$
	
	$f^*(\theta ) = \bar{\tau} f(\tau )$
	
	$\sigma_\theta^2=\dfrac{\sigma^2} {\bar{\tau} }^2$
	
\end{frame}

\begin{frame}
	Ячеечная модель:
	\begin{equation}
		\dfrac{1} {m} = \sigma^2_\theta
	\end{equation}
	Диффузионная модель:
	\begin{equation}
		\sigma^2_\theta = \dfrac {2}{Pe} - \dfrac{2 (1-e^{-Pe}  )} {Pe^2}
	\end{equation}
	
	Алгоритм импульсного ввода индикатора:
	\begin{itemize}
		\item Проводится эксперимент методом импульсного ввода индикатора
		\item Определяется кривая отклика
		\item По кривой отклика находится функция распределения
		\item Определяются параметры модели
	\end{itemize}
	
\end{frame}

\begin{frame}
	\begin{figure}[h]
		\includegraphics[width=9cm]{l4-cells2.png}
	\end{figure}
	\begin{equation}
		V_i \dfrac { dC_i } { d \tau } = \dot {V} C_{i-1} + e C_{ i+1 } -(\dot{V} +e  ) C_i
	\end{equation}

	\begin{equation}
		M_2^\theta=1+\dfrac { m(1-x)^2-2x(1-x^m) } { m^2 (1-x)^2 }
	\end{equation}
	где $M $ --- безразмерный центральный момент
	
	$f=\dfrac {e}{\dot{V}}$, 	$x=\dfrac {f} { 1+f }$
	\begin{equation}
	M_3^\theta=1+\dfrac {2} {m^2} + \dfrac{ 6x(1+3^m)+3m(1-x^2) } { m^2 (1-x)^2 } - \dfrac { 12(1+x)(1-x^m) } { m^3 (1-x^3) }
	\end{equation}
\end{frame}

\begin{frame}
\frametitle{Комбинированные модели}
	\begin{figure}[h]
		\includegraphics[width=9cm]{l4-comb.png}
	\end{figure}
\end{frame}

\begin{frame}
	Застойные зоны определяются из соотношения:
	\begin{equation}
		\bar{ \tau_U} = \dfrac{ \int \tau C(\tau) d \tau } { \int C(\tau) d \tau }  \neq \dfrac{V_{ap}} {\dot{V}} = \bar{\tau}
	\end{equation}
	\begin{equation}
		\bar{\tau} = \dfrac{V_{ap}} {\dot{V}} = \dfrac {V_{short}} {\dot{ V}}+ \dfrac{V_{dead}} {\dot{V}} = \bar{\tau}_u + \dfrac{V_{dead}} {\dot{V}}
	\end{equation}		
	$V_{short}$, $V_{dead}$ --- объем проточной и застойной зон
	$bar \tau_U < \dfrac {V_{ap}}  {\dot{V}}$

\end{frame}

\begin{frame}
	\frametitle{Bypass}
	\begin{figure}[h]
		\includegraphics[width=9cm]{l4-bypass.png}
	\end{figure}
	Случай, когда индикатор не попадает в байпас:
	\begin{equation}
		\bar {\tau_U} = \dfrac { \int \tau C(\tau) d \tau }  { \int C(\tau) d \tau }  = \dfrac {V_{ap}} {\dot{V_2}}
	\end{equation}	
	
	\begin{equation}
		\bar{ \tau_U} = \dfrac {V_{ap}} { \left( 1- \dfrac {\dot{V_1}} {\dot V} \right) \dot V }
	\end{equation}
\end{frame}

\begin{frame}
Доля байпассного потока:
\begin{equation}
	a=1- \dfrac { \bar \tau } { \bar \tau_U }
\end{equation}
	\begin{figure}[h]
		\includegraphics[width=9cm]{l4-bypass2.png}
	\end{figure}
	\begin{figure}[h]
		\includegraphics[width=5cm]{l4-rasp.png}
	\end{figure}
\end{frame}

\begin{frame}
	Массы поступающие в байпасный поток и аппарат:
	$m_1= \dot{V} \int_0^{\tau_1} C( \tau ) d\tau$
	
	
	$m_2=\dot{V} \int_{\tau_1}^{\infty} C( \tau ) d\tau$
	
	
	$m_1=\dfrac { (m_1+m_2) \dot V_1 } { \dot{V}} = (m_1+m_2) a$
	
	$m_2=\dfrac { (m_1+m_2) \dot V_2 } {\dot{V}} =(m_1+m_2)( 1-a )$
\end{frame}

\begin{frame}
	\frametitle{Комбинированные потоки из параллельно соединенных зон }
	\begin{figure}[h]
		\includegraphics[width=8cm]{l4-comb2.png}
	\end{figure}
	
	$\dot {V_1} C_1 + \dot {V_2} C_2 = \dot{V} C$
	
	$C=C_1 \dfrac { \dot{ V_1} } { \dot{V} } + C_2 \dfrac{ \dot {V_2} } { \dot{V} }$
\end{frame}


\begin{frame}
	
	\begin{figure}[h]
		\includegraphics[width=8cm]{l4-rasp2.png}
	\end{figure}
	
\end{frame}

\begin{frame}
	\frametitle{Ориентировочные области применения различных моделей структуры потока в аппарате}
	
	\begin{tabular}{| p{2cm} | p{8cm} | }
		\hline
		МИВ & Трубчатые аппараты с большим соотношением длинны к диаметру  \\ \hline
		МИС & Цилиндрические аппараты со сферическим дном в условиях интенсивного перемешивания, аппараты с отражательными перегородками, барботажные аппараты   \\ \hline
		ЯМ & Каскады реакторов с мешалками, тарельчатые колонны, аппараты с псевдоожиженным слоем  \\ \hline
		ЯМ с обратными потоками & Тарельчатые и секционированные насадочные аппараты, пульсационные аппараты \\ \hline
		ДМ & Трубчатые аппараты, аппараты колонного типа с насадкой, и с осевым рассеиванием вещества \\ \hline
	\end{tabular}
\end{frame}
\lecture{Лекция 5}{lec5}
\section{Моделирование теплообменных процессов}

\begin{frame}
	\frametitle{Теплообмен}
	Задача: определение поле температуры теплоносителей $T(\tau, x, y, z)$
	
	Теплообменник типа труба в трубе:
	\begin{figure}[h]
		\includegraphics[width=8cm]{l5-tepl1.png}
	\end{figure}
\end{frame}

\begin{frame}
\frametitle{Уравнение теплового баланса}
Уравнение теплового баланса для первого горячего теплоносителя:
\begin{equation}
	\upsilon_1 T_1 \rho_1 c_{p1} S_1 - \upsilon_1 (T_1-dT_1) \rho_1 c_{p1} S_1 -q_{12} dF =0
\end{equation}
Элементарный объем: $dV_1=S_1 dx$

Тепловой поток межфазной поверхности:
\begin{equation}
	q_{12}=K ( T_1 - T_2 )
\end{equation}
K --- коэффициент теплопередачи

\begin{equation}
	\upsilon_1  \dfrac{ dT_1 }  { dx } = - \dfrac { K (T_1 -T_2)  }  { \rho_1 c_{p1} } \dfrac { d F } { S_1 dx }
\end{equation}
$ \dfrac {dF} { d x } = \dfrac { P L } {L} = P$
, где $P$ --- периметр теплообмена
\end{frame}

\begin{frame}
Система уравнений для двух теплоносителей:

\begin{equation}
\left\lbrace 
\begin{gathered} 
\dfrac {dT_1} {dx} = \dfrac {-K (T_1-T_2)} {G_1 c_{p1}} P \\
\dfrac {dT_2} {dx} = \dfrac {K (T_1-T_2)} {G_2 c_{p2}} P
\end{gathered} 
\right.
\end{equation}
где $G$ --- массовый расход
Граничные условия:
\begin{itemize}
\item Прямоток (задача Коши)
$T_1( 0 ) =T_{1 in}$,  $T_2( 0 ) =T_{2 in}$
\item Противоток (краевая задача)
$T_1( 0 ) =T_{1 in}$,  $T_1( L ) =T_{1 in}$

\end{itemize}
\end{frame}

\begin{frame}
	\frametitle{Противоток}
	Тепловой баланс второго потока в случае противотока:
	\begin{equation}
	-\upsilon_2 T_2 \rho_2 c_{p2} S_2 + \upsilon_2 (T_2-d T_2) \rho_2 c_{p2} S_2 +q_{12} dF =0
	\end{equation}
	
	\begin{equation}
	\left\lbrace 
	\begin{gathered} 
	\dfrac {dT_1} {dx} = \dfrac {-K (T_1-T_2)} {G_1 c_{p1}} P \\
	\dfrac {dT_2} {dx} = \dfrac {-K (T_1-T_2)} {G_2 c_{p2}} P
	\end{gathered} 
	\right.
	\end{equation}
	
\end{frame}

\begin{frame}
\frametitle{Диффузионная модель}
Уравнение теплового баланса для диффузионной модели:
%\begin{equation}
	\begin{multline}
	\upsilon_1 T_1 \rho_1 c_{p1} S_1 -D_L \dfrac {dT} {dx} S_1 \rho_1 c_{p1} - \upsilon_1 (T_1-dT_1) \rho_1 c_{p1} S_1 + \\   +D_L \dfrac {d(T_1+dT_1)} {dx} S_1 \rho_1 c_{p1}-q_{12} dF =0
	\end{multline}
%\end{equation}

\begin{equation}
\left\lbrace 
\begin{gathered} 
\upsilon_1 \dfrac {dT_1} {dx} = D_L1 \dfrac {d^2 T_1} {dx^2} - \dfrac {K(T_1-T_2) P} {\rho_1 c_{p1} S_1} \\
\upsilon_2 \dfrac {dT_1} {dx} = D_L2 \dfrac {d^2 T_2} {dx^2} + \dfrac {K(T_1-T_2) P} {\rho_2 c_{p2} S_2} 
\end{gathered} 
\right.
\end{equation}

\end{frame}

\begin{frame}
	\frametitle{Граничные условия}
Для решения системы записываются 2 граничных условия для каждого уравнения:
 
$x=0$ ; \quad $\upsilon_1 ( T_{in} - T_1 ) = -D_{L1} \left.  \dfrac { d T_1 } { dx }  \right| _{ x=0 }$

$x=L$ \quad $ \left. \dfrac { d T_1 } { dx }  \right|_{ x=L } =0$


\begin{figure}[h]
	\includegraphics[width=8cm]{l5-gr1.png}
\end{figure}
\end{frame}

\begin{frame}
	\frametitle{Противоток}
	\begin{equation}
	\left\lbrace 
	\begin{gathered} 
	\upsilon_1 \dfrac {dT_1} {dx} = D_L1 \dfrac {d^2 T_1} {dx^2} - \dfrac {K(T_1-T_2) P} {\rho_1 c_{p1} S_1} \\
	\upsilon_2 \dfrac {dT_1} {dx} = - D_L2 \dfrac {d^2 T_2} {dx^2} - \dfrac {K(T_1-T_2) P} {\rho_2 c_{p2} S_2} 
	\end{gathered} 
	\right.
	\end{equation}
	Для решения системы записываются 2 граничных условия для каждого уравнения:
	
	$x=0$ ; \quad $\upsilon_1 ( T_{in1} - T_1 ) = -D_{L1} \left.  \dfrac { d T_1 } { dx }  \right| _{ x=0 }$
	
	$x=L$ \quad $ \left. \dfrac { d T_1 } { dx }  \right|_{ x=L } =0$
	
	$x=L$ ; \quad $\upsilon_2 ( T_{in2} - T_2 ) = -D_{L2} \left.  \dfrac { d T_2 } { dx }  \right| _{ x=L }$
	
	$x=0$ \quad $ \left. \dfrac { d T_2 } { dx }  \right|_{ x=0 } =0$
	
\end{frame}

\begin{frame}
	\frametitle{МИС}
	Уравнение теплового баланса для МИС:
	\begin{equation}
		T_1-T_{in 1}=\dfrac { K(T_1 - T_{P} )F} { G_1 c_{p1} }
	\end{equation}
	
	\begin{figure}[h]
		\includegraphics[width=5cm]{l5-mis.png}
	\end{figure}
\end{frame}

\begin{frame}
	\frametitle{Ячеечная модель}
	Уравнение теплового баланса для ячеечной модели:
	\begin{equation}
		G c_p ( T_{1,j} - T_{1,j-1} ) = K_j \dfrac {F} {m} ( T_P - T_{1,j} ) \quad j=1,m
	\end{equation}
\end{frame}


\begin{frame}
	\frametitle{Сравнение различных моделей}
	\begin{figure}[h]
	\includegraphics[width=10cm]{l5-models.png}
	\end{figure}
\end{frame}

\begin{frame}
	\frametitle{Коэффициент теплопередачи}
	\begin{equation}
		K = \dfrac {q}  { F \Delta T }
	\end{equation}
	\begin{equation}
		K = \dfrac {1} { \dfrac{1} {\alpha_1} + \sum{\dfrac {\delta} {\lambda}} + \dfrac {1} {\alpha_2} }
	\end{equation}
	, где $\alpha$ --- коэффициент теплоотдачи
	
	Коэффициент теплоотдачи зависит от:
	\begin{itemize}
		\item Режима течения $Re$, (скорость, вязкость, форма аппарата)
		\item Теплофизических свойств $Pr$ (теплоемкость, теплопроводность …)
		\item Геометрии аппарата
	\end{itemize}
	
\end{frame}

\begin{frame}
	Коэффициент теплоотдачи:
	\begin{equation}
		Nu = \dfrac { \alpha l } {\lambda}
	\end{equation}
	Если движение в трубе (канале) носит характер переходного режима, т.е. $Re = 2300 - 10000$, то критерий Нуссельта:
	
	Для ламинарного движения ( Re < 2300):
	
	$Nu = 0.08 Re^{0.9} Pr^{0.43}$
	
	Для ламинарного движения ( Re < 2300):
	$Nu =a ( Re Pr )^{0.2} ( Gr Pr )^{0.1}$
	где $a$ --- множитель (для горизонтальных труб d = 0,74; для вертикальных труб a= 0,85)
\end{frame}


\end{document}